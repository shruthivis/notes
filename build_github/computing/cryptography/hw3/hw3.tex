\documentclass[10pt]{article}
\usepackage{amssymb, algorithm2e, amsthm, amsmath}

% Use something like:
% % Use something like:
% % Use something like:
% \input{../../macros}

% groupings of objects.
\newcommand{\set}[1]{\left\{ #1 \right\}}
\newcommand{\seq}[1]{\left(#1\right)}
\newcommand{\ang}[1]{\langle#1\rangle}
\newcommand{\tuple}[1]{\left(#1\right)}

% numerical shortcuts.
\newcommand{\abs}[1]{\left| #1\right|}
\newcommand{\floor}[1]{\left\lfloor #1 \right\rfloor}
\newcommand{\ceil}[1]{\left\lceil #1 \right\rceil}

% linear algebra shortcuts.
\newcommand{\change}{\Delta}
\newcommand{\norm}[1]{\left\| #1\right\|}
\newcommand{\dprod}[1]{\langle#1\rangle}
\newcommand{\linspan}[1]{\langle#1\rangle}
\newcommand{\conj}[1]{\overline{#1}}
\newcommand{\gradient}{\nabla}
\newcommand{\der}{\frac{d}{dx}}
\newcommand{\lap}{\Delta}
\newcommand{\kron}{\otimes}
\newcommand{\nperp}{\nvdash}

\newcommand{\mat}[1]{\left( \begin{smallmatrix}#1 \end{smallmatrix} \right)}

% derivatives and limits
\newcommand{\partder}[2]{\frac{\partial #1}{\partial #2}}
\newcommand{\partdern}[3]{\frac{\partial^{#3} #1}{\partial #2^{#3}}}

% Arrows
\newcommand{\diverge}{\nearrow}
\newcommand{\notto}{\nrightarrow}
\newcommand{\up}{\uparrow}
\newcommand{\down}{\downarrow}
% gets and gives are defined!

% ordering operators
\newcommand{\oleq}{\preceq}
\newcommand{\ogeq}{\succeq}

% programming and logic operators
\newcommand{\dfn}{:=}
\newcommand{\assign}{:=}
\newcommand{\co}{\ co\ }
\newcommand{\en}{\ en\ }


% logic operators
\newcommand{\xor}{\oplus}
\newcommand{\Land}{\bigwedge}
\newcommand{\Lor}{\bigvee}
\newcommand{\finish}{$\Box$}
\newcommand{\contra}{\Rightarrow \Leftarrow}
\newcommand{\iseq}{\stackrel{_?}{=}}


% Set theory
\newcommand{\symdiff}{\Delta}
\newcommand{\union}{\cup}
\newcommand{\inters}{\cap}
\newcommand{\Union}{\bigcup}
\newcommand{\Inters}{\bigcap}
\newcommand{\nullSet}{\phi}

% graph theory
\newcommand{\nbd}{\Gamma}

% Script alphabets
% For reals, use \Re

% greek letters
\newcommand{\eps}{\epsilon}
\newcommand{\del}{\delta}
\newcommand{\ga}{\alpha}
\newcommand{\gb}{\beta}
\newcommand{\gd}{\del}
\newcommand{\gf}{\phi}
\newcommand{\gF}{\Phi}
\newcommand{\gl}{\lambda}
\newcommand{\gm}{\mu}
\newcommand{\gn}{\nu}
\newcommand{\gr}{\rho}
\newcommand{\gs}{\sigma}
\newcommand{\gt}{\theta}
\newcommand{\gx}{\xi}

\newcommand{\sw}{\sigma}
\newcommand{\SW}{\Sigma}
\newcommand{\ew}{\lambda}
\newcommand{\EW}{\Lambda}

\newcommand{\Del}{\Delta}
\newcommand{\gD}{\Delta}
\newcommand{\gG}{\Gamma}
\newcommand{\gO}{\Omega}
\newcommand{\gL}{\Lambda}
\newcommand{\gS}{\Sigma}

% Formatting shortcuts
\newcommand{\red}[1]{\textcolor{red}{#1}}
\newcommand{\blue}[1]{\textcolor{blue}{#1}}
\newcommand{\htext}[2]{\texorpdfstring{#1}{#2}}

% Statistics
\newcommand{\distr}{\sim}
\newcommand{\stddev}{\sigma}
\newcommand{\covmatrix}{\Sigma}
\newcommand{\mean}{\mu}
\newcommand{\param}{\gt}
\newcommand{\ftr}{\phi}

% General utility
\newcommand{\todo}[1]{\footnote{TODO: #1}}
\newcommand{\exclaim}[1]{{\textbf{\textit{#1}}}}
\newcommand{\tbc}{[\textbf{Incomplete}]}
\newcommand{\chk}{[\textbf{Check}]}
\newcommand{\oprob}{[\textbf{OP}]:}
\newcommand{\core}[1]{\textbf{Core Idea:}}
\newcommand{\why}{[\textbf{Find proof}]}
\newcommand{\opt}[1]{\textit{#1}}


\DeclareMathOperator*{\argmin}{arg\,min}
\DeclareMathOperator{\rank}{rank}
\newcommand{\redcol}[1]{\textcolor{red}{#1}}
\newcommand{\bluecol}[1]{\textcolor{blue}{#1}}
\newcommand{\greencol}[1]{\textcolor{green}{#1}}


\renewcommand{\~}{\htext{$\sim$}{~}}


% groupings of objects.
\newcommand{\set}[1]{\left\{ #1 \right\}}
\newcommand{\seq}[1]{\left(#1\right)}
\newcommand{\ang}[1]{\langle#1\rangle}
\newcommand{\tuple}[1]{\left(#1\right)}

% numerical shortcuts.
\newcommand{\abs}[1]{\left| #1\right|}
\newcommand{\floor}[1]{\left\lfloor #1 \right\rfloor}
\newcommand{\ceil}[1]{\left\lceil #1 \right\rceil}

% linear algebra shortcuts.
\newcommand{\change}{\Delta}
\newcommand{\norm}[1]{\left\| #1\right\|}
\newcommand{\dprod}[1]{\langle#1\rangle}
\newcommand{\linspan}[1]{\langle#1\rangle}
\newcommand{\conj}[1]{\overline{#1}}
\newcommand{\gradient}{\nabla}
\newcommand{\der}{\frac{d}{dx}}
\newcommand{\lap}{\Delta}
\newcommand{\kron}{\otimes}
\newcommand{\nperp}{\nvdash}

\newcommand{\mat}[1]{\left( \begin{smallmatrix}#1 \end{smallmatrix} \right)}

% derivatives and limits
\newcommand{\partder}[2]{\frac{\partial #1}{\partial #2}}
\newcommand{\partdern}[3]{\frac{\partial^{#3} #1}{\partial #2^{#3}}}

% Arrows
\newcommand{\diverge}{\nearrow}
\newcommand{\notto}{\nrightarrow}
\newcommand{\up}{\uparrow}
\newcommand{\down}{\downarrow}
% gets and gives are defined!

% ordering operators
\newcommand{\oleq}{\preceq}
\newcommand{\ogeq}{\succeq}

% programming and logic operators
\newcommand{\dfn}{:=}
\newcommand{\assign}{:=}
\newcommand{\co}{\ co\ }
\newcommand{\en}{\ en\ }


% logic operators
\newcommand{\xor}{\oplus}
\newcommand{\Land}{\bigwedge}
\newcommand{\Lor}{\bigvee}
\newcommand{\finish}{$\Box$}
\newcommand{\contra}{\Rightarrow \Leftarrow}
\newcommand{\iseq}{\stackrel{_?}{=}}


% Set theory
\newcommand{\symdiff}{\Delta}
\newcommand{\union}{\cup}
\newcommand{\inters}{\cap}
\newcommand{\Union}{\bigcup}
\newcommand{\Inters}{\bigcap}
\newcommand{\nullSet}{\phi}

% graph theory
\newcommand{\nbd}{\Gamma}

% Script alphabets
% For reals, use \Re

% greek letters
\newcommand{\eps}{\epsilon}
\newcommand{\del}{\delta}
\newcommand{\ga}{\alpha}
\newcommand{\gb}{\beta}
\newcommand{\gd}{\del}
\newcommand{\gf}{\phi}
\newcommand{\gF}{\Phi}
\newcommand{\gl}{\lambda}
\newcommand{\gm}{\mu}
\newcommand{\gn}{\nu}
\newcommand{\gr}{\rho}
\newcommand{\gs}{\sigma}
\newcommand{\gt}{\theta}
\newcommand{\gx}{\xi}

\newcommand{\sw}{\sigma}
\newcommand{\SW}{\Sigma}
\newcommand{\ew}{\lambda}
\newcommand{\EW}{\Lambda}

\newcommand{\Del}{\Delta}
\newcommand{\gD}{\Delta}
\newcommand{\gG}{\Gamma}
\newcommand{\gO}{\Omega}
\newcommand{\gL}{\Lambda}
\newcommand{\gS}{\Sigma}

% Formatting shortcuts
\newcommand{\red}[1]{\textcolor{red}{#1}}
\newcommand{\blue}[1]{\textcolor{blue}{#1}}
\newcommand{\htext}[2]{\texorpdfstring{#1}{#2}}

% Statistics
\newcommand{\distr}{\sim}
\newcommand{\stddev}{\sigma}
\newcommand{\covmatrix}{\Sigma}
\newcommand{\mean}{\mu}
\newcommand{\param}{\gt}
\newcommand{\ftr}{\phi}

% General utility
\newcommand{\todo}[1]{\footnote{TODO: #1}}
\newcommand{\exclaim}[1]{{\textbf{\textit{#1}}}}
\newcommand{\tbc}{[\textbf{Incomplete}]}
\newcommand{\chk}{[\textbf{Check}]}
\newcommand{\oprob}{[\textbf{OP}]:}
\newcommand{\core}[1]{\textbf{Core Idea:}}
\newcommand{\why}{[\textbf{Find proof}]}
\newcommand{\opt}[1]{\textit{#1}}


\DeclareMathOperator*{\argmin}{arg\,min}
\DeclareMathOperator{\rank}{rank}
\newcommand{\redcol}[1]{\textcolor{red}{#1}}
\newcommand{\bluecol}[1]{\textcolor{blue}{#1}}
\newcommand{\greencol}[1]{\textcolor{green}{#1}}


\renewcommand{\~}{\htext{$\sim$}{~}}


% groupings of objects.
\newcommand{\set}[1]{\left\{ #1 \right\}}
\newcommand{\seq}[1]{\left(#1\right)}
\newcommand{\ang}[1]{\langle#1\rangle}
\newcommand{\tuple}[1]{\left(#1\right)}

% numerical shortcuts.
\newcommand{\abs}[1]{\left| #1\right|}
\newcommand{\floor}[1]{\left\lfloor #1 \right\rfloor}
\newcommand{\ceil}[1]{\left\lceil #1 \right\rceil}

% linear algebra shortcuts.
\newcommand{\change}{\Delta}
\newcommand{\norm}[1]{\left\| #1\right\|}
\newcommand{\dprod}[1]{\langle#1\rangle}
\newcommand{\linspan}[1]{\langle#1\rangle}
\newcommand{\conj}[1]{\overline{#1}}
\newcommand{\gradient}{\nabla}
\newcommand{\der}{\frac{d}{dx}}
\newcommand{\lap}{\Delta}
\newcommand{\kron}{\otimes}
\newcommand{\nperp}{\nvdash}

\newcommand{\mat}[1]{\left( \begin{smallmatrix}#1 \end{smallmatrix} \right)}

% derivatives and limits
\newcommand{\partder}[2]{\frac{\partial #1}{\partial #2}}
\newcommand{\partdern}[3]{\frac{\partial^{#3} #1}{\partial #2^{#3}}}

% Arrows
\newcommand{\diverge}{\nearrow}
\newcommand{\notto}{\nrightarrow}
\newcommand{\up}{\uparrow}
\newcommand{\down}{\downarrow}
% gets and gives are defined!

% ordering operators
\newcommand{\oleq}{\preceq}
\newcommand{\ogeq}{\succeq}

% programming and logic operators
\newcommand{\dfn}{:=}
\newcommand{\assign}{:=}
\newcommand{\co}{\ co\ }
\newcommand{\en}{\ en\ }


% logic operators
\newcommand{\xor}{\oplus}
\newcommand{\Land}{\bigwedge}
\newcommand{\Lor}{\bigvee}
\newcommand{\finish}{$\Box$}
\newcommand{\contra}{\Rightarrow \Leftarrow}
\newcommand{\iseq}{\stackrel{_?}{=}}


% Set theory
\newcommand{\symdiff}{\Delta}
\newcommand{\union}{\cup}
\newcommand{\inters}{\cap}
\newcommand{\Union}{\bigcup}
\newcommand{\Inters}{\bigcap}
\newcommand{\nullSet}{\phi}

% graph theory
\newcommand{\nbd}{\Gamma}

% Script alphabets
% For reals, use \Re

% greek letters
\newcommand{\eps}{\epsilon}
\newcommand{\del}{\delta}
\newcommand{\ga}{\alpha}
\newcommand{\gb}{\beta}
\newcommand{\gd}{\del}
\newcommand{\gf}{\phi}
\newcommand{\gF}{\Phi}
\newcommand{\gl}{\lambda}
\newcommand{\gm}{\mu}
\newcommand{\gn}{\nu}
\newcommand{\gr}{\rho}
\newcommand{\gs}{\sigma}
\newcommand{\gt}{\theta}
\newcommand{\gx}{\xi}

\newcommand{\sw}{\sigma}
\newcommand{\SW}{\Sigma}
\newcommand{\ew}{\lambda}
\newcommand{\EW}{\Lambda}

\newcommand{\Del}{\Delta}
\newcommand{\gD}{\Delta}
\newcommand{\gG}{\Gamma}
\newcommand{\gO}{\Omega}
\newcommand{\gL}{\Lambda}
\newcommand{\gS}{\Sigma}

% Formatting shortcuts
\newcommand{\red}[1]{\textcolor{red}{#1}}
\newcommand{\blue}[1]{\textcolor{blue}{#1}}
\newcommand{\htext}[2]{\texorpdfstring{#1}{#2}}

% Statistics
\newcommand{\distr}{\sim}
\newcommand{\stddev}{\sigma}
\newcommand{\covmatrix}{\Sigma}
\newcommand{\mean}{\mu}
\newcommand{\param}{\gt}
\newcommand{\ftr}{\phi}

% General utility
\newcommand{\todo}[1]{\footnote{TODO: #1}}
\newcommand{\exclaim}[1]{{\textbf{\textit{#1}}}}
\newcommand{\tbc}{[\textbf{Incomplete}]}
\newcommand{\chk}{[\textbf{Check}]}
\newcommand{\oprob}{[\textbf{OP}]:}
\newcommand{\core}[1]{\textbf{Core Idea:}}
\newcommand{\why}{[\textbf{Find proof}]}
\newcommand{\opt}[1]{\textit{#1}}


\DeclareMathOperator*{\argmin}{arg\,min}
\DeclareMathOperator{\rank}{rank}
\newcommand{\redcol}[1]{\textcolor{red}{#1}}
\newcommand{\bluecol}[1]{\textcolor{blue}{#1}}
\newcommand{\greencol}[1]{\textcolor{green}{#1}}


\renewcommand{\~}{\htext{$\sim$}{~}}


% Use something like:
% % Use something like:
% % Use something like:
% \input{../../amsartMacros}

\newtheorem{thm}{Theorem}[subsection]
\newtheorem{cor}[thm]{Corollary}
\newtheorem{lem}[thm]{Lemma}
\newtheorem{fact}[thm]{Fact}
\newtheorem{claim}[thm]{Claim}

\newtheorem{assumption}[thm]{Assumption}

\newtheorem{defn}[thm]{Definition}

\theoremstyle{remark}
\newtheorem{alg}[thm]{Algorithm}
\newtheorem*{notation}{Notation}
\newtheorem*{rem}{Remark}
\newtheorem*{hint}{Hint}
\newtheorem*{ack}{Acknowledgement}

\newtheorem{question}[thm]{Question}
\newtheorem{answer}[thm]{Answer}


\newtheorem{thm}{Theorem}[subsection]
\newtheorem{cor}[thm]{Corollary}
\newtheorem{lem}[thm]{Lemma}
\newtheorem{fact}[thm]{Fact}
\newtheorem{claim}[thm]{Claim}

\newtheorem{assumption}[thm]{Assumption}

\newtheorem{defn}[thm]{Definition}

\theoremstyle{remark}
\newtheorem{alg}[thm]{Algorithm}
\newtheorem*{notation}{Notation}
\newtheorem*{rem}{Remark}
\newtheorem*{hint}{Hint}
\newtheorem*{ack}{Acknowledgement}

\newtheorem{question}[thm]{Question}
\newtheorem{answer}[thm]{Answer}


\newtheorem{thm}{Theorem}[subsection]
\newtheorem{cor}[thm]{Corollary}
\newtheorem{lem}[thm]{Lemma}
\newtheorem{fact}[thm]{Fact}
\newtheorem{claim}[thm]{Claim}

\newtheorem{assumption}[thm]{Assumption}

\newtheorem{defn}[thm]{Definition}

\theoremstyle{remark}
\newtheorem{alg}[thm]{Algorithm}
\newtheorem*{notation}{Notation}
\newtheorem*{rem}{Remark}
\newtheorem*{hint}{Hint}
\newtheorem*{ack}{Acknowledgement}

\newtheorem{question}[thm]{Question}
\newtheorem{answer}[thm]{Answer}


%opening
\title{Cryptography: Answers to Homework 3}
\author{vishvAs vAsuki}

\begin{document}

\maketitle

\begin{rem}
 No Collaborators unless specifically acknowledged in individual solutions.
\end{rem}


\section{Answer to question 1}
The answer is simply the Lamport one time signature scheme, about which I read in the Wikipedia.

\subsection*{Some notation}
Let the message space be contained in $\set{0, 1}^{k}$. Alternatively, you can think of this space as the space to which all messages are hashed to by a collision resistant hash function, prior to signing.

Let G be the prime order group of order p. Let g be some generator of G.

\subsection*{The one-time signature scheme Sig}
\begin{itemize}
\item Setup(l): For $i \in \set{1, .. k}$ and $j \in {0, 1}$, pick numbers $y_{i,j}$ uniformly at random from G. Compute numbers $z_{i,j} = g^{y_{i,j}} \mod p$. The verfication key $VK = (z_{i, j})$. Signing key $SK = (y_{i,j})$.
\item Sign(m): Let $m_{i}$ represent the ith bit of m. Then, signature \\
$s = (y_{1, m_{1}} .. y_{k, m_{k}})$.
\item Verify(m, s): The following checks are done for every number in the vector s: $g^{y_{i,m_{i}}} \mod p \iseq z_{i, m_{i}}$. If all tests succeed, the s is considered to be a valid signature for m.
\end{itemize}

\subsection*{Proof of unforgeability under one-time signature scheme}
Consider the following security game. Let A be the Discrete log assumption challeger. Let B be the Sig challenger/ Discrete log assumption attacker. Let C be an attacker against the Sig scheme.

\begin{itemize}
 \item A challenges B to find the a in $b = g^{a} \mod p$, given g, p and b.
 \item B runs the setup algorithm described above. B then replaces a randomly picked $z_{i, j}$ with b. So, $z_{i, j} = b$ now.
 \item B tells C the VK.
 \item C asks B to sign some message of its choice $m'$.
 \item B returns the signature $s'$, if $m'_{i} \neq j$. Here j is one of the indices of the number $z_{i,j}$ which was replaced with b. Otherwise, B aborts. As $z_{i,j}$ was randomly chosen, this happens with probability 1/2.
 \item C forges a valid signature s for message $m\neq m'$ with some non negligible probability p.
 \item m differs from m' by atleast 1 bit. If one of the changed bits is i, and if s is valid, B obtains the solution for $b = g^{a} \mod p$, $y_{i, j} = a$ from the ith number in s. As i, j were randomly chosen, this happens with probability at least 1/k.
 \item In such a case, B can respond to A's challenge by sending it a.
\end{itemize}

Thus, B successfully responds to A's challenge with probability at least $\frac{p}{2k}$, which is non negligible.

\section{Answer to question 2}
We want to show a separation between weak signatures and existentially unforgeable signatures. In the weak definition, the attacker must tell the challenger all the messages he wants signed before seeing the public key.

It is obvious that every existentially unforgeable signature is also a weak signature. To show the separation between these two classes, we will therefore construct a weak signature scheme Sig-W which is not existentially unforgeable.

Take any standard signature scheme, Sig, with the algorithms Setup, Sign and Verify.

% We consider two schemes, the \textbf{second} of which appears more likely to be correct.

\subsection*{Scheme 1}
\begin{ack}
 This scheme emerged during a discussion with Rashid Kaleem.
\end{ack}
We build the algorithms of $Sig_{W}$ as follows:

\begin{itemize}
 \item $Setup_W(\gl, m_{1}, .. m_{n})$ :
 \subitem As this is the setup function of a weak signature scheme, we are assuming that it will know beforehand the messages $m_{1}, .. m_{n}$, whose signatures the adversary will see.
 \subitem Call Setup($\gl$). This returns the public key VK and the secret key SK. Set $VK_{W} = VK, SK_W = (SK, m_{1}, .. m_{n}, m')$, where $m'$ is picked randomly from the message space such that $m' \notin (m_{1}, .. m_{n})$.

\item $Sign_{W}(m, SK_W)$ : 
 \subitem If $m \in (m_{1}, .. m_{n})$, return Sign(m, SK). Otherwise, return Sign(m', SK).
\item $Verify_{W}(m, s, VK_W)$:
 \subitem Simply return the result of $Verify(m, s, VK)$
\end{itemize}

\subsubsection*{$Sig_W$ is a weak signature.}
To see this, suppose that there existed an attacker C which is able to break $Sig_W$ under the weak signature security game. We will see that C can be used to build an attacker B which is successful against the strong signature scheme, Sig. Consider the following security game, where A is the challgenger for Sig.

\begin{itemize}
 \item A runs Setup, and announces VK to B.
 \item B simulates the functioning of the weak signature scheme $Sig_W$.
 \subitem It gathers from C the messages it wants signed: $(m_{1}, .. m_{n})$.
 \subitem B announces $VK_{W} = VK$ to C.
 \item B gets signatures for $(m_{1}, .. m_{n})$ from A and passes it on to C.
 \item C produces a forgery on message $m \notin (m_{1}, .. m_{n})$ with some non negligible probability p.
 \item B then transmits C's attempt at forgery to A. Thus, B succeeds with non negligible probability p.
\end{itemize}

\subsubsection*{$Sig_W$ is not existentially unforgeable.}
Under $Sig_W$, all $m \notin (m_{1}, .. m_{n})$ have the same signature s. Under the definition of existential unforgeability, the attacker is allowed to make a polynomial number of signature queries, after seeing the verification key VK. So, an attacker C would simply need to see the signatures of n+2 distinct messages. Then, C could trawl through the signatures and find the signature s which appears atleast twice. Then, it picks a message m at random from the message space. Then, if M is the size of the message space, with probability at least $\frac{n+2}{M}$, s is a forged signature for m.

\subsubsection*{A potential problem}
Is the proof that $Sig_W$ is a weak signature valid? It may not be, because the attacker C is not guaranteed to work on such a weird distribution.

% \subsection*{Scheme 2}
% \begin{ack}
%  The idea behind the following scheme is entirely due to Sanjam Garg. He also pointed out the potential problem with scheme 1.
% \end{ack}
% 
% We build the algorithms of $Sig_{W}$ as follows:
% 
% \begin{itemize}
%  \item $Setup_W(\gl)$ :
%  \subitem Call Setup($\gl$). This returns the public key VK and the secret key SK. Use the same key pair.
% \item $Sign_{W}(m, SK_W)$ : 
%  \subitem Pick a random number r. Return $r, s_{1} = sign(m \xor r, SK), s_{2} = sign(r, SK)$.
% \item $Verify_{W}(m, r, s_{1}, s_{2}, VK_W)$:
%  \subitem Simply return $Verify(m \xor r, s_{1}, VK) \et Verify(r, s_{2}, VK)$.
% \end{itemize}
% 
% \subsubsection*{$Sig_W$ is a weak signature.}
% To see this, suppose that there existed an attacker C which is able to break $Sig_W$ under the weak signature security game. We will see that C can be used to build an attacker B which is successful against the strong signature scheme, Sig. Consider the following security game, where A is the challgenger for Sig.
% 
% \begin{itemize}
%  \item A runs Setup, and announces VK to B.
%  \item B simulates the functioning of the weak signature scheme $Sig_W$.
%  \subitem It gathers from C the messages it wants signed: $(m_{1}, .. m_{n})$.
%  \subitem B announces $VK_{W} = VK$ to C.
%  \item B picks $(r_{1}, .. r_{n})$ randomly, gets signatures for $(m_{1} \xor r_{1}, .. m_{n} \xor r_{n}), (r_{1}, .. r_{n})$ from A, constructs signatures under $Sig_{W}$ and passes it on to C.
%  \item C produces a forgery on message $m \notin (m_{1}, .. m_{n})$ with some non negligible probability p.
%  \item B then transmits C's attempt at forgery to A. Thus, B succeeds with non negligible probability p.
% \end{itemize}
% 
% \subsubsection*{$Sig_W$ is not existentially unforgeable.}
% Here is a successful attack against $Sig_W$.
% 
% Obtain $sign_w(m, SK) = r, s_{1} = sign(m \xor r, SK), s_{2} = sign(r, SK)$.
% 
% Obtain $sign_w(r, SK) = r', s_{1}' = sign(r \xor r', SK), s_{2}' = sign(r', SK)$.
% 
% Produce the forgery: $sign_w(m \xor r', SK) = r' \xor r, s_{1}'' = sign(m \xor r' \xor r \xor r', SK) = s_{1}, s_{2}'' = sign(r \xor r', SK) = s_{2}'$.
% 
% The forgery attempt fails if r' = r. But, as these numbers are picked randomly, this happens with negligible probability.

\section{Answer to question 3}
\subsection*{A precise definition for a pseudorandom bit generator}
Let n be the security parameter. For making a precise argument, we will use k(n) and l(n) instead of k and l mentioned in the question.

$G:\set{0, 1}^{k(n)} \to \set{0, 1}^{l(n)}$ for $l(n)>k(n)$ is called pseudorandom if, for all polynomial time (possibly randomized) algorithms A, and for all polynomials q(n), $\exists n_{0}: n>n_{0} \implies |Pr_{x \in \set{0, 1}^{k(n)}}(A(G(x))) - Pr_{y \in \set{0, 1}^{l(n)}}(A(y))| \leq \frac{1}{q(n)}$.

\subsection*{The generator G'}
Consider the function $G':\set{0, 1}^{2k(n)} \to \set{0, 1}^{2l(n)}$ which is evaluated by executing G on the first k(n) input bits and then on the next k(n) input bits. We will show that G' is also pseudorandom bit generator.

\subsection*{Proof}
Note that this proof implicitly, but not explicitly, uses the hybrid technique.

Suppose, for the sake of contradiction, that G' is not a pseudorandom bit generator. Consider any polynomial time algorithm A.

For each i, let $t_{i}$ denote the probability that A(y)=1 when the first i bits of y are the first i bits of G'(x), evaluated over a random input x; and the remaining bits of y are truly random.

If G' were not a pseudorandom generator, $1/q(n) \leq |t_{0} - t_{2l(n)}| \leq |t_{0} - t_{l(n)}| + |t_{l(n)} - t_{2l(n)}|$ for some polynomial q(n) and for arbitrarily large n. As G is a pseudorandom generator, $|t_{0} - t_{l(n)}| \leq (2q(n))^{-1}$ for sufficiently large n. So, we have: $(2q(n))^{-1} \leq |t_{l(n)} - t_{2l(n)}|$ for arbitrarily large n.

With A, one can build a randomized algorithm B which can tell if it has been supplied truly random bits, or bits from G. This is done as follows: Given $x \in {0, 1}^{l(n)}$, B first runs G by supplying it k(n) truly random bits. It prepends the l(n) bits it gets from G to x, and gets the bit-string y. B then runs A on y. As $(2q(n))^{-1} \leq |t_{l(n)} - t_{2l(n)}|$, B behaves noticably differently on truly random x, and on x generated from G.

The existence of B then leads us to a contradiction of our assumption that G is a pseudorandom bit generator. Thus, B cannot exist. Thus, G' is a pseudorandom bit generator if G is.

% \bibliographystyle{plain}
% \bibliography{../colt}


\end{document}