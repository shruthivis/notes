\documentclass[10pt]{amsart}
\usepackage{amssymb, algorithm2e}

% Use something like:
% % Use something like:
% % Use something like:
% \input{../../macros}

% groupings of objects.
\newcommand{\set}[1]{\left\{ #1 \right\}}
\newcommand{\seq}[1]{\left(#1\right)}
\newcommand{\ang}[1]{\langle#1\rangle}
\newcommand{\tuple}[1]{\left(#1\right)}

% numerical shortcuts.
\newcommand{\abs}[1]{\left| #1\right|}
\newcommand{\floor}[1]{\left\lfloor #1 \right\rfloor}
\newcommand{\ceil}[1]{\left\lceil #1 \right\rceil}

% linear algebra shortcuts.
\newcommand{\change}{\Delta}
\newcommand{\norm}[1]{\left\| #1\right\|}
\newcommand{\dprod}[1]{\langle#1\rangle}
\newcommand{\linspan}[1]{\langle#1\rangle}
\newcommand{\conj}[1]{\overline{#1}}
\newcommand{\gradient}{\nabla}
\newcommand{\der}{\frac{d}{dx}}
\newcommand{\lap}{\Delta}
\newcommand{\kron}{\otimes}
\newcommand{\nperp}{\nvdash}

\newcommand{\mat}[1]{\left( \begin{smallmatrix}#1 \end{smallmatrix} \right)}

% derivatives and limits
\newcommand{\partder}[2]{\frac{\partial #1}{\partial #2}}
\newcommand{\partdern}[3]{\frac{\partial^{#3} #1}{\partial #2^{#3}}}

% Arrows
\newcommand{\diverge}{\nearrow}
\newcommand{\notto}{\nrightarrow}
\newcommand{\up}{\uparrow}
\newcommand{\down}{\downarrow}
% gets and gives are defined!

% ordering operators
\newcommand{\oleq}{\preceq}
\newcommand{\ogeq}{\succeq}

% programming and logic operators
\newcommand{\dfn}{:=}
\newcommand{\assign}{:=}
\newcommand{\co}{\ co\ }
\newcommand{\en}{\ en\ }


% logic operators
\newcommand{\xor}{\oplus}
\newcommand{\Land}{\bigwedge}
\newcommand{\Lor}{\bigvee}
\newcommand{\finish}{$\Box$}
\newcommand{\contra}{\Rightarrow \Leftarrow}
\newcommand{\iseq}{\stackrel{_?}{=}}


% Set theory
\newcommand{\symdiff}{\Delta}
\newcommand{\union}{\cup}
\newcommand{\inters}{\cap}
\newcommand{\Union}{\bigcup}
\newcommand{\Inters}{\bigcap}
\newcommand{\nullSet}{\phi}

% graph theory
\newcommand{\nbd}{\Gamma}

% Script alphabets
% For reals, use \Re

% greek letters
\newcommand{\eps}{\epsilon}
\newcommand{\del}{\delta}
\newcommand{\ga}{\alpha}
\newcommand{\gb}{\beta}
\newcommand{\gd}{\del}
\newcommand{\gf}{\phi}
\newcommand{\gF}{\Phi}
\newcommand{\gl}{\lambda}
\newcommand{\gm}{\mu}
\newcommand{\gn}{\nu}
\newcommand{\gr}{\rho}
\newcommand{\gs}{\sigma}
\newcommand{\gt}{\theta}
\newcommand{\gx}{\xi}

\newcommand{\sw}{\sigma}
\newcommand{\SW}{\Sigma}
\newcommand{\ew}{\lambda}
\newcommand{\EW}{\Lambda}

\newcommand{\Del}{\Delta}
\newcommand{\gD}{\Delta}
\newcommand{\gG}{\Gamma}
\newcommand{\gO}{\Omega}
\newcommand{\gL}{\Lambda}
\newcommand{\gS}{\Sigma}

% Formatting shortcuts
\newcommand{\red}[1]{\textcolor{red}{#1}}
\newcommand{\blue}[1]{\textcolor{blue}{#1}}
\newcommand{\htext}[2]{\texorpdfstring{#1}{#2}}

% Statistics
\newcommand{\distr}{\sim}
\newcommand{\stddev}{\sigma}
\newcommand{\covmatrix}{\Sigma}
\newcommand{\mean}{\mu}
\newcommand{\param}{\gt}
\newcommand{\ftr}{\phi}

% General utility
\newcommand{\todo}[1]{\footnote{TODO: #1}}
\newcommand{\exclaim}[1]{{\textbf{\textit{#1}}}}
\newcommand{\tbc}{[\textbf{Incomplete}]}
\newcommand{\chk}{[\textbf{Check}]}
\newcommand{\oprob}{[\textbf{OP}]:}
\newcommand{\core}[1]{\textbf{Core Idea:}}
\newcommand{\why}{[\textbf{Find proof}]}
\newcommand{\opt}[1]{\textit{#1}}


\DeclareMathOperator*{\argmin}{arg\,min}
\DeclareMathOperator{\rank}{rank}
\newcommand{\redcol}[1]{\textcolor{red}{#1}}
\newcommand{\bluecol}[1]{\textcolor{blue}{#1}}
\newcommand{\greencol}[1]{\textcolor{green}{#1}}


\renewcommand{\~}{\htext{$\sim$}{~}}


% groupings of objects.
\newcommand{\set}[1]{\left\{ #1 \right\}}
\newcommand{\seq}[1]{\left(#1\right)}
\newcommand{\ang}[1]{\langle#1\rangle}
\newcommand{\tuple}[1]{\left(#1\right)}

% numerical shortcuts.
\newcommand{\abs}[1]{\left| #1\right|}
\newcommand{\floor}[1]{\left\lfloor #1 \right\rfloor}
\newcommand{\ceil}[1]{\left\lceil #1 \right\rceil}

% linear algebra shortcuts.
\newcommand{\change}{\Delta}
\newcommand{\norm}[1]{\left\| #1\right\|}
\newcommand{\dprod}[1]{\langle#1\rangle}
\newcommand{\linspan}[1]{\langle#1\rangle}
\newcommand{\conj}[1]{\overline{#1}}
\newcommand{\gradient}{\nabla}
\newcommand{\der}{\frac{d}{dx}}
\newcommand{\lap}{\Delta}
\newcommand{\kron}{\otimes}
\newcommand{\nperp}{\nvdash}

\newcommand{\mat}[1]{\left( \begin{smallmatrix}#1 \end{smallmatrix} \right)}

% derivatives and limits
\newcommand{\partder}[2]{\frac{\partial #1}{\partial #2}}
\newcommand{\partdern}[3]{\frac{\partial^{#3} #1}{\partial #2^{#3}}}

% Arrows
\newcommand{\diverge}{\nearrow}
\newcommand{\notto}{\nrightarrow}
\newcommand{\up}{\uparrow}
\newcommand{\down}{\downarrow}
% gets and gives are defined!

% ordering operators
\newcommand{\oleq}{\preceq}
\newcommand{\ogeq}{\succeq}

% programming and logic operators
\newcommand{\dfn}{:=}
\newcommand{\assign}{:=}
\newcommand{\co}{\ co\ }
\newcommand{\en}{\ en\ }


% logic operators
\newcommand{\xor}{\oplus}
\newcommand{\Land}{\bigwedge}
\newcommand{\Lor}{\bigvee}
\newcommand{\finish}{$\Box$}
\newcommand{\contra}{\Rightarrow \Leftarrow}
\newcommand{\iseq}{\stackrel{_?}{=}}


% Set theory
\newcommand{\symdiff}{\Delta}
\newcommand{\union}{\cup}
\newcommand{\inters}{\cap}
\newcommand{\Union}{\bigcup}
\newcommand{\Inters}{\bigcap}
\newcommand{\nullSet}{\phi}

% graph theory
\newcommand{\nbd}{\Gamma}

% Script alphabets
% For reals, use \Re

% greek letters
\newcommand{\eps}{\epsilon}
\newcommand{\del}{\delta}
\newcommand{\ga}{\alpha}
\newcommand{\gb}{\beta}
\newcommand{\gd}{\del}
\newcommand{\gf}{\phi}
\newcommand{\gF}{\Phi}
\newcommand{\gl}{\lambda}
\newcommand{\gm}{\mu}
\newcommand{\gn}{\nu}
\newcommand{\gr}{\rho}
\newcommand{\gs}{\sigma}
\newcommand{\gt}{\theta}
\newcommand{\gx}{\xi}

\newcommand{\sw}{\sigma}
\newcommand{\SW}{\Sigma}
\newcommand{\ew}{\lambda}
\newcommand{\EW}{\Lambda}

\newcommand{\Del}{\Delta}
\newcommand{\gD}{\Delta}
\newcommand{\gG}{\Gamma}
\newcommand{\gO}{\Omega}
\newcommand{\gL}{\Lambda}
\newcommand{\gS}{\Sigma}

% Formatting shortcuts
\newcommand{\red}[1]{\textcolor{red}{#1}}
\newcommand{\blue}[1]{\textcolor{blue}{#1}}
\newcommand{\htext}[2]{\texorpdfstring{#1}{#2}}

% Statistics
\newcommand{\distr}{\sim}
\newcommand{\stddev}{\sigma}
\newcommand{\covmatrix}{\Sigma}
\newcommand{\mean}{\mu}
\newcommand{\param}{\gt}
\newcommand{\ftr}{\phi}

% General utility
\newcommand{\todo}[1]{\footnote{TODO: #1}}
\newcommand{\exclaim}[1]{{\textbf{\textit{#1}}}}
\newcommand{\tbc}{[\textbf{Incomplete}]}
\newcommand{\chk}{[\textbf{Check}]}
\newcommand{\oprob}{[\textbf{OP}]:}
\newcommand{\core}[1]{\textbf{Core Idea:}}
\newcommand{\why}{[\textbf{Find proof}]}
\newcommand{\opt}[1]{\textit{#1}}


\DeclareMathOperator*{\argmin}{arg\,min}
\DeclareMathOperator{\rank}{rank}
\newcommand{\redcol}[1]{\textcolor{red}{#1}}
\newcommand{\bluecol}[1]{\textcolor{blue}{#1}}
\newcommand{\greencol}[1]{\textcolor{green}{#1}}


\renewcommand{\~}{\htext{$\sim$}{~}}


% groupings of objects.
\newcommand{\set}[1]{\left\{ #1 \right\}}
\newcommand{\seq}[1]{\left(#1\right)}
\newcommand{\ang}[1]{\langle#1\rangle}
\newcommand{\tuple}[1]{\left(#1\right)}

% numerical shortcuts.
\newcommand{\abs}[1]{\left| #1\right|}
\newcommand{\floor}[1]{\left\lfloor #1 \right\rfloor}
\newcommand{\ceil}[1]{\left\lceil #1 \right\rceil}

% linear algebra shortcuts.
\newcommand{\change}{\Delta}
\newcommand{\norm}[1]{\left\| #1\right\|}
\newcommand{\dprod}[1]{\langle#1\rangle}
\newcommand{\linspan}[1]{\langle#1\rangle}
\newcommand{\conj}[1]{\overline{#1}}
\newcommand{\gradient}{\nabla}
\newcommand{\der}{\frac{d}{dx}}
\newcommand{\lap}{\Delta}
\newcommand{\kron}{\otimes}
\newcommand{\nperp}{\nvdash}

\newcommand{\mat}[1]{\left( \begin{smallmatrix}#1 \end{smallmatrix} \right)}

% derivatives and limits
\newcommand{\partder}[2]{\frac{\partial #1}{\partial #2}}
\newcommand{\partdern}[3]{\frac{\partial^{#3} #1}{\partial #2^{#3}}}

% Arrows
\newcommand{\diverge}{\nearrow}
\newcommand{\notto}{\nrightarrow}
\newcommand{\up}{\uparrow}
\newcommand{\down}{\downarrow}
% gets and gives are defined!

% ordering operators
\newcommand{\oleq}{\preceq}
\newcommand{\ogeq}{\succeq}

% programming and logic operators
\newcommand{\dfn}{:=}
\newcommand{\assign}{:=}
\newcommand{\co}{\ co\ }
\newcommand{\en}{\ en\ }


% logic operators
\newcommand{\xor}{\oplus}
\newcommand{\Land}{\bigwedge}
\newcommand{\Lor}{\bigvee}
\newcommand{\finish}{$\Box$}
\newcommand{\contra}{\Rightarrow \Leftarrow}
\newcommand{\iseq}{\stackrel{_?}{=}}


% Set theory
\newcommand{\symdiff}{\Delta}
\newcommand{\union}{\cup}
\newcommand{\inters}{\cap}
\newcommand{\Union}{\bigcup}
\newcommand{\Inters}{\bigcap}
\newcommand{\nullSet}{\phi}

% graph theory
\newcommand{\nbd}{\Gamma}

% Script alphabets
% For reals, use \Re

% greek letters
\newcommand{\eps}{\epsilon}
\newcommand{\del}{\delta}
\newcommand{\ga}{\alpha}
\newcommand{\gb}{\beta}
\newcommand{\gd}{\del}
\newcommand{\gf}{\phi}
\newcommand{\gF}{\Phi}
\newcommand{\gl}{\lambda}
\newcommand{\gm}{\mu}
\newcommand{\gn}{\nu}
\newcommand{\gr}{\rho}
\newcommand{\gs}{\sigma}
\newcommand{\gt}{\theta}
\newcommand{\gx}{\xi}

\newcommand{\sw}{\sigma}
\newcommand{\SW}{\Sigma}
\newcommand{\ew}{\lambda}
\newcommand{\EW}{\Lambda}

\newcommand{\Del}{\Delta}
\newcommand{\gD}{\Delta}
\newcommand{\gG}{\Gamma}
\newcommand{\gO}{\Omega}
\newcommand{\gL}{\Lambda}
\newcommand{\gS}{\Sigma}

% Formatting shortcuts
\newcommand{\red}[1]{\textcolor{red}{#1}}
\newcommand{\blue}[1]{\textcolor{blue}{#1}}
\newcommand{\htext}[2]{\texorpdfstring{#1}{#2}}

% Statistics
\newcommand{\distr}{\sim}
\newcommand{\stddev}{\sigma}
\newcommand{\covmatrix}{\Sigma}
\newcommand{\mean}{\mu}
\newcommand{\param}{\gt}
\newcommand{\ftr}{\phi}

% General utility
\newcommand{\todo}[1]{\footnote{TODO: #1}}
\newcommand{\exclaim}[1]{{\textbf{\textit{#1}}}}
\newcommand{\tbc}{[\textbf{Incomplete}]}
\newcommand{\chk}{[\textbf{Check}]}
\newcommand{\oprob}{[\textbf{OP}]:}
\newcommand{\core}[1]{\textbf{Core Idea:}}
\newcommand{\why}{[\textbf{Find proof}]}
\newcommand{\opt}[1]{\textit{#1}}


\DeclareMathOperator*{\argmin}{arg\,min}
\DeclareMathOperator{\rank}{rank}
\newcommand{\redcol}[1]{\textcolor{red}{#1}}
\newcommand{\bluecol}[1]{\textcolor{blue}{#1}}
\newcommand{\greencol}[1]{\textcolor{green}{#1}}


\renewcommand{\~}{\htext{$\sim$}{~}}


% Use something like:
% % Use something like:
% % Use something like:
% \input{../../amsartMacros}

\newtheorem{thm}{Theorem}[subsection]
\newtheorem{cor}[thm]{Corollary}
\newtheorem{lem}[thm]{Lemma}
\newtheorem{fact}[thm]{Fact}
\newtheorem{claim}[thm]{Claim}

\newtheorem{assumption}[thm]{Assumption}

\newtheorem{defn}[thm]{Definition}

\theoremstyle{remark}
\newtheorem{alg}[thm]{Algorithm}
\newtheorem*{notation}{Notation}
\newtheorem*{rem}{Remark}
\newtheorem*{hint}{Hint}
\newtheorem*{ack}{Acknowledgement}

\newtheorem{question}[thm]{Question}
\newtheorem{answer}[thm]{Answer}


\newtheorem{thm}{Theorem}[subsection]
\newtheorem{cor}[thm]{Corollary}
\newtheorem{lem}[thm]{Lemma}
\newtheorem{fact}[thm]{Fact}
\newtheorem{claim}[thm]{Claim}

\newtheorem{assumption}[thm]{Assumption}

\newtheorem{defn}[thm]{Definition}

\theoremstyle{remark}
\newtheorem{alg}[thm]{Algorithm}
\newtheorem*{notation}{Notation}
\newtheorem*{rem}{Remark}
\newtheorem*{hint}{Hint}
\newtheorem*{ack}{Acknowledgement}

\newtheorem{question}[thm]{Question}
\newtheorem{answer}[thm]{Answer}


\newtheorem{thm}{Theorem}[subsection]
\newtheorem{cor}[thm]{Corollary}
\newtheorem{lem}[thm]{Lemma}
\newtheorem{fact}[thm]{Fact}
\newtheorem{claim}[thm]{Claim}

\newtheorem{assumption}[thm]{Assumption}

\newtheorem{defn}[thm]{Definition}

\theoremstyle{remark}
\newtheorem{alg}[thm]{Algorithm}
\newtheorem*{notation}{Notation}
\newtheorem*{rem}{Remark}
\newtheorem*{hint}{Hint}
\newtheorem*{ack}{Acknowledgement}

\newtheorem{question}[thm]{Question}
\newtheorem{answer}[thm]{Answer}


%opening
\title{Computational Learning Theory: Answers to Homework 3}
\author{vishvAs vAsuki}

\begin{document}

\maketitle

\begin{rem}
 Thank you for the wonderful assignment. The solution to the first two parts of problem 4 were especially pleasing.
\end{rem}

\section{}
\begin{thm}
C is efficiently PAC learnable. Then, there is an efficient algorithm A such that: given a sample of size m labelled according to $c \in C$, A, with probability 1-d finds a hypothesis h that is consistent with all the m points and has size poly(n, size(c), log(m)).
\end{thm}
\begin{proof}
Proof by construction.

As C is efficiently PAC learnable, there exists an efficient algorithm L which can PAC learn C. So, given error parameter $\eps = 0.25$ and confidence parameter $d'$; L is capable of producing, with probability $1-d'$, a hypothesis h with $error(h) \leq \eps$. The advantage of h over random guessing is $g = 0.5 - error(h) \geq 0.5 - \eps$.

Given a sample S and confidence parameter d, the algorithm A uses ADAboost algorithm to repeatedly invoke algorithm L and produce a hypothesis consistent with S.

During this process, A always uses L to get a hypothesis h with $error(h) \leq \eps = 0.25$ with confidence $1-d'$, where $d'$ will be specified later. Note that the specified $\eps$ determines g.

After $\frac{\ln m}{2 g^{2}}$ steps, A produces a hypothesis h of size $|h| = O(\frac{\ln m}{2 g^{2}})|c|$ which is consistent with S. (This follows from the analysis of Adaboost done in class.)

Now we find out what $d'$ must be in order for A to succeed with probability 1-d. A can fail only when one or more of the $\frac{\ln m}{2 g^{2}}$ invocations of L fail to produce an $\eps$ close hypothesis. From the union bound, this happens with probability $\frac{\ln m}{2 g^{2}} d'$. Thus, when $d' = \frac{2d g^{2}}{\ln m}$, A succeds with probability 1-d.

\end{proof}

\section{}
\begin{rem}
f and h are $\set{\pm 1}$ boolean functions.
\end{rem}

\subsection{a}
\begin{thm}
For any distribution D, h is a weak hypothesis for f with advantage g if and only if $E_{x \distr D}[h(x)f(x)]\geq 2g$.
\end{thm}
\begin{proof}
\begin{eqnarray*}
E_{x \distr D}[h(x)f(x)] &=& Pr_{x\distr D}[h(x)f(x)= 1] - Pr_{x\distr D}[h(x)f(x)= -1]\\
&=& 1-2Pr_{x\distr D}[h(x)f(x)= -1]\\
\end{eqnarray*}


h is a weak hypothesis for f if and only if $Pr_{x\distr D}[h(x)f(x)= -1] \leq 2^{-1}+g$; which happens if and only if $1-2Pr_{x\distr D}[h(x)f(x)= -1] \geq 2g$; which happens if and only if $E_{x \distr D}[h(x)f(x)]\geq 2g$.
\end{proof}

\subsection{b}
\begin{thm}
f is an LTF. $f=sgn(w.x)$; $w\in Z^{n}$; $W = \sum |w_{i}|$. Assume: $\forall x, \dprod{w,x} \neq 0$. For any distribution D, there exists an $x_{i}$ such that $|E_{x\distr D}[f(x)x_{i}]|\geq W^{-1}$.
\end{thm}
\begin{proof}
Assumption: $w_{i} \neq 0$. This assumption can be removed by not considering any $x_{i}$ corresponding to $w_{i} = 0$.

Now, we will see that an absurd thing will happen if we suppose that $\forall i : |E_{x\distr D}[f(x)x_{i}]| < W^{-1}$.

Then:
\begin{eqnarray*}
W^{-1} &>& |E_{x\distr D}[f(x)x_{i}]|\\
|w_{i}|W^{-1} &>& |w_{i}||E_{x\distr D}[f(x)x_{i}]|\\
&\geq& w_{i}E_{x\distr D}[f(x)x_{i}]\\
&=& E_{x\distr D}[f(x)w_{i}x_{i}]\\
\therefore \sum_{i}|w_{i}|W^{-1} &>& \sum_{i}E_{x\distr D}[f(x)w_{i}x_{i}]\\
1 &>& E_{x\distr D}[f(x)\sum_{i}w_{i}x_{i}]\\
\end{eqnarray*}

$|\dprod{w,x}|\geq 1$ due to $w_{i}$ being non zero integers, and due to the fact that \\
$\forall x, \dprod{w,x} \neq 0$.

So, $E_{x\distr D}[f(x)\sum_{i}w_{i}x_{i}] \geq 1$ as $\dprod{w,x}$ and f(x) always agree on sign. Thus we have reached an absurdity.
\end{proof}

\subsection{c}
\begin{question}
What is the weak learner?
\end{question}
\begin{answer}
Let WL be the weak learner. We have shown earlier that \\
$|E_{x\distr D}[f(x)x_{i}]|\geq W^{-1}$. So, for each of the n bits, WL finds the bit \\
$argmax_{i} |E_{x\distr D}[f(x)x_{i}]|$. If $\max_{i} E_{x\distr D}[f(x)x_{i}] > 0$, it uses the corresponding bit $x_{i}$ as its weak hypothesis h; otherwise it uses $-x_{i}$.

Note that h has advantage $(2W)^{-1}$. Also, using the Hoeffding ineuqality, for any i, $E_{x\distr D}[f(x)x_{i}] = Pr_{x \distr D}[f(x)x_{i} = 1] - Pr_{x \distr D}[f(x)x_{i} = -1]$ can be estimated whp to arbitrarily high accuracy by taking a sufficiently large polynomial sized sample.
\end{answer}

\begin{question}
How do we apply a boosting algorithm?
\end{question}
\begin{answer}
We take WL as a black box, and simply apply any convinient boosting algorithm. Eg: ADABoost.
\end{answer}

\begin{question}
What is the output hypothesis?
\end{question}
\begin{answer}
If ADABoost is used for boosting, the output hypothesis reduces to a halfspace; as all the hypotheses the weak learner returns are of the form $\pm x_{i}$!
\end{answer}

\begin{question}
For what values of W do we get a polynomial time algorithm?
\end{question}
\begin{answer}
The individual hypotheses produced by WL are guaranteed to have an advantage of $(2W)^{-1}$. So, boosting makes sense only when W is sub exponential. Also, when ADA boost is used, the number of iterations required is polynomial in W. So, for values of W which are polynomial, we get a polynomial time algorithm.
\end{answer}


\section{}
\begin{notation}
$p_{S}$ denotes the parity function $\chi_{S}$.
\end{notation}


\begin{thm}
$\sum_{|S|\geq d} \hat{f}(S)^{2} \leq \eps$. $Pr_{x}(g(x)\neq f(x)) = \eta$. Then, $E[(g(x)-\sum_{|S|<d}\hat{g}(S)p_{S}(x))^{2}] \leq E[(g(x)-\sum_{|S|<d}\hat{f}(S)p_{S}(x))^{2}]$.
\end{thm}
\begin{proof}
Using the Fourier expansion of f and g, we see the following:

\begin{eqnarray*}
E[(g(x)-\sum_{|S|<d}\hat{g}(S)p_{S}(x))^{2}] &=& \norm{\sum_{|S|\geq d}\hat{g}(S)p_{S}(x))}^{2}\\
&=& \sum_{|S|\geq d} \hat{g}(S)^{2}\\
\end{eqnarray*}


\begin{eqnarray*}
E[(g(x)-\sum_{|S|<d}\hat{f}(S)p_{S}(x))^{2}] &=& \norm{\sum_{|S|\geq d}\hat{g}(S)p_{S}(x))+ \sum_{|S|<d}(\hat{g}(S)-\hat{f}(S))p_{S}(x)}^{2}\\
&=& \sum_{|S|\geq d} \hat{g}(S)^{2} + \norm{\sum_{|S|<d}(\hat{g}(S)-\hat{f}(S))p_{S}(x)}^{2}\\
\end{eqnarray*}

Thence the result.
\end{proof}

\begin{cor}
$\sum_{|S|\geq d} \hat{g}(S)^{2} \leq O(\eta + \eps)$.
\end{cor}
\begin{defn}
$f_{<d} = \sum_{|S|<d}\hat{f}(S)p_{S}(x)$. $f_{\geq d} = \sum_{|S| \geq d}\hat{f}(S)p_{S}(x)$.
\end{defn}
\begin{proof}

Note that:
\begin{eqnarray*}
\norm{g-f}^{2} &=& E_{x}[(g(x)-f(x))^{2}]\\
&=& 4Pr_{x}(f(x) \neq g(x))\\
&=& 4\eta \\
\end{eqnarray*}


Using the theorem proved earlier:
\begin{eqnarray*}
\sum_{|S|\geq d} \hat{g}(S)^{2} &\leq& \norm{g - f_{<d}}^{2}\\
&=& \norm{g - f + f_{\geq d}}^{2}\\
&=& \norm{g - f}^{2} + \norm{f_{\geq d}}^{2} + 2\dprod{f_{\geq d}, g-f}\\
&\leq& \norm{g - f}^{2} + \norm{f_{\geq d}}^{2} + 2\norm{g - f}\norm{f_{\geq d}}\\
&\leq& 4\eta + \eps + 4\sqrt{\eta\eps}\\
&\leq& 4\eta + \eps + 4 \max (\eta, \eps)\\
&=& O(\eta + \eps)\\
\end{eqnarray*}
\end{proof}


\section{}
\begin{thm}
f is a monotone boolean function. Flipping a bit $x_{i}$ from 1 to -1 cannot cause f(x) to flip from -1 to 1. Then, $I_{i}(f) = \hat{f}(\set{x_{i}})$.
\end{thm}
\begin{proof}
Let $x' \in \set{1,-1}^{n-1}$ be a variable corresponding to parts of the string x without $x_{i}$. Let $g(x_{i}, x') = f(x)$ for all values of $x_{i}$ and $x'$.

Note that, for any $x'$, $g(-1,x') = 1 \et g(1,x') = -1$ can never happen as g and f are montonic. So, the only way $g(-1,x') = -g(1,x')$ can happen is when $g(-1,x') = -1 \et g(1,x') = 1$. Also, for this reason, $E_{x'}[g(1,x') - g(-1,x')] = 2 Pr_{x'}(g(1,x') = 1 \et g(-1,x') = -1 ) $.

So,
\begin{eqnarray*}
I_{i}(f) &=& Pr_{x}[f(x) \neq f(x^{(i)})]\\
&=& Pr_{x'}[g(1,x') \neq g(-1,x')]\\
&=& Pr_{x'}(g(1,x') = 1 \et g(-1,x') = -1 )\\
&=& 2^{-1}E_{x'}[g(1,x') - g(-1,x')]\\
&=& 2^{-1}(E_{x'}[g(1,x')] - E_{x'}[g(-1,x')])\\
\end{eqnarray*}


But,
\begin{eqnarray*}
\hat{f}(\set{i}) &=& E_{x}[f(x)x_{i}]\\
&=& Pr_{x}(x_{i}=1)E_{x}[f(x)|x_{i}=1] - Pr_{x}(x_{i}=-1)E_{x}[f(x)|x_{i}=-1]\\
&=& 2^{-1}(E_{x'}[g(1,x')] - E_{x'}[g(-1,x')])\\
&=& I_{i}(f)\\
\end{eqnarray*}

\end{proof}

\begin{cor}
The sum of influences of any monotone function is at most $\sqrt{n}$.
\end{cor}
\begin{proof}
This follows from the inequality between 1-norm and 2-norm.

$\sum_{i}I_{i}(f) = \sum_{i} \hat{f}(\set{i}) \leq \sqrt{n}\sum_{i} (\hat{f}(\set{i})^{2})^{1/2} \leq \sqrt{n}$.
\end{proof}

\begin{cor}
Sum of influences of the majority function is $\approx n(\frac{2}{(n-1)\pi})^{0.5}$.
\end{cor}
\begin{proof}
Consider the influence of a single variable. Flipping a single variable can make a difference only when there is a tie between the votes of the remaining variables. This can happen with probability $2^{-(n-1)}\frac{(n-1)!}{(\frac{n-1}{2})!(\frac{n-1}{2})!}$. Using Stirling's approximation and multiplying by n; we get the above mentioned estimate.

\end{proof}


% \bibliographystyle{plain}
% \bibliography{../colt}


\end{document}
