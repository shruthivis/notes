\section{Pairwise Discrete Graphical Models}

Let $\support_\svert= \{u \in \vertex : (\svert,u) \in \edge\}$ be the set of all neighbors of the node $\svert$ in the graph and $\support^c_\svert = \vertex \backslash \support_\svert$. Notice that $\|\theta^*_{\svert u}\|_0 = 0$ for all $u \in \support^c_\svert$. Fixing $\svert\in\vertex$, and defining $\Theta^*_{\backslash \svert}$ as before, let $\support^{(ex)}_\svert$ be the index set of parameters $\{\theta^*_{\svert t;jk} \neq 0 \}$ in $\Theta^*_{\backslash \svert}$. When clear from context, we will overload notation and again use $\support_\svert$ for this index set.

\noindent Let $\fisher^* = \mathbb{E} \left[ \nabla^2 \log\left(\mathbb{P}_{\Theta^*_{\backslash\svert}} \left[X_\svert\big|X_{\backslash\svert}\right] \right)\right]$ be the population Fisher information matrix. Note that $\fisher^* \in \real^{(m-1)^2 (p-1) \times (m-1)^2 (p-1)}$. Similarly, let $\fisher^n=\frac{1}{n}\sum_{i=1}^n\nabla^2\ell^{(i)}\left(\Theta_{\backslash\svert}; \Data\right)$ be the sample Fisher information matrix.\\

\noindent Define $\J^*=\mathbb{E}\left[\I\left[x_{t_2}=k_2\right] \I\left[x_{t_1}=k_1\right]^T\right] \in \mathbb{R}^{(m-1)(p-1)\times(m-1)(p-1)}$. Accordingly, define $\J^n$ to be the empirical mean of the same quantity over $n$ drawn samples. In the proofs (specifically in analyzing the derivative of the Hessian of the log-likelihood function), we will actually need control over $\Im^* := \J^*\otimes\mathbf{1}_{(m-1)\times (m-1)}$, the Kronecker product of $\J^*$ and matrix of all ones (which would be of the size of $\fisher^*$). But by properties of Kronecker products, we have $\Lambda_{\max}(\Im^*)=\Lambda_{\max}(\J^*)$, so that it suffices to impose assumptions on the maximum eigen values of $\J^*$ and $\J^n$.\\

\subsection{Assumptions}

We begin by stating the assumptions imposed on the true model. We note that similar sufficient conditions have been imposed in papers analyzing Lasso~\citep{WainwrightLasso} and block-regularization methods~\cite{NWJoint,Obozinski10}.


\begin{itemize}
\item [(A1)] {\bf Invertibility: } $\Lambda_{\min}\left(\fisher^*_{\support_\svert\support_\svert}\right)\geq C_{\min}>0$.

\item [(A2)] {\bf Incoherence: } $\norm{\fisher^*_{\support^c_\svert\support_\svert}\left(\fisher^*_{\support_\svert\support_\svert}\right)^{-1}}{\infty,2}\leq \frac{1-2\alpha}{\sqrt{d_\svert}}$ for some $\alpha\in\left(0,\frac{1}{2}\right)$.

\item [(A3)] {\bf Boundedness: } $\Lambda_{\max}\left(\J^*\right)\leq D_{\max}<\infty$.\\
\end{itemize}


\noindent The next lemma states that imposing these assumptions on the population quantities implies analogous conditions on the sample statistics with high probability.
\begin{lemma}
Assumptions (A1)-(A3) on the population Fisher information matrix yield the following (analogous) properties on the empirical Fisher information matrix:
\begin{itemize}
\item [(B1)] $\mathbb{P}\left[\Lambda_{\min}\left(\fisher^n_{\support_\svert\support_\svert}\right)< C_{\min}-\epsilon\right]\\
\;\;\leq 2\,\exp(-\frac{1}{8}(\epsilon\sqrt{n}-\sqrt{d_\svert})^2+\log((m-1)^2d_\svert))$.

\item [(B2)] 
$\mathbb{P}\left[\norm{\fisher^n_{\support^c_\svert\support_\svert}\left(\fisher^n_{\support_\svert\support_\svert}\right)^{-1}}{\infty,2} > \frac{1-\alpha}{\sqrt{d_\svert}}+\epsilon\right]\\
\;\; \leq 6 \, \exp(-\frac{1}{8} (\bar{C}_{\min} (\frac{\alpha}{3\sqrt{d_\svert}} + \epsilon) \sqrt{n} \\
\quad -	(1 + \frac{\sqrt{d_\svert}}{C_{\min}^2\sqrt{n}} )  \sqrt{d_\svert})^2 + \log((m-1)^2(p-1))).
$

\item [(B3)] $\mathbb{P}\left[\Lambda_{\max}\left(\J^n\right)> D_{\max}+\epsilon\right]\leq 2\exp\Big(-\frac{1}{8}(\epsilon\sqrt{n}-\sqrt{d_\svert})^2+\log\big((m-1)^2d_\svert\big)\Big)$.\\
\end{itemize} 
\label{Concentration_Lemma}
\end{lemma}


\subsection{Main Theorem}

We can now state our main result on the sparsistency of the group-sparse regularized estimator.
\begin{theorem}\label{ThmPairwise}
Consider a discrete graphical model of the form~\eqref{EqnGenMRFPairwise} with parameters $\Theta^*$ and associated edge set $\edge$ such that conditions (A1)-(A3) are satisfied. Suppose the regularization parameter satisfies
\begin{equation}
\regpar_n\geq\frac{8(2-\alpha)}{\alpha}\left(\sqrt{\frac{\log(p-1)}{n}}+\frac{m-1}{4\sqrt{n}}\right).
\end{equation}
Then, there exist positive constants $K$, $c_1$ and $c_2$ such that if the number of samples $n$ scales as
\begin{equation}
n\geq K (m-1)^2 d_\svert^{\,2} \log\left((m-1)^2(p-1)\right),
\end{equation}
then with probability $1-c_1\exp(-c_2\regpar_n^2n)$ we are guaranteed
\begin{itemize}
\item [(a)] For each node $\svert\in\vertex$, the $\ell_1/\ell_2$ regularized logistic regression \eqref{EqnPairwiseGroup} has a unique solution and hence specifies a neigborhood $\widehat{\N}(\svert)$.
\item [(b)] For each node $\svert\in\vertex$ correctly excludes all edges not in the true neighborhood $\N(\svert)$. Moreover, it includes all edges $(\svert,t)$ such that $\norm{\theta^*_{\svert t;jk}}{2}\geq\frac{10}{C_{\min}}\lambda_n$.\\
\end{itemize}

\end{theorem}



Before sketching the proof outline, we first state some lemmas characterizing the solution of \eqref{EqnGenMRFPairwise}. 
\begin{lemma} [{\bf Optimality Conditions}]
Any optimal primal-dual pair $(\hat{\Theta}_{\backslash \svert},\hat{\dual}_{\backslash \svert})$ of \eqref{EqnGenMRFPairwise} satisfies
\label{NecessaryOptCondition}
\begin{enumerate}
\item  {\bf (Stationary Condition).}
\begin{align}\label{EqnStatCondPairwise}
\nabla\ell\left(\hat{\Theta}_{\backslash \svert}\right) + \regpar_n \hat{\dual}_{\backslash \svert} = 0.
\end{align}

\item {\bf (Dual Feasibility).} $\hat{\dual}_{\backslash \svert}$ is equal to the subgradient $\partial \| \hat{\Theta}_{\backslash \svert}\|_{1,2}$ so that
for any $u \in \vertex \backslash \svert$,
\begin{enumerate}
\item if $(\hat{\Theta}_{\backslash \svert})_{u;jk} \neq 0$ for some $j,k$ then 
\[(\hat{\dual}_{\backslash \svert})_{u} = \frac{(\hat{\Theta}_{\backslash \svert})_{u}}{\|(\hat{\Theta}_{\backslash \svert})_{u}\|_{2}}.\]
\item if the entire group $(\hat{\Theta}_{\backslash \svert})_{u} = 0$, then $\|(\hat{\dual}_{\backslash \svert})_{u}\|_{2} \le 1$.
\end{enumerate}

\end{enumerate}
\end{lemma}

The next lemma states that structure recovery is guaranteed if the dual is \emph{strictly} feasible. 
\begin{lemma}[{\bf Strict Dual Feasibility}]
\label{LemSuffOptCond} 
Suppose that there exists an optimal primal-dual pair $\left(\hat{\Theta}_{\backslash \svert},\hat{\dual}_{\backslash \svert}\right)$ for \eqref{EqnPairwiseGroup} such that $\left\|\left(\hat{\dual}_{\backslash \svert}\right)_{\support^c_\svert}\right\|_{\infty,2}<1$. Then, any optimal primal solution $\widetilde{\Theta}_{\backslash \svert}$ satisfies $\left(\widetilde{\Theta}_{\backslash \svert}\right)_{\support^c_\svert}=\mathbf{0}$. Moreover, if the Hessian sub-matrix $\left[\nabla^2\ell\left(\hat{\Theta}_{\backslash \svert}\right)\right]_{\support_\svert\support_\svert}\succ 0$ then $\hat{\Theta}_{\backslash \svert}$ is the unique optimal solution.\\
\label{SuffOptCond}
\end{lemma}

We are now ready to sketch the proof of Theorem~\ref{ThmPairwise}.
\begin{proof}
\vskip0.05in
\myparagraph{Part (a)} The proof proceeds by a primal-dual witness technique, and consists of the construction of a feasible primal-dual pair in the following two steps:
\begin{itemize}
\item [(i)] {\bf Primal Candidate using an oracle subproblem}: Let $\hat{\Theta}_{\backslash \svert}$ be the optimal solution of the restricted problem
\begin{equation}
\label{OracleEqnPairwiseGroup}
\hat{\Theta}_{\backslash \svert}=\arg\min_{\left(\Theta_{\backslash \svert}\right)_{\support_\svert^c}=\mathbf{0}} \biggr\{ \ell(\Theta_{\backslash \svert}; \Data) +
	\regpar_\numobs \norm{\Theta_{\backslash \svert}}{1,2}
	\biggr\}.
\end{equation}

\item [(ii)] {\bf Dual Candidate from Stationary Optimality Condition}: 
For any column $u\in\support_\svert$ set $\left(\hat{\dual}_{\backslash \svert}\right)_{u}=\frac{1}{\left\|\left(\hat{\Theta}_{\backslash \svert}\right)_u\right\|_2}\left(\hat{\Theta}_{\backslash \svert}\right)_u$.\\
Set $\left(\hat{\dual}_{\backslash \svert}\right)_{\support_\svert^c}$ from the stationary condition \eqref{NecessaryOptCondition}.
\end{itemize}

\myparagraph{Showing Strict Dual Feasibility}
By construction, the $(\hat{\Theta}_{\backslash \svert}, \hat{\dual}_{\backslash \svert})$ pair satisfies the stationary condition~\eqref{EqnStatCondPairwise}. It remains to show that the the dual 
$\hat{\dual}_{\backslash \svert}$ is strictly feasible. We show that this holds, and also that the solution is unique, with high probability in Lemma~\ref{OptCertificate}.

\myparagraph{Part (b)} By uniqueness of the solution shown in part [(a)], the method excludes all edges that are not in the set of edges. To show that all correct edges are included, i.e., to show the correct correct sign recovery, it suffices to show that
\begin{equation}
\norm{\widehat{\Theta}_{\support_\svert}-\widehat{\Theta}^*_{\support_\svert}}{\infty,2} \leq\frac{\theta_{\min}}{2},
\nonumber
\end{equation} 
where, $\displaystyle\theta_{\min}=\min_{t\in\vertex\backslash\{\svert\}}\norm{\theta_{\svert t;jk}}{2}$. 

We provide an $\|\cdot\|_{\infty,2}$ bound on the error in \eqref{L_inf_2_bound}, from which
\begin{equation}
\begin{aligned}
\frac{2}{\theta_{\min}}\norm{\widehat{\Theta}_{\support_\svert}-\widehat{\Theta}^*_{\support_\svert}}{\infty,2}
&\leq\frac{2}{\theta_{\min}}\frac{5}{C_{\min}}\regpar_n\\
&\leq 1,
\end{aligned}
\nonumber
\end{equation}
provided that $\theta_{\min}>\frac{10}{C_{\min}}\regpar_n$.

\end{proof}
