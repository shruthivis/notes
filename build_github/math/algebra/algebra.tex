\documentclass[oneside, article]{memoir}

\usepackage{amsmath, amssymb}
\usepackage{hyperref, graphicx, verbatim, listings, multirow, subfigure}
\usepackage{algorithm, algorithmic}
% \usepackage[bottom]{footmisc}
\lstset{breaklines=true}
\setcounter{tocdepth}{3}

% Lets verbatim and verb environments automatically break lines.
\makeatletter
\def\@xobeysp{ }
\makeatother
% \lstset{breaklines=true,basicstyle=\ttfamily}

% Configuration for the memoir class.
\renewcommand{\cleardoublepage}{}
% \renewcommand*{\partpageend}{}
\renewcommand{\afterpartskip}{}
\maxsecnumdepth{subsubsection} % number subsections
\maxtocdepth{subsubsection}

\addtolength{\parindent}{-5mm}
% Packages not included:
% For multiline comments, use caption package. But this conflicts with hyperref while making html files.
% subfigure conflicts with use with memoir style-sheet.

% Use something like:
% % Use something like:
% % Use something like:
% \input{../../macros}

% groupings of objects.
\newcommand{\set}[1]{\left\{ #1 \right\}}
\newcommand{\seq}[1]{\left(#1\right)}
\newcommand{\ang}[1]{\langle#1\rangle}
\newcommand{\tuple}[1]{\left(#1\right)}

% numerical shortcuts.
\newcommand{\abs}[1]{\left| #1\right|}
\newcommand{\floor}[1]{\left\lfloor #1 \right\rfloor}
\newcommand{\ceil}[1]{\left\lceil #1 \right\rceil}

% linear algebra shortcuts.
\newcommand{\change}{\Delta}
\newcommand{\norm}[1]{\left\| #1\right\|}
\newcommand{\dprod}[1]{\langle#1\rangle}
\newcommand{\linspan}[1]{\langle#1\rangle}
\newcommand{\conj}[1]{\overline{#1}}
\newcommand{\gradient}{\nabla}
\newcommand{\der}{\frac{d}{dx}}
\newcommand{\lap}{\Delta}
\newcommand{\kron}{\otimes}
\newcommand{\nperp}{\nvdash}

\newcommand{\mat}[1]{\left( \begin{smallmatrix}#1 \end{smallmatrix} \right)}

% derivatives and limits
\newcommand{\partder}[2]{\frac{\partial #1}{\partial #2}}
\newcommand{\partdern}[3]{\frac{\partial^{#3} #1}{\partial #2^{#3}}}

% Arrows
\newcommand{\diverge}{\nearrow}
\newcommand{\notto}{\nrightarrow}
\newcommand{\up}{\uparrow}
\newcommand{\down}{\downarrow}
% gets and gives are defined!

% ordering operators
\newcommand{\oleq}{\preceq}
\newcommand{\ogeq}{\succeq}

% programming and logic operators
\newcommand{\dfn}{:=}
\newcommand{\assign}{:=}
\newcommand{\co}{\ co\ }
\newcommand{\en}{\ en\ }


% logic operators
\newcommand{\xor}{\oplus}
\newcommand{\Land}{\bigwedge}
\newcommand{\Lor}{\bigvee}
\newcommand{\finish}{$\Box$}
\newcommand{\contra}{\Rightarrow \Leftarrow}
\newcommand{\iseq}{\stackrel{_?}{=}}


% Set theory
\newcommand{\symdiff}{\Delta}
\newcommand{\union}{\cup}
\newcommand{\inters}{\cap}
\newcommand{\Union}{\bigcup}
\newcommand{\Inters}{\bigcap}
\newcommand{\nullSet}{\phi}

% graph theory
\newcommand{\nbd}{\Gamma}

% Script alphabets
% For reals, use \Re

% greek letters
\newcommand{\eps}{\epsilon}
\newcommand{\del}{\delta}
\newcommand{\ga}{\alpha}
\newcommand{\gb}{\beta}
\newcommand{\gd}{\del}
\newcommand{\gf}{\phi}
\newcommand{\gF}{\Phi}
\newcommand{\gl}{\lambda}
\newcommand{\gm}{\mu}
\newcommand{\gn}{\nu}
\newcommand{\gr}{\rho}
\newcommand{\gs}{\sigma}
\newcommand{\gt}{\theta}
\newcommand{\gx}{\xi}

\newcommand{\sw}{\sigma}
\newcommand{\SW}{\Sigma}
\newcommand{\ew}{\lambda}
\newcommand{\EW}{\Lambda}

\newcommand{\Del}{\Delta}
\newcommand{\gD}{\Delta}
\newcommand{\gG}{\Gamma}
\newcommand{\gO}{\Omega}
\newcommand{\gL}{\Lambda}
\newcommand{\gS}{\Sigma}

% Formatting shortcuts
\newcommand{\red}[1]{\textcolor{red}{#1}}
\newcommand{\blue}[1]{\textcolor{blue}{#1}}
\newcommand{\htext}[2]{\texorpdfstring{#1}{#2}}

% Statistics
\newcommand{\distr}{\sim}
\newcommand{\stddev}{\sigma}
\newcommand{\covmatrix}{\Sigma}
\newcommand{\mean}{\mu}
\newcommand{\param}{\gt}
\newcommand{\ftr}{\phi}

% General utility
\newcommand{\todo}[1]{\footnote{TODO: #1}}
\newcommand{\exclaim}[1]{{\textbf{\textit{#1}}}}
\newcommand{\tbc}{[\textbf{Incomplete}]}
\newcommand{\chk}{[\textbf{Check}]}
\newcommand{\oprob}{[\textbf{OP}]:}
\newcommand{\core}[1]{\textbf{Core Idea:}}
\newcommand{\why}{[\textbf{Find proof}]}
\newcommand{\opt}[1]{\textit{#1}}


\DeclareMathOperator*{\argmin}{arg\,min}
\DeclareMathOperator{\rank}{rank}
\newcommand{\redcol}[1]{\textcolor{red}{#1}}
\newcommand{\bluecol}[1]{\textcolor{blue}{#1}}
\newcommand{\greencol}[1]{\textcolor{green}{#1}}


\renewcommand{\~}{\htext{$\sim$}{~}}


% groupings of objects.
\newcommand{\set}[1]{\left\{ #1 \right\}}
\newcommand{\seq}[1]{\left(#1\right)}
\newcommand{\ang}[1]{\langle#1\rangle}
\newcommand{\tuple}[1]{\left(#1\right)}

% numerical shortcuts.
\newcommand{\abs}[1]{\left| #1\right|}
\newcommand{\floor}[1]{\left\lfloor #1 \right\rfloor}
\newcommand{\ceil}[1]{\left\lceil #1 \right\rceil}

% linear algebra shortcuts.
\newcommand{\change}{\Delta}
\newcommand{\norm}[1]{\left\| #1\right\|}
\newcommand{\dprod}[1]{\langle#1\rangle}
\newcommand{\linspan}[1]{\langle#1\rangle}
\newcommand{\conj}[1]{\overline{#1}}
\newcommand{\gradient}{\nabla}
\newcommand{\der}{\frac{d}{dx}}
\newcommand{\lap}{\Delta}
\newcommand{\kron}{\otimes}
\newcommand{\nperp}{\nvdash}

\newcommand{\mat}[1]{\left( \begin{smallmatrix}#1 \end{smallmatrix} \right)}

% derivatives and limits
\newcommand{\partder}[2]{\frac{\partial #1}{\partial #2}}
\newcommand{\partdern}[3]{\frac{\partial^{#3} #1}{\partial #2^{#3}}}

% Arrows
\newcommand{\diverge}{\nearrow}
\newcommand{\notto}{\nrightarrow}
\newcommand{\up}{\uparrow}
\newcommand{\down}{\downarrow}
% gets and gives are defined!

% ordering operators
\newcommand{\oleq}{\preceq}
\newcommand{\ogeq}{\succeq}

% programming and logic operators
\newcommand{\dfn}{:=}
\newcommand{\assign}{:=}
\newcommand{\co}{\ co\ }
\newcommand{\en}{\ en\ }


% logic operators
\newcommand{\xor}{\oplus}
\newcommand{\Land}{\bigwedge}
\newcommand{\Lor}{\bigvee}
\newcommand{\finish}{$\Box$}
\newcommand{\contra}{\Rightarrow \Leftarrow}
\newcommand{\iseq}{\stackrel{_?}{=}}


% Set theory
\newcommand{\symdiff}{\Delta}
\newcommand{\union}{\cup}
\newcommand{\inters}{\cap}
\newcommand{\Union}{\bigcup}
\newcommand{\Inters}{\bigcap}
\newcommand{\nullSet}{\phi}

% graph theory
\newcommand{\nbd}{\Gamma}

% Script alphabets
% For reals, use \Re

% greek letters
\newcommand{\eps}{\epsilon}
\newcommand{\del}{\delta}
\newcommand{\ga}{\alpha}
\newcommand{\gb}{\beta}
\newcommand{\gd}{\del}
\newcommand{\gf}{\phi}
\newcommand{\gF}{\Phi}
\newcommand{\gl}{\lambda}
\newcommand{\gm}{\mu}
\newcommand{\gn}{\nu}
\newcommand{\gr}{\rho}
\newcommand{\gs}{\sigma}
\newcommand{\gt}{\theta}
\newcommand{\gx}{\xi}

\newcommand{\sw}{\sigma}
\newcommand{\SW}{\Sigma}
\newcommand{\ew}{\lambda}
\newcommand{\EW}{\Lambda}

\newcommand{\Del}{\Delta}
\newcommand{\gD}{\Delta}
\newcommand{\gG}{\Gamma}
\newcommand{\gO}{\Omega}
\newcommand{\gL}{\Lambda}
\newcommand{\gS}{\Sigma}

% Formatting shortcuts
\newcommand{\red}[1]{\textcolor{red}{#1}}
\newcommand{\blue}[1]{\textcolor{blue}{#1}}
\newcommand{\htext}[2]{\texorpdfstring{#1}{#2}}

% Statistics
\newcommand{\distr}{\sim}
\newcommand{\stddev}{\sigma}
\newcommand{\covmatrix}{\Sigma}
\newcommand{\mean}{\mu}
\newcommand{\param}{\gt}
\newcommand{\ftr}{\phi}

% General utility
\newcommand{\todo}[1]{\footnote{TODO: #1}}
\newcommand{\exclaim}[1]{{\textbf{\textit{#1}}}}
\newcommand{\tbc}{[\textbf{Incomplete}]}
\newcommand{\chk}{[\textbf{Check}]}
\newcommand{\oprob}{[\textbf{OP}]:}
\newcommand{\core}[1]{\textbf{Core Idea:}}
\newcommand{\why}{[\textbf{Find proof}]}
\newcommand{\opt}[1]{\textit{#1}}


\DeclareMathOperator*{\argmin}{arg\,min}
\DeclareMathOperator{\rank}{rank}
\newcommand{\redcol}[1]{\textcolor{red}{#1}}
\newcommand{\bluecol}[1]{\textcolor{blue}{#1}}
\newcommand{\greencol}[1]{\textcolor{green}{#1}}


\renewcommand{\~}{\htext{$\sim$}{~}}


% groupings of objects.
\newcommand{\set}[1]{\left\{ #1 \right\}}
\newcommand{\seq}[1]{\left(#1\right)}
\newcommand{\ang}[1]{\langle#1\rangle}
\newcommand{\tuple}[1]{\left(#1\right)}

% numerical shortcuts.
\newcommand{\abs}[1]{\left| #1\right|}
\newcommand{\floor}[1]{\left\lfloor #1 \right\rfloor}
\newcommand{\ceil}[1]{\left\lceil #1 \right\rceil}

% linear algebra shortcuts.
\newcommand{\change}{\Delta}
\newcommand{\norm}[1]{\left\| #1\right\|}
\newcommand{\dprod}[1]{\langle#1\rangle}
\newcommand{\linspan}[1]{\langle#1\rangle}
\newcommand{\conj}[1]{\overline{#1}}
\newcommand{\gradient}{\nabla}
\newcommand{\der}{\frac{d}{dx}}
\newcommand{\lap}{\Delta}
\newcommand{\kron}{\otimes}
\newcommand{\nperp}{\nvdash}

\newcommand{\mat}[1]{\left( \begin{smallmatrix}#1 \end{smallmatrix} \right)}

% derivatives and limits
\newcommand{\partder}[2]{\frac{\partial #1}{\partial #2}}
\newcommand{\partdern}[3]{\frac{\partial^{#3} #1}{\partial #2^{#3}}}

% Arrows
\newcommand{\diverge}{\nearrow}
\newcommand{\notto}{\nrightarrow}
\newcommand{\up}{\uparrow}
\newcommand{\down}{\downarrow}
% gets and gives are defined!

% ordering operators
\newcommand{\oleq}{\preceq}
\newcommand{\ogeq}{\succeq}

% programming and logic operators
\newcommand{\dfn}{:=}
\newcommand{\assign}{:=}
\newcommand{\co}{\ co\ }
\newcommand{\en}{\ en\ }


% logic operators
\newcommand{\xor}{\oplus}
\newcommand{\Land}{\bigwedge}
\newcommand{\Lor}{\bigvee}
\newcommand{\finish}{$\Box$}
\newcommand{\contra}{\Rightarrow \Leftarrow}
\newcommand{\iseq}{\stackrel{_?}{=}}


% Set theory
\newcommand{\symdiff}{\Delta}
\newcommand{\union}{\cup}
\newcommand{\inters}{\cap}
\newcommand{\Union}{\bigcup}
\newcommand{\Inters}{\bigcap}
\newcommand{\nullSet}{\phi}

% graph theory
\newcommand{\nbd}{\Gamma}

% Script alphabets
% For reals, use \Re

% greek letters
\newcommand{\eps}{\epsilon}
\newcommand{\del}{\delta}
\newcommand{\ga}{\alpha}
\newcommand{\gb}{\beta}
\newcommand{\gd}{\del}
\newcommand{\gf}{\phi}
\newcommand{\gF}{\Phi}
\newcommand{\gl}{\lambda}
\newcommand{\gm}{\mu}
\newcommand{\gn}{\nu}
\newcommand{\gr}{\rho}
\newcommand{\gs}{\sigma}
\newcommand{\gt}{\theta}
\newcommand{\gx}{\xi}

\newcommand{\sw}{\sigma}
\newcommand{\SW}{\Sigma}
\newcommand{\ew}{\lambda}
\newcommand{\EW}{\Lambda}

\newcommand{\Del}{\Delta}
\newcommand{\gD}{\Delta}
\newcommand{\gG}{\Gamma}
\newcommand{\gO}{\Omega}
\newcommand{\gL}{\Lambda}
\newcommand{\gS}{\Sigma}

% Formatting shortcuts
\newcommand{\red}[1]{\textcolor{red}{#1}}
\newcommand{\blue}[1]{\textcolor{blue}{#1}}
\newcommand{\htext}[2]{\texorpdfstring{#1}{#2}}

% Statistics
\newcommand{\distr}{\sim}
\newcommand{\stddev}{\sigma}
\newcommand{\covmatrix}{\Sigma}
\newcommand{\mean}{\mu}
\newcommand{\param}{\gt}
\newcommand{\ftr}{\phi}

% General utility
\newcommand{\todo}[1]{\footnote{TODO: #1}}
\newcommand{\exclaim}[1]{{\textbf{\textit{#1}}}}
\newcommand{\tbc}{[\textbf{Incomplete}]}
\newcommand{\chk}{[\textbf{Check}]}
\newcommand{\oprob}{[\textbf{OP}]:}
\newcommand{\core}[1]{\textbf{Core Idea:}}
\newcommand{\why}{[\textbf{Find proof}]}
\newcommand{\opt}[1]{\textit{#1}}


\DeclareMathOperator*{\argmin}{arg\,min}
\DeclareMathOperator{\rank}{rank}
\newcommand{\redcol}[1]{\textcolor{red}{#1}}
\newcommand{\bluecol}[1]{\textcolor{blue}{#1}}
\newcommand{\greencol}[1]{\textcolor{green}{#1}}


\renewcommand{\~}{\htext{$\sim$}{~}}


\title{General sets and operations}
\author{vishvAs vAsuki}

\begin{document}
\maketitle
\tableofcontents

\part{Prelude}
\chapter{Notation}
'Algebra about strucure and equalities. Analysis about inequalities.'

Entire/ Integral function is holo morphic (differentiable) on entire complex plane.

\chapter{Research Themes}
Properties of various algebraic objects, their relationships.

\section{Characterization of research effort}
See linear algebra survey, complexity theory survey.

\chapter{Algebra techniques}
\section{Symbol manipulation, abstraction}
Algebra is about correct reasoning: symbol manipulation according to some rules.

\subsection{Abstraction: sub expressions}
It is about abstraction by means of techniques such as \textbf{change of variables}.

Look for structure in expressions, understand and if necessary, abstract away sub-expressions with new variables: very important! \exclaim{Too many symbols can slow down cognition!}

Use properties of algebraic objects well, keep a list of such properties handy.

\section{Common proof techniques}
Induction. Direct inference. 

\subsection{Contradiction}
\textbf{Diagonalization}: Order all elements, make a new element which differs from every other element.


\part{Sets}

\chapter{Set S}
\section{Specification}
Use vectors $\in \set{0,1}^{n}$. Or use an indicator function: $I_{S}(x) = 1$ if $x \in S$.

\subsection{Variants}
Multiset/ Bag: set with repeats. Class: set of sets.

\section{Operations}
$\union, \inters, -, \symdiff$; universal set U.

\subsection{Product}
Set (Cartesian) product of sets, $A \times B = \set{(a, b)| a \in A \land b \in B}$.

Subsets of product of sets, or relations, are considered elsewhere.

\subsection{Disjoint union}
A union of disjoint sets, with each element subscripted by the set it originates from.

\subsection{Properties}
Connections to logic: De Morgan laws.

\section{Impossible sets}
S : set of all sets which are not members of themselves: See if S is a member of itself.

\section{Metric spaces and topology}
See topology survey.

\section{Partition of set S}
A mutually disjoint $\set{S_i}$ such that $\union_i S_i = S$.

\chapter{Orders over set S}
\section{Total order}
All $(x, y) \in S$ comparable.

\subsection{Minimum element}
$\forall y \in S: x\leq y$.

\section{Partial order}
Aka Posets: Partially Ordered sets. Maybe some $(x, y) \in S$ not comparable. Eg: vectors, componentwise $<$. Visualize with Hasse diagrams.

\subsection{Minimal element}
$\forall y \in S: x\leq y \implies x = y$. Note difference from minimum. Eg: in triangle joining (1,1), (0, 1), (1, 0): line joining the last 2 pts are minimal elements.


\section{Bounds on A, subset of S}
Upper and lower bounds: need not be in A.

Supremum or least upper bound or GLB or join. Infimum or greatest lower bound or LUB or meet.

\subsection{Difference from max and min}
$\max(A) \in A \subseteq S$ always, but $\sup(A) \notin A, \sup(A) \in S$ possible.

\subsection{Supremum property in S}
S with supremum property: For any $A \subseteq S: \exists \sup A \in S$.

If S has supremum property, it has infemum property: For any subset $A \neq S$, take S-A and find its supremum.

\subsection{Examples}
Q does not have supremum property, but R does. See complex analysis survey.

\section{Lattices}
Sets where every pair has a same supremum and same infemum: Diamond.

\section{Well founded order}
Take any 'decreasing chain' $a<b<c..$: it must be finite. So, $a<a$ not allowed, no cycles too. Not total or partial order: no transitivity etc required.

Thence get 'well founded set'. Minimal elements exist.

Eg: $(N, <)$, (strings, psurveyix), (strings, subsequence), (trees, subtree). Lexicographic ordering of $((a, b), <)$ is well founded if $<$ well founded over dom(a) and dom(b).

\subsection{Mathematical induction proofs generalized}
(Noether). If $\forall y:: x>y \land p(y) \implies p(x)$, then $\forall x: p(x)$; note: base cases subsumed here. Strictly more powerful than induction on natural numbers: consider lexicographic ordering.

\chapter{Algebras over sets}
\section{Boolean algebra over set X}
Bounded lattice where every element has complement, which is distributive. Eg: power sets wrt inclusion, propositional logic.

\section{Sigma algebra S over set X}
Aka Borel algebra. Let $2^{X}$ be the set of powersets of $X$. Let $S \subseteq 2^{X}$ be closed under countable unions and complementation.

Note that $S$ is also \exclaim{closed under intersection} due to these conditions: due to De Morgan's laws.

Sigma algebra is a  Boolean algebra defined over $S$ with the inclusion operation, extended to unbounded sets.

\subsection{Importance}
The sigma algebra is useful in defining a measurable space.

\chapter{Size}
\section{Measurable space}
Suppose that $S$ is a $\sigma$ algebra over $X$. $(X, S)$ is called a measurable space. Every member of $S$ is a measurable set.

$(S, 2^{S})$ is a common measurable space.

\subsection{Importance}
This notion is useful because it enumerates the sets whose size we want to measure.

\subsection{Product space}
You can take two measurable spaces $(S_1, F_1), (S_2, F_2)$, and, by set product get a bigger measurable space $(S_1 \times S2, F_1 \times F_2)$.

\section{Measure}
\subsection{Minimal definition}
$m:S\to [0, \infty]$, with $m(\phi)=0$ and the countable additivity property ($A \inters B= \nullSet \implies m(A \union B) = m(A) + m(B)$) is called a measure on $X$ of subsets $S$.

$(X, S, m)$ is called a measure space.

\subsubsection{Motivation}
This measure of size generalizes concepts such as volume/ area/ box measure, mass, time. Especially important measures are the box measure and the probability measure.

\subsection{Special classes}
\subsubsection{General additivity}
For some measures $m$, any bunch of mutually disjoint sets $S_i$: $m(\union_i S_i) = \sum_i S_i$. This is stronger than countable additivity.

\subsubsection{Finite measures}
If $X$ is a countable union of finite measure sets, $m$ is $\gs-$finite. This is a very common property.

\subsubsection{Signed measure}
If $m(x) < 0$ is allowed, then $m$ is a signed measure.

\subsection{Null set, almost-everywhereness}
If $m(T) = 0$, then $T$ is called a $m-$null set.

A property (eg: $f(x)>0$) holds 'almost everywhere' if the set of elements for which the property does not hold is a null set. Eg: 'Almost always' in applications of probability.

\subsection{Size of the union and intersection}
\subsubsection{Inclusion/ exclusion principle}
The following holds for any measure which is finite on the sets involved. $|\union_{i \in V} S_i| = \sum_i |S_i| - \sum_{i \neq j}|S_i \inters S_j| + .. = \sum_{T \subseteq V} (-1)^{|T|+1}|\inters_{i \in T}S_i|$.

\subsubsection{Bounds}
Thence, we have the union upper bound: $m(A \union B) \leq m(A) + m(B)$. In the case of probabilistic analysis, this is very useful. Aka Boole's inequality.

Intersection lower bound: $m(A \inters B) \geq m(A) + m(B) - 1$. Aka Bonferroni's inequality. \pf{$m(A \union B) \leq 1$ with the inclusion-exclusion principle.}

By mathematical induction, $m(\inters_{i \in (1, n)} A_i) \geq \sum_i m(A_i) - (n-1)$.

\subsubsection{Generalization}
(Mobius inversion lemma). Got functions on sets f, g. $[\forall A \subseteq V: f(A) = \sum_{B \subseteq A} g(B)] \equiv [g(B) = \sum_{B \subseteq A}(-1)^{A-B}f(B)]$. \chk Easy algebraic proof.

\section{Counting measure}
The counting measure of $A$, $|A|$, equals the number of elements in a set. 

\subsection{Counting, combinatorics}
See probability survey.

\section{Cardinality}
The concept of Cardinal numbers extends the notion of the counting measure to compare the sizes of even infinite sets. For finite sets, the cardinality equals the counting measure.

\subsection{Comparison by bijection}
Comparison of cardinalities of A and B can be made by making bijections, even if they're $\infty$ sets: 'Equinumerousness'.

\subsection{Hierarchy of cardinal numbers}
Consider the power set $P(S)$. $|P(S)|>|S|$.

\section{Cardinalities of compared to N}
Cardinality (or power) of the continuum $c = |R|$; $\aleph_{0} = |N|$. Continuum hypothesis: $\exists c'?: \aleph_{0}< c'<c$.

\subsection{Countability}
$S$ is countable if it can be mapped to $N$.

Countable unions of countable $S$ still countable: write any $S$ as a row vector, see their sequence as a matrix, draw a zig-zag line to cover all matrix elements. Similarly, see countability of Q.

If $S$ is countable, $S^{n}$ is countable: use induction: if $S^{k-1}$ countable, for every $a\in S$, $\set{ba: b\in S^{k-1}}$ is countable; So, their union is also countable.

\subsection{Show uncountability}
Use Cantor's diagonalization.

\subsection{Infinite (sub)sets' cardinality}
(Dedekind): $S$ is $\infty$ iff $\exists A\set S$ with same cardinality as $S$. \pf{Finite $S$ can't have such a proper subset. If $|S|= \infty$, get countably $\infty$ $S'$; map to $N$ with function $f$; but map $n\in N$ to $n+1$ with function g, do $f^{-1}$.}

\section{Product measure}
Consider the product of two measure spaces: $\set{(S_i, F_i, m_i) | i \in \set{1, 2}}$. The product measure: $m(E_1, E_2) = m_1(E_1) \times m_2(E_2) : \forall E_i \in S_i$.

By induction, one can define product measure for the product of arbitrary number of measure spaces.

\subsection{Importance}
This measure finds application in defining, for example, measures for $R^{n}$ based on the common box measure for $R$; and in considering measures over the product of ranges of multiple random variables.

\subsection{Extension to bigger sigma algebra}
Consider the product space with an expanded sigma algebra $T$ such that $(F_1 \times F_2) \subseteq T \subseteq 2^{S_1} \times 2^{S_2}$.

For $A \in T$, one can use the product measure $m$ to define the minimum cover measure $m'(A) = \inf \set{\sum_i m(B_i) : A \subseteq \union_i B_i}$. One can show that this obeys required properties like countable additivity.

This is aka Lebesgue measure.

\subsubsection{Importance}
It forms a natural basis for defining and studying the box integral over product of multiple measure spaces.

\section{Connecting measures}
\subsection{Absolute continuity}
Consider two $\gs-$finite measures $m, n$. $m$ is absolutely continuous with $n$ - or $m$ is dominated by $n$ - or $m << n$ if $\forall t \in S: n(t) = 0 \implies m(t) = 0$.

This is equivalent to a definition reminiscent of absolute continuity of functions $n, m$: $\forall \eps, \exists \gd: \forall t: n(t) \leq \gd \implies m(t) \leq \eps$. Prior definition implies this because if there $\exists \eps, t: m(t) \geq \eps \land (n(t) \leq \gd \forall \gd)$ then for that t, $m(t) \geq \eps$ while $n(t) = 0$.

This notion is important in defining the inter-measure derivative.

\subsection{Inter-measure Derivative}
Aka Radon-Nikodym derivative. For measures $m<<n$ over $(X, S)$, a theorem by Radon/ Nikodym says that $\exists f:X \to [0, \infty]: m(t) = \int_{x \in t} f(x) dn$, and that this $f$ is unique almost everywhere wrt $n$. \why

Note that $m<<n$ is necessary: otherwise, for the event $E$ where $n(E) = 0, m(E) \neq 0$, there is no $f$ such that: $m(E) = \int_{E} f(x) dn$.

This concept is important in defining probability density functions of random variables.

\part{Relations}
\chapter{Relations and functions}
\section{Relations among n sets}
\subsection{Definition}
It is a subset of $A_1 \times .. A_n$.

It is a binary relation between $A_1 \times .. A_{n-1}$ and $A_n$.

\subsection{Binary relation R on (A, B)}
Aka dyadic relation.

\subsubsection{Definitions/ views}
A relation is fully defined by a subset $G \subseteq A \times B$, called the graph of $R$. So, it is a $R = (A, B , G)$.

It corresponds to a function $2^{A} \to 2^{B}$, and to the characteristic function $A \times B \to \set{0, 1}$.

It is a general many-to-many relationship : a directed graph involving the sets.

\subsubsection{(Co)Domain}
$A$ is the domain/ set of departure. $B$ is the co-domain/ set of destination.

Domain of definition is $\set{a: a \in A, \exists b \in B: aRb}$.

range(R) is the subset of $B$ related by $R$ to some element in $A$.

\subsubsection{Totality}
If ran(f) = codomain(f), f is onto / surjective/ right-total.

If domain of definition = domain, f is left-total.

A correspondence: a binary relation that is both left-total and surjective.

\subsection{Endo-relations}
A relation where the domain = co-domain.

The set of endo-relations is same as the set of directed graphs.

\subsubsection{Equivalence}
Equivalence relations: Reflexive ($a R a$), symmetric ($a R b \implies b R a$), transitive ($a R b \land b R c \implies a R c$).

The set of symmetric relations is the set of undirected graphs.

Equivalence class determined by a set of elements $S$ and equivalence relation $R$ is the set of all elements related to elements in $S$ by $R$.

\subsubsection{Congruence}
Complement: $A \times B - R$.

Restricting domain/ codomain of the relation, we get other (left/ right) restricted relations.

\subsubsection{Reduction and closure}
Equivalence relation which preserves certain algebraic operators. Eg: Modulo arithmetic preserves +, *, -.

\subsection{Functions on relation R: A to B}
Ensuring or removing all cases of reflexivity, symmetry and transitivity, we get closures and reductions of relations.

\subsubsection{Inverse}
$R^{-1}(b) = \set{a: a\in A, R(a) = b}$.

\section{Functions/ transformation f}
\subsection{Partial function A to B}
Aka functional, right unique.

\subsubsection{Definition}
It is a special binary relation, where every element in $A$ is mapped to at most one element in $B$.

\subsubsection{(Co)domain sets}
The domain of definition is also called the preimage. The range is also called the image.

\subsection{(Total) function}
A function is a partial function which is left-total.

A function acts. Like an electrical circuit with an input and an output.

\subsection{Types}
If every element in B has at most one preimage, $f$ is said to be One to one / injection.

A bijective function is both injective and surjective.

Also see survey on Analysis of functions over fields.

\subsection{Vector nature}
A finite domain function can be seen as a vector. So can an $\infty$ domain function. See functional analysis survey.

\subsection{Domain: Interesting locations}
\subsubsection{Level Set}
$\set{x | f(x) = c}$. A 2d contour line for 3d function. See linear algebra survey for geometric properties.

Kernel is the 0 level set.

\subsubsection{Fixed point}
f(w) = w.

\subsection{Traits of functions from X to X}
Idempotence: $f^{n}(x) = f(x)$. Nilpotence: $\exists n: f^{n}(x) = 0$.

\subsection{Measurable function}
Consider a function $f: X_1 \to X_2$, where $(X_i, S_i) \forall i \in \set{1, 2}$ are measurable spaces.

$f$ is a measurable function if the preimage $f^{-1}(s \in S_2)$ is a measurable set: a member of $S_1$. (This is analogous to definition of continuous functions over metric spaces.) So, it preserves some structure - but not fully: not every member of $S_1$ is represented in $S_2$ - only a subset is.

This notion is important in defining box integrals and random variables.

\subsection{Function/ model family}
Suppose that $f:X \times W \to Y$. Suppose that $w \in W$ are designated parameters, and $x \in X$ is designated the independent variable. Then, \\$\set{f_w:X \to Y = f(x, w)| w \in W}$ is a parametrized family of functions.

Such function families occur frequently, for example, in machine learning.

\section{Sequence of maps to metric space}
Consider $E \subseteq S$.

\subsection{Pointwise convergence on E}
$f_{n} \to f$ pointwise if $\forall x \in E, \eps, \exists N: n > N \implies d(f_n(x), f(x)) < \eps$. Visualize geometrically as a sequence of curves which get closer and closer at different rates at different points.

\subsection{Uniform convergence on E}
$f_{n} \to f$ if $\forall \eps, \exists N: n > N \implies\\
 \forall x \in E\ d(f_n(x), f(x)) < \eps$. $f_{n} \to f$ uniformly $\equiv$ $\sup_{x \in E} d(f_{n}(x), f(x)) \to 0$.

Visualize geometrically as a sequence of curves which get closer and closer at all points.

Cauchy criterion: $\forall n, m>N, x: d(f_{n}(x), f_{m}(x)) < \eps$.

\subsection{Interesting functions}
Point function: f(x) = 1 only if x = a, f(x) = 0 elsewhere.

\subsubsection{Important functions over R and C}
Includes polynomials over fields. See complex analysis survey.

\subsubsection{Sequence over S}
$f:N\to S$; $\set{a_{i}}_{i=1}^{\infty}$.

Subsequence: $\set{a_{j_{i}}}_{i=1}^{\infty}$: $\set{j_{i}}$ monotonically increasing. $(1^{k})$ not subsequence of N.

For topological properties, see topology survey.

\subsection{Randomized function}
For any set $S$, and set of random variables RV: $f: S \to RV$. RV $f(x)$ independent of $f(y)$ and of previous runs.

\subsubsection{Functions over vector spaces}
See linear algebra survey.

\subsubsection{Functions defined over convex and affine spaces}
See linear algebra survey.

\subsection{Function families and parameters}
Functions with a certain form(ula). An member function over $\set{x}$ actually specified by the parameter t. f(x, t).

\subsection{Operators}
See functional analysis survey.

\chapter{Category theory}
\section{Abstraction}
Aka general abstract nonsense. Abstract from sets and relations to categories and morphisms.

\section{Category}
(Class ob(C) of objects, morphisms or arrows hom(C), composition op: $\cdot$) with $\cdot$ identity, associative $\cdot$. Category Eg: Set, $Vect_{k}$.

Small category: aka CAT: both ob(C) and hom(C) are sets, rather than classes.

\section{Morphisms}
Homomorphism: A structure (identity, inverse elements, and binary ops) preserving funciton f: f(x)=3x preserves addition. Isomorphism: both f and $f^{-1}$ are homomorphisms. Endomorphism: homomorphism of a mathematical object to itself. Automorphism is both isomorphism and an endomorphism.

\section{Functors}
Structure preserving mapping between categories and their morphisms.

\part{Sets with operations}
\chapter{Group}
\section{Semigroup, monoid}
\textbf{Semigroup}: $<S,+>$: closed under binary operation +. \textbf{Monoid}: semigroup with identity element e.

\subsection{Function characteristics}
Consider functions on ordered semigroups. Some of these have some notable properties.

\subsubsection{Subadditivity}
$f(a + b) \leq f(a) + f(b)$.

\section{Group G}
\textbf{Group} (G): monoid with inverses. Commutative group. Cayley tables.

No element can have 2 inverses: $a_1^{-1}aa_2^{-1} = a_2^{-1} = a_1^{-1}$. $(ab)^{-1} = b^{-1}a^{-1}$. Unique solution for ax=b: $x=a^{-1}b$.

For examples $Z_{n}^{+}$ and $Z_{n}^{*}$, see Number Theory survey.

\subsection{Order of a group}
Number of elements in the group, $\phi(G)$.

\subsection{Subgroups}
$H \leq G$. Eg: p prime: $\set{\pm 1} \leq Z_{p}^{*}$.

\subsubsection{Cosets of subgroup}
Left coset of subgroup H containing g: gH or g+H; may not be group. Also, right coset of H containing g. Normal subgroup: N for which gN = Ng. Eg: 2Z or 2+Z has 2 cosets: evens and odds.

\subsubsection{Cardinality}
(Lagrange): If $H \leq G: |H| | |G|$: Take $a \in G-H$; then $aH \inters H = \nullSet$; repeat with $a' \in G-H-aH$ etc.. So, if $H < G$, $|H| \leq |G|/2$.

So, this is an easy partial-test to see if H is a group.

\subsubsection{Quotient/ factor group}
$G/N$: cosets; with Coset product: (aN)(bN) = abNN = abN; eN identity. Eg: Z/nZ isomorphic to $\{0, .. n-1\}, \oplus_n$.

\textbf{Product group}: G*H.

\subsection{Multiplicative order ord(a) of element a}
$ord(a) = argmin_{n}: a^{n}=e$. $ord(A) | \phi(G)$.

\subsection{Cyclic group G generated by a}
Every $a \in G$ generates some subgroup of G.

G is cyclic if some generator generates G. Then G is non degenerate. Eg: $Z_{4}$; $\omega$ in $\omega^{n}=1$.

\subsubsection{Number of generators}
If there is a generator g, there are at least $\phi(Z_{\phi(G)}^{*})$ of them: $Z_{\phi(G)}^{*}$ excludes all numbers which divide $\phi(G)$; so for any $a \in Z_{\phi(G)}^{*}$, can't write $(g^{a})^{b} = e$ for any $b < \phi(G)$.

\subsubsection{Periodic group}
Every element has finite order. All finite groups are periodic.

\section{Group homomorphism}
It(a) maps elements of two groups (G,H) : $a(g.h)=a(g).a(h)$. Image a(G). \textbf{Kernel} of homomorphism: ker(a) = G elements mapped to $1_{H}$. Isomorphic groups: homomorphism is invertible. ker(a) and a(G) measure closness to homomorphism. ker(a) is a normal subgroup. a(G) isomorphic to G/ker(a).

\chapter{Special groups}
\section{Symmetric group on X}
$S_{X}$ or Sym(X) or $S_{n}$ is a group of permutations/ bijective functions on X, under composition. Not commutative for $n>2$. \textbf{Transposition} only switches 2 elements. Every permutation f is a product of transpositions. Even and odd permutations. The product is not unique, but oddness is same: Consider number of pairs $i<j$, where $f(j)<f(i)$. Sign of Permutation: Sgn(f) is +1 or -1. Cycle.

\section{Elliptic curve groups}
See topology survey.

\section{Bilinear groups}
Groups with efficiently computable bilinear maps. $G_{T}$: target group; $g_{1}, g_{2}$ generators of $G_{1}$ and $G_{2}$. Bilinear map/ pairing operation: $p: G_1 \times G_2 \to G_{T}$. Not necessarily 1 to 1.

Bilinearity property: $p(g_{1}^{a}, g_{2}^{b}) = p(g_{1}, g_{2})^{ab}$; can be seen as bilinear map amongst exponents: $p'(a, b) = ab$. $p(xz,y) = p(z,y)p(x,y)$.

Can efficiently compute bilinear map $Z_{p} \times Z_{p} \to Z_{p}$. \why

No efficient way to make multilinear maps known.


\chapter{Ring}
\section{Ring}
$<set, *, +>$: generalizes $<Z, *, +>$. Division ring.

\subsection{Ideal I of Ring R}
Eg: Even numbers, multiples of 3 or 4. \textbf{Principle Ideal} is generated by 1 number.

\subsection{Polynomial ring}
The set of polynomials with coefficients taken from a field is  a commutative ring. (Z/2Z)(t).

\section{Field}
Division ring with commutative *. Eg: Q, R, C; not Z.

For prime p: GF(p) or Z/pZ or $F_{p}$ or $Z_{p}$: contains both additive, multiplicative subgroup ($F^{*}$); Euclid's alg proves inverse for latter. $Z/p^{n}Z : n>1$ not a field.

Size of any finite field is a prime power (Find proof); A finite field is a vector space in n dimensions. 2 equisized finite fields are isomorphic.

\section{Polynomial representation of \htext{$GF(p^n)$}{GF p n}}
Eg: $GF(p^{2}): (Z/2Z)(t)/ (t^{2}+1)$ is a finite field. The elements are from the polynomial ring. Operations are performed modulo the polynomial.

\section{Ordered field}
Field which is also an ordered set, with $x+y < x+z$ if $y<z$ and $xy>0$ if both above 0.

So, $x^{2}>0$; multiplication by +ve (but not -ve) x maintains inequality direction; for $0<x<y$, $0<y^{-1}<x^{-1}$.

\section{Linear algebra over a field}
See linear algebr survey.

%\bibliographystyle{plain}
%\bibliography{colt}

\end{document}
