Social network analysis has attracted increasing attention in recent years. In many social networks, besides friendship links amongst users, the phenomenon of users associating themselves with groups or communities is common. Thus, two networks exist simultaneously: the friendship network among users, and the affiliation network between users and groups. In this paper, we tackle the affiliation recommendation problem, where the task is to predict or suggest new affiliations between users and communities, given the current state of the friendship and affiliation networks. More generally, affiliations need not be community affiliations---they can be a user's taste, so affiliation recommendation algorithms have applications beyond community recommendation. In this paper, we show that information from the friendship network can indeed be fruitfully exploited in making affiliation recommendations. Using a simple way of combining these networks, we suggest two models of user-community affinity for the purpose of making affiliation recommendations: one based on graph proximity, and another using latent factors to model users and communities. 
We explore the affiliation recommendation algorithms suggested by these models and evaluate these algorithms on two real world networks---Orkut and Youtube. In doing so, we motivate and propose a way of evaluating recommenders, by measuring how good the top 50 recommendations are for the average user, and demonstrate the importance of choosing the right evaluation strategy. The algorithms suggested by the graph proximity model turn out to be the most effective. We also introduce scalable versions of these algorithms, and demonstrate their effectiveness. This use of link prediction techniques for the purpose of affiliation recommendation is, to our knowledge, novel.
