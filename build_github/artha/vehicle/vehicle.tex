\documentclass[oneside, article]{memoir}
\usepackage{amsmath, amssymb}
\usepackage{hyperref, graphicx, verbatim, listings, multirow, subfigure}
\usepackage{algorithm, algorithmic}
% \usepackage[bottom]{footmisc}
\lstset{breaklines=true}
\setcounter{tocdepth}{3}

% Lets verbatim and verb environments automatically break lines.
\makeatletter
\def\@xobeysp{ }
\makeatother
% \lstset{breaklines=true,basicstyle=\ttfamily}

% Configuration for the memoir class.
\renewcommand{\cleardoublepage}{}
% \renewcommand*{\partpageend}{}
\renewcommand{\afterpartskip}{}
\maxsecnumdepth{subsubsection} % number subsections
\maxtocdepth{subsubsection}

\addtolength{\parindent}{-5mm}
% Packages not included:
% For multiline comments, use caption package. But this conflicts with hyperref while making html files.
% subfigure conflicts with use with memoir style-sheet.

% Use something like:
% % Use something like:
% % Use something like:
% \input{../../macros}

% groupings of objects.
\newcommand{\set}[1]{\left\{ #1 \right\}}
\newcommand{\seq}[1]{\left(#1\right)}
\newcommand{\ang}[1]{\langle#1\rangle}
\newcommand{\tuple}[1]{\left(#1\right)}

% numerical shortcuts.
\newcommand{\abs}[1]{\left| #1\right|}
\newcommand{\floor}[1]{\left\lfloor #1 \right\rfloor}
\newcommand{\ceil}[1]{\left\lceil #1 \right\rceil}

% linear algebra shortcuts.
\newcommand{\change}{\Delta}
\newcommand{\norm}[1]{\left\| #1\right\|}
\newcommand{\dprod}[1]{\langle#1\rangle}
\newcommand{\linspan}[1]{\langle#1\rangle}
\newcommand{\conj}[1]{\overline{#1}}
\newcommand{\gradient}{\nabla}
\newcommand{\der}{\frac{d}{dx}}
\newcommand{\lap}{\Delta}
\newcommand{\kron}{\otimes}
\newcommand{\nperp}{\nvdash}

\newcommand{\mat}[1]{\left( \begin{smallmatrix}#1 \end{smallmatrix} \right)}

% derivatives and limits
\newcommand{\partder}[2]{\frac{\partial #1}{\partial #2}}
\newcommand{\partdern}[3]{\frac{\partial^{#3} #1}{\partial #2^{#3}}}

% Arrows
\newcommand{\diverge}{\nearrow}
\newcommand{\notto}{\nrightarrow}
\newcommand{\up}{\uparrow}
\newcommand{\down}{\downarrow}
% gets and gives are defined!

% ordering operators
\newcommand{\oleq}{\preceq}
\newcommand{\ogeq}{\succeq}

% programming and logic operators
\newcommand{\dfn}{:=}
\newcommand{\assign}{:=}
\newcommand{\co}{\ co\ }
\newcommand{\en}{\ en\ }


% logic operators
\newcommand{\xor}{\oplus}
\newcommand{\Land}{\bigwedge}
\newcommand{\Lor}{\bigvee}
\newcommand{\finish}{$\Box$}
\newcommand{\contra}{\Rightarrow \Leftarrow}
\newcommand{\iseq}{\stackrel{_?}{=}}


% Set theory
\newcommand{\symdiff}{\Delta}
\newcommand{\union}{\cup}
\newcommand{\inters}{\cap}
\newcommand{\Union}{\bigcup}
\newcommand{\Inters}{\bigcap}
\newcommand{\nullSet}{\phi}

% graph theory
\newcommand{\nbd}{\Gamma}

% Script alphabets
% For reals, use \Re

% greek letters
\newcommand{\eps}{\epsilon}
\newcommand{\del}{\delta}
\newcommand{\ga}{\alpha}
\newcommand{\gb}{\beta}
\newcommand{\gd}{\del}
\newcommand{\gf}{\phi}
\newcommand{\gF}{\Phi}
\newcommand{\gl}{\lambda}
\newcommand{\gm}{\mu}
\newcommand{\gn}{\nu}
\newcommand{\gr}{\rho}
\newcommand{\gs}{\sigma}
\newcommand{\gt}{\theta}
\newcommand{\gx}{\xi}

\newcommand{\sw}{\sigma}
\newcommand{\SW}{\Sigma}
\newcommand{\ew}{\lambda}
\newcommand{\EW}{\Lambda}

\newcommand{\Del}{\Delta}
\newcommand{\gD}{\Delta}
\newcommand{\gG}{\Gamma}
\newcommand{\gO}{\Omega}
\newcommand{\gL}{\Lambda}
\newcommand{\gS}{\Sigma}

% Formatting shortcuts
\newcommand{\red}[1]{\textcolor{red}{#1}}
\newcommand{\blue}[1]{\textcolor{blue}{#1}}
\newcommand{\htext}[2]{\texorpdfstring{#1}{#2}}

% Statistics
\newcommand{\distr}{\sim}
\newcommand{\stddev}{\sigma}
\newcommand{\covmatrix}{\Sigma}
\newcommand{\mean}{\mu}
\newcommand{\param}{\gt}
\newcommand{\ftr}{\phi}

% General utility
\newcommand{\todo}[1]{\footnote{TODO: #1}}
\newcommand{\exclaim}[1]{{\textbf{\textit{#1}}}}
\newcommand{\tbc}{[\textbf{Incomplete}]}
\newcommand{\chk}{[\textbf{Check}]}
\newcommand{\oprob}{[\textbf{OP}]:}
\newcommand{\core}[1]{\textbf{Core Idea:}}
\newcommand{\why}{[\textbf{Find proof}]}
\newcommand{\opt}[1]{\textit{#1}}


\DeclareMathOperator*{\argmin}{arg\,min}
\DeclareMathOperator{\rank}{rank}
\newcommand{\redcol}[1]{\textcolor{red}{#1}}
\newcommand{\bluecol}[1]{\textcolor{blue}{#1}}
\newcommand{\greencol}[1]{\textcolor{green}{#1}}


\renewcommand{\~}{\htext{$\sim$}{~}}


% groupings of objects.
\newcommand{\set}[1]{\left\{ #1 \right\}}
\newcommand{\seq}[1]{\left(#1\right)}
\newcommand{\ang}[1]{\langle#1\rangle}
\newcommand{\tuple}[1]{\left(#1\right)}

% numerical shortcuts.
\newcommand{\abs}[1]{\left| #1\right|}
\newcommand{\floor}[1]{\left\lfloor #1 \right\rfloor}
\newcommand{\ceil}[1]{\left\lceil #1 \right\rceil}

% linear algebra shortcuts.
\newcommand{\change}{\Delta}
\newcommand{\norm}[1]{\left\| #1\right\|}
\newcommand{\dprod}[1]{\langle#1\rangle}
\newcommand{\linspan}[1]{\langle#1\rangle}
\newcommand{\conj}[1]{\overline{#1}}
\newcommand{\gradient}{\nabla}
\newcommand{\der}{\frac{d}{dx}}
\newcommand{\lap}{\Delta}
\newcommand{\kron}{\otimes}
\newcommand{\nperp}{\nvdash}

\newcommand{\mat}[1]{\left( \begin{smallmatrix}#1 \end{smallmatrix} \right)}

% derivatives and limits
\newcommand{\partder}[2]{\frac{\partial #1}{\partial #2}}
\newcommand{\partdern}[3]{\frac{\partial^{#3} #1}{\partial #2^{#3}}}

% Arrows
\newcommand{\diverge}{\nearrow}
\newcommand{\notto}{\nrightarrow}
\newcommand{\up}{\uparrow}
\newcommand{\down}{\downarrow}
% gets and gives are defined!

% ordering operators
\newcommand{\oleq}{\preceq}
\newcommand{\ogeq}{\succeq}

% programming and logic operators
\newcommand{\dfn}{:=}
\newcommand{\assign}{:=}
\newcommand{\co}{\ co\ }
\newcommand{\en}{\ en\ }


% logic operators
\newcommand{\xor}{\oplus}
\newcommand{\Land}{\bigwedge}
\newcommand{\Lor}{\bigvee}
\newcommand{\finish}{$\Box$}
\newcommand{\contra}{\Rightarrow \Leftarrow}
\newcommand{\iseq}{\stackrel{_?}{=}}


% Set theory
\newcommand{\symdiff}{\Delta}
\newcommand{\union}{\cup}
\newcommand{\inters}{\cap}
\newcommand{\Union}{\bigcup}
\newcommand{\Inters}{\bigcap}
\newcommand{\nullSet}{\phi}

% graph theory
\newcommand{\nbd}{\Gamma}

% Script alphabets
% For reals, use \Re

% greek letters
\newcommand{\eps}{\epsilon}
\newcommand{\del}{\delta}
\newcommand{\ga}{\alpha}
\newcommand{\gb}{\beta}
\newcommand{\gd}{\del}
\newcommand{\gf}{\phi}
\newcommand{\gF}{\Phi}
\newcommand{\gl}{\lambda}
\newcommand{\gm}{\mu}
\newcommand{\gn}{\nu}
\newcommand{\gr}{\rho}
\newcommand{\gs}{\sigma}
\newcommand{\gt}{\theta}
\newcommand{\gx}{\xi}

\newcommand{\sw}{\sigma}
\newcommand{\SW}{\Sigma}
\newcommand{\ew}{\lambda}
\newcommand{\EW}{\Lambda}

\newcommand{\Del}{\Delta}
\newcommand{\gD}{\Delta}
\newcommand{\gG}{\Gamma}
\newcommand{\gO}{\Omega}
\newcommand{\gL}{\Lambda}
\newcommand{\gS}{\Sigma}

% Formatting shortcuts
\newcommand{\red}[1]{\textcolor{red}{#1}}
\newcommand{\blue}[1]{\textcolor{blue}{#1}}
\newcommand{\htext}[2]{\texorpdfstring{#1}{#2}}

% Statistics
\newcommand{\distr}{\sim}
\newcommand{\stddev}{\sigma}
\newcommand{\covmatrix}{\Sigma}
\newcommand{\mean}{\mu}
\newcommand{\param}{\gt}
\newcommand{\ftr}{\phi}

% General utility
\newcommand{\todo}[1]{\footnote{TODO: #1}}
\newcommand{\exclaim}[1]{{\textbf{\textit{#1}}}}
\newcommand{\tbc}{[\textbf{Incomplete}]}
\newcommand{\chk}{[\textbf{Check}]}
\newcommand{\oprob}{[\textbf{OP}]:}
\newcommand{\core}[1]{\textbf{Core Idea:}}
\newcommand{\why}{[\textbf{Find proof}]}
\newcommand{\opt}[1]{\textit{#1}}


\DeclareMathOperator*{\argmin}{arg\,min}
\DeclareMathOperator{\rank}{rank}
\newcommand{\redcol}[1]{\textcolor{red}{#1}}
\newcommand{\bluecol}[1]{\textcolor{blue}{#1}}
\newcommand{\greencol}[1]{\textcolor{green}{#1}}


\renewcommand{\~}{\htext{$\sim$}{~}}


% groupings of objects.
\newcommand{\set}[1]{\left\{ #1 \right\}}
\newcommand{\seq}[1]{\left(#1\right)}
\newcommand{\ang}[1]{\langle#1\rangle}
\newcommand{\tuple}[1]{\left(#1\right)}

% numerical shortcuts.
\newcommand{\abs}[1]{\left| #1\right|}
\newcommand{\floor}[1]{\left\lfloor #1 \right\rfloor}
\newcommand{\ceil}[1]{\left\lceil #1 \right\rceil}

% linear algebra shortcuts.
\newcommand{\change}{\Delta}
\newcommand{\norm}[1]{\left\| #1\right\|}
\newcommand{\dprod}[1]{\langle#1\rangle}
\newcommand{\linspan}[1]{\langle#1\rangle}
\newcommand{\conj}[1]{\overline{#1}}
\newcommand{\gradient}{\nabla}
\newcommand{\der}{\frac{d}{dx}}
\newcommand{\lap}{\Delta}
\newcommand{\kron}{\otimes}
\newcommand{\nperp}{\nvdash}

\newcommand{\mat}[1]{\left( \begin{smallmatrix}#1 \end{smallmatrix} \right)}

% derivatives and limits
\newcommand{\partder}[2]{\frac{\partial #1}{\partial #2}}
\newcommand{\partdern}[3]{\frac{\partial^{#3} #1}{\partial #2^{#3}}}

% Arrows
\newcommand{\diverge}{\nearrow}
\newcommand{\notto}{\nrightarrow}
\newcommand{\up}{\uparrow}
\newcommand{\down}{\downarrow}
% gets and gives are defined!

% ordering operators
\newcommand{\oleq}{\preceq}
\newcommand{\ogeq}{\succeq}

% programming and logic operators
\newcommand{\dfn}{:=}
\newcommand{\assign}{:=}
\newcommand{\co}{\ co\ }
\newcommand{\en}{\ en\ }


% logic operators
\newcommand{\xor}{\oplus}
\newcommand{\Land}{\bigwedge}
\newcommand{\Lor}{\bigvee}
\newcommand{\finish}{$\Box$}
\newcommand{\contra}{\Rightarrow \Leftarrow}
\newcommand{\iseq}{\stackrel{_?}{=}}


% Set theory
\newcommand{\symdiff}{\Delta}
\newcommand{\union}{\cup}
\newcommand{\inters}{\cap}
\newcommand{\Union}{\bigcup}
\newcommand{\Inters}{\bigcap}
\newcommand{\nullSet}{\phi}

% graph theory
\newcommand{\nbd}{\Gamma}

% Script alphabets
% For reals, use \Re

% greek letters
\newcommand{\eps}{\epsilon}
\newcommand{\del}{\delta}
\newcommand{\ga}{\alpha}
\newcommand{\gb}{\beta}
\newcommand{\gd}{\del}
\newcommand{\gf}{\phi}
\newcommand{\gF}{\Phi}
\newcommand{\gl}{\lambda}
\newcommand{\gm}{\mu}
\newcommand{\gn}{\nu}
\newcommand{\gr}{\rho}
\newcommand{\gs}{\sigma}
\newcommand{\gt}{\theta}
\newcommand{\gx}{\xi}

\newcommand{\sw}{\sigma}
\newcommand{\SW}{\Sigma}
\newcommand{\ew}{\lambda}
\newcommand{\EW}{\Lambda}

\newcommand{\Del}{\Delta}
\newcommand{\gD}{\Delta}
\newcommand{\gG}{\Gamma}
\newcommand{\gO}{\Omega}
\newcommand{\gL}{\Lambda}
\newcommand{\gS}{\Sigma}

% Formatting shortcuts
\newcommand{\red}[1]{\textcolor{red}{#1}}
\newcommand{\blue}[1]{\textcolor{blue}{#1}}
\newcommand{\htext}[2]{\texorpdfstring{#1}{#2}}

% Statistics
\newcommand{\distr}{\sim}
\newcommand{\stddev}{\sigma}
\newcommand{\covmatrix}{\Sigma}
\newcommand{\mean}{\mu}
\newcommand{\param}{\gt}
\newcommand{\ftr}{\phi}

% General utility
\newcommand{\todo}[1]{\footnote{TODO: #1}}
\newcommand{\exclaim}[1]{{\textbf{\textit{#1}}}}
\newcommand{\tbc}{[\textbf{Incomplete}]}
\newcommand{\chk}{[\textbf{Check}]}
\newcommand{\oprob}{[\textbf{OP}]:}
\newcommand{\core}[1]{\textbf{Core Idea:}}
\newcommand{\why}{[\textbf{Find proof}]}
\newcommand{\opt}[1]{\textit{#1}}


\DeclareMathOperator*{\argmin}{arg\,min}
\DeclareMathOperator{\rank}{rank}
\newcommand{\redcol}[1]{\textcolor{red}{#1}}
\newcommand{\bluecol}[1]{\textcolor{blue}{#1}}
\newcommand{\greencol}[1]{\textcolor{green}{#1}}


\renewcommand{\~}{\htext{$\sim$}{~}}

\title{Car and driving}
\author{vishvAs}

\begin{document}
\maketitle

\part{Car purchase}
\chapter{Identifying target}
\section{Target identification}
\subsection{About usage}
\subsubsection{Used vs new}
Used cars, even those that are only one year old, are 20 to 30 percent cheaper than new cars. You'll save money on insurance.

Most new cars are sold with a three-year/36,000-mile warranty. Therefore, if you buy a car that is from one to three years old, with less than 36,000 miles on the odometer, it will still be under the factory warranty.

\subsubsection{Usage type}
Single owned or double owned cars are preferable. Family owned cars good. Rental vehicles are reputed to be badly used.

Modified cars, and the speedometers on them are less trustworthy.

Cars from the south, where there is no snow, are preferable. Single owner, family cars are more desirable, as they are more likely to be well taken care of.

\subsubsection{Certification}
If it is a certified used car, there is no reason to take it to a mechanic.

\subsection{About various models}

Japanese cars are considered the most reliable, long-lasting and fuel efficient. American cars are reputed to be more powerful, but less reliable.

German cars are considered somewhat reliable, but the cost of parts is high. Honda part costs are lower. Mazda part costs are higher. Getting parts for discontinued models may be tougher.

Manual transmission cars are cheaper and more fuel efficient, but they are harder to sell.

Some American cars are imitations of Japanese cars, and hence are likely to be more reliable than average. Chevrolet Prizm is an imitation of Toyota Corolla.  If you are considering a Toyota Camry you should also look at the Honda Accord, Nissan Altima, or Mitsubishi Galant. Two Edmunds.com editors recently shopped in the family sedan class. They found that two-year-old Camrys and Accords were about \$3,000 more than comparable 626s and Galants.

Consumer reports indicates that, for less than 6000\$, the following used cars have better than average verdicts: (Honda civic and Toyota Corolla seem to be the perennial favorites.) Check MSN reliability ratings, howstuffworks, Edmunds customer reviews.

\begin{itemize}
\item Chevrolet Prizm 98-00
\item Ford Escort 99, 01, 02
\item Ford Mustang V6 98
\item Ford Ranger 2WD 98-99
\item Honda Civic 98
\item Hyundai Accent 03 (The ratings for the 02 model seem to have gone down within the past year. A disturbing trend.)
\item Mazda B series 2WD or Protege 98-00
\item Mercury Tracer 99
\item Nissan Altima 4 cylinder 98
\item Nissan Frontier 4 cylinder 98
\item Nissan Sentra 99
\item Saturn SL 99, 01
\item Saturn SW 98
\item Subaru Legacy 98
\item Toyota Corolla 98, 99
\item Toyota Echo 00
\end{itemize}

\subsection{Target description}

Target reputed, fuel efficient model from first or second owner. Do not buy heavily/ inconsistantly used cars. Do not buy cars with major/ costly problems.

Try to buy cars which do not require comprehensive insurance. Many cars were dented by hailstorms in Austin in 2008 and 2009. [Ref]

\section{Good places to look for cars}
\subsection{Private sellers}
\subitem If you buy from a private party, the negotiation process is less stressful.
\subitem Reasonable prices.

\subsubsection{Listings}
See craigslist and newspaper listings.

\subsection{New car dealerships}
\subitem Dealerships usually get these cars at rock-bottom prices. If you make a low offer, but one that gives them some profit, you just might get a great deal.
\subitem They sometimes offer certified cars, perhaps with manufacturer guarantee.

\subsection{Used car lots}
Car max sells certified cars.

\subitem Airport boulevard in Austin has many dealerships; bus 10 goes there.

\subsection{Used car websites}
\subitem carmax.com
\subitem cars.com
\subitem craigslist.com. Look for deals by searching for "must sell".
\subitem autotrader.com
\subitem Edmunds.com Used Vehicle Listings.

\subsection{Rental car sell-outs}
Rental car companies often sell their cars after operating them for some time. But see caveat about rental cars.

\subsection{Auctions}
Repossessed cars are often auctioned. \tbc

\section{Anecdotal data}
2007 observations: Manju bought a 2001 Civic with 78000 miles on it, for 8500\$. Manoj bought a 1999 car 6500\$. I might be able to get 98 Civic for less than 4500\$. John Root bought a Toyota Corolla 2000 car for 3500\$.

Anecdotal data: Dan drives the Suzuki DR650 motorcycle. Bought new for \$5000. It has been around for a long time, and is considered reliable. He drives it on highways for short distances.
About accessories:

\chapter{Inspection}
\section{Evaluating a car}
\subsection{Questions to ask on phone or in person}
Know Make/ Model, Milage (Ask reason for low/ high milage), number of doors, engine type, transmission type, whether it has leather or cloth upholstery, VIN number. What add-ons does it have?

Ask if he has the title, whether it is a salvaged title. Are you the first owner?

In case of dealer: Is it a trade-in or is it a lease return? Have any major parts been replaced? (Be cautious of buying a car that has had major repairs such as transmission rebuilds, valve jobs or engine overhauls.)

Are any repairs needed?

Has it been in an accident?

Is there anything else I need to know?

What is the asking price?

Also see questions to ask in person.

\subsection{Remote checks}

Check the vehicle's history at carfax.com or autocheck using the VIN. Buy the 1 month unlimited checks deal. Consider ownership, accident, milage record. Note that these records are incomplete, and do not encompass important records (eg: police) in many states. But it has in the past revealed failed emissions and safety tests.

Check KBB, Edmunds' True Market Value, \\
NADAguides.com and CarMax value. Edmunds gives the lowest price, usually. Also check trade-in (and Private Party and Dealer Retail) value in these, especially if negotiating with a dealer.

For motorcycles: KBB (trade-in and retail), NADAguides, MC news guide.

If purchasing a new car, check the inventory price.

Read dealership reviews.

\subsection{Questions to ask}

Judge consistency of usage using the right questions.

Is the title clear? Do you have the title?

Gauge trustworthiness of owner. Age, profession, (teen) kids? What car is he buying next?

Any problems with the car?

Why are you selling the car?

Does the car have any rust? 

Does all the electrical and mechanical systems work?

When was timing belt replaced? Any engine or transmission problems?

\subsection{Documents to ask for}
\begin{itemize}
\item The title: check ownership.
\item Inspection sticker.
\item Registration sticker.
\item Ask for 30K, 60K and 90K service records. Records for transmission fluid change after every 30K. Note whether the car has had oil changes at regular intervals (at every 5,000 to 7,500 miles).
\item Some sellers carry evaluation certificates.
\item Determine miles travelled. Calculate miles used per year.
\end{itemize}


\subsection{Visual Inspection checklist}

Try to inspect during the day. Try to arrange your test drive so that you start the engine when it is completely cold. Some cars are harder to start when they are dead cold and, when doing so, will reveal chronic problems.
Check for signs of accident, burglary, parts replacement:
Carfax and other records are not at all complete.

\subsubsection{Dents}
Are any of the doors replaced? Look for misalignment, original door color under the seals: it should usually be grey, if replaced, there can be color mismatch.

Look for signs of repainting: does one side have a slightly different shade compared to the other? Are nuts painted over (especially under the hood and the boot doors)?
    *

\subsubsection{Other items}
\begin{itemize}
\item Does it offer enough headroom? Legroom? Are the gauges and controls conveniently positioned?
\item Rust spots: These are often concentrated around the tyre frames, door frame. It is often painted over in order to hide it.
\item worn out tires
\item Non functioning locks
\item Dents, scratches, broken glass.
\item Signs of accident.
\item Wipers.
\item Lights and indicators. Higher frequency sound by the turn indicator indicates that a bulb is not working well.
\item Alignment of tires: Look from behind and front.
\item Door alignment and Window alignment. Can cause wind noise, ac inefficiency.
\item Cargo space/ boot.
\item Uneven tread within tires. (Eg: outer edge of drivers side rear tire and the inside edge of the passenger rear tire)
\item Start the car, insert finger in tail pipe and check to see if the emission is greasy / oily: indicated problem with emission or engine.
\item Start the car, check fluid levels.
\item Start the car, walk around, check for smoke coming from the sides: could indicate broken tailpipe.
\item Start the car, open the hood, check for abnormalities.
\item Start the car, place it in neutral, mildly press the accelerator and the break at once. If there is a problem in engine mounting, there will be vibration.
\end{itemize}

\subsection{Driving test}

Turn off the radio before you begin driving.

On the test drive, take your time and be sure to simulate the conditions of your normal driving patterns. If you do a lot of highway driving, be sure to go on the highway and take the car up to 65 mph. If you go into the mountains, test the car on a steep slope.

\begin{itemize}
\item Guages.
\item AC.
\item Acceleration from stop.
\item Passing acceleration (Does it downshift quickly and smoothly?)
\item Hill-climbing power
\item Visibility (Check for blind spots)
\item Engine sound.
\item Non-Smooth gear shifting.
\item Odd sounds.
\item Wheel alignment: Does the car keep going straight without need for compensation? Leave the steering wheel briefly.
\item Is the steering wheel loose?
\item Breaks.
\item Cornering, sharp turns.
\item Suspension (How does it ride?)
\end{itemize}

\subsection{Mechanic check}

After making an offer, but before making the deal, get it evaluated by a mechanic.

AutoPI will come to the spot and check the vehicle. They charge 125\$ for a 600 point check, which lasts about 45 minutes. There is a discount for UFCU members. Good reviews heard.

Firestone charges 50\$. Bad reviews heard.


\chapter{Finalizing, other info}
\section{Negotiation}

The foundation of successful negotiation is information.

Negotiate over any defects you notice during inspections.

It is possible to buy new cars below the inventory price: Dealerships get paid by manufacturer to display cars for 3 months after release. Also, during certain times of the year, dealerships may be keen to get rid of old cars in order to stock new ones.
In negotiating with dealers: Whilst negotiating over used cars, start at little over the trade-in value. Be prepared to walk out. Used car salesmen are often more experienced than new car salesmen. Don't include your old car's trade-in or financing during the negotiations. Once the price is negotiated, they can be brought up to check if the dealership can beat the bank loan rate.

Other dealership negotiation tips from Edmunds.com:
\begin{itemize}
\item Only enter into negotiations with a salesperson you feel comfortable with
\item Make an opening offer that is low, but in the ballpark
\item Decide ahead of time how high you will go and leave when your limit's reached
\item Walk out — this is your strongest negotiating tool
\item Be patient — plan to spend an hour or more negotiating
\item Leave the dealership if you get tired or hungry
\item Don't be distracted by pitches for related items such as extended warranties or anti-theft devices
\item Expect a "closer" (another salesman you've haven't previously dealt with) to try to improve the deal before you reach a final price.
\end{itemize}

Dealership tricks: wait extended periods of time during the negotiations, good guy/ bad guy and straight out lying and manipulation.

\subsection{Closing the deal}

"Trust your instincts. If you have any doubts about the vehicle or the seller, don't buy."

Leave a deposit, get a receipt acknowledging this and promising not to sell the car till you return in a couple of days.

Make payment, but get owner's signature on the title before the money changes hands. Pay with cashier's check if possible.

In dealerships: In most states, it will contain the cost of the vehicle, a documentation fee, a smog fee, a small charge for a smog certificate, sales tax and license fees (also known as DMV fees). Make sure you understand the charges and question the appearance of any significant, sudden additions to the contract. If any repair work is required, and has been promised by the dealer, get it in writing in a "Due Bill." Make sure the temporary registration has been put in the proper place.

Get insurance. Make sure you have insurance for the car you just bought before you drive it away.

Get car registration. Pay (6.5\%) tax: write down the official price, or say it is a gift. Application for Texas Certificate of Title (Form 130-U). Apply for title. Visit the County tax office with seller. The clerk will tell you whether there is any salvage or other issue with the vehicle.

\section{References}
Edmunds guide.

\part{Owning, maintaining a car}
\chapter{Insurance}
\section{Insurance}
Can buy it online: Geico.com or Progressive or State Farm or AAA.

Can add or remove comprehensive coverage depending on evaluation of risk during journey.

Non-commuter / pleasure cars merit lower insurance rates - especially because they travel fewer miles. Negotiation should be tried.

\subsection{Insurance coverage}
\subsubsection{Permissive use}
Insurance coverage travels with an automobile and not the driver. Therefore, it is irrelevant who was driving an automobile. Anyone you permit to use your car is covered.

It won't apply if you loan it for work/ commercial use. It won't apply to drivers living with you who are explicitly removed from car usage.

If you allow an unlicensed or inexperienced driver to operate your vehicle then its possible the car insurer may opt to try and deny any claim for damages. 

Permissive use coverage often only extends the bare minimum state insurance requirements to anyone else driving the car.

\section{Claims}
\subsection{At accident scene}
In case of an accident, exchange insurance information.  Better yet, if you have a cell-phone camera, photograph the damage, the drivers license, insurance details. Call the highway patrol/ police, ensure that a accident or incident report is filed, and get the officer and case ID. Copy that information.

\subsection{Filing a claim}
If the other person is at fault, file a claim, for which you call their phone number, give them the facts about the accident, report damages and injuries. Write down the claim number, the identity of the team handling the claim and follow up periodically.

\subsubsection{Determining liability}
The insurance company will then determine the liability (who is at fault, and to what extant). In order to do this, they may consult the other parties involved, view the police report.

\subsubsection{Damage assessment}
Once the liability is favorably determined, the insurance sends a person to inspect the damage. They then calculate the min(repair costs, value of the car prior to the incident). They then provide the repair costs (within the limits of the policy) if that is the smaller amount, or they will offer to buy the car in the other case. In the latter case, they offer to sell it back for a substantially lower amount (If a car is 'totaled', it is wise to reject this offer). In any case they will provide the treatment costs.

\chapter{Car maintenance}
\section{Oil change}
\subsection{Oil change intervals}
The 3000 mile oil change interval had a scientific basis when engines used non-multi-weight, non-detergent oil. There are still vehicles that need 3K oil changes, but it's not because the oil goes bad after 3K miles Eg: Saturn SK.

Dark oil does not indicate the need for an oil change. The way modern detergent motor oil works is that minute particles of soot are suspended in the oil. These minute particles pose no danger to your engine, but they cause the oil to darken. A non-detergent oil would stay clearer. If you never changed your oil, eventually the oil would no longer be able to suspend any more particles in the oil and sludge would form. Fortunately, by following the manufacturer's recommended oil change interval, you are changing your oil long before the oil has become saturated. The only real way to determine whether oil is truly in need of changing is to have an oil analysis performed. it's acceptable to err heavily on the safe side and simply follow the manufacturer's recommended change interval for severe service. [Ref]

Generally, severe service consists of operating the vehicle in a very muddy or dusty areas (because dust particles get through the air filter and contaminate the oil more quickly), operating the vehicle in a very hot areas (heat breaks down oil more quickly), using the vehicle only for short trips in cold weather (the moisture in the oil never gets vaporized), or using the vehicle for towing or when carrying a car-top carrier. If you primarily do freeway driving in moderate weather you do not fall into the severe service category.

Different countries have different maintenance schedules, even for the same car. At least part of the reason is due to the differences in fuel. 

\subsection{Oils and filters}

Virtually all modern multi-weight oils are detergent oils. Detergent oil, cleans the soot of the internal engine parts and suspends the soot particles in the oil. The particles are too small to be trapped by the oil filter and stay in the oil until you change it. These particles are what makes the oil turn darker.

The viscosity of multi-weight motor oil is specified using two numbers. The first number is the viscosity when the oil is cold. This is followed by the letter W (which stands for winter, not weight), which is followed by the number that indicates the viscosity when the oil is at operating temperature. The higher the number the thicker the oil. In order to protect an engine at start time, the oil needs to be thinner when cold so it flows freely. Viscosity modifiers are added to the base stock to make the oil flow better when cold, without making the oil too thin when hot. In warm climates, 10W30 is usually an acceptable alternative to the preferred 5W30 and may be used without measurable adverse effects.

Synthetic oils withstand higher temperatures before breaking down, and have more base stock and less viscosity modifiers. In short, synthetic may give you the peace of mind of knowing that you are using an oil that is far better than necessary for your vehicle, but it won't reduce wear or extend the life of the engine. They don't give a significant extended change interval.

Filter should be changed at every oil change.

Do not use any oil additives. They provide no benefit and can interfere and react with the additives already present in the oil.

\subsection{Oil change places}

Oil changes are pretty inexpensive when done at a reputable repair shop or dealer. Most dealers offer oil change specials that cost less than the quick-change oil places, and the dealers do a better job and use better filters. Another advantage of having it done at a repair shop or dealer is that you have solid legal proof of the date and mileage when the oil change took place.

\begin{itemize}
\item Check Champion Toyota website for coupons.
\item 1 to 1.5 hours wait is standard. Have reading material handy.
\end{itemize}

Some quick-lube places have been known to offer advertised specials that use SAE 30 oil, as opposed to 5W30 or 10W30. Pay the extra for the proper oil, or better yet avoid merchants that try to pull this kind of thing because it's an indicator that they are less than honest. Insist on the oil that is specified on your filler cap and in your manual.

Pumping the oil out through the dipstick hole instead of removing the drain plug is bad.

Never let a quick-lube place do any mechanical work on your vehicle.. They do no use journeyman mechanics.

You really want to bring your own filter, from the dealer, with you when you go to a quick-lube place. They may take \$1 or \$2 off the price if you do this but don't count on it.

Do not let a quick-lube place change or add any fluids other than oil. No transmission fluid, no brake fluid, no power steering fluid, no antifreeze, no oil additives, no fuel additives. It is just too easy for them to use the wrong fluid and cause permanent damage to your vehicle.

Engine flushes pump heated solvent through your engine, supposedly to wash away sludge. But regular oil changes with detergent oil already take care of the sludge problem.

\section{Fuel}

6/4/2009: Car milage seems to be roughly 25 mpg.

\section{Brake}
Front brakes: pads and rotors. Back brakes: shoes and drums.

\subsection{Examination}
Screetching: metal is embedded into the brake pads to make a sound when the pad is worn out.

While braking, does the vehicle swerve to one side?

If there is thumping or if there is slight screetching, the brake rotor is unevenly worn. Usually, this is solved by turning/ machining the rotor for evenness. This does not cause a safety problem.

\subsection{History}
Got back brakes stocked from the dealer.

Midas, April 2010: Got front brake pads replaced, but refused to machine the rotors for evenness. Was told that the brake pads will last between 30k (city driving) to 60k miles (highway driving). Warranty: Z55330030294 guarantees that I don't have to .

\subsection{Brake change places}
Buy parts at the dealership, get labor done elsewhere, in places like Brakecheck.

\begin{itemize}
\item Midas : 90\$ labor 2010, will refuse to replace brake pads if rotors are too worn out to be machined and I refuse to replace them.
\item Austin's finest alignment and brake quoted 75\$ for labor.
\item Dura Tune refused to use brake pads brought in by customers, as they make money on the pads and have warranty on them. Otherwise quoted 160 for rear brakes.
\item Brakecheck told me on phone they won't do it. But, praveeNa had said that they do it.
\end{itemize}

\section{Battery}
\subsection{Useful tools and materials}
jump-start cables. Portable battery for jump-starting: if it is strong (outputs good 12 volt DC voltage) and is kept charged. (Can rely on alternator for charging only if you drive a really long time.)

Multimeter to check battery, alternator etc..

'Maintenance free' batteries usually use better materials, are better sealed.

\subsection{Problem indicators and response}
Weak engine-start. Even if battery is fine, the alternator may not be outputting sufficient electricity.

If the battery is dead, fix the problem: get jump-started or towed to a repair facility. Make sure that the interior/ exterior lights have not been left on, the door has not been left open/ improperly closed etc.. - otherwise you will end up with a dead battery again.

\subsection{Terminals}
Keep the terminals corrosion free: use a spray and metal brush to keep them clean, apply high temperature grease to prevent corrosion. Unclean exterior can reduce power-flow.

\subsection{Checking goodness}
For checking battery, the car should be turned off - otherwise electricity is being produced by the alternator. This is possible only when the battery is sufficiently charged. It should be able to output 12 volts.

To check alternator, rev up the engine. The alternator, which charges the battery, should output 13.6 to 14 volts. When the engine is being accelerated, it should produce around 88 amps.

\subsection{History}
September 2010: The same battery had died. Had driven the car without problems just 4 days before. Used AAA to get a jump-start. Went to Autozone, where I found that the water level was very low - replaced water, managed to start the car again. Bought a maintenance-free battery at Sears Auto-center. That again died by Wednesday: discovered later that the interior light switch was left on. Called AAA again and got it replaced.

April 2010, Midas: Battery (in use since 2003, as indicated by the string 'C3') was found to be good, but exterior cleaning  was recommended as corrosion was present.

\section{Wheel alignment}
\subsection{Adjustment parameters}
Camber: angle to the road.

Toe: Angle from the vertical plane in the direction of the car.

Caster: Angle of the load relative to the tire.

Some (cheap tire-shop) mechanics are unable to adjust camber - which probably requires removal of wheels.

\section{Reputed mechanics}
\subsection{Special repairs}
3 parameters: Ability, honesty, expensiveness.

In case body shops are required, ask mechanics you trust.

\subsection{Negotiating with mechanics}
Get at least 3 estimates for any needed repair. Be sure that the repair shop guarantees their work and that they are ASE Certified in that area of repair.

Even if you are getting a free check, you have right to every courtesy as a customer, and for the promised service in full. Be forceful but polite in demanding it. Remember that you can ask to see the boss.

Often in mechanic chains, the mechanic fills out a form with detailed measurements. The sales-person may not give this to you (Example: alignment print-out.). Demand this.

Belonging to a wide-spread mechanic chain does not necessarily ensure quality. True about Goodyear Gemini auto shop; and also about Jiffy Lube as noted from reviews.

\section{Tire}
The tires are tubeless! Slow leaks are probably caused by nails stuck in the tyre.

\subsection{Examination}
\begin{itemize}
\item Check for uneven tread. Do tire rotation.
\item Check tyre air pressure often: 30 psi in general, 60 psi for the spare. Tires loose 2 psi every month. Maybe inflate slightly more. N2 is not required.
\end{itemize}

\subsection{Good leak-fixing shops}
Leal's tires at North Lamar and North Loop charged 11\$ on Jul 18 2011 to fix a slow leak - no wait was involved : it was a well-tuned process which used many electric tools (uncommon where I come from), and the mechanic was very skilled.

\section{Wipers}
\subsection{Examination}
Operate wipers, check for streaks. Wipers which leave streaks are dangerous - they leave stains when used on a hot day to clean the glass; this leads to decreased visiblity.

\subsection{Solution}
Wipers are easy to replace - do it on your own.

\section{Fluid levels}
Check for fluid levels, leaks.

Check for engine leaks under car. Note how to check engine oil (black), engine coolant (bright greenish yellow), brake fluid (clear, with slight brown tinge), washer fluid, power steering fluid (clear, with slight brown tinge), automatic transmission fluid (pink) levels.

Engine oil must be changed by a mechanic.

Change transmission fluid, break oil, engine coolant.

Start your own oil change, maintenance schedule.

Check your oil level at least every other fill-up.

\part{Car operation}
\chapter{Parking safety and security}
\chapter{Parking safety and security}
\section{Parallel Parking}
Park within 1 foot of the curb, but don't take the wheels too close. When opening the door, ensure that other vehicles going along the road are not obstructed.

\section{Angular Parking}
\subsection{Approach positioning}
Realize that rear of the car follows a smaller arc during sharp pivot-like turns. Just because the front part of the car did not hit a pillar does not necessarily mean that the rear of the car will do the same. Align the car with the parking spot properly before moving in.

\subsection{Final positioning}
If there are cars in adjacent slots, don't use the lines on the ground for guidance. Instead, aim to have your car body parallel to the adjacent car. Ensure that you have enough space to open the door.

\section{Location}
Also see the section on burglary avoidance, for choosing a secure parking location.

Remember the location of the car in the parking lot.
\subitem Try to park on the same side and the same row every time, if possible.
\subitem After you park, deliberately state and memorize its location.

\section{Burglary of contents}
\subsection{Reduce attractiveness, increase difficulty}
Hide things. Never leave plastic bags etc.. in plain sight - Thieves can think that it contains something precious.

Park the car in well lit areas, visible from windows of apartments, with high pedestrian traffic. Do not park next to the dumpster etc.., where the view is obstructed etc..

\subsection{Minimize damage in case of break-in}
Hide precious things. Hide tools near the spare tyre, below the trunk floor.

\subsection{Prosecuting the thief}
Maintain a record of the serial numbers of the items you store in the car: just photograph it, for example. Thieves try to sell these things in a pawn shop; and they run these serial numbers by the police.

\part{Bicycle}
\chapter{Features}
\section{Frame}
The bike frame is made either with attention to reduce weight (aluminium frames, without shock absorbers) or to increase sturdiness (as in case of mountain bikes). 

\section{Handle-bar and seat}
\subsection{Relative orientation and posture}
The seat and handle-bar height and angle, to a large extant determine posture while cycling. They determine what portion of the weight is borne by the arms, wrists, back, bottom etc.. The seat height also determines the extant to which the leg straightens while pedaling.

\subsection{Handle bar supports}
The handle bar may provide various provisions for supporting the weight of the body - including elbow pads, wrist-grips etc..

\subsection{Seat softness}
Softer seats provide more comfort and cushioning - though they are liable to absorbing water in a rain. Tieable seat covers made of materials like gel or memory foam may be used for additional softness.

\section{Wheel}
Important features include circumference, truing - the closeness to being a perfect circle. Bikes sold in supermarkets often undergo very rough handling, causing the truing to be defective.

Double-walled rims can undergo great punishment without loosing truing.

The wheel encloses the tube. Around the rim is a strip of rubber used to ensure that the spokes don't poke the tube.

\subsection{Axle}
The 'axle' which holds the wheel should ideally have as little friction as possible.

\section{Tire}
Tire breadth determines grip.

\section{Brakes}
Brush brakes tend to be less effective in rainy season than disk brakes. The latter are also less affected by loss of truing.

\subsection{Brake-application}
Brake-application has several interfaces. Most common is a pair of levers on either side of the handle-bar. In USA, the left lever controls the front brake and the right lever controls the back brake. In Europe and Asia, the reverse is true.

Brake is sometimes applied using a mechanism so that pedaling backwards applies the brake.

\section{Gear}
\subsection{Gear size and rotation speed}
Gears employed affect the speed with which the cycle chain completes a rotation. Use of larger gears attached to the pedal increase this rate. Use of smaller gears attached to the rear wheel increase the rate at which the rear wheel turns relative to the chain.

\subsection{Gear shifting}
Gear shifting is done using either the grip-shift or the lever shift mechanism. There is usually one mechanism for the front  and the rear gears - on the left and right handles respectively in USA. In US bikes gear shifting only works while pedaling.

Gear shifting happens by shifting the chain laterally. In case of some bikes, shifting into certain combination of front and rear gears results in the chain brushing against the shifting mechanism - this is resolved by shifting the gear appropriately. Some bikes avoid this by making those combinations impossible/ ineffective.

\section{Categories}
A road bike is often lighter frame (making carrying easier), with a larger wheel to allow for traveling greater distance with lesser pedaling, and with a very thin tire inflated to such a high level that often only a thin line touches the ground (thereby reducing friction).

A mountain bike has a sturdier frame, with good shock absorption abilities, with relatively smaller wheels, a broader tire.

Hybrids combine mountain bikes and road bikes. They have a broader wheel while maintaining a thin tread on road due to high inflation.

\chapter{Purchasing}
\section{Assembly quality}
When buying a non-lower-end bike, it is best to visit a shop where they are assembled by a good mechanic.

Supermarkets like walmart employ low-payed, low-trained mechanics to assemble bikes. They do a less thorough job. Eg: Lost truing results in repeated tire distensions, gears don't function properly.

\section{Usage}
Used Bikes certified and inspected by mechanics are better.

\chapter{Cycling}
\section{Posture}
While sitting upright: (5 degrees backward from the vertical). Also, even while holding the handle, ensure that the spinal chord behind the stomach is not concave but convex. 

You should be able to keep a slight bend in your elbows and not feel stretched out when holding the handlebars.

\subsection{Seat height and angle}
'You want to have the bicycle seat set to a height that allows your leg to extend until it is almost completely straight when you are sitting on the seat. There should be only a slight bend to the knee when your foot is on the pedal in the bottom position. This will maximize power and minimize fatigue.'

\subsection{Seat orientation relative to handlebar}
The height and distance of the handle-bar relative to the seat determines posture and weight distribution. A handle-bar too close to the seat post results in a concave bend (rather than convex) in the back-bone, resulting in back-ache.

Saddle should point around 10 to 15 degrees downward - that has been shown to reduce lower back pain by altering the default posture.

\section{Safety}
Use helmet and reflective gear. The bike should be equipped with front and rear reflectors.

\section{Cadence}
Cadence (pedal-pushes per minute) should be maintained constant irrespective of the rate of rotation of wheels (up or down). This is accomplished by proper use of gears - shifting to low gears when at slow speed (as in climbing an incline) and high gears while at high speed.

\section{Carrying stuff}
Carrying heavy items in a back-pack can be strenuous for the back. Carrying them in a front-pack - especially while riding upright leaning 5 degrees backwards is much less strenuous.

Various baskets and racks can be attached to the rear or front axles. Some that attach to only the seat post or the front handle are limited in the weight they can carry. There are also various pouches meant for carrying light items, which are attached to the seat post and handle-bar.

Heavy items carried attached to the handle decrease the manoeuvrability of the bike.

Provisions are usually made to attach racks by providing screw-holes in the frame near the rear wheel axle and under the seat post.

\section{Commute itihAsa}
Did a 3-mile commute by mountain bike to work in Mountain view (Jan 2012). Experienced back ache with seat height higher than handle-bar, and at roughly the same level as handle bar. Changed the height to be slightly lower than handle bar, tilted it forward a little - to no effect.

Switched to a hybrid bike for a month. I felt slightly better initially, but eventually, reduced pain returned - necessitating rest after the commute.

Then I switched to sandIp's road bike (bought at target for 250\$) with an aluminium frame and without no shock absorbers. It was so light that I could lift it with my left arm. Greater speed, almost complete lack of pain were the result.

I rode the hybrid bike again after that for 3 days, where I tried to imitate the posture experienced with sandIp's road bike. Before that, I tried a lagaam to the handlebar, enabling me to ride while maintaining a good back posture (5 degrees backward from the vertical). Also, even while holding the handle, I ensured that the spinal chord behind the stomach was not concave but convex. There was no pain as a result.

\part{Water vehicles}
\chapter{Stand up paddling}
The important thing is to keep the center of gravity on the center of the board.

\chapter{Paddling}
Greater paddling on one side or greater resistence or even reverse paddling to the other produces a turn. When there are multiple paddlers they may cooperate to make turns more rapid and avoid unintended turns.

\chapter{Wind-surfing}
\section{Situational awareness}
In stationary water, the direction of the wind may be determined from the waves. Or one may use the sail.

\section{Preparing the boat}
While sailing, the fin under the center board should be down in the water. The aft fin/ skeg is always down.

\section{Mechanics}
The sail and the fin operate on the same principle as an airplane wing. Critical forces are lift and drag. The lift generated by the sail is at an angle to the orientation of the board defined by the relative orientation of the sail. 

When both the fin and the sail are aligned, the component of the lift perpendicular to the board is negated to some extant by the opposite lift generated in the water by the fin. When the center of effort of the sail is forward or backward relative to the center fin, a turn results.

One cannot sail closer than 45 degrees into the wind.

\section{Controls}
The feet are almost always on the center-line. The back should always (mostly) face the wind.

\subsection{Use body weight}
The sail with the mast is very heavy. To lift the sail out of the water, one has to use the body weight. The body weight should balance the weight of the sail - if the sail is to one side of the board, one should lean slightly back to the other side.

\subsection{Static positioning}
Hold the mast with the sail facing windward, with feet on either side of the mast. One gets into this position upon lifting the sail from the water.

One can turn the board under the feet using baby steps in the feet while using the sail against the wind for resistance. Thus, one can turn the board and arrange to be on the board's left or right side while sailing.

\subsection{Sailing}
Here, you hold the boom, with feet aft of the mast. To go straight, you align the center of effort on the line perpendicular to the center fin.

\subsubsection{Turning}
To turn windward, you move the sail relatively backward. To turn leeward, move the sail relatively forward. One needn't remember this - one can experiment on the spot and turn where one needs to.

\subsubsection{Dealing with gust}
Just release the hand towards the aft, rather than be pulled with the sail and loose balance.

\section{Gross Movements}
Reaching: Travel perpendicular to the wind. Running: Travel windward. Working: travel leeward at an angle.

\end{document}
