\documentclass[11pt]{article}
% \usepackage[left=3cm,top=1.5cm,right=3cm,bottom=3cm]{geometry}
\usepackage{caption}
\usepackage{amsmath, amssymb}
\usepackage{hyperref, graphicx, verbatim, listings, multirow, subfigure}
\usepackage{algorithm, algorithmic}
% \usepackage[bottom]{footmisc}
\lstset{breaklines=true}
\setcounter{tocdepth}{3}

% Lets verbatim and verb environments automatically break lines.
\makeatletter
\def\@xobeysp{ }
\makeatother
% \lstset{breaklines=true,basicstyle=\ttfamily}

\textheight 8.8in
\textwidth 6.4in
\oddsidemargin +0.05in
\evensidemargin +0.05in
%\textheight 9.0in
%\textwidth 6.5in
%\oddsidemargin 0in
%\evensidemargin 0in
\topmargin 0in
\headheight 0in
\headsep 0in

\thispagestyle{empty}

% Use something like:
% % Use something like:
% % Use something like:
% \input{../../macros}

% groupings of objects.
\newcommand{\set}[1]{\left\{ #1 \right\}}
\newcommand{\seq}[1]{\left(#1\right)}
\newcommand{\ang}[1]{\langle#1\rangle}
\newcommand{\tuple}[1]{\left(#1\right)}

% numerical shortcuts.
\newcommand{\abs}[1]{\left| #1\right|}
\newcommand{\floor}[1]{\left\lfloor #1 \right\rfloor}
\newcommand{\ceil}[1]{\left\lceil #1 \right\rceil}

% linear algebra shortcuts.
\newcommand{\change}{\Delta}
\newcommand{\norm}[1]{\left\| #1\right\|}
\newcommand{\dprod}[1]{\langle#1\rangle}
\newcommand{\linspan}[1]{\langle#1\rangle}
\newcommand{\conj}[1]{\overline{#1}}
\newcommand{\gradient}{\nabla}
\newcommand{\der}{\frac{d}{dx}}
\newcommand{\lap}{\Delta}
\newcommand{\kron}{\otimes}
\newcommand{\nperp}{\nvdash}

\newcommand{\mat}[1]{\left( \begin{smallmatrix}#1 \end{smallmatrix} \right)}

% derivatives and limits
\newcommand{\partder}[2]{\frac{\partial #1}{\partial #2}}
\newcommand{\partdern}[3]{\frac{\partial^{#3} #1}{\partial #2^{#3}}}

% Arrows
\newcommand{\diverge}{\nearrow}
\newcommand{\notto}{\nrightarrow}
\newcommand{\up}{\uparrow}
\newcommand{\down}{\downarrow}
% gets and gives are defined!

% ordering operators
\newcommand{\oleq}{\preceq}
\newcommand{\ogeq}{\succeq}

% programming and logic operators
\newcommand{\dfn}{:=}
\newcommand{\assign}{:=}
\newcommand{\co}{\ co\ }
\newcommand{\en}{\ en\ }


% logic operators
\newcommand{\xor}{\oplus}
\newcommand{\Land}{\bigwedge}
\newcommand{\Lor}{\bigvee}
\newcommand{\finish}{$\Box$}
\newcommand{\contra}{\Rightarrow \Leftarrow}
\newcommand{\iseq}{\stackrel{_?}{=}}


% Set theory
\newcommand{\symdiff}{\Delta}
\newcommand{\union}{\cup}
\newcommand{\inters}{\cap}
\newcommand{\Union}{\bigcup}
\newcommand{\Inters}{\bigcap}
\newcommand{\nullSet}{\phi}

% graph theory
\newcommand{\nbd}{\Gamma}

% Script alphabets
% For reals, use \Re

% greek letters
\newcommand{\eps}{\epsilon}
\newcommand{\del}{\delta}
\newcommand{\ga}{\alpha}
\newcommand{\gb}{\beta}
\newcommand{\gd}{\del}
\newcommand{\gf}{\phi}
\newcommand{\gF}{\Phi}
\newcommand{\gl}{\lambda}
\newcommand{\gm}{\mu}
\newcommand{\gn}{\nu}
\newcommand{\gr}{\rho}
\newcommand{\gs}{\sigma}
\newcommand{\gt}{\theta}
\newcommand{\gx}{\xi}

\newcommand{\sw}{\sigma}
\newcommand{\SW}{\Sigma}
\newcommand{\ew}{\lambda}
\newcommand{\EW}{\Lambda}

\newcommand{\Del}{\Delta}
\newcommand{\gD}{\Delta}
\newcommand{\gG}{\Gamma}
\newcommand{\gO}{\Omega}
\newcommand{\gL}{\Lambda}
\newcommand{\gS}{\Sigma}

% Formatting shortcuts
\newcommand{\red}[1]{\textcolor{red}{#1}}
\newcommand{\blue}[1]{\textcolor{blue}{#1}}
\newcommand{\htext}[2]{\texorpdfstring{#1}{#2}}

% Statistics
\newcommand{\distr}{\sim}
\newcommand{\stddev}{\sigma}
\newcommand{\covmatrix}{\Sigma}
\newcommand{\mean}{\mu}
\newcommand{\param}{\gt}
\newcommand{\ftr}{\phi}

% General utility
\newcommand{\todo}[1]{\footnote{TODO: #1}}
\newcommand{\exclaim}[1]{{\textbf{\textit{#1}}}}
\newcommand{\tbc}{[\textbf{Incomplete}]}
\newcommand{\chk}{[\textbf{Check}]}
\newcommand{\oprob}{[\textbf{OP}]:}
\newcommand{\core}[1]{\textbf{Core Idea:}}
\newcommand{\why}{[\textbf{Find proof}]}
\newcommand{\opt}[1]{\textit{#1}}


\DeclareMathOperator*{\argmin}{arg\,min}
\DeclareMathOperator{\rank}{rank}
\newcommand{\redcol}[1]{\textcolor{red}{#1}}
\newcommand{\bluecol}[1]{\textcolor{blue}{#1}}
\newcommand{\greencol}[1]{\textcolor{green}{#1}}


\renewcommand{\~}{\htext{$\sim$}{~}}


% groupings of objects.
\newcommand{\set}[1]{\left\{ #1 \right\}}
\newcommand{\seq}[1]{\left(#1\right)}
\newcommand{\ang}[1]{\langle#1\rangle}
\newcommand{\tuple}[1]{\left(#1\right)}

% numerical shortcuts.
\newcommand{\abs}[1]{\left| #1\right|}
\newcommand{\floor}[1]{\left\lfloor #1 \right\rfloor}
\newcommand{\ceil}[1]{\left\lceil #1 \right\rceil}

% linear algebra shortcuts.
\newcommand{\change}{\Delta}
\newcommand{\norm}[1]{\left\| #1\right\|}
\newcommand{\dprod}[1]{\langle#1\rangle}
\newcommand{\linspan}[1]{\langle#1\rangle}
\newcommand{\conj}[1]{\overline{#1}}
\newcommand{\gradient}{\nabla}
\newcommand{\der}{\frac{d}{dx}}
\newcommand{\lap}{\Delta}
\newcommand{\kron}{\otimes}
\newcommand{\nperp}{\nvdash}

\newcommand{\mat}[1]{\left( \begin{smallmatrix}#1 \end{smallmatrix} \right)}

% derivatives and limits
\newcommand{\partder}[2]{\frac{\partial #1}{\partial #2}}
\newcommand{\partdern}[3]{\frac{\partial^{#3} #1}{\partial #2^{#3}}}

% Arrows
\newcommand{\diverge}{\nearrow}
\newcommand{\notto}{\nrightarrow}
\newcommand{\up}{\uparrow}
\newcommand{\down}{\downarrow}
% gets and gives are defined!

% ordering operators
\newcommand{\oleq}{\preceq}
\newcommand{\ogeq}{\succeq}

% programming and logic operators
\newcommand{\dfn}{:=}
\newcommand{\assign}{:=}
\newcommand{\co}{\ co\ }
\newcommand{\en}{\ en\ }


% logic operators
\newcommand{\xor}{\oplus}
\newcommand{\Land}{\bigwedge}
\newcommand{\Lor}{\bigvee}
\newcommand{\finish}{$\Box$}
\newcommand{\contra}{\Rightarrow \Leftarrow}
\newcommand{\iseq}{\stackrel{_?}{=}}


% Set theory
\newcommand{\symdiff}{\Delta}
\newcommand{\union}{\cup}
\newcommand{\inters}{\cap}
\newcommand{\Union}{\bigcup}
\newcommand{\Inters}{\bigcap}
\newcommand{\nullSet}{\phi}

% graph theory
\newcommand{\nbd}{\Gamma}

% Script alphabets
% For reals, use \Re

% greek letters
\newcommand{\eps}{\epsilon}
\newcommand{\del}{\delta}
\newcommand{\ga}{\alpha}
\newcommand{\gb}{\beta}
\newcommand{\gd}{\del}
\newcommand{\gf}{\phi}
\newcommand{\gF}{\Phi}
\newcommand{\gl}{\lambda}
\newcommand{\gm}{\mu}
\newcommand{\gn}{\nu}
\newcommand{\gr}{\rho}
\newcommand{\gs}{\sigma}
\newcommand{\gt}{\theta}
\newcommand{\gx}{\xi}

\newcommand{\sw}{\sigma}
\newcommand{\SW}{\Sigma}
\newcommand{\ew}{\lambda}
\newcommand{\EW}{\Lambda}

\newcommand{\Del}{\Delta}
\newcommand{\gD}{\Delta}
\newcommand{\gG}{\Gamma}
\newcommand{\gO}{\Omega}
\newcommand{\gL}{\Lambda}
\newcommand{\gS}{\Sigma}

% Formatting shortcuts
\newcommand{\red}[1]{\textcolor{red}{#1}}
\newcommand{\blue}[1]{\textcolor{blue}{#1}}
\newcommand{\htext}[2]{\texorpdfstring{#1}{#2}}

% Statistics
\newcommand{\distr}{\sim}
\newcommand{\stddev}{\sigma}
\newcommand{\covmatrix}{\Sigma}
\newcommand{\mean}{\mu}
\newcommand{\param}{\gt}
\newcommand{\ftr}{\phi}

% General utility
\newcommand{\todo}[1]{\footnote{TODO: #1}}
\newcommand{\exclaim}[1]{{\textbf{\textit{#1}}}}
\newcommand{\tbc}{[\textbf{Incomplete}]}
\newcommand{\chk}{[\textbf{Check}]}
\newcommand{\oprob}{[\textbf{OP}]:}
\newcommand{\core}[1]{\textbf{Core Idea:}}
\newcommand{\why}{[\textbf{Find proof}]}
\newcommand{\opt}[1]{\textit{#1}}


\DeclareMathOperator*{\argmin}{arg\,min}
\DeclareMathOperator{\rank}{rank}
\newcommand{\redcol}[1]{\textcolor{red}{#1}}
\newcommand{\bluecol}[1]{\textcolor{blue}{#1}}
\newcommand{\greencol}[1]{\textcolor{green}{#1}}


\renewcommand{\~}{\htext{$\sim$}{~}}


% groupings of objects.
\newcommand{\set}[1]{\left\{ #1 \right\}}
\newcommand{\seq}[1]{\left(#1\right)}
\newcommand{\ang}[1]{\langle#1\rangle}
\newcommand{\tuple}[1]{\left(#1\right)}

% numerical shortcuts.
\newcommand{\abs}[1]{\left| #1\right|}
\newcommand{\floor}[1]{\left\lfloor #1 \right\rfloor}
\newcommand{\ceil}[1]{\left\lceil #1 \right\rceil}

% linear algebra shortcuts.
\newcommand{\change}{\Delta}
\newcommand{\norm}[1]{\left\| #1\right\|}
\newcommand{\dprod}[1]{\langle#1\rangle}
\newcommand{\linspan}[1]{\langle#1\rangle}
\newcommand{\conj}[1]{\overline{#1}}
\newcommand{\gradient}{\nabla}
\newcommand{\der}{\frac{d}{dx}}
\newcommand{\lap}{\Delta}
\newcommand{\kron}{\otimes}
\newcommand{\nperp}{\nvdash}

\newcommand{\mat}[1]{\left( \begin{smallmatrix}#1 \end{smallmatrix} \right)}

% derivatives and limits
\newcommand{\partder}[2]{\frac{\partial #1}{\partial #2}}
\newcommand{\partdern}[3]{\frac{\partial^{#3} #1}{\partial #2^{#3}}}

% Arrows
\newcommand{\diverge}{\nearrow}
\newcommand{\notto}{\nrightarrow}
\newcommand{\up}{\uparrow}
\newcommand{\down}{\downarrow}
% gets and gives are defined!

% ordering operators
\newcommand{\oleq}{\preceq}
\newcommand{\ogeq}{\succeq}

% programming and logic operators
\newcommand{\dfn}{:=}
\newcommand{\assign}{:=}
\newcommand{\co}{\ co\ }
\newcommand{\en}{\ en\ }


% logic operators
\newcommand{\xor}{\oplus}
\newcommand{\Land}{\bigwedge}
\newcommand{\Lor}{\bigvee}
\newcommand{\finish}{$\Box$}
\newcommand{\contra}{\Rightarrow \Leftarrow}
\newcommand{\iseq}{\stackrel{_?}{=}}


% Set theory
\newcommand{\symdiff}{\Delta}
\newcommand{\union}{\cup}
\newcommand{\inters}{\cap}
\newcommand{\Union}{\bigcup}
\newcommand{\Inters}{\bigcap}
\newcommand{\nullSet}{\phi}

% graph theory
\newcommand{\nbd}{\Gamma}

% Script alphabets
% For reals, use \Re

% greek letters
\newcommand{\eps}{\epsilon}
\newcommand{\del}{\delta}
\newcommand{\ga}{\alpha}
\newcommand{\gb}{\beta}
\newcommand{\gd}{\del}
\newcommand{\gf}{\phi}
\newcommand{\gF}{\Phi}
\newcommand{\gl}{\lambda}
\newcommand{\gm}{\mu}
\newcommand{\gn}{\nu}
\newcommand{\gr}{\rho}
\newcommand{\gs}{\sigma}
\newcommand{\gt}{\theta}
\newcommand{\gx}{\xi}

\newcommand{\sw}{\sigma}
\newcommand{\SW}{\Sigma}
\newcommand{\ew}{\lambda}
\newcommand{\EW}{\Lambda}

\newcommand{\Del}{\Delta}
\newcommand{\gD}{\Delta}
\newcommand{\gG}{\Gamma}
\newcommand{\gO}{\Omega}
\newcommand{\gL}{\Lambda}
\newcommand{\gS}{\Sigma}

% Formatting shortcuts
\newcommand{\red}[1]{\textcolor{red}{#1}}
\newcommand{\blue}[1]{\textcolor{blue}{#1}}
\newcommand{\htext}[2]{\texorpdfstring{#1}{#2}}

% Statistics
\newcommand{\distr}{\sim}
\newcommand{\stddev}{\sigma}
\newcommand{\covmatrix}{\Sigma}
\newcommand{\mean}{\mu}
\newcommand{\param}{\gt}
\newcommand{\ftr}{\phi}

% General utility
\newcommand{\todo}[1]{\footnote{TODO: #1}}
\newcommand{\exclaim}[1]{{\textbf{\textit{#1}}}}
\newcommand{\tbc}{[\textbf{Incomplete}]}
\newcommand{\chk}{[\textbf{Check}]}
\newcommand{\oprob}{[\textbf{OP}]:}
\newcommand{\core}[1]{\textbf{Core Idea:}}
\newcommand{\why}{[\textbf{Find proof}]}
\newcommand{\opt}[1]{\textit{#1}}


\DeclareMathOperator*{\argmin}{arg\,min}
\DeclareMathOperator{\rank}{rank}
\newcommand{\redcol}[1]{\textcolor{red}{#1}}
\newcommand{\bluecol}[1]{\textcolor{blue}{#1}}
\newcommand{\greencol}[1]{\textcolor{green}{#1}}


\renewcommand{\~}{\htext{$\sim$}{~}}


%opening
% \title{Social and biological network analysis}
% \author{}
% \date{}

\begin{document}
% \maketitle
% \tableofcontents

\newpage
\section{Introduction and Problem description: Social and biological network analysis}
\begin{enumerate}
 \item Mostly a rank one perturbation of a linear transformation applied to the whitepaper.
 \item Biological problem and its great importance.
 \subitem A fundamental problem in biology is to identify the genes which affect the expression of physiological traits of organisms, or phenes. This problem is extremely important from the healthcare perspective. \exclaim{Finding cures for hard to cure diseases.}
Use Edward Marcotte work as example.
 \subitem Besides the healthcare perspective, it is interesting in its own right: Understanding the mechanism of life.
 \item \exclaim{We benefit from social network analysis perspective.} We can use ideas from recommendation systems and link prediction.
 \subitem Remark about the expertise of the PI's lab in social network analysis, paricularly in the use of multiple sources of infromation.
 \item We can further the state of art when it comes to using multiple sources of information. This will have applications beyond biological networks.
 \item \exclaim{Commercial applications} Recommendation systems.
 \item Claim that this has general application in many other fields.
\end{enumerate}


\section{Data Description}
\subsection{Terminology}

\subsection{Introduction to social and biological networks}
\begin{enumerate}
\item Background on social and biological networks. 
\item Cite examples.
\item Display the picture of yeast network.
\end{enumerate}

\subsection{Biological networks: Description and Sources}
\begin{enumerate}
\item Introduce gene-gene and gene-phene networks.
\subitem Describe an example phenotype.
\item Present a table of species involved.
\item Formally define the biological network (bipartite graph).
\item Mention the sources of the biological data set.
\end{enumerate}

\subsection{Preprocessing the data}
\begin{enumerate}
\item Explain the retention of only the orthologous genes in other species.
\item Present the data set sizes.
\end{enumerate}

\subsection{Biological networks and social networks: an analogy}
\begin{enumerate}
\item State the analogy between social and biological networks.
\item Reason why the analogy is helpful.
\end{enumerate}

\subsection{Interesting observations}
\begin{enumerate}
\item Remark some of the interesting observations in the data set.
\item Present some statistics of the data set.
\end{enumerate}

\subsection{Notation}
\begin{enumerate}
\item Table of data matrices and symbols.
\end{enumerate}

\subsection{Other potential sources}
\begin{enumerate}
 \item Introduce orthologous relationships.
 \item Mention about association between phenes.
\end{enumerate}

\section{Prior research}
\begin{enumerate}
 \item How were genes associated with phenes without the use of informatics? By Genome-Wide association studies (GWAS).
  \subitem Explain how it works. Note that this is possible due to the completion of Human Genome project in 2003.
  \subsubitem Cite references for GWA.
  \subitem Explain how it is conducted in detail. State that this is a laborious process.
  \subitem Explain its other weaknesses.
   \subsubitem GWA studies are necessarily hypothesis-free: that is they search the entire genome for associations rather than focusing on small candidate areas.
   \subsubitem The common variants identified by GWAS contribute very little of value to individual disease risk predictions over existing clinical markers for most common diseases. The technology for surveying rare variants is only just becoming feasible.
 \item Mention previous research in the analysis of large networks by others, including Prof Dhillon. (Copy parts from whitepaper.)
 \item Mention recent research by Marcotte etal in gathering data, leveraging statistical models to identify gene-phene links. Mention the potential of techniques from network analysis etc.. in this problem.
 \item Describe the link prediction problem in social networks. Address the problem of using side information.
 \item Describe some research in recommender systems, compressive sensing. Address the problem of using side information.
\end{enumerate}



\section{Proposed research}
\subsection{Problem formulation}
\paragraph*{The network link identification perspective}
\begin{enumerate}
  \item Begin by describing the gene-phene link prediction problem.
  \item Describe it as a link prediction problem on a network.
  \subitem Describe related problems in social network analysis.
  \item Describe the closely connected goal of figuring out ways to use multiple sources of information in recommendation systems, network analysis.
  \subitem Point out recent work in this area by the PI's lab.
\end{enumerate}

\subsubsection{Formulation as a ranking problem}
\begin{enumerate}
  \item Formulate the problem using adjascency matrices.
  \item Pose it as an item recommendation problem, which inturn can be posed as a ranking problem.
  \subitem Draw parallels with recommender systems, link prediction/ social network analysis.
  \subitem Describe score generators. This formulation makes the use of score generators convenient.
\end{enumerate}

\subsubsection{Other formulations}
\begin{enumerate}
 \item Probabilistic modelling of gene-phene connections.
 \item The outlier detection perspective.
 \item Explain prior social network link prediction research.
\end{enumerate}



\subsection{Proposed Approaches}
\subsubsection{Matrix completion approaches}
\begin{enumerate}
 \item Pose the problem of generating scores for the purpose of ranking genes as a matrix completion problem.
 \item Further pose the problem as one of low rank matrix approximation. Mention PI's work in this area.
 \item Mention challenges faced in incorporating additional sources of information. Mention PI's work in this area.
  \subitem Eg: Using the giant matrix.
\end{enumerate}

\subsubsection{Graph proximity approaches}
\begin{enumerate}
 \item Describe graph proximity approaches to score generation.
 \item Describe similarity measures: common neighbors, Katz measure. Mention PI's work in developing new similarity measures (Eg: Supervised Katz).
 \subitem Describe ways of extending the similarity measures to bipartite graphs.
 \item Mention challenges faced in incorporating additional sources of information. Mention PI's work in this area.
  \subitem Eg: Using the giant matrix.
\end{enumerate}

\subsubsection{Other approaches}
\begin{enumerate}
 \item Explore other possibilities, especially outlier detection and Probabilistic modelling. Give a probabilistic modelling approach to low rank approximation as an example.
 \item Propose to combine various predictions: worked well in Netflix.
\end{enumerate}

\subsubsection{Contributions to machine learning}
\begin{enumerate}
\item As we try these approaches, we will discover new insights about the power and design of machine learning algorithms to exploit multiple sources of information.
\end{enumerate}

\subsection{Identifying new data sources}
\begin{enumerate}
 \item Try to explore the use of hitherto unused sources of information for gene-phene link prediction.
\end{enumerate}
 

\subsection{Evaluation}
\begin{enumerate}
 \item Motivate the importance of good evaluation.
  \subitem Wet-lab experiments tend to be costly and time consuming.
  \subitem Also essential for using an enseble of predictors.
\end{enumerate}

\subsubsection{Analytical evaluation}
\begin{enumerate}
 \item Describe evaluation. Make predictions about gene-phene connections. How good are these?
 \item Describe analytical evaluation strategies. Describe ROC, AUC, maybe even area under the truncated ROC.
\end{enumerate}

\subsubsection{Enabling wetlab experiments}
\begin{enumerate}
 \item Enable wetlab experiments to check these predictions. Collaborate with Marcotte and Wallingford. Cite recent success attained by Marcotte.
 \item Explore the question about how limited resources can be allocated for wetlab experiments to verify gene-phene link predictions. Decision theory research.
\end{enumerate}

\subsection{Development of bioinformatics tools}
\begin{enumerate}
 \item Produce software biologists in general can use. Include visualization software example besides gene-phene link prediction.
\end{enumerate}



\section{Preliminary research, results}
\subsection{Introduction}
\begin{enumerate}
\item Recall the problem.
\item State what the preliminary experiments suggest.
\end{enumerate}

\subsection{Recent work}
\begin{enumerate}
\item Briefly state PI's recent research in social networks, link prediction in particular.
\item State how it is relevant to the problem at hand.
\end{enumerate}

\subsection{Describe preliminary experiments}
\begin{enumerate}
\item Explain path-based approaches: Use Graph Proximity Model.
\subitem Katz-based predictors: Katz on Human gene-gene interaction network, Katz on bipartite graphs.
\item Explain factor-based approaches: Use Latent Factor Model.
\subitem Factor-based predictors: SVD approximation; learning $\lambda$ and $d$ (low rank).
\end{enumerate}

\subsection{Results}
\begin{enumerate}
\item Present the summary of results: Short discussion.
\item Sample AUC plot.
\end{enumerate}

\section{Project timeline}
\begin{enumerate}
 \item Year 1: Implementations in Matlab of predictors based on graph proximity and latent factor models. Generating early gene-phene connection predictions. Theoretical work on the development of network analysis algorithms, recommender systems.
 \item Year 2:  Exploring the use of probabilistic modelling and outlier detection to solve the problem. Wet-lab experiments to check predictions. Further development of network analysis algorithms and recommender systems. Application to various other problems.
 \item Year 3: Explore various ways of combining multiple predictors to produce excellent predictions. Make software for other biologists to use in identifying new connections as new gene-phene data comes to light. Release other software tools for use by biologists. Try to explore the use of hitherto unused sources of information for gene-phene link prediction.
\end{enumerate}


\bibliographystyle{plain}
\bibliography{../report/references}

\end{document}
