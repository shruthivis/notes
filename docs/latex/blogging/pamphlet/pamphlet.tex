\subsection{Link prediction}
Use microscopic evolution model here; or observe link formation statistics (degree distribution of new linkers etc..) and try to replicate these statistics while making predictions.

credits: \verb jhe~Ngdo~Ng .

\subsubsection{Using affiliation networks}
See if affiliation nw can be used to predict links. Maybe bias assigned to affiliation nw observations can be learned from using the corresponding social-nw evolution data.

credits: \verb jhe~Ngdo~Ng , pratIka.

\subsubsection{Experiments}
Acquire data for evolving social networks, with corresponding affiliation networks. Perhaps Yin Zhang's students are trawling various social networking websites.


\subsection{Partitioning/ clustering bi-partite graphs}
Extend current graph partitioning algorithms to have soft clusters.

Try extending clustering algorithms by Brian and Trajan to co-cluster bi-partitite graphs.

credits: pratIka.

\subsection{Co-Clustering affiliation networks}
credits: pratIka.

\subsubsection{Ideas}
\begin{itemize}
 \item Use special properties of these networks to improve clustering.
 \item Try to use information from corresponding social network to better cocluster the affiliation nw.
\end{itemize}

\subsubsection{Actions}
\begin{itemize}
 \item Acquire affilation network data to experiment with: ones with community names.
 \item Consider previous work on co-clustering bipartite graphs.
 \item Consider previous work on clustering social networks in general. May give you ideas.
\end{itemize}

\subsubsection{Experiments}
\begin{itemize}
 \item Try implementing the simple dhillon alg.
 \item Try applying the Trajan algorithm.
 \item Try the algorithm where ye cluster users and clusters by iteratively folding the affiliation nw to get user nw/ community nw and then using clusters in one nw to cluster the other nw.
\end{itemize}

\subsection{Clustering nodes in social networks}
Cluster nodes in social networks. Use special properties of these networks to improve clustering. Try to use affiliation nw here.

credits: pratIka.

\section{Deferred research projects}
Construct a graph generative model which, besides having good static and temporal properties displayed by graphs evolved using other models, also shows hierarchical community structure.

\begin{itemize}
\item Can you extend the affiliation networks model to do this?
\item This can be done if the idea to use affiliation networks algorithm to generate clusters and then interconnect them works.
\end{itemize}

Given the k*k q(xhat, yhat) matrix, which specifies the edge densities between various node clusters, make a generative model which produces graphs with that clustering structure, such that it possesses various static and temporal properties of real world graphs.

\begin{itemize}
\item A trivial solution exists if the diagonal entries heavily dominate others. Is this well motivated?
\end{itemize}

Consider the problem of learning a decision tree with a 1/n correlated parity, given a noisy parity learner which only works with constant noise.
