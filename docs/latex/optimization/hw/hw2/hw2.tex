\documentclass{article}
% Configuration for the memoir class.
\renewcommand{\cleardoublepage}{}
% \renewcommand*{\partpageend}{}
\renewcommand{\afterpartskip}{}
\maxsecnumdepth{subsubsection} % number subsections
\maxtocdepth{subsubsection}

\addtolength{\parindent}{-5mm}
% Packages not included:
% For multiline comments, use caption package. But this conflicts with hyperref while making html files.
% subfigure conflicts with use with memoir style-sheet.

% Use something like:
% % Use something like:
% % Use something like:
% \input{../../macros}

% groupings of objects.
\newcommand{\set}[1]{\left\{ #1 \right\}}
\newcommand{\seq}[1]{\left(#1\right)}
\newcommand{\ang}[1]{\langle#1\rangle}
\newcommand{\tuple}[1]{\left(#1\right)}

% numerical shortcuts.
\newcommand{\abs}[1]{\left| #1\right|}
\newcommand{\floor}[1]{\left\lfloor #1 \right\rfloor}
\newcommand{\ceil}[1]{\left\lceil #1 \right\rceil}

% linear algebra shortcuts.
\newcommand{\change}{\Delta}
\newcommand{\norm}[1]{\left\| #1\right\|}
\newcommand{\dprod}[1]{\langle#1\rangle}
\newcommand{\linspan}[1]{\langle#1\rangle}
\newcommand{\conj}[1]{\overline{#1}}
\newcommand{\gradient}{\nabla}
\newcommand{\der}{\frac{d}{dx}}
\newcommand{\lap}{\Delta}
\newcommand{\kron}{\otimes}
\newcommand{\nperp}{\nvdash}

\newcommand{\mat}[1]{\left( \begin{smallmatrix}#1 \end{smallmatrix} \right)}

% derivatives and limits
\newcommand{\partder}[2]{\frac{\partial #1}{\partial #2}}
\newcommand{\partdern}[3]{\frac{\partial^{#3} #1}{\partial #2^{#3}}}

% Arrows
\newcommand{\diverge}{\nearrow}
\newcommand{\notto}{\nrightarrow}
\newcommand{\up}{\uparrow}
\newcommand{\down}{\downarrow}
% gets and gives are defined!

% ordering operators
\newcommand{\oleq}{\preceq}
\newcommand{\ogeq}{\succeq}

% programming and logic operators
\newcommand{\dfn}{:=}
\newcommand{\assign}{:=}
\newcommand{\co}{\ co\ }
\newcommand{\en}{\ en\ }


% logic operators
\newcommand{\xor}{\oplus}
\newcommand{\Land}{\bigwedge}
\newcommand{\Lor}{\bigvee}
\newcommand{\finish}{$\Box$}
\newcommand{\contra}{\Rightarrow \Leftarrow}
\newcommand{\iseq}{\stackrel{_?}{=}}


% Set theory
\newcommand{\symdiff}{\Delta}
\newcommand{\union}{\cup}
\newcommand{\inters}{\cap}
\newcommand{\Union}{\bigcup}
\newcommand{\Inters}{\bigcap}
\newcommand{\nullSet}{\phi}

% graph theory
\newcommand{\nbd}{\Gamma}

% Script alphabets
% For reals, use \Re

% greek letters
\newcommand{\eps}{\epsilon}
\newcommand{\del}{\delta}
\newcommand{\ga}{\alpha}
\newcommand{\gb}{\beta}
\newcommand{\gd}{\del}
\newcommand{\gf}{\phi}
\newcommand{\gF}{\Phi}
\newcommand{\gl}{\lambda}
\newcommand{\gm}{\mu}
\newcommand{\gn}{\nu}
\newcommand{\gr}{\rho}
\newcommand{\gs}{\sigma}
\newcommand{\gt}{\theta}
\newcommand{\gx}{\xi}

\newcommand{\sw}{\sigma}
\newcommand{\SW}{\Sigma}
\newcommand{\ew}{\lambda}
\newcommand{\EW}{\Lambda}

\newcommand{\Del}{\Delta}
\newcommand{\gD}{\Delta}
\newcommand{\gG}{\Gamma}
\newcommand{\gO}{\Omega}
\newcommand{\gL}{\Lambda}
\newcommand{\gS}{\Sigma}

% Formatting shortcuts
\newcommand{\red}[1]{\textcolor{red}{#1}}
\newcommand{\blue}[1]{\textcolor{blue}{#1}}
\newcommand{\htext}[2]{\texorpdfstring{#1}{#2}}

% Statistics
\newcommand{\distr}{\sim}
\newcommand{\stddev}{\sigma}
\newcommand{\covmatrix}{\Sigma}
\newcommand{\mean}{\mu}
\newcommand{\param}{\gt}
\newcommand{\ftr}{\phi}

% General utility
\newcommand{\todo}[1]{\footnote{TODO: #1}}
\newcommand{\exclaim}[1]{{\textbf{\textit{#1}}}}
\newcommand{\tbc}{[\textbf{Incomplete}]}
\newcommand{\chk}{[\textbf{Check}]}
\newcommand{\oprob}{[\textbf{OP}]:}
\newcommand{\core}[1]{\textbf{Core Idea:}}
\newcommand{\why}{[\textbf{Find proof}]}
\newcommand{\opt}[1]{\textit{#1}}


\DeclareMathOperator*{\argmin}{arg\,min}
\DeclareMathOperator{\rank}{rank}
\newcommand{\redcol}[1]{\textcolor{red}{#1}}
\newcommand{\bluecol}[1]{\textcolor{blue}{#1}}
\newcommand{\greencol}[1]{\textcolor{green}{#1}}


\renewcommand{\~}{\htext{$\sim$}{~}}


% groupings of objects.
\newcommand{\set}[1]{\left\{ #1 \right\}}
\newcommand{\seq}[1]{\left(#1\right)}
\newcommand{\ang}[1]{\langle#1\rangle}
\newcommand{\tuple}[1]{\left(#1\right)}

% numerical shortcuts.
\newcommand{\abs}[1]{\left| #1\right|}
\newcommand{\floor}[1]{\left\lfloor #1 \right\rfloor}
\newcommand{\ceil}[1]{\left\lceil #1 \right\rceil}

% linear algebra shortcuts.
\newcommand{\change}{\Delta}
\newcommand{\norm}[1]{\left\| #1\right\|}
\newcommand{\dprod}[1]{\langle#1\rangle}
\newcommand{\linspan}[1]{\langle#1\rangle}
\newcommand{\conj}[1]{\overline{#1}}
\newcommand{\gradient}{\nabla}
\newcommand{\der}{\frac{d}{dx}}
\newcommand{\lap}{\Delta}
\newcommand{\kron}{\otimes}
\newcommand{\nperp}{\nvdash}

\newcommand{\mat}[1]{\left( \begin{smallmatrix}#1 \end{smallmatrix} \right)}

% derivatives and limits
\newcommand{\partder}[2]{\frac{\partial #1}{\partial #2}}
\newcommand{\partdern}[3]{\frac{\partial^{#3} #1}{\partial #2^{#3}}}

% Arrows
\newcommand{\diverge}{\nearrow}
\newcommand{\notto}{\nrightarrow}
\newcommand{\up}{\uparrow}
\newcommand{\down}{\downarrow}
% gets and gives are defined!

% ordering operators
\newcommand{\oleq}{\preceq}
\newcommand{\ogeq}{\succeq}

% programming and logic operators
\newcommand{\dfn}{:=}
\newcommand{\assign}{:=}
\newcommand{\co}{\ co\ }
\newcommand{\en}{\ en\ }


% logic operators
\newcommand{\xor}{\oplus}
\newcommand{\Land}{\bigwedge}
\newcommand{\Lor}{\bigvee}
\newcommand{\finish}{$\Box$}
\newcommand{\contra}{\Rightarrow \Leftarrow}
\newcommand{\iseq}{\stackrel{_?}{=}}


% Set theory
\newcommand{\symdiff}{\Delta}
\newcommand{\union}{\cup}
\newcommand{\inters}{\cap}
\newcommand{\Union}{\bigcup}
\newcommand{\Inters}{\bigcap}
\newcommand{\nullSet}{\phi}

% graph theory
\newcommand{\nbd}{\Gamma}

% Script alphabets
% For reals, use \Re

% greek letters
\newcommand{\eps}{\epsilon}
\newcommand{\del}{\delta}
\newcommand{\ga}{\alpha}
\newcommand{\gb}{\beta}
\newcommand{\gd}{\del}
\newcommand{\gf}{\phi}
\newcommand{\gF}{\Phi}
\newcommand{\gl}{\lambda}
\newcommand{\gm}{\mu}
\newcommand{\gn}{\nu}
\newcommand{\gr}{\rho}
\newcommand{\gs}{\sigma}
\newcommand{\gt}{\theta}
\newcommand{\gx}{\xi}

\newcommand{\sw}{\sigma}
\newcommand{\SW}{\Sigma}
\newcommand{\ew}{\lambda}
\newcommand{\EW}{\Lambda}

\newcommand{\Del}{\Delta}
\newcommand{\gD}{\Delta}
\newcommand{\gG}{\Gamma}
\newcommand{\gO}{\Omega}
\newcommand{\gL}{\Lambda}
\newcommand{\gS}{\Sigma}

% Formatting shortcuts
\newcommand{\red}[1]{\textcolor{red}{#1}}
\newcommand{\blue}[1]{\textcolor{blue}{#1}}
\newcommand{\htext}[2]{\texorpdfstring{#1}{#2}}

% Statistics
\newcommand{\distr}{\sim}
\newcommand{\stddev}{\sigma}
\newcommand{\covmatrix}{\Sigma}
\newcommand{\mean}{\mu}
\newcommand{\param}{\gt}
\newcommand{\ftr}{\phi}

% General utility
\newcommand{\todo}[1]{\footnote{TODO: #1}}
\newcommand{\exclaim}[1]{{\textbf{\textit{#1}}}}
\newcommand{\tbc}{[\textbf{Incomplete}]}
\newcommand{\chk}{[\textbf{Check}]}
\newcommand{\oprob}{[\textbf{OP}]:}
\newcommand{\core}[1]{\textbf{Core Idea:}}
\newcommand{\why}{[\textbf{Find proof}]}
\newcommand{\opt}[1]{\textit{#1}}


\DeclareMathOperator*{\argmin}{arg\,min}
\DeclareMathOperator{\rank}{rank}
\newcommand{\redcol}[1]{\textcolor{red}{#1}}
\newcommand{\bluecol}[1]{\textcolor{blue}{#1}}
\newcommand{\greencol}[1]{\textcolor{green}{#1}}


\renewcommand{\~}{\htext{$\sim$}{~}}


% groupings of objects.
\newcommand{\set}[1]{\left\{ #1 \right\}}
\newcommand{\seq}[1]{\left(#1\right)}
\newcommand{\ang}[1]{\langle#1\rangle}
\newcommand{\tuple}[1]{\left(#1\right)}

% numerical shortcuts.
\newcommand{\abs}[1]{\left| #1\right|}
\newcommand{\floor}[1]{\left\lfloor #1 \right\rfloor}
\newcommand{\ceil}[1]{\left\lceil #1 \right\rceil}

% linear algebra shortcuts.
\newcommand{\change}{\Delta}
\newcommand{\norm}[1]{\left\| #1\right\|}
\newcommand{\dprod}[1]{\langle#1\rangle}
\newcommand{\linspan}[1]{\langle#1\rangle}
\newcommand{\conj}[1]{\overline{#1}}
\newcommand{\gradient}{\nabla}
\newcommand{\der}{\frac{d}{dx}}
\newcommand{\lap}{\Delta}
\newcommand{\kron}{\otimes}
\newcommand{\nperp}{\nvdash}

\newcommand{\mat}[1]{\left( \begin{smallmatrix}#1 \end{smallmatrix} \right)}

% derivatives and limits
\newcommand{\partder}[2]{\frac{\partial #1}{\partial #2}}
\newcommand{\partdern}[3]{\frac{\partial^{#3} #1}{\partial #2^{#3}}}

% Arrows
\newcommand{\diverge}{\nearrow}
\newcommand{\notto}{\nrightarrow}
\newcommand{\up}{\uparrow}
\newcommand{\down}{\downarrow}
% gets and gives are defined!

% ordering operators
\newcommand{\oleq}{\preceq}
\newcommand{\ogeq}{\succeq}

% programming and logic operators
\newcommand{\dfn}{:=}
\newcommand{\assign}{:=}
\newcommand{\co}{\ co\ }
\newcommand{\en}{\ en\ }


% logic operators
\newcommand{\xor}{\oplus}
\newcommand{\Land}{\bigwedge}
\newcommand{\Lor}{\bigvee}
\newcommand{\finish}{$\Box$}
\newcommand{\contra}{\Rightarrow \Leftarrow}
\newcommand{\iseq}{\stackrel{_?}{=}}


% Set theory
\newcommand{\symdiff}{\Delta}
\newcommand{\union}{\cup}
\newcommand{\inters}{\cap}
\newcommand{\Union}{\bigcup}
\newcommand{\Inters}{\bigcap}
\newcommand{\nullSet}{\phi}

% graph theory
\newcommand{\nbd}{\Gamma}

% Script alphabets
% For reals, use \Re

% greek letters
\newcommand{\eps}{\epsilon}
\newcommand{\del}{\delta}
\newcommand{\ga}{\alpha}
\newcommand{\gb}{\beta}
\newcommand{\gd}{\del}
\newcommand{\gf}{\phi}
\newcommand{\gF}{\Phi}
\newcommand{\gl}{\lambda}
\newcommand{\gm}{\mu}
\newcommand{\gn}{\nu}
\newcommand{\gr}{\rho}
\newcommand{\gs}{\sigma}
\newcommand{\gt}{\theta}
\newcommand{\gx}{\xi}

\newcommand{\sw}{\sigma}
\newcommand{\SW}{\Sigma}
\newcommand{\ew}{\lambda}
\newcommand{\EW}{\Lambda}

\newcommand{\Del}{\Delta}
\newcommand{\gD}{\Delta}
\newcommand{\gG}{\Gamma}
\newcommand{\gO}{\Omega}
\newcommand{\gL}{\Lambda}
\newcommand{\gS}{\Sigma}

% Formatting shortcuts
\newcommand{\red}[1]{\textcolor{red}{#1}}
\newcommand{\blue}[1]{\textcolor{blue}{#1}}
\newcommand{\htext}[2]{\texorpdfstring{#1}{#2}}

% Statistics
\newcommand{\distr}{\sim}
\newcommand{\stddev}{\sigma}
\newcommand{\covmatrix}{\Sigma}
\newcommand{\mean}{\mu}
\newcommand{\param}{\gt}
\newcommand{\ftr}{\phi}

% General utility
\newcommand{\todo}[1]{\footnote{TODO: #1}}
\newcommand{\exclaim}[1]{{\textbf{\textit{#1}}}}
\newcommand{\tbc}{[\textbf{Incomplete}]}
\newcommand{\chk}{[\textbf{Check}]}
\newcommand{\oprob}{[\textbf{OP}]:}
\newcommand{\core}[1]{\textbf{Core Idea:}}
\newcommand{\why}{[\textbf{Find proof}]}
\newcommand{\opt}[1]{\textit{#1}}


\DeclareMathOperator*{\argmin}{arg\,min}
\DeclareMathOperator{\rank}{rank}
\newcommand{\redcol}[1]{\textcolor{red}{#1}}
\newcommand{\bluecol}[1]{\textcolor{blue}{#1}}
\newcommand{\greencol}[1]{\textcolor{green}{#1}}


\renewcommand{\~}{\htext{$\sim$}{~}}

% Use something like:
% % Use something like:
% % Use something like:
% \input{../../amsartMacros}

\newtheorem{thm}{Theorem}[subsection]
\newtheorem{cor}[thm]{Corollary}
\newtheorem{lem}[thm]{Lemma}
\newtheorem{fact}[thm]{Fact}
\newtheorem{claim}[thm]{Claim}

\newtheorem{assumption}[thm]{Assumption}

\newtheorem{defn}[thm]{Definition}

\theoremstyle{remark}
\newtheorem{alg}[thm]{Algorithm}
\newtheorem*{notation}{Notation}
\newtheorem*{rem}{Remark}
\newtheorem*{hint}{Hint}
\newtheorem*{ack}{Acknowledgement}

\newtheorem{question}[thm]{Question}
\newtheorem{answer}[thm]{Answer}


\newtheorem{thm}{Theorem}[subsection]
\newtheorem{cor}[thm]{Corollary}
\newtheorem{lem}[thm]{Lemma}
\newtheorem{fact}[thm]{Fact}
\newtheorem{claim}[thm]{Claim}

\newtheorem{assumption}[thm]{Assumption}

\newtheorem{defn}[thm]{Definition}

\theoremstyle{remark}
\newtheorem{alg}[thm]{Algorithm}
\newtheorem*{notation}{Notation}
\newtheorem*{rem}{Remark}
\newtheorem*{hint}{Hint}
\newtheorem*{ack}{Acknowledgement}

\newtheorem{question}[thm]{Question}
\newtheorem{answer}[thm]{Answer}


\newtheorem{thm}{Theorem}[subsection]
\newtheorem{cor}[thm]{Corollary}
\newtheorem{lem}[thm]{Lemma}
\newtheorem{fact}[thm]{Fact}
\newtheorem{claim}[thm]{Claim}

\newtheorem{assumption}[thm]{Assumption}

\newtheorem{defn}[thm]{Definition}

\theoremstyle{remark}
\newtheorem{alg}[thm]{Algorithm}
\newtheorem*{notation}{Notation}
\newtheorem*{rem}{Remark}
\newtheorem*{hint}{Hint}
\newtheorem*{ack}{Acknowledgement}

\newtheorem{question}[thm]{Question}
\newtheorem{answer}[thm]{Answer}

\lstset{language=matlab}


%opening
\title{Non Linear Programming: Homework 2}
\author{vishvAs vAsuki}

\begin{document}

\maketitle

\section{2.12 Identification of convex sets}
\subsection{Hint}
For problem 2.12, give a brief justification for each of your answers. With the possible exception of part (g), you shouldn't need more than a sentence or two.

\subsection{a}
A slab, $\set{x: \ga \leq a^{T}x \leq \gb}$, is convex. This is because it is the intersection of two half-spaces, which are convex sets.

\subsection{b}
A rectangle, $\set{x:\ga_i \leq x_i \leq \gb_i}$, is convex. It is the intersection of convex sets. This is because it is the intersection of halfspaces formed by vertical and horizontal hyperplanes, which are convex sets.

\subsection{c}
A wedge, which is in the interesection of two halfspaces, is convex as it is the intersection of two convex sets.

\subsection{d}
Consider the set S of points which are closer to a given point p than to a given set Y. It is the intersection of halfspaces, or in degenerate cases, even the entire space. So, it is convex.

\subsection{e}
Take sets S and T. Consider the set X of points closer to S than to T. Note that $S \subseteq X$. This is not convex. Consider S and T to be 2 concentric rings in $R^{2}$.

\begin{ack}
Got this example from discussion with nAgarAjan naTarAjan. 
\end{ack}

\subsection{f}
Take the set $X = \set{x : x + S_2 \subseteq S_1}$, with $S_1$ convex. This set is convex. If $S_2$ is larger than $S_1$, $X = \emptyset$. Else, we can use the definition of convexity to show that X is convex.

\subsection{g}
Take the set $X = \set{x: \norm{x - a} \leq \gt \norm{x - b}}$, with $a\neq b$ and $\gt \in [0, 1]$. Consider the boundary cases: at $\gt = 1$, X is a halfspace, and at $\gt = 0$, it is a line. For intermediate values, it seems to be a hyperbola-like. By geometric intuition, I conjecture that this is a convex set.

\section{2.17 (parts b and c only). Image of polyhedral sets under perspectve function}
\subsection{Hint}
Be careful on problem 2.17! Each has three different possibilities depending upon specific values of f, g, or h.

$P(C) = \set{v/t:(v, t) \in C, t>0}$.

\subsection{b Hyperplane}
$C = \set{(v, t): f^{T}v + gt  = h}; f, g \neq 0$.

$P(C) = \set{v: tf^{T}v = h - gt, \forall t>0}$. If $g>0$: $P(C) = \set{v: tf^{T}v \leq h}$. This is a halfspace below the hyperplane $f^{T}v = h$. If $g<0$: $P(C) = \set{v: tf^{T}v \geq h}$. This is a halfspace above the hyperplane $f^{T}v = h$.

\subsection{c Halfspace}
$C = \set{(v, t): f^{T}v + gt \leq h}; f, g \neq 0$.

$P(C) = \set{v: tf^{T}v \leq h - gt, \forall t>0}$. Depending on whether g is +ve or -ve, P(C) becomes the same halfspaces described in part (b).

\section{2.19: Preimages of convex sets under linear fractional functions.}
\subsection{The problem setup}
$f(x) = \frac{(Ax + b)}{c^{T}x + d}$, $dom(f) = \set{x: c^{T}x +d > 0}$.

\subsection{a Halfspace}
$C = \set{y: g^{T}y \leq h}, g\neq 0$.

\begin{eqnarray*}
g^{T}f(x) &\leq& h\\
g^{T}\frac{(Ax + b)}{c^{T}x + d} &\leq& h\\
(g^{T}Ax + g^{T}b) &\leq& h(c^{T}x + d)\\
g^{T}A x&\leq& h(c^{T}x + d) - g^{T}b\\
\end{eqnarray*}

So, $f^{-1}(C) = \set{x: g^{T}A x \leq h(c^{T}x + d) - g^{T}b}$, a halfspace.

\subsection{b Polyhedron}
$C = \set{y: Gy \leq h}$. Look at C as an intersection of halfspaces \\
$C_i = \set{y: g_{i,:}^{T}y \leq h_i}, g_{i, :}\neq 0$. Using the previous result, we have:
\begin{equation}
f^{-1}(C_i) =  \set{x: g_{i,:}^{T}A x \leq h_i(c^{T}x + d) - g^{T}b}\\
f^{-1}(C) = \inters_i f^{-1}(C_i).
\end{equation}
The inverse image is also a polyhedron.

\subsection{c Ellipsoid}
$C = \set{y: y^{T}P^{-1}y \leq 1}, P \in S_{++}^{n}$.
\begin{eqnarray*}
f(x)^{T}P^{-1}f(x) &\leq& h\\
(Ax + b)^{T}P^{-1}(Ax + b) &\leq& h(c^{T}x + d)^{2}\\
\end{eqnarray*}
So, $f^{-1}(C) = \set{x: (Ax + b)^{T}P^{-1}(Ax + b) \leq h(c^{T}x + d)^{2}}$, some ellipsoid in $R^{m}$. Note that this ellipsoid is displaced from 0.

\subsection{d Solution set of linear matrix inequality}
$C = \set{y: \sum y_i A_i \leq B}; A_i, B \in S^{p}$.

Let $1_i$ be an $n \times p^{2}$ matrix with ith row being 1's and all other elements being 0. Let $h(A_i)$ be the vector derived by stacking up the elements of $A_i$ in any some way. Let $A = [h(A_1)  .. h(A_n)]$. So, $C = \set{y: Ay \leq h(B)}$.

This is a polyhedron. As described in answer to a previous question, $f^{-1}(C)$ is also a polyhedron.

\section{2.20 or 2.22: +ve solutions of linear equations}
\begin{lem}
$\forall x: Ax = b \implies c^{T}x = d$ iff $\exists l: c = A^{T}l, d = b^{T}l$.
\end{lem}
\begin{proof}
If $\exists l: c = A^{T}l, d = b^{T}l$, we have that, $\forall x: Ax = b$, $ c^{T}x = l^{T}Ax = l^{T}b = d$.

The proving the implication in the reverse direction is more interesting. Suppose that $\forall x: Ax = b \implies c^{T}x = d$. Take any $x_n$ in the null space of A; ie: $Ax_n = 0$. If Ax = b, we have that $A(x + x_n) = b$. So, we also have that $c^{T}(x + x_n) = d$. So, c is orthogonal to the null space of A. But, the component of $R^{m}$ orthogonal to this is just the row space of A, which is $R(A^{T})$. So, c can be expressed as a linear combination of A's rows: $\exists l: c = A^{T}l$. Thence, we also derive $l^{T}b = l^{T}Ax = c^{T}x = d$.
\end{proof}

\begin{notation}
When we use inequalities with vectors, we mean to use elementwise inequalities.
\end{notation}


\begin{thm}
pa: There exists x satisfying $x > 0, Ax = b$.

pb: $\nexists l: A^{T}l \geq 0, A^{T}l \neq 0, b^{T}l \leq 0$.

Then $pa \equiv pb$.
\end{thm}
\begin{proof}
Consider $\lnot pa$. Suppose that $\exists l: A^{T}l = c \geq 0, b^{T}l = d \leq 0$. Then, using the lemma proved earlier, we see that this is equivalent to saying $\forall x: Ax = b \implies c^{T}x = d \leq 0 \implies \lnot( x > 0)$. So, $\lnot pa \equiv \lnot pb$.
\end{proof}



\section{2.23 Strict separability}
The following are disjoint closed non-strictly separable convex sets.

$A = \set{x :x\in R^{2}, x_2 \leq 1}$.

$B = \set{x :x\in R^{2}, x_2 \geq 1/x_1}$.

\begin{ack}
Heard this example from nAgarAjan naTarAjan. 
\end{ack}

\section{2.24 Supporting hyperplanes}
\subsection{a}
$S = \set{x \in R_+^2: x_1 x_2 \geq 1}$. Boundary of S, $bnd(S) = \set{x \in R_+^2: x_1 x_2 = 1}$. Take halfspaces defined by supporting hyperplanes at $x \in bnd(S)$: $H(x) = \set{y \in R_+^2: \frac{(y_2 - x_2)}{y_1 - x_1} \geq -x_1^{-2}}$. $Then, S = \inters_{x \in bnd(S)} H(x)$.

\subsection{b}
$C = \set{x \in R^{n}: \norm{x}_\infty \leq 1}$. Boundary $bnd(C) = \set{x : \max(x) = 1}$. Take $x \in bnd(C)$. Then, $\forall i: x_i = 1$, define $H_i = \set{y \in R^{n}: y_i = 1}$. These are the supporting hyperplanes of C at x.

% \bibliographystyle{plain}
% \bibliography{../linAlg}

\end{document}
