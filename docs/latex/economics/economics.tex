\documentclass[oneside, article]{memoir}
\usepackage{amsmath, amssymb}
\usepackage{hyperref, graphicx, verbatim, listings, multirow, subfigure}
\usepackage{algorithm, algorithmic}
% \usepackage[bottom]{footmisc}
\lstset{breaklines=true}
\setcounter{tocdepth}{3}

% Lets verbatim and verb environments automatically break lines.
\makeatletter
\def\@xobeysp{ }
\makeatother
% \lstset{breaklines=true,basicstyle=\ttfamily}

% Configuration for the memoir class.
\renewcommand{\cleardoublepage}{}
% \renewcommand*{\partpageend}{}
\renewcommand{\afterpartskip}{}
\maxsecnumdepth{subsubsection} % number subsections
\maxtocdepth{subsubsection}

\addtolength{\parindent}{-5mm}
% Packages not included:
% For multiline comments, use caption package. But this conflicts with hyperref while making html files.
% subfigure conflicts with use with memoir style-sheet.

% Use something like:
% % Use something like:
% % Use something like:
% \input{../../macros}

% groupings of objects.
\newcommand{\set}[1]{\left\{ #1 \right\}}
\newcommand{\seq}[1]{\left(#1\right)}
\newcommand{\ang}[1]{\langle#1\rangle}
\newcommand{\tuple}[1]{\left(#1\right)}

% numerical shortcuts.
\newcommand{\abs}[1]{\left| #1\right|}
\newcommand{\floor}[1]{\left\lfloor #1 \right\rfloor}
\newcommand{\ceil}[1]{\left\lceil #1 \right\rceil}

% linear algebra shortcuts.
\newcommand{\change}{\Delta}
\newcommand{\norm}[1]{\left\| #1\right\|}
\newcommand{\dprod}[1]{\langle#1\rangle}
\newcommand{\linspan}[1]{\langle#1\rangle}
\newcommand{\conj}[1]{\overline{#1}}
\newcommand{\gradient}{\nabla}
\newcommand{\der}{\frac{d}{dx}}
\newcommand{\lap}{\Delta}
\newcommand{\kron}{\otimes}
\newcommand{\nperp}{\nvdash}

\newcommand{\mat}[1]{\left( \begin{smallmatrix}#1 \end{smallmatrix} \right)}

% derivatives and limits
\newcommand{\partder}[2]{\frac{\partial #1}{\partial #2}}
\newcommand{\partdern}[3]{\frac{\partial^{#3} #1}{\partial #2^{#3}}}

% Arrows
\newcommand{\diverge}{\nearrow}
\newcommand{\notto}{\nrightarrow}
\newcommand{\up}{\uparrow}
\newcommand{\down}{\downarrow}
% gets and gives are defined!

% ordering operators
\newcommand{\oleq}{\preceq}
\newcommand{\ogeq}{\succeq}

% programming and logic operators
\newcommand{\dfn}{:=}
\newcommand{\assign}{:=}
\newcommand{\co}{\ co\ }
\newcommand{\en}{\ en\ }


% logic operators
\newcommand{\xor}{\oplus}
\newcommand{\Land}{\bigwedge}
\newcommand{\Lor}{\bigvee}
\newcommand{\finish}{$\Box$}
\newcommand{\contra}{\Rightarrow \Leftarrow}
\newcommand{\iseq}{\stackrel{_?}{=}}


% Set theory
\newcommand{\symdiff}{\Delta}
\newcommand{\union}{\cup}
\newcommand{\inters}{\cap}
\newcommand{\Union}{\bigcup}
\newcommand{\Inters}{\bigcap}
\newcommand{\nullSet}{\phi}

% graph theory
\newcommand{\nbd}{\Gamma}

% Script alphabets
% For reals, use \Re

% greek letters
\newcommand{\eps}{\epsilon}
\newcommand{\del}{\delta}
\newcommand{\ga}{\alpha}
\newcommand{\gb}{\beta}
\newcommand{\gd}{\del}
\newcommand{\gf}{\phi}
\newcommand{\gF}{\Phi}
\newcommand{\gl}{\lambda}
\newcommand{\gm}{\mu}
\newcommand{\gn}{\nu}
\newcommand{\gr}{\rho}
\newcommand{\gs}{\sigma}
\newcommand{\gt}{\theta}
\newcommand{\gx}{\xi}

\newcommand{\sw}{\sigma}
\newcommand{\SW}{\Sigma}
\newcommand{\ew}{\lambda}
\newcommand{\EW}{\Lambda}

\newcommand{\Del}{\Delta}
\newcommand{\gD}{\Delta}
\newcommand{\gG}{\Gamma}
\newcommand{\gO}{\Omega}
\newcommand{\gL}{\Lambda}
\newcommand{\gS}{\Sigma}

% Formatting shortcuts
\newcommand{\red}[1]{\textcolor{red}{#1}}
\newcommand{\blue}[1]{\textcolor{blue}{#1}}
\newcommand{\htext}[2]{\texorpdfstring{#1}{#2}}

% Statistics
\newcommand{\distr}{\sim}
\newcommand{\stddev}{\sigma}
\newcommand{\covmatrix}{\Sigma}
\newcommand{\mean}{\mu}
\newcommand{\param}{\gt}
\newcommand{\ftr}{\phi}

% General utility
\newcommand{\todo}[1]{\footnote{TODO: #1}}
\newcommand{\exclaim}[1]{{\textbf{\textit{#1}}}}
\newcommand{\tbc}{[\textbf{Incomplete}]}
\newcommand{\chk}{[\textbf{Check}]}
\newcommand{\oprob}{[\textbf{OP}]:}
\newcommand{\core}[1]{\textbf{Core Idea:}}
\newcommand{\why}{[\textbf{Find proof}]}
\newcommand{\opt}[1]{\textit{#1}}


\DeclareMathOperator*{\argmin}{arg\,min}
\DeclareMathOperator{\rank}{rank}
\newcommand{\redcol}[1]{\textcolor{red}{#1}}
\newcommand{\bluecol}[1]{\textcolor{blue}{#1}}
\newcommand{\greencol}[1]{\textcolor{green}{#1}}


\renewcommand{\~}{\htext{$\sim$}{~}}


% groupings of objects.
\newcommand{\set}[1]{\left\{ #1 \right\}}
\newcommand{\seq}[1]{\left(#1\right)}
\newcommand{\ang}[1]{\langle#1\rangle}
\newcommand{\tuple}[1]{\left(#1\right)}

% numerical shortcuts.
\newcommand{\abs}[1]{\left| #1\right|}
\newcommand{\floor}[1]{\left\lfloor #1 \right\rfloor}
\newcommand{\ceil}[1]{\left\lceil #1 \right\rceil}

% linear algebra shortcuts.
\newcommand{\change}{\Delta}
\newcommand{\norm}[1]{\left\| #1\right\|}
\newcommand{\dprod}[1]{\langle#1\rangle}
\newcommand{\linspan}[1]{\langle#1\rangle}
\newcommand{\conj}[1]{\overline{#1}}
\newcommand{\gradient}{\nabla}
\newcommand{\der}{\frac{d}{dx}}
\newcommand{\lap}{\Delta}
\newcommand{\kron}{\otimes}
\newcommand{\nperp}{\nvdash}

\newcommand{\mat}[1]{\left( \begin{smallmatrix}#1 \end{smallmatrix} \right)}

% derivatives and limits
\newcommand{\partder}[2]{\frac{\partial #1}{\partial #2}}
\newcommand{\partdern}[3]{\frac{\partial^{#3} #1}{\partial #2^{#3}}}

% Arrows
\newcommand{\diverge}{\nearrow}
\newcommand{\notto}{\nrightarrow}
\newcommand{\up}{\uparrow}
\newcommand{\down}{\downarrow}
% gets and gives are defined!

% ordering operators
\newcommand{\oleq}{\preceq}
\newcommand{\ogeq}{\succeq}

% programming and logic operators
\newcommand{\dfn}{:=}
\newcommand{\assign}{:=}
\newcommand{\co}{\ co\ }
\newcommand{\en}{\ en\ }


% logic operators
\newcommand{\xor}{\oplus}
\newcommand{\Land}{\bigwedge}
\newcommand{\Lor}{\bigvee}
\newcommand{\finish}{$\Box$}
\newcommand{\contra}{\Rightarrow \Leftarrow}
\newcommand{\iseq}{\stackrel{_?}{=}}


% Set theory
\newcommand{\symdiff}{\Delta}
\newcommand{\union}{\cup}
\newcommand{\inters}{\cap}
\newcommand{\Union}{\bigcup}
\newcommand{\Inters}{\bigcap}
\newcommand{\nullSet}{\phi}

% graph theory
\newcommand{\nbd}{\Gamma}

% Script alphabets
% For reals, use \Re

% greek letters
\newcommand{\eps}{\epsilon}
\newcommand{\del}{\delta}
\newcommand{\ga}{\alpha}
\newcommand{\gb}{\beta}
\newcommand{\gd}{\del}
\newcommand{\gf}{\phi}
\newcommand{\gF}{\Phi}
\newcommand{\gl}{\lambda}
\newcommand{\gm}{\mu}
\newcommand{\gn}{\nu}
\newcommand{\gr}{\rho}
\newcommand{\gs}{\sigma}
\newcommand{\gt}{\theta}
\newcommand{\gx}{\xi}

\newcommand{\sw}{\sigma}
\newcommand{\SW}{\Sigma}
\newcommand{\ew}{\lambda}
\newcommand{\EW}{\Lambda}

\newcommand{\Del}{\Delta}
\newcommand{\gD}{\Delta}
\newcommand{\gG}{\Gamma}
\newcommand{\gO}{\Omega}
\newcommand{\gL}{\Lambda}
\newcommand{\gS}{\Sigma}

% Formatting shortcuts
\newcommand{\red}[1]{\textcolor{red}{#1}}
\newcommand{\blue}[1]{\textcolor{blue}{#1}}
\newcommand{\htext}[2]{\texorpdfstring{#1}{#2}}

% Statistics
\newcommand{\distr}{\sim}
\newcommand{\stddev}{\sigma}
\newcommand{\covmatrix}{\Sigma}
\newcommand{\mean}{\mu}
\newcommand{\param}{\gt}
\newcommand{\ftr}{\phi}

% General utility
\newcommand{\todo}[1]{\footnote{TODO: #1}}
\newcommand{\exclaim}[1]{{\textbf{\textit{#1}}}}
\newcommand{\tbc}{[\textbf{Incomplete}]}
\newcommand{\chk}{[\textbf{Check}]}
\newcommand{\oprob}{[\textbf{OP}]:}
\newcommand{\core}[1]{\textbf{Core Idea:}}
\newcommand{\why}{[\textbf{Find proof}]}
\newcommand{\opt}[1]{\textit{#1}}


\DeclareMathOperator*{\argmin}{arg\,min}
\DeclareMathOperator{\rank}{rank}
\newcommand{\redcol}[1]{\textcolor{red}{#1}}
\newcommand{\bluecol}[1]{\textcolor{blue}{#1}}
\newcommand{\greencol}[1]{\textcolor{green}{#1}}


\renewcommand{\~}{\htext{$\sim$}{~}}


% groupings of objects.
\newcommand{\set}[1]{\left\{ #1 \right\}}
\newcommand{\seq}[1]{\left(#1\right)}
\newcommand{\ang}[1]{\langle#1\rangle}
\newcommand{\tuple}[1]{\left(#1\right)}

% numerical shortcuts.
\newcommand{\abs}[1]{\left| #1\right|}
\newcommand{\floor}[1]{\left\lfloor #1 \right\rfloor}
\newcommand{\ceil}[1]{\left\lceil #1 \right\rceil}

% linear algebra shortcuts.
\newcommand{\change}{\Delta}
\newcommand{\norm}[1]{\left\| #1\right\|}
\newcommand{\dprod}[1]{\langle#1\rangle}
\newcommand{\linspan}[1]{\langle#1\rangle}
\newcommand{\conj}[1]{\overline{#1}}
\newcommand{\gradient}{\nabla}
\newcommand{\der}{\frac{d}{dx}}
\newcommand{\lap}{\Delta}
\newcommand{\kron}{\otimes}
\newcommand{\nperp}{\nvdash}

\newcommand{\mat}[1]{\left( \begin{smallmatrix}#1 \end{smallmatrix} \right)}

% derivatives and limits
\newcommand{\partder}[2]{\frac{\partial #1}{\partial #2}}
\newcommand{\partdern}[3]{\frac{\partial^{#3} #1}{\partial #2^{#3}}}

% Arrows
\newcommand{\diverge}{\nearrow}
\newcommand{\notto}{\nrightarrow}
\newcommand{\up}{\uparrow}
\newcommand{\down}{\downarrow}
% gets and gives are defined!

% ordering operators
\newcommand{\oleq}{\preceq}
\newcommand{\ogeq}{\succeq}

% programming and logic operators
\newcommand{\dfn}{:=}
\newcommand{\assign}{:=}
\newcommand{\co}{\ co\ }
\newcommand{\en}{\ en\ }


% logic operators
\newcommand{\xor}{\oplus}
\newcommand{\Land}{\bigwedge}
\newcommand{\Lor}{\bigvee}
\newcommand{\finish}{$\Box$}
\newcommand{\contra}{\Rightarrow \Leftarrow}
\newcommand{\iseq}{\stackrel{_?}{=}}


% Set theory
\newcommand{\symdiff}{\Delta}
\newcommand{\union}{\cup}
\newcommand{\inters}{\cap}
\newcommand{\Union}{\bigcup}
\newcommand{\Inters}{\bigcap}
\newcommand{\nullSet}{\phi}

% graph theory
\newcommand{\nbd}{\Gamma}

% Script alphabets
% For reals, use \Re

% greek letters
\newcommand{\eps}{\epsilon}
\newcommand{\del}{\delta}
\newcommand{\ga}{\alpha}
\newcommand{\gb}{\beta}
\newcommand{\gd}{\del}
\newcommand{\gf}{\phi}
\newcommand{\gF}{\Phi}
\newcommand{\gl}{\lambda}
\newcommand{\gm}{\mu}
\newcommand{\gn}{\nu}
\newcommand{\gr}{\rho}
\newcommand{\gs}{\sigma}
\newcommand{\gt}{\theta}
\newcommand{\gx}{\xi}

\newcommand{\sw}{\sigma}
\newcommand{\SW}{\Sigma}
\newcommand{\ew}{\lambda}
\newcommand{\EW}{\Lambda}

\newcommand{\Del}{\Delta}
\newcommand{\gD}{\Delta}
\newcommand{\gG}{\Gamma}
\newcommand{\gO}{\Omega}
\newcommand{\gL}{\Lambda}
\newcommand{\gS}{\Sigma}

% Formatting shortcuts
\newcommand{\red}[1]{\textcolor{red}{#1}}
\newcommand{\blue}[1]{\textcolor{blue}{#1}}
\newcommand{\htext}[2]{\texorpdfstring{#1}{#2}}

% Statistics
\newcommand{\distr}{\sim}
\newcommand{\stddev}{\sigma}
\newcommand{\covmatrix}{\Sigma}
\newcommand{\mean}{\mu}
\newcommand{\param}{\gt}
\newcommand{\ftr}{\phi}

% General utility
\newcommand{\todo}[1]{\footnote{TODO: #1}}
\newcommand{\exclaim}[1]{{\textbf{\textit{#1}}}}
\newcommand{\tbc}{[\textbf{Incomplete}]}
\newcommand{\chk}{[\textbf{Check}]}
\newcommand{\oprob}{[\textbf{OP}]:}
\newcommand{\core}[1]{\textbf{Core Idea:}}
\newcommand{\why}{[\textbf{Find proof}]}
\newcommand{\opt}[1]{\textit{#1}}


\DeclareMathOperator*{\argmin}{arg\,min}
\DeclareMathOperator{\rank}{rank}
\newcommand{\redcol}[1]{\textcolor{red}{#1}}
\newcommand{\bluecol}[1]{\textcolor{blue}{#1}}
\newcommand{\greencol}[1]{\textcolor{green}{#1}}


\renewcommand{\~}{\htext{$\sim$}{~}}



%opening
\title{Economics: Quick reference}
\author{vishvAs vAsuki}

\begin{document}
\maketitle

\chapter{Themes}
For game theory, see game theory ref.

\part{Macroeconomics}
\chapter{Goods and services}
\section{Public goods}
\subsection{Traits}
\subsubsection{Non-exclusion}
Once the good has been provided - possibly with the support of certain people, there is no good way of excluding others (including unconnected 'free-riders') from using it.

\subsubsection{Non-rivalry}
Use of the good by a person does not make it less enjoyable or worthwhile for other users. 

\subsection{Examples}
Fresh air, pleasant weather. More controversial examples: roads.

\subsection{Funding}
Provision of public goods is often done by the government.

\subsubsection{Private sponsorship}
A limited group of consumers (aka the K-group) could sponsor public goods - and tolerate free-riders.

\subsection{Media}
Information in the form of songs, visual performances, books which were once private, is increasingly being digitally copied and distributed using computers - mainly by enthusiasts acting against the wishes of publishers, who hold the copyright, and often with the approval of the original artists.

\subsubsection{Videos}
Websites like youtube allow sharing videos. Copyrighted work may be sought using services like projectfreetv etc..

\subsubsection{Books}
bookz or ebooks channels on the undernet (good as of 2004-2011).

\section{Private}
Aka market goods. Eg: food.

\subsection{Access vs ownership}
For low-idling items, like a microwave: it is better to own the item. High idling items, like a power drill, car, jewels are better rented.

\subsection{Price and rationing}
\subsubsection{Rationing}
Rationing is a mechanism for distributing goods to people.

\subsubsection{Price rationing}
Price balances demand and supply in the market. The pricing of goods with money is described in the money section.

\paragraph{Industrial vs investment demand}
Prices are driven often by current demand and supply: aka industrial demand. Prices may also be driven by speculation about future demand and supply relative to other goods/ currency: aka investment demand. Investment demand can be very fickle, and speculation may be mistaken.

\paragraph{Elasticity of demand}
The efficacy of increasing price in decreasing demand varies. In case of elastic demand/ prices, demand smoothly decreases with increasing price.

In case of inelastic demand, consumers are willing to pay higher price without decreasing demand. This is especially true in case essential commodities - like petroleum or food; or in case of addictive commodities - like psychotropic drugs.

\subsubsection{Non-price rationing}
Artificially restricts demand to keep price below the equilibrium. Ensures that rich folk don't get everything in the time of shortages. Ration coupons, books, queues, taxes.

\subsection{Swaps/ barter}
Internet has facilitated barter: Eg: swaptree for books.

\chapter{Money}
\section{Money and value}
The value of goods and services, for various people, keeps changing in accordance with demand and supply for it. Money is a common way of expressing this value numerically.

\subsection{Value of money, goods}
If more value has been created in the economy, if no new money is created/ the amount of money in the system remains the same, then the value of money increases, and there is deflation. On the other hand, if the total value of the economy decreases relative to the amount of money in the system, there is inflation/ devaluation of money.

\subsection{Currency}
Paper currency and coins made from cheap metals is in use as currency.

\section{Creation/ destruction}
The federal bank/ reserve (Fed) of a country is responsible for creating and destroying money (devaluation). Thus, it has power to determine the value of money. It creates money by buying/ selling objects (usually government securities) from entities (usually banks) and paying for it using new money or by destroying the money it receives.

\subsection{Interest rates}
Banks, which suddenly end up with more or less money, react by lowering or raising the interest rates on loans and on savings accounts.

\subsection{Economic behavior altered}
The amount of money in the economy does not change the total actual value of the economy. But, it changes how economic entities behave.

Injecting money into the economy is a common way for the Fed to revitalize or spur growth in the economy. This comes from the fact that there is more money in banks to borrow from. 

\section{Counterfeiting}
Creating fake money others will not honor, counterfeiters cheat those they transact with.

\section{Currency History}
Different civilizations used different forms of money. Yap people used huge stones. Other civilizations used gold, copper and other coins. Chinese started issuing paper money.

\subsection{Gold standard}
Western economies started issuing paper money after the industrial revolution, but they initially were backed by gold (aka gold standard). One could take paper money to a government agency and exchange it for gold.

This was dangerous at a time when gold was no longer common currency, and gold hoarding resulted in reduced investment in the economy. But this limited the government's ability to create new money corresponding to spur greater investment and growth. This caused the severity of the great depression in 1930's.

\chapter{Economic growth}
\section{Economic value indicators}
Certain numbers reveal the current evaluation of the economy. The health of the economy can be roughly viewed in terms of the financial health and confidence of 4 major components: consumers, banks, non-bank businesses, government.

Eg: Prices of critical commodities (like oil), value of index funds, growth in real estate prices, government debt, public debt.

\subsection{Gross domestic product}
\subsubsection{As a summary}
This is a measure of the total gross income of the country.

\subsubsection{Comparison with GPA}
Like the GPA in case of academia, it is a good summary, but it does not include finer details which could be gleaned from the whole data, such as how rich an average person feels.

\subsubsection{Avoiding overcounting}
To avoid double-counting and to include all services and goods which go into making a certain product, only end-consumption is considered. To avoid over-counting, money spent on imported goods is not counted - that contributes to the exporting country's GDP.

Goods produced during previous years is not included in the GDP - even if they continue to be consumed. Instead, an imputed rent is included in the calculation.

\subsubsection{Environmental costs}
The costs to the environment from economic activity leading to growth is not included in the GDP. These costs arise from decreased life expectancy, greater sickness, lesser tourist pleasure etc.. Cost to life expectancy can be calculated based on the higher salary yielded by increased risk to life (around 60\$ more for .0001 greater risk in 2011).

On the other hand efforts at spending to repair environmental damage gets counted - eg: environmental damage from BP's oil spill is not included, but money spent on clean-up effort is!

\section{Booms and busts}
Economies generally keep cycling through periods of booms and busts (described below). This is due to heuristic way in which demand and supply are balanced.

\subsection{Boom}
To meet demand, industry in certain sectors grow - often on borrowed money, counting on future profits. Industries indeed produce profit for some time, but production keeps increasing until production exceeds demand.

During this time, the industries' economic value - or at least their evaluation by the securities market- grows. Their operating cost increases because they invest more and hire more workers. Banks are more willing to lend at low interest rates.

Consumers, being able to find employment and loans, feel richer and more confident in their spending.

Note that a boom is different from a bubble - which is a valuation error, rather than a overproduction problem.

\subsection{Bust}
Aka recession.

\subsubsection{Market causes and reactions}
When production exceeds demand, some industries go bankrupt, defaulting on their loans. Prices of securities fall, because the market has by now learned that the demand for product(s) is lower than the supply.

\subsubsection{Reaction of banks}
If loans are too big, lender banks fail, or start hoarding capital in order to compensate for losses by becoming reluctant to make new loans.

\paragraph{Role of government}
The government's budget deficit has an impact on the interest rate on government debt, which inturn affects the rate of interest charged by banks in their lending.

\subsubsection{Business response}
Due to impeded entrepreneurship, GDP growth reduces drastically. Due to bankrupt industries, the economy may even shrink! Businesses which remain solvent respond by cutting costs (including laying of people), selling their goods and services for cheaper to attract consumers. Due to the cost cutting, companies may in fact make record profits!

\subsubsection{Consumer behavior}
Consumers feel poor due to falling prices and fears of unemployment, and don't have confidence to spend.

\subsubsection{Impact on prices}
As consumers have less money to spend, prices may fall.

Because businesses do not expand their services, projected demand for oil actually falls. This leads to decreased price of oil and oil-related enterprises.

\subsubsection{Recovery}
To recover from a bust, consumer confidence must be restored, government should cut deficit, banks' willingness to lend should increase, and businesses should feel confident to hire more people.

\subsubsection{International Contagion}
When a certain country is experiencing a recession, it can adversely affect others due to many reasons.

Countries whose businesses (especially banks) had lent to recession-hit businesses and governments experience greater risk. This triggers reduction in lending by banks.

Countries with otherwise healthy businesses which export goods to recession hit countries also face reduced profitability due to reduced demand and due to decrease in value of the recession-hit country's currency.

\section{Investment}
GDP growth usually requires investment in business enterprise. But, investments may have returns either in the short-term or in the long-term. Investment which responds to consumer demand yields immediate benefits, and is often favored.

Investments vary in quality: for example, a rich government investment firm can increase GDP by just building cities which will never be occupied in the desert.

\subsection{Consumer demand and public debt}
Individuals borrow money, and they have to repay them. The purchasing power of people correspondingly decreases.

\subsection{High public debt}
When the money they collectively borrow is close to the (projected) GDP (national money production), there is trouble - it means that people are collectively incapable of repaying their debts.

This has happened in USA in 1929 and in 2009; thus Americans have financed their high individual standard of living with borrowed money they cannot repay. This eventually may lead to most individuals in USA being less worthy of credit and investment - leading to economic depression.

\subsection{Global savings}
Increased prosperity in many parts of the world has resulted in greater amount of money becoming available for investment. IMF tracks this number - it exceeds the world's GDP 5-fold, at 83 trillion \$ in 2009.

Investment managers seek to utilize this money by investing in something which yields good return and low risk, with emphasis being on one or the other depending on their recent experience (and partially based on their long-term education). Their willingness to take reasonable risk is important to spur continued GDP growth.

\section{Disadvantages}
\subsection{Social costs}
A society often pays an excessive price for economic growth in the form of environmental pollution, eroding social/ cultural values, compassion.

\subsection{Inflation}
The injection of new money by governments to favor economic growth leads to inflation/ currency devaluation, which disproportionately affects the poor, for the following reason.

The poor spend a greater fraction of their income on food/ raw materials, whereas rich tend to spend a greater fraction of their wealth on services. During inflation, the price of commodities grows at a much greater rate than the price of services.

\subsubsection{Growth imbalance}
Thus, with growth, the monetary imbalance between the rich and the poor tends to grow.

\section{Effect of local disasters}
Suppose that a natural disaster like a hurricane in the gulf coast of USA or a earthquake + tsunami + nuclear leak in a part of Japan (2011) occurs. Regardless of where such a disaster occurs, much wealth is destroyed, and the economy of the affected region is severely set back.

If the country where the disaster occurs is advanced - like Japan or USA, effect on the GDP may not even be noticeable both in the short and the long term. This is because, these countries have redundant economic units (roads/ industries) which compensate for the destruction; and they have savings or lines of credit which they can use to rebuild - such investment increases, rather than decrease the GDP.

But, if the country is poor (eg: Haiti earthquake 2010), and the affected region is economically important, redundancy is low, and the entire economy suffers as a result.

\section{Financial crises}
Growth in economic value slows down drastically sometimes. The immediate cause for this disruption is a drop in investment/ money-lending due to varied reasons.

\subsection{1929: great depression}
\subsubsection{Importance}
A transformative event in world (especially US) history: disagreement about the goodness of new deal set lasted throughout the following century as rivalry between Democrats and Republicans in politics.

\subsubsection{Boom + bubbles}
In the 1920's, there was over-investment and under-consumption - there was a boom and a bubble. The bubble burst.

\subsubsection{Bust, credit crisis}
There was over-indebtedness. Some banks crashed. Depositors withdrew money from other banks.  So, remaining banks became reluctant to lend money. On Oct 29 1929, Stock market crashed.

\subsubsection{Deflationary cycle}
In response to rising unemployment and falling income, businesses - in order to sell goods - had to reduce prices. The reduced prices led to further decrease in wages; which in turn led to reduced prices and reduced profits, which in-turn fed the credit crisis. Thus vicious cycles were formed

Commodity prices, especially food (and therefore farmland) fell; and farmers got much poorer. Disparity in wealth between the rich and the poor had grown.

\subsubsection{Hoarding savings}
Some banks refused to accept new deposits, because they were unable to find satisfactory investments. So, people, for psychological reason, wanted to convert their currency to gold instead - which was possible because the currency was exchangeable for gold on surrender to the government.

\subsubsection{Wrong fixes}
To discourage hoarding of savings in the form of gold and to encourage investments, the inexperienced Fed tightened credit - this raised interest rates both for savings accounts and for loans - even though the opposite was required to encourage banks to lend money. Thus, the status quo continued.

International trade broke down: tariffs were increased in order to protect the profitability of local industry.

\subsubsection{Ditching gold standard}
UK, USA ditched the gold-standard - to discourage gold hoarding and perhaps because of the inadequacy of the gold reserve. This allowed currency to be printed to encourage banks to lend. Also, as a way of injecting money into the economy, the government started buying back gold at higher prices with newly printed money.

This was critical. Countries which ditched the gold standard later got out of the depression later.

\subsubsection{New deal 1}
FD Roosevelt took over from Hoover, promised vigorous experimentation to find some way out. He kept citizens informed through a radio show, stopped prohibition of liquor and was popular.

Farmers were paid not to grow crops - to raise food prices. Working hours and conditions favorable to workers was encouraged: growth which was favorable for the rich was slowed down. The huge unemployed population was employed by the government to build infrastructure. Banks, with government aid, were set to operate soundly; and small deposits were guaranteed by the government .

\subsection{2008: sub-prime mortgages}
\subsubsection{Bad credit}
At the most basic level, a lot of bad loans were made to consumers: not just in buying homes, but also in general.

\paragraph{Eager investors}
In 2000, US federal bank indicated that government bonds will continue to offer low interest rates - at 1\%; so investors managing global savings - which had recently grown due to rapid economic growth - eager to get profit despite risk sought investment avenues.

\paragraph{Bad mortgage-backed securities}
At this time, people in the US financial industry invented mortgage-backed securities, by which investors could buy a share of ownership in loans. There was so much supply of investor money and such short supply of worthy investments, that people started making bad loans to people incapable of paying back in order to sell shares in these loans and make profit in the short term. Borrowers lied about their incomes - often encouraged to do so by lenders. Housing prices were also increasing, emboldening lenders into thinking that foreclosure would be a way to remain profitable even if borrowers defaulted.

\subsubsection{Credit default swaps}
Investors buying risky mortgage-backed securities got insurance/ bet that these securities will fail; and this insurance business was neither public nor regulated. So, insurers such as AIG failed.

\subsubsection{Consequences}
In 2007, people started defaulting, real estate prices fell. Investors lost much money.

\paragraph{Bank collapse}
Banks (especially investment banks who created and held on to mortgage-backed securities) which invested in bad mortgages were insolvent (or solvent only in paper because they refused to liquidate their toxic assets at current market value).

\paragraph{Credit crisis}
Money market mutual funds, which gave out short term loans to businesses (commercial paper market) partially suffered because of loans to those who invested in mortgage-backed securities, and such short term lending froze due to a panic. Investors who managed the global savings pool became very risk-averse.

So, this led to a big slow-down in investment as both types of investors were not willing to lend money -either to regain solvency in case of banks or due to risk aversion. This credit-crisis hit countries like Iceland, which were now unable to low-interest loans.

\paragraph{Effect on industry}
Shrinking markets caused industries such as GM to fail.

\paragraph{Effect on consumers}
Consumers lost jobs, owed huge debt, felt poorer and reduced spending.

\paragraph{Profit}
Some mutual funds (like Magnetar) profited from this crisis because they understood, encouraged and bet against risky securities using credit default swaps.

Employees of investment banks who created the heavily traded risky \\mortgage-backed securities and ignored their fiduciary duty towards the investor got rich from bonuses earned from fees received by the bank.

\subsubsection{Mitigation}
Central banks created a lot of new money in a huge effort to encourage investment/ lending and thus growth; and to finance efforts to rescue failing businesses.

Banks and industries were rescued - Eg: by US government subsidizing people who bought toxic assets from banks at slightly higher prices. 

\subsubsection{Devaluation}
Commodity prices (esp: oil price) increased.

\chapter{Government}
\section{Services}
Government agencies or companies hired by them are responsible for providing basic infrastructure - public health-care, roads, water, security, regulation, environment protection, foreign policy, arbitration - to the public.

\section{Design}
\subsection{Branches}
Government has 3 main arms: Legislative (to make laws and oversee policies), Executive (to provide services as designed at a high level by the legislature) and Judiciary (to take care of grievances, punish wrong-doing).

\subsection{Importance}
Government design determines the type and the quality of the services provided by the government. Bad governments are often a result of bad design.

\subsection{Legislature}
Legislatures have great power because they determine the laws/ policies; the executive and judiciary merely execute them.

Legislatures vary in the extant to which they fear the people rather than the people fearing them. The more the legislature fears the people, the better quality of life people often achieve.

\subsubsection{Lobbying}
Special interest groups (serving the interest of foreign nations, or industry segments, for example) try to influence, often very successfully, the laws and policies designed by the legislature in order to be favorable to them.

They may do these by propaganda and dialogue to honestly convince that the ideas they support are good - for not just the legislator. But, they also often resort to appealing to the self-interest rather than the legislator's duty (except perhaps using a broader justification for the importance of staying in power) by supplying money for campaigns or by bribery.

\section{Formation}
\subsection{Monarchy}
In the past monarchy, based on armed or religous support, was common. Succession was based on very narrow nepotism - this often led to fragile states whose fortunes varied wildly over the generations: See nepotism section of business succession for details. 

\subsection{Democracy}
In democracies, legislatures are constituted by election. \tbc

\section{Income}
\subsection{Taxation}
In exchange for government services, people and businesses pay taxes on their incomes/ gains (any increase in wealth). So, the government collects taxes on behalf of its citizenry.

\subsubsection{Taxation rate and service level}
Some highly 'socialist' countries (Eg: Scandinavian countries, France etc..) tend to have high taxation rates, and provide excellent services to their citizens in return. Other countries, like USA, have low taxation rates and in return provide low-quality public services.

\subsubsection{Capital gains}
The quirks of capital gains tax is explained elsewhere.

\subsection{Tax rate discrimination}
\subsubsection{Based on residence}
Income Tax rates are often different for legal residents and temporary visitors. IN USA, one is declared a US resident for tax purposes if he passes a 'substantial presence test' (not counting days under exempt status- eg: foreign student/ scholar).

\subsubsection{Progressive taxation}
Entities earning more money (sufficiently large to fall in different bins/ 'tax brackets') are usually taxed at a higher rate than others. This is essentially a means of income distribution - this is often done to increase the average well-being of citizens.

\subsection{Sales tax rate discrimination}
Some goods attract higher sales taxes than others. This is different from the tax paid by manufacturers on their income.

This is motivated by the fact that these goods (eg: electricity produced from fossil fuels, tobacco, alcohol) come at a cost to the environment or society or health which would not be otherwise factored into the item price.

\subsection{Tax code ideals}
\subsubsection{Simplicity and loopholes}
Economists accross the board agree that the tax code should be simple. The government wishing to earn money well and be fair should eliminate major loopholes: including concessions to the middle class such as mortgage tax exemption (which is discrimination against renters), health insurance contributions by employers etc..

\subsubsection{Focus on consumption and pollution}
Still others argue that corporate - perhaps even individual - income tax should be eliminated in order to spur the economic growth; instead consumption and pollution should be taxed.

\subsection{Natural resources}
Natural resources - such as petroleum or minerals - form another source of income to the government.

\subsubsection{Renter economies}
For many countries- eg: Saudi Arabia and Libya - they form the major part (98\% in Lybia) of the economy. So, the government in such countries - especially if it involves undemocratic self-perpetuation - need not provide good services to its people/ industries as long as they maintain the natural resource exploitation industry in good health. They just provide enough to keep people calm - not necessarily economically productive.

Resource-rich countries which break this curse tend to find a way to ensure that no single entity has control of resources.

\subsection{Import tariff}
Aka customs duty. This is tax levied on the price of imported items.

\subsubsection{Design factors}
Complicated tariff rules form as a result of lobbying by industry groups within the country (aka protectionism by the government), while the desire to increase exports lead to international agreements requiring lowering tariffs.

It can also be a international politics instrument.

\subsubsection{Enforcement}
Manufacturers could be lying about the make of their products in order to incur lower tariffs. Or, smugglers could be importing goods without paying tariffs.

Inspectors and scientists working at ports test products and ensure that tariffs are applied.

\section{Budget deficits}
Many cities, states and countries outside have greater expenditure than revenue, a gap which is often projected to widen as time passes.

\subsection{Quantification}
Government deficit is often measured as a percentage of the GDP, which is much larger than the actual income of the government alone. Anything over 10\% is considered too high.

\subsection{Expectations without willingness to spend}
Often people expect unrealistically good public services from the government, without the willingness to spend a proportionate fraction of their income. So, politicians in many countries are elected based on promise to increase or maintain public services while reducing (rather than increasing) taxation - this results in the perpetuation of a budget crisis. 

This is especially the case in countries where the affected taxpayers have significant say in choosing their rulers.

\subsection{Long term solution}
The long term solution involves reducing expenditures (downsizing government services), increased taxation, later retirement age etc..

\subsubsection{Increase burden on the rich}
In the case of democracies and socialist countries, one solution is to place a higher tax burden on the rich. But, people in countries characterized by a phobia of socialism or by general optimism where people are rich or expect to become rich, this is rejected.

\subsection{Short term solution}
Temporary measures to combat it include borrowing money by issuing bonds, or, in the case of countries - creating money as explained elsewhere.

\subsubsection{Sovereign debt}
Treasury bonds are issued by national governments to borrow money, they return the principal with a very small interest.

\subsection{Defaults}
Sovereign debt is generally considered to be very secure - as the government can repay debt under normal circumstances by printing new money.

But in some cases, the country is so indebted that printing money will lead to currency devaluation/ inflation with very bad socio-economic consequences. So, sometimes countries default on their commitment to pay back debtors.

\subsubsection{Default Consequences}
When a government defaults on its debts, it can only acquire loans under much harsher, tougher conditions.

\subsubsection{Rescue}
Troubled governments, not wanting to default - or wanting to default gracefully, are often rescued by friendly countries/ organizations like IMF and EU lending money, under conditions which discourage future budget deficits - such as raising interest rates on future government bonds, and other long-term solutions mentioned earlier

\section{Corruption, inefficiency}
Public servants and those who otherwise depend on public funding fall into unethical practices like accepting bribes, lying, delay and coercion because their seniors compel them to do so - either explicitly or unwittingly as a result of the unrealistic or wrong expectations, or because of a desire for personal profit, or for job security.

\subsection{Bad expectations}
For example, during the Chinese famine, officials presented Mao with fake fields and yields during his inspections. NYPD cops, whose anti-crime activity was being tracked using statistics, registered fewer cases, and were coerced into unnecessary pat-downs, tickets and arrests.

\section{Law enforcement}
Considered in the society survey.

\section{Health care}
Quality of health care varies widely amongst nations, and within nations.

\subsection{Universal health care}
Universal Health Care in many countries is free or low cost. 

\subsubsection{Quality}
The quality in many countries is good - In France there are even house-calls. In other countries beset with inefficient executive branches, like India, public health care is often low quality - though even there, there are islands of high quality.

\subsubsection{Delays, overuse}
Free health care is often overused, leading to resource scarcity given that a limited amount of money is alloted for health care. This can lead to long delays in access to health care procedures or specialists (dentists, MRI scans) - in 2010, this was a problem for UK's NHS and Canada's health service.

\subsection{Insurance based care}
\subsubsection{Systemic Inefficiency}
\paragraph{Useless treatments}
Insurance is a lousy way to pay, as it separates the consumer from the cost. Also, it results in care providers being paid per-procedure, rather than a salary. This encourages wasteful, useless treatments. 1/3 of the treatment provided in USA is wasteful.

The wrong belief among patients that more care implies lower sickness, cognitive defects in doctors who don't rely on research and the fear of law-suits also lead to these useless treatments - but these are not specific to the insurance-system.

\paragraph{Administrative overhead}
Claims processing tends to be a complicated procedure, with codes for diseases etc.., this results in a big portion of medical expenses being spent on paper-pushing.

Also, there is the overhead of negotiating with drug companies and health-care providers.

\paragraph{Partial interest in reducing cost}
If health-care gets too cheap, less money flows through the insurance company as premiums, and there may be fewer customers. Also, insurance companies want to make a profit - so, people necessarily spend more than in the case of government health services.

\paragraph{Uninsured people}
These are turned away - In USA, they are often dropped to homeless person shelters disoriented due to inability to pay or lack of insurance.

However, countries like USA offers Medicare insurance to the needy; but this is not automatic.

Low cost clinics like Rediclinic and minute-clinic are cropping up in USA to serve patients with common non-chronic illnesses.

\subsubsection{USA}
\paragraph{Comparison with other countries}
This system is followed in USA, with unsatisfactory results. USA spends more money per capita on health-care than most countries in the world.

Yet, Infant mortality and life expectancy are worse than many other nations, including Cuba, Canada, France and UK. A documentary called Sicko has more criticism.

\paragraph{Employer choice}
Employers are encouraged to provide health-insurance for employees, because they get a tax-break. So, patients are further removed from the payment service.

\paragraph{Forecast}
This system is held in place by conservative politicians and by lobbying by drug and insurance companies; but even they are now (2011) realizing that unless they change slightly to keep everyone reasonably happy, a health-care revolution will replace them - health care costs are raising at alarming levels, and this is having a huge effect on slowing down the economy and reducing the quality of life of people.

Other options for health care, which are customer driven rather than employer driven, are cropping up.

\section{Education}
\subsection{Schools}
Most countries provide free universal compulsory primary education for children. Quality varies.

\subsection{Tuition}
Access to prestigious universities/ schools may be limited, so there is often a competitive exam; for which there often exists special tuition/ cram schools.

\subsection{Higher education}
In some countries (like Scandinavia, France), government heavily subsidizes higher education; while in others students must mostly rely on other sources, and accrue huge debts.

\section{Business attraction}
\subsection{Luring away businesses}
More businesses mean more tax money, and prosperity of the nation. So, territories (cities, states or countries) try to lure businesses from each other - they usually have special agencies called 'Economic development agencies' which do this.

They sometimes offer tax-breaks in the initial years to attract businesses. They also try to achieve concentration of talent by offering pre-setup infrastructure and by setting up special economic zones where companies will get tax benefits.

\subsection{Entrepreneurship}
\subsubsection{Effect of taxes}
Entrepreneurship rates are high in the socialist countries of Scandinavia -despite high taxes. This could be due to the fact that the government, with its excellent services, removes the financial risk associated with failure.


\chapter{Business regulation}
Businesses, left to themselves, often operate to increase their own profit at the expense of the environment and the general economy.

Regulation about working conditions are described in another chapter.

\section{Competition promotion}
Governments then try to ensure that no company gets a monopoly on any market, as monopoly reduces progress and keeps the prices up.

\subsection{Illegal activities}
\subsubsection{Disadvantaging competition}
Microsoft, in the 1990's, bundling the internet explorer browser with its operating system, disadvantaged rival browsers and gained monopoly.

\subsubsection{Cartel formation}
Treating customers as the enemy, and competitors as partners, companies collude and resolve not to compete with each other by amicably dividing up the market and fixing the price. This drains customer money, and decreases GDP growth.

Eg: In 1995, a handful of American, Korean and Japanese manufacturers of lysene colluded to fix the price; were exposed and were fined.

\subsubsection{Interoperability prevention}
Microsoft was fined by EU for not providing adequate documentation for the various protocols used by its products.

\subsubsection{Mergers/ acquisition for monopoly}
\tbc

\subsection{Regulation}
Government regulation designed to preserve and promote competition is called anti-trust/ competition law. Violations are met with severe fines in USA.

Sometimes, the government breaks up huge monopolies by breaking up the business into competing units.

\section{Foreign political regulation}
\subsection{Boycotts}
\subsubsection{Economic embargos}
\tbc

\subsubsection{Travel restrictions}
\tbc

\section{Credit rating of individuals}
Good credit rating implies lower assessment of risk from lenders, renters and employers. Thence implies lower interest rates on loans.

In USA, one can check yearly free reports from Experian, TransUnion and Equifax at equally spaced times during the year.

\subsection{Growth}
Have around 2 credit cards, maintain a high level of available capacity in each card. This contributes to increase in credit score.


\chapter{Negotiable instruments trading}
Here we consider basic (possibly mutli-step) actions involving various negotiable instruments in the market.

\section{Negotiable instruments}
Aka Securities. These are objects which promise payment to whoever holds the object. They include: stocks/ equities (which represent ownership in various businesses/ funds), bonds (loans), derivatives (whose value is determined based on the value of other instruments).

\subsection{Creating securities}
Investment bankers help raise capital for various businesses by creating bonds and stocks. They get big fees for it.

\subsubsection{Underwriting}
They often take the responsibility of selling such securities by purchasing it from the business and holding on to them - usually until they can sell it.

\subsubsection{Fiduciary duty}
When securities are created, investment bankers have a duty towards the investor who may buy it to inform them fully and clearly about the risks involved in the security.

Yet, they collect fees from the businesses which want to create the security; a large fraction of which is distributed as bonus to the individuals working on these deals. So, investment bankers may often neglect this duty.

\subsection{Bonds}
These are documents in which borrowers acknowledge that they owe money to the holder of the document (lender), with the terms of repayment. The lender usually gets some return on investment, which may vary with the company's performance. Hence, bonds are often traded: other people can buy the loan.

Often, bonds yield periodic interests, the size of which corresponds to the risk in the bond.

Bonds may be short-term or long-term.

\subsubsection{Mortgage backed securities}
Individuals get mortgages from certain businesses - loans with their homes as collateral. These lenders then bunch together such loans and sell ownership of these loans to others who then may split it up and sell shares of these loans on the stock market.

\subsubsection{Insurance/ betting against}
Aka Credit default swap, hedging. When there is a large lending, the lender (or someone else!) may acquire insurance to cover credit-defaults. The insurers may inturn provide the guarantee by relying on an insurance they themselves bought, and so on. So, there can possibly be a chain of insurers, and the risk can be pushed to the bottom of the chain. If these bankruptcy insurances are not public knowledge, it is possible that the insurer may be taking a risk it cannot handle in case of a huge financial depression.

In the sub-prime mortgage crisis of 2006, companies like Magnetor profited by encouraging the creation of and demand for more risky financial instruments, and betting that they will fail/ go down in value (shorting)! Here, hedging became baiting!


\section{Value, valuers}
\subsection{Beliefs and market positions}
Agents, in trading securities, bet on change in market's evaluation of securities. Thus, the market price of securities is adjusted continually.

If agents believe that a security will increase in value, they are said to act from a bullish/ long position. If the reverse is true, they are said to act from a bearish/ short position.

Agents may also believe a certain company will outperform or underperform other companies in the sector.

\subsection{Jitters}
It is natural that the market tries to converge on the value of a security only gradually, as they process information about it through trades and bets. Thus, there are short term jitters in the stock price curve, which are different from the long-term booms and busts. Jitters result either from temporary change in sentiment of the traders while assimilating new information, or it can be the result of a large trade increasing the supply of certain securities relative to the demand.

Long-term booms and busts also a natural consequence of the heuristic nature of matching demand and supply.

\subsubsection{Human instincts and biases}
The market, being composed of humans, benefits and suffers from strengths and weaknesses of human instincts. Behavioral economics models can be used to explain the quirks of market evaluation.

\paragraph{Risk aversion}
Humans are subject to risk aversion: they are more averse to loosing than they yearn for profits. This explains why security prices are fast to drop, but slow to rise: investors, eager to stop loss are more eager to sell, and they are more cautious in buying in the future.

\paragraph{Recency bias}
Recent news about a security tends to affect evaluation more severely than it perhaps should. Ideally, long-term information about the asset should be a very important part of the evaluation.

\subsection{Comparative measures}
\subsubsection{P/E ratio}
P/E or price/ earning ratio measures the size of a security's price relative to the associated dividend. This is often used to identify high dividend yield stocks, which constitute attractive investments. One can think of the units as being years it takes to earn enough dividend to equal the price.

Comparison can even be between the states of a derivative at different points in time: Considering the P/E curve and the stock price together can be useful.

\subsubsection{Price / book value (P/B) ratio}
Here one considers the total asset value (aka book value) of a company and sees if the price is reasonable compared to it.

\subsubsection{Risks}
Securities which constitute riskier investments have lower value compared to less risky securities yielding similar returns. Judging investment risks is described elsewhere.

\section{Trading infrastructure}
\subsection{Securities Exchanges}
Stock markets provide a regulated/ facilitated environment for carrying out securities transactions/ trades. It operates in fixed trading hours - beyond those hours, brokers may trade on their amongst themselves in extended trading hours.

\subsection{Brokerage}
Every trade is accomplished by agents/ brokers authorized to trade in the stock market - usually on behalf of their customers. A fraction of each trade is usually collected by the stock exchange as fees. A fraction is also collected from the buyers by the brokerage agency.

\subsubsection{Regulation}
They are regulated by rules framed by the Securities and Exchanges Commission (SEC) in USA.

Front-running of clients by their brokers is illegal. Also, while assigning option exercises sent to them by the exchange, they should use some equitable mechanism.

\subsubsection{Using leverage}
Aka Margin trading. One can act on one's belief about a security using money borrowed from an agent - in the hopes that one can sell it, repay the loan and make a profit on top of it.

\subsubsection{Quality}
Brokerages vary in the fees they charge for trades, inactivity, money transfers.

They vary in the account opening process: minimum money required, ease, acceptance of international/ foreign traders.

They also vary in their ability to reinvest dividend without extra fees, interest paid for funds in 'sweep accounts' holding uninvested cash.

They vary in the types of orders one can place, their quality of execution, the UI, speed of money transfers, whether investments are insured by a government agency (SPIC in USA).

zecco seems to have issues in order execution, severe ACH transfer delays, horrendous customer service. just2trade seems cheap and reliable, but it has very restrictive policies on options trading: eg: it would not allow selling cash covered puts. OptionsHouse seemed more flexible, with a good customer service, if slightly costlier.

\subsection{Insider trading}
Trading based on non-public information - like board meetings - which can especially profit some people more than others, is discouraged and made illegal, mainly based on disguised jealousy.


\section{Buying and selling}
Basic actions in a stock market are to buy or sell a security. The securities exchange has programs/ machines which effectively communicate the price of various stocks based on what prices people are asking/ willing to pay for them - by recording these bids.

\subsection{Market price order}
A market order is an order to buy or sell a stock at the current market price.

\subsubsection{Market price trigger}
Aka Stop order. A stop order is an order to buy or sell a stock once the price of the stock reaches a specified price (the stop price).  When the specified price is reached, your order becomes a market order.

With such orders, one can trigger buying or selling; this is a common technique used by agents for preventing excessive loss.

The disadvantage is that prices could fall rapidly between the trigger and the trade, so that securities are sold at very low prices - this happened in the 2010 Flash crash.

\subsubsection{Market opening/ closing trigger}
These orders become a market order at the opening/ closing time of the market.

\subsection{Price Limit order}
One may want to buy or sell at or beyond a preset price - but this is only possible when there exist sellers or buyers willing to transact at that price. So, these actions are conducted at an appropriate time by computer programs acting on behalf of the investors.

Price limit order by an agent ensures that the transaction occurs only if it is not too unfavorable to the agent.

\subsubsection{Market price trigger}
Aka Stop-limit order. Once the stop price is reached, the stop-limit order becomes a limit order.

Stop-limit orders improve on stop orders in that: not only do they act on a similar trigger, but they are processed only if the price limit is met for the transaction.

\subsection{Order size and price}
Small 'sell' orders trade for the highest available bid in the market, whereas large 'sell' orders may exhaust high-price bids and may actually sell at a lower average price.

\section{Multi-step Trading strategies}
\subsection{Duration}
The duration required for the entire trade varies. For example, while holding a long position, one can hold on to a stock for a short time - as in day-trading; or one can turn it into a long term investment.

\subsection{Buy, sell}
One can buy a security at a certain time, in the hopes that it will increase significantly in monetary value - at which point one can sell the security to make a profit.

\subsection{Borrow, sell, buy, return}
Aka Shorting.

Or one can borrow a security at a certain time, sell it immediately, and buy it in the future at a hopefully lower price to return the borrowed security, - aka shorting. So, one is betting that a certain security will loose value. Or, one can do this by buying credit-default swaps.

\subsection{Sector ETF + company trading.}
Aka Pairs trading. Here agents act on the belief that a certain company will outperform or underperform relative to other companies in the sector.

They take advantage of the outperformance belief by holding a short position of a certain amount on the sector ETF and holding a long position of the same amount in the company in question. Similarly, they take advantage of belief in underperformance by holding a short position on the company and a long position on the sector ETF.

\section{Future trading contract}
\subsection{Anticipated trade}
One agent makes a deal with another to buy a certain amount of an 'underlying' security on a specified future date at a certain 'strike' price. To satisfy this deal, the seller is obliged to buy the security - if he does not already hold it - on or before the due-date.

\subsubsection{Motivation}
The seller is betting that the market price of the security on or before the due-date will be lower than the set price; and the buyer is betting that the market price will exceed the set-price on or after the due-date. So, both stand to make a profit according to their beliefs.

\subsubsection{Effect on evaluation}
If one ignores market jitters, the beliefs of the buyer and seller are mildly contradictory, in that believes that there will be a fall in the price of the security, whereas they buyer believes that there will be a rise in the price of the security.

These publicly stated beliefs have an effect on the evaluation of securities - speculation about the future can cause currency or commodity prices to rise even during the present time.

\section{Buy/sell options}
\subsection{Definition}
The buyer of an option gains the right (not obligation) to buy or sell (depending on the option type) an underlying security (100 shares in case of a standard option) at a specified strike price per share by a certain expiration day in case of American-style options (which we consider here), or exactly on a certain day in case of European-style options.

The seller of such an option is called a writer.

\subsubsection{Premium}
The price paid for buying an option is called a premium: this insurance term being used because one is essentially guaranteeing purchase or sale at a certain price despite risk of loss.

Its size depends on the terms of the contract (due date, underlying security, strike price). An option is worth nothing if it has expired.

The expected market price of the underlying security remaining the same, premium decreases with time as uncertainty decreases with time.

\subsubsection{Expiration day}
Option chains expire weekly, monthly or quarterly. Most commonly, in USA, they expire monthly on the third Friday/ Thursday of every month.

A limited number of weekly options are being tried out.

\subsection{Notation}
The running example below is XOP call option expiring on 20 Aug 2011, with strike price 48.

Most common notation: 48 Aug call (symbol XOP is mentioned elsewhere).

Sometimes a 5 letter alphabetical symbol is used, with letters indicating the stock, the expiry month, the strike price.

Another notation is XOP   110820C00048000.



\subsection{Options exchange}
All options purchases, sales and exercises are done through the centralized options exchange: Options clearing corporation (OCC) in USA.

\subsection{Exercising options}
The holder of an option can exercise his right to buy/ sell the underlying security by informing his broker of this intent before a certain deadline agreed between them.

When exercising an option, one usually needs to pay a brokerage for selling/ buying stock.

\subsubsection{Effective time}
The option-holder, upon exercising the option, is deemed to have purchased or sold the underlying security at the end of the same trading day: so theoretically he has the right to use the money/ security he gets immediately, and he has the right to any dividend which may be given the next day onwards.

\subsubsection{Settlement type}
At exercise, the option contract specifies the manner in which the contract is to be settled: physical (Eg: many equity options) or cash to buy (Eg: index tracking security options).

\subsubsection{Clearing, assignment}
This intent is communicated to the exchange by the broker/ clearing agent, which then randomly picks a broker who has sold such an option and decides that this entity must fulfill the obligation. The obligated broker will be informed by next day and the securities/ money is deducted from his account.

The broker in turn internally passes the obligation on an equitably chosen client on whose behalf the option contract was written. This process is called assignment.



\subsubsection{Frequency}
Around 12\% of all option contracts are exercised: people often prefer to avoid the hassle of  exercising options by selling away their options near the expiry date.

People are also motivated to exercise options just before dividend is expected to be distributed.

\subsection{Motivation}
The motivation for either party to enter the contract is their differing opinions about the price of the underlying security in the stretch of time upto the expiration time.

The seller of a call option believes that with significant probability, the market price during this time will not exceed the (strike-price - premium).

The seller of a put option believes that the price during this time is unlikely to fall below the strike-price - premium. 

\subsection{Basic trades}
Options which grant the right to buy are called call options; those that grant the right to sell are called put options. Selling or buying an option is, as usual, denotes taking a short or long position on the option.

\subsubsection{Risk and return}
On any given day, the market value ($m$) of the underlying security at the end of trading on that day, an option (bought with premium $p$) is said to be 'in the money' or 'out of the money' depending on whether exercising it at strike price $s$ leads to a profit or a loss for the holder of the option ($|s-m|-p$). Correspondingly, the option writer looses or gains the same amount.

Since the option holder may choose not to exercise the option in case doing so would lead to a loss, the maximum loss he undergoes is $p$, while the returns can potentially be $\infty$ in case of call options or $|s|$ in case of put options. The option writer, correspondingly has limited opportunity for profit $|p|$ (which happens when the option is not exercised), and a correspondingly big potential loss.

\subsubsection{Covered options}
Since the put option writer only stands to loose (is obliged to pay) at most $|s|$, even risk averse brokerages usually allow cash covered puts. This is usually a good option in case one intends to invest a certain security at a low enough price.

Similarly, brokerages easily allow folks to write call options when they already own the underlying security in sufficient quantities.

\subsection{Vertical Spreads}
\subsubsection{Definition}
Here one buys and sells a put or call option over the underlying security, and at the same expiration date at different strike prices ($s_b$ and $s_s$), with premiums $p_b, p_s$.

\subsubsection{Exercising strategy}
Since an option is both bought and sold, the person holding the spread is obligated in case the sold option is exercised. The idea is to be able to meet this obligation whenever necessary by exercising the bought option, hopefully while making a profit.

This being the case, good triggers for exercising the bought option would be a] the expiration day, when the bought option is 'in the money', and/ or b] the sold option has been exercised.

\subsubsection{Plots: inference}
Consider a coordinate system with horizontal axis being time, and vertical axis being price, with 3 curves: two horizontal lines corresponding to $y = s_b$ and $y = s_s$, and the market value $m + p_s - p_b$. From this, one can reason quickly and correctly about a] the value of benefiting from the option on a certain expiration day (a vertical line in the plot, leading to the name 'vertical' spreads).

It is also easy to visualize risk and return with a plot of profit against market price.

\subsubsection{Risk and return}
One is reducing the investment (and the risked loss) while limiting potential returns.

If both options are in the money, returns are $|s_s - s_b| - (p_b - p_s)$. If just the purchased option is in the money, returns are $|m - s_b| - (p_b - p_s)$. If both the options are out of money, loss is $(p_b - p_s)$.  When only the sold option is in the money, loss is $|m - s_s| -(p_b - p_s) \leq |s_b - s_s| -(p_b - p_s)$.

Thus, in all cases, profit and loss are limited. Potential loss is significant only when it happens that only the sold option is in the money: in other cases, loss is merely $(p_b - p_s)$. This significant loss eventuality is non existent in case of bullish call spreads and bearish put spreads.

\subsubsection{Types}
In case of a call spread, one buys and sells call options; put spreads are similarly defined. Depending on whether $s_s \geq s_b$, call spreads are called bullish or bearish (the reason for which is clear upon considering potential profit and loss). Put spreads are bearish or bullish depending on whether $s_b < s_s$.

\subsection{Straddles}
In a straddle, one buys/ sells put and call options (at a combined premium $p$) for the same security at the same strike price $s$, with the same expiry date.

If the options are bought, we have a long straddle. If the options are sold, we have a short straddle. In case of a long straddle, one is essentially betting that the market price will move outside the interval $[s - p, s + p]$. In the case of a short straddle, one is betting that the reverse will happen.

\subsubsection{Risk and returns}
WIth a long straddle, potential loss is atmost $p$, while potential returns are $\infty$. In case of a short straddle, the potential returns are atmost $p$, while potential loss is unlimited.

\chapter{Commodities}
\section{Price trends}
Commodity prices are highly sensitive to (speculation about) the supply situation (relative to demand). This leads to great volatility.

\subsection{Long term investment suitability}
Historically, most commodities have only keep up with inflation. In the long term, Investing in commodities is essentially betting against human ingenuity .

\section{Oil}
Oil is used in distribution of goods and services. Only a few nations produce it, and they form a cartel - OPEC - which can often arbitrarily increase price as a way to bargain for concessions from their buyers. But, the cartel is slightly inefficient because various producers often violate agreements due to self interests.

When its price increases, the value of all common goods increases.

\section{Precious metals}
There are mainly used as a store of value. Though they may have industrial use, their demand is dominated by the former. It is often difficult to get the fundamental value of precious metals. For example, Gold is in a 400 year price increase: but since its fundamental value is unknown, it can't be easily said to be a boom.

\section{Rare earth elements}
\tbc

\chapter{Critical goods}
The availability of items determine the value of others. A reduction in their supply leads to various crises.

\section{Foreign currency}
If an entity in country B wants to buy something from an entity in country A, it needs to pay in some currency used by country A: it could still have a choice among a set of currencies like \$, yuan etc.

\subsection{Shortage}
Sometimes there is a shortage in country A of currency accepted by country B's entities.

\subsubsection{Cause}
Trade deficit is usually the cause: the country imports more than it exports. For example: due to a drop in tourism which in-turn could be due to higher oil prices. 

\subsubsection{Consequences}
So due to the imbalance in supply and demand, foreign currency is suddenly more valuable. So, the country may be unable to import essential food items.

\subsection{Response}
\subsubsection{Immediate loan}
Typically countries attempt to solve this problem by getting a foreign currency loan from an agency such as the IMF, which then imposes certain conditions to try to ensure that such foreign currency shortage does not re-occur.

\subsubsection{Long-term solution}
These conditions aim to reduce the reliance of the country on foreign goods by making people of the country collectively poorer, which is accomplished by currency devaluation or by reducing public services so as to increase individual expenditure or by wage cuts. These conditions often lead to social unrest.

\part{Behavior of economic agents}
\chapter{Organizations}
\section{Mission}
Some organizations are focused on earning money for their owners by pursuing and growing a certain business.

Charitable organizations are focused on gathering resources (usually from donations) and spending them to promote social good by a ceratin type of activity. They are usually managed by a board of trustees.

\chapter{Businesses: general mechanism}
\section{Ownership}
At any given point in time, it is clear from documentation/ declaration who the owners of a business are, what portion of the business they own.

Owners usually get a portion of profit as 'dividends'. They have a say in appointing the highest level/ board of executives/ managers who run the company on their behalf.

\subsection{Shares/ Equity}
\subsubsection{Public offerings}
Existing owners or executives can effectively divide the company's ownership into a certain number of shares. A common way of expanding the ownership of a company is to sell these shares to the general public at a price which is either preset or determined automatically by an auction. This process is called the initial public offering/ IPO.

\subsubsection{Dividing profit}
Companies offer a dividend to owners by proportionally dividing the annual profits. This makes ownership of the company attractive, which becomes useful when the owners decide to sell more shares to raise more money.

High dividend yield companies offer a big dividend relative to the share value - they thus constitute attractive investment.

\subsubsection{Market capitalization}
A rough way to calculate the value of a company is to measure it by multiplying the number of shares by their market price. Depending on the size of the market capitalization, companies are classified into small, medium and large capital (cap) companies.

\subsubsection{Privileges}
Shares grant varying privileges: voting rights, priority in receiving payment in the event of liquidation of assets after bankruptcy.

\subsection{Succession}
In many firms, a small group of people - eg: a family - owns a large fraction of the shares. Oftentimes, it was one of them (or their ancestor) who started the business, and one or more of them are actively running it.

\subsubsection{Professional management}
Sometimes the owners appoint professional managers/ CEO's to run the company. Sometimes, they pick managers from their family. This is more common in countries where social contracts are weakly enforced, so trust outside the family is low.

\subsubsection{Nepotism}
Statistics reveal that succession involving nepotism in general results in poorer performance. This is often because the person given the responsibility of running the enterprise is not as capable of running a business as their predecessors - either due to lack of talent or due to lack of energy and drive to work hard. Also focusing on one's family artificially removes access to top talent outside the family.

Japanese firms often rely on adoption to solve this problem.

\section{Starting and funding business}
Entrepreneurship.

\subsection{Funding business}
The effect of governments on business operations and running is described elsewhere.

\subsubsection{Prototyping}
People with a good business idea may get 'seed money' from an 'angel investor' to make a prototype. The angel investors may well be governments or university funds.

\subsubsection{Launch}
Then, they pitch it to venture capitalists who provide two rounds of funding in return for a share in the company.

\subsubsection{Expansion}
A third round of funding, called the mezzannine funding, is then sought for a possible expansion phase, usually after the company is profitable.

Later, the companies often get money from the general public at large, through stock market, in return for a share in the company's ownership, as necessary.

\subsubsection{Long-term Borrowing}
This is often done by issuing bonds at the stock exchange or taking regular loans.

\subsubsection{Short-term borrowing}
For day-to-day operations, businesses may need money, but they may not hold any reserve capital for that purpose - all of it may be invested in expansion.

So, depending on the day-to-day financial situation companies \\borrow money from the 'commercial paper market', where various entities (especially money market mutual funds) lend money over the short-term. Commercial paper is essentially notes acknowledging these short-term, usually safe loans.

\subsubsection{Leverage}
Sometimes entities are allowed to be liable for much more money than they are worth. This is especially true in dealing with stocks and bonds, where people can make especially risky maneuvers to get high returns: entities are allowed to buy stocks with the agents money, insurances involved in credit-swap make huge guarantees.

\section{Bankruptcy}
When a company declares bankruptcy, it says that it is incapable of meeting all of its financial commitments/ liabilities. All of its agreements with trade-unions are then also canceled.

The business's assets can be liquidated (seized, valued and sold) to satisfy its liabilities. Also, the company may be taken over and restructured; and the new owner can repay outstanding debts.

\subsection{Rescue}
\subsubsection{Motivation}
The government is often motivated to rescue failing businesses to prevent negative consequences like job loss, reduction in national pride etc..

'Distress investors' are motivated by the prospects of taking ownership of a well established enterprise cheaply, if they believe they can make it profitable.

\subsubsection{Overpaying for bad assets}
One way rescue can happen is by buying or by encouraging others in buying toxic assets/ investments of the business at a sufficiently high price to ensure that the company regains solvency - the drawback is that the investor (tax-payer/ government usually) is then stuck with an unsound investment.

\subsubsection{Stock injection}
Another way is to invest in the business - become one of its owners by replenishing the business's capital.

As part of doing so, it is often essential to ensure that future investments are sounder, so the management is often to be replaced or stress tests are conducted, where businesses are allowed to fail if they fail to perform even after a rescue.

\subsubsection{Disadvantages}
Businesses which are rescued from collapse as a consequence of mistakes tend to make more bad moves in the future - especially if the management is not replaced. This happened with Japanese banks in 1990's.

\chapter{Labor}
\section{Hiring in businesses}
Academic hiring follows a different pattern.

\subsection{Selectivity}
Hiring people is generally easier than getting rid of them. So, when possible, businesses create interview processes which have low 'false positive' rates - even at the cost of high 'false negative' rates.

\subsection{Process}
Often, specialized recruiters are used to make initial contacts with people - in order to get an idea of where how the candidate may fit the company's needs. That person then may pitch the candidate to potential hirers.

\subsubsection{Screening rounds}
There is often a written/ programming/ modeling test or a technical phone screen, and successful candidates are then invited for an in-person interviews (travel expenses usually being paid by the hirer).

\subsubsection{Technical interviews}
In each of these stages, multiple persons (including potential hiring managers and peers) are used to reduce margin of error. In interviews, potential employers probe the candidate's knowledge, skills and personality. Ideally, interviewers should allow the candidate some time to think, try to minimize pressure - especially if 'working under pressure' is not a job requirement.

\subsubsection{Big-boss interview}
The final interview is usually with the highest-level boss, where a simple technical question may be asked. The reason for this final interview is unclear - except perhaps the company wanting to rely on the 'feeling' of a senior person.

Research seems to show that in such qualitative interviews' outcomes are decided in the first few moments of the interview!

\subsection{Candidate's strategy}
The candidate's strategy is to present his skills, knowledge and professional attitude to the interviewers; and to gather information/ impressions to judge whether the organization - especially the team he will be working with - will match his expectations.

\subsubsection{Attitude}
The professional attitude to be presented includes calmness, the ability to collect and compose oneself and one's proposals. A social aspect is being friendly, with pleasant body language described elsewhere.

\subsubsection{Answers and limits}
When his knowledge and skills are being probed in live-communication, the candidate should answer to the best of his ability - taking time to think if necessary. He may not be able to answer all questions - especially if he is seeking work which is at the edge of his abilities.

\subsubsection{Judging potential employers}
This is done both by reading online reviews (Eg: glassdoor.com ), by asking questions to potential peers/ managers, and by observing their interactions with you and each other.

\subsubsection{Salary negotiation}
Know how much you can get both for the job in a particular area in general and from the company in particular, by visiting online surveys (eg: glassdoor.com ). Use this to answer questions about the 'expected salary'.

\section{Motivation}
Motivating a hired employee to do his job involves ensuring that his pleasure/ satisfaction (perhaps even pain) coincides with benefits and loss of the company. 

The best case is when the employee, by nature, cares deeply about doing his job well - which is why good hiring is important.

Another alternative is to do this using rewards and threats. Research shows that for menial/ mechanical tasks, this works; but for creative work, this approach is counter-productive.

\section{Variation parameters}
\subsection{Employment term}
Employment may be short term or long term. Laws usually guarantee minimum daily wages.

\subsection{Mobility}
Contracts or laws may limit a worker's mobility.

Indentured laborers have agreed to serve for a certain period of time under certain conditions in exchange for money/ transportation paid in advance.

In case of slavery, the laborer is considered the property of the owner/ employer; and employment changes only when the slave is sold.

\subsection{Working conditions}
Some countries, by law limit working hours (36 hours in France, t be compensated when violated), specify minimum vacation (5 weeks in France, often exceeded), unlimited paid sick-leave (heavily subsidized by government in France), maternity/ fraternity leave (with maid service in France), heavily subsidized day-care.

Other countries, such as USA; and certain industries, are more friendly towards employers rather than employees.

\subsection{Remuneration}
Remuneration is paid in various forms: cash, stock options, stock allocation, matching contributions to retirement savings and charities etc..

\section{Labor unions}
Workers collectively have greater bargaining power while negotiating terms of employment (salary, benefits, leave, penalties) with their employer. To this end, they form trade unions.

\subsection{Breakdown of negotiations}
When negotiations with employers break down, they go on strike - that is halt work without pay, thereby causing loss to the employers. Employers sometimes try to get around the problem by hiring new workers who will accept poorer conditions and work - which often leads to violence and damage, and hence is an unattractive option.

\subsection{Administration, financing, membership}
Trade unions are usually run by members elected from among the employees. Not every employee is required to be a member of the union or participate in their strikes, but if a union exists and does bargaining on their behalf, even non-members have to pay a bargaining-fee to the union. Trade union elections tend to be very politically charged, and in many cases, they often coerce and punish disagreement and non-membership by illegally means - including arranging undesirable assignments and violence.

\chapter{Intermediary trading}
Basic negotiable securities tradings actions are considered elsewhere.

\section{Role in markets}
Intermediary traders buy and sell items, holding them for a short time while it changes hands between long-term investors. In doing so, they make a profit.

So, they focus strongly on the guessing and taking advantage of the motives of investors - aka front-running investors.

\subsection{Liquidity, cost}
The benefit they provide is in bringing greater liquidity to assets, buy supplying a sea of ready purchasing and selling power. So, an investor wanting to find a seller or a buyer has a easier time.

The disadvantage is that they increase the costs to investors: sellers receive less while buyers would pay more than if they were dealing directly with each other.

\section{Quote}
This includes the current asking and bidding price, bid size etc..

\section{Charts}
These are especially useful for highly liquid items Eg: negotiable securities. Portfolio decisions can be based on looking at the chart for the appropriate time period. With a chart, one can observe features like support and resistence levels. Thence, one can decide whether it is a good time to buy or sell an equity.

This is because charts contain in them the evaluations of large fund managers etc.. But that can be misleading in case of bubbles.

\subsection{Price vs time}
\subsubsection{Stock price: piecewise line}
On this chart, the main plot would be the stock price as a function of time. When the time axis is finely grained, one uses a piecewise linear function to show price movements.

\subsubsection{Stock price: candlesticks}
For some analysis it is more convenient to consider grosser discretizations of time: eg: daily. Since the stock price will have shown variations during this time period, one may represent many important features, like starting price, closing price, price range within a standard deviation: eg: using candlestick like shapes instead of points. Depending on whether price rose or fell during a time period, these candlesticks may be shaded differently.

\subsubsection{Trend lines}
On the same chart, it is useful to show a pair of trend lines/ Bulliger bands which connect highs and lows at grosser time periods. This helps understand how the resistance to price increases and support in the face of price decreases have varied over time, while ignoring short-term variations.

\paragraph{Support/ resistence levels}
Support and resistance levels in a stock's price chart correspond to price-ranges at which a twice or more times a low or high was observed. These include within them information about the fundamental value of a stock, as determined by fund managers.

\subsubsection{Moving averages}
Plots of k-day moving averages, for k = 15 to 200, brings out a smoother curve ignoring local jitters. This is helpful for long/ intermediate term investors.

\subsubsection{Futures price}
Here, one plots the average strike price against time together with the actual security price. If future pricing were perfect, these lines would coincide; but this is not the case. The current price is affected by its speculated price, while the speculated price is affected by the current price. So, these curves tend to converge.

If the speculated price converges from above, it is called contango. If the speculated price converges from below \tbc.

\subsection{Running strength index (RSI)}
This plots a ratio which measures bullishness (0 to 1) against time. Anything above .5 is considered strongly bullish. Extereme RSI's outside say [.3, .7] indicates oversoldness: ie the market is probably overreacting, one-sided, in a herd mentality. This is used as a signal by some traders to bet against the trend. 

\subsubsection{Moving average convergance/  divergence(MACD)}
This plots two exponential weighted k-day moving averages for, say, k = 12 and 26 against time. 

These together indicate the momentum of price, and can also indicate bullishness and bearishness levels. This is mainly of interest to traders rather than long term investors.

\subsection{Volume histogram}
\tbc

\section{Patterns}
Traders have come to assign names to various patterns in the movement of prices. The idea is to use patterns in predicting future price movements: ie, determine the trend.

\subsection{Reliability}
Reliability of patterns can be judge using the duration of the developing pattern. However, such reliance is dangerous in case of a bubble.

\subsection{Types}
Patterns may be short term or long term. They may be continuation patterns or reversals depending on the overall price movement (taking away the jitters).

\subsection{Ticks}
These are very short term patterns  - where just the last two distinctly priced trades (possibly at the end of small time-periods) are considered. A security price which is observed to be increasing is said to be uptick. The reverse is called a down-tick.

\subsection{Triangle}
The trendlines connecting highs and lows converge to form a triangle. So, there is a contraction in the price range. This is usually a continuation trend.

\subsubsection{Ascending}
Here, the highs are roughly the same level, while the lows keep getting higher. This is a sign of increasing demand sufficient to overcome existing skepticism about it. This usually indicates rising price.

\subsubsection{Descending}
Here, the lows are roughly the same level, while the highs keep decreasing. This usually indicates falling price.

\subsubsection{Symmetric}
Here, highs keep decreasing while lows keep increasing to converge at an intermediate price.

\subsubsection{Wedge}
In this pattern, trend-lines connecting highs and lows respectively seem to converge, but they move in the same direction. This is a continuation pattern - indicating acceleration or deceleration.

\subsection{Head and shoulders}
The price vs time graph shows price going towards (either increasing or decreasing) a certain level (the neck-line), and then 3 spikes beyond this level, the second of which (the head) is taller than the others.

This is a reversal pattern.

\section{Equity trading}
\subsection{Insurance against loss}
Traders are especially sensitive to short term price variations; eg: they would not like to leave their money tied up in securities for a long time, unlike investors. So, they often buy and sell options to guard against adverse price movements.

\section{Options trading}
Basic options trading actions are considered elsewhere.

\subsection{Strike price from chart}
For highly volatile stocks, a good way to set the strike price might be to look at the support or resistence level.

\tbc

\chapter{Consumption}
Effects of macro-economic circumstance on consumption levels is described elsewhere.

\section{Conspicuous consumption}
This is done mainly for the purpose of increasing one's status in society (aka 'status signaling'). This is aka 'keeping up with the joneses'.

\subsection{Green consumption}
People might want to similarly enhance status among environmentally conscious people by making conspicuous green choices - even though a less conspicuous investment may have benefited the ecosystem more. Eg: Installing solar panels on street side of the house, Buying the distinctive Toyota Prius rather than other inconspicuous hybrid vehicles around 2010's.

\chapter{Investing}
\section{Core idea}
Investment involves financing/ owning businesses - perhaps by buying stocks or bonds, with the aim of getting a good return on investment and with little risk of loss.

\section{Future Discount factor}
For most people and organizations, $x$ amount of money today is better than $kx$ amount of money one year from now, where $k > 1$. One reason for this is inflation.

\subsection{Among individuals}
Discount factor is much steeper in the short-term than in the long term. Many people will say I prefer x now rather than 1.25 x one month from now. But they will say I would rather get 1.25x 13 months from now rather than x 12 months form now.

\subsection{Organizations}
In calculating the tradeoff in investing towards avoiding an anticipated disaster vs fixing damages once a disaster happens, the discount factor is often used.

The discount factor that is to be used is controvertial. Some argue that it must be tied to the inflation rate or the government borrowing rate. But, USA keeps it at an unrealistic level of 1.07 as of 2012. This causes them not to invest as much in disaster prevention, fighting global climate change etc..

\section{Profit/ risk potential}
Information is fundamental in judging the quality of  an investment. At the very least, one should clearly know the bounds on potential profit and loss.

\subsection{Dividend yield vs growth}
There are two avenues for profiting from an investment: a share in profits earned, increase in value of the share in the business - which may be turned into a profit by selling it to someone else.

Value stocks which have shown high dividend (distributed profit) yield relative to capitalization are attractive. Growth stocks, even if they have low dividend yield, have a potential to grow in value in the future - it is possible that undistributed profits have been invested to such fuel growth. 'Balanced businesses' lie between these two extremes.

\subsubsection{Total returns}
Total returns adds (dividend/ interest) income and capital/ value appreciation.

One can judge the potential for returns by observing the total returns over a period of time trailing from the current date (week/ trimester/ year).

\subsection{Risk and return-advantage types}
Risk in the investment is the potential to loose money. Risk is paired with advantage: investors promised a bigger return will tolerate some risk.

\subsubsection{Credit default risk}
This risk is particularly relevant for bonds. This is the risk of businesses not repaying their loans.

The advantage is that riskier bonds yield bigger returns.

\paragraph{Credit rating agencies}
Sometimes, the ultimate owners of bonds sometimes do not have a clear idea about the risk involved.

Agencies such as Moody's evaluate various investment options (including government bonds) and assign a letter grade indicating the risk involved; AAA being the best. Credit rating assigned to Japan actually means Japanese government bonds.

\subparagraph{Trustworthiness}
These ratings are highly trusted by investors who manage global savings. They are also trusted by governments, which require firms which hold possibly big liabilities (insurance firms, banks) to invest a certain fraction of their capital in highly rated bonds/ loans.

But, their evaluation methodology is not flawless. For example, before the 2008 sub-prime mortgage crisis, they wrongly rated US mortgage-backed securities as being a safe investment- because they used the wrong historical data pertaining to mortgages with better guarantees, and because they assumed that real estate prices will continue to increase.

\subsubsection{Interest rate increase risk}
Suppose that the interest rate on government bonds increases (perhaps due to investor concerns about budget deficits or due to the action of the federal bank to combat inflation). Then, bond prices, especially of long term bonds, fall in response to the decreased demand. People can simply invest capital in the newly high-yielding government bonds or banks: so that for existing bonds to compete with them in yield, they have to fall in value.

\subsubsection{Concentration risk}
Investing in a small number of businesses is considerably riskier than investing in a diverse variety of businesses. But, focusing on a particular fast growing business segment promises higher returns.

'Exposure' to a variety of businesses (diversity in investment) reduces risk.

\paragraph{Underlying security capitalization}
In case of investment funds, this can be measured in terms of the capitalization of the underlying securities.

The safety of a fund is proportional to the market capitalization of the component stocks (i.e: what portion of the economy it represents.)

For example, S\&P 100 tracks 100 important companies in various sectors, S\&P 500 tracks 500 companies. The prices of these index funds form a way of tracking the health of the economy - Eg NSE, BSE in India, Dow Jones etc..

\subsubsection{Small business risk}
Aka small capitalization risk. Small businesses constitute riskier investments - they are not as entrenched as bigger businesses which may out-compete them; and they may fail due to short-term reasons which would not affect larger companies with their greater access to resources.

Larger businesses are better able to benefit from faster growth in emerging markets.

\subsubsection{Foreign currency risk}
If the returns from a business/ security is in a foreign currency, that currency may appreciate in value relative to the investor's currency, leading to lower value reaching the investor than initially expected.

\subsubsection{Emerging markets risk}
There is a cost of lower skilled work-force, corruption, worse regulation and bad infrastructure.

Emerging markets often have the advantage of higher growth.

\subsubsection{Geographical risk}
Certain geographical regions have more socio-political instability than others - causing businesses in those regions to be more susceptible to disruptions. 

\subsubsection{Leverage risk}
Businesses which use leverage to conduct transactions are especially vulnerable to short term losses.

\subsection{Risk measures}
One component of this is to observe past performance: though it is not a guarantee of future returns.

\subsubsection{Correlation}
The correlation of the value of the security with the relevant economy / sector - perhaps measured using the value of the corresponding index - is informative. An investment whose return is not highly correlated with the relevant economy / sector brings diversity to the portfolio, and helps reduce risk.

$R^2$ is a common correlation measure - with values ranging [0, 1].

\subsubsection{Curve based}
The ratio of past upticks to downticks, yield in past years (especially bad years) help compare risk.

\subsubsection{Volatility}
Also, the standard deviation in stock-prices (together with highs and lows) in a certain time period describes volatility. Volatility, if it does not co-occur with monotonic increase, indicates risk in short term trading.

\subsection{Security analysis}
Security analysts gather information, including non-public information, to advice people on trading strategies to maximize their profit. They run a special risk of being charged with insider trading.

Analyst reports and news articles are sometimes available for free on the internet.

\subsubsection{Mutual fund Ratings}
Mutual funds are rated by agencies which examine their past performance and holdings: Eg Morningstar.

\section{Portfolio design}
How to pick good investments?

Pick a simple asset allocation model, stick to it, rebalance occasionally.

See http://www.bogleheads.org/forum/viewtopic.php?t=798.

Switch to fixed income when retiring.

\subsection{Investment horizon}
One should be clear about the investment horizon: when does one intend to withdraw money from the fund, with what frequency etc..

\subsubsection{Connection to risk}
Long term investors (eg: very young people) are capable of taking greater risks than short-term investors (eg: older people, about-to-retire people); because they can afford to ride out market turmoil.

\subsubsection{Long term investing}
One should not be fooled by market jitters (or perhaps even by medium term booms and busts) in deciding on worthiness of securities; though jitters can be exploited later, while purchasing it.

\subsection{Betting on broad swathes of the economy}
Abbreviations: Total Stock Market: TSM, SV: Small Value.

Model: TSM core 5/8, SV tilt 3/8 (\~ Fama French three factor model of US fund returns). Repeat internationally. The allocation was checked with FF factors published on Ken French's website .

References:  Rick Ferri here: https://www.bogleheads.org/forum/viewtopic.php?t=154130 and
 https://www.forbes.com/sites/rickferri/2014/07/17/to-tilt-or-not-to-tilt/#54ba6f4d4986

See kreNa-vikalpAni .


\subsection{Value investing}
One judges the intrinsic value of a security, and buys it if the market value is less than this by a significant 'margin of safety'. One particularly focuses on value - rather than growth - businesses.

\subsubsection{Identifying value securities}
This can be difficult, despite the use of comparative measures of value, like P/E. Thus, it is advantageous to invest in areas which you understand - Or one can copy investments by reputed value-investor funds.

Usually, equities of small companies not yet discovered by the market fit this description.

One can ask the question: 'Will I be willing to privatize this business?'

\subsubsection{Focus on differentiation}
In addition, one may focus on well-differentiated, rather than generic, companies: this ensures that the underlying product retains an advantage in capturing customers even if competitors catch up. This was exploited by Warren Buffet, when he invested in Coca Cola.

\subsection{Tuning portfolio based on situations}
\subsubsection{Credit crisis}
Increase allocation on big capital companies - they have access to more resources/ capital.

Government bailout will devalue currency - however slightly: so bet on foreign currency or commodities.

\subsubsection{Raw material price increase}
\paragraph{Know the cause}
It is important to understand whether the price increase is indeed long-lasting and sustainable. To do this, one must consider the causes: is the mismatch between demand and supply current, or is it speculative (The latter is more fickle - especially if it occurs despite current surplus)? Is the speculated price increase due to speculation of currency devaluation? 

\paragraph{Action}
If the price has good potential to increase, it would be a good idea to focus on stocks/ funds focusing on those sectors or on markets/ countries where they are produced and sold.

\subsubsection{Common under-performance}
In a large study in 1991-1996, active traders (avg portfolio turnover was 75\%) performed much worse than the market returns (5\% difference ); they tended to trade mostly common stocks (with very few value stocks). Some individuals did much better, others did much worse, but the large majority of individual investors would have done better by taking a nap rather than by acting on their ideas. It is the cost of trading and the frequency of trading, not portfolio selections, that explain the poor investment performance. One model to explain this is that this excessive trading emanates from investor overconfidence.


\section{Common avenues}
\subsection{Broad investment in chosen sectors}
\subsubsection{Picking a sector}
Different sectors show different patterns of growth over a long period of time.
\begin{itemize}
 \item Cyclical: Real estate, basic materials, consumer items like cars, financial services.
 \item Sensitive: Energy (including petroleum), technology, communication services, industrials. These show greater variation depending on current macroeconomic circumstances.
 \item Defensive (Monotonically small growth): Healthcare, essential consumer items (like food), utilities (electricity, water).
\end{itemize}

\subsubsection{ETF investing}
One can invest in ETF's tracking index funds. Similarly, one can invest in ETF's tracking the bond and real-estate markets. To pick an ETF, consider its portfolio and deduce its potential profit/ risk.

\subsection{Currency trading}
Trade in precious metals (gold and silver) can be included in this category: investment/ demand in them is correlated with stability and people's discomfort with paper currency.

\subsection{Bonds}
Some allocation to bonds, through an ETF perhaps, reduces volatility. Bonds are specially susceptible to the interest-rate-increase risk.

\subsubsection{Bond ladder}
Bonds have a fixed maturity date. So, investors often build a ladder of bonds such that one bond or the other matures in a given time slot. This brings greater liquidity to the investment.

\section{Purchase and selling}
\subsection{Deciding a fair price}
In case of funds, one can consider the net asset value of the stock. Even ETF's, which usually track the NAV closely due to arbitrage, show some tracking error.

\subsection{Using options}
Selling put options is a profitable way of investing in a certain security at a low enough price. Thus, one can try to buy the equity at the desired price, while simultaneously collecting the premium for selling the option.

Similarly call options can be used while selling an equity.

\subsection{Memorylessness}
Non-professional investors like to lock in their gains; they sell “winners,” stocks whose prices have gone up, and they hang on to their losers. In the short run going forward recent winners tend to do better than recent losers, so individuals sell the wrong stocks. Professional investors react more selectively to news, they apply predetermined strategies/ decision processes rather than relying on emotional reactions.

Memorylessness is important in good investing. Past transactions should not inform investment decisions; only current positions and beliefs about future price changes should. Thence, one will not hang on to loosers and impulsively sell winners.

\subsection{Timing}
Timing can be accomplished using triggers (eg: limit orders) described elsewhere.

\subsubsection{Bubble-bursts}
Bubble-bursts constitute good time to buy-up high value securities.

\subsubsection{Short-term jitters}
While trading securities (especially in buying and selling), one can take advantage of short-term jitters in the market. Thus, one can take advantage of others' mistakes in evaluating a security - or traders' sentiments. But, one should not have excessive reliance on jitters to make profits.

\subsubsection{Criticism}
Some value investors argue that no attention should be paid to timing. They say that focusing on 'when to buy' detracts one from the more fundamental question of 'what to buy'.

\section{Investment/ mutual funds}
\subsection{Relying on experts or algorithms}
One can mitigate risks by relying on experts or well-tested algorithms to manage funds. In USA, publicly traded companies which follow the value investing philosophy include: Berkshire Hathaway, BBTEX, Boyar, Tweedy Borwne, YACKX.

Index funds for a given economic sector simply invest in major companies in proportion to their market value. One relys on the less riskier supposition that the entire sector is likely to grow, rather than any single security belonging to that sector.

\subsection{Fees}
Both index and human-managed funds need to pay brokerage costs. Managed funds usually have greater overhead costs, as they need to pay fund managers.

Over a long period of time, index funds seem to be equal to or better than most managed funds in terms of return on investment/ growth - excepting those following a good long-term investment philosophy.

\subsubsection{Fee types}
Various forms of fees are charged: these are separate from trading expenses. The simplest is the administration fees.

To encourage stability in their pool of money by discouraging excessive money-movement, they often charge fees for purchase and redemption of fund shares, and exchange of shares between funds.

\paragraph{Sales loads}
A fraction of the investment: front-end load and back-end loads are often imposed at the time of purchase and redemption respectively. These are different from fixed purchase and redemption fees mostly independent of the size of the investment.

This is often to pay brokers who sell mutual fund shares, and encourage long-term investment. For example, back end loads (aka class B shares in the fund) often decrease with time. No-load funds (aka class C shares), being more fluid, carry higher administrative charges.

\subsection{Rating}
Analysts rate the risk/ profit potential of various funds using a 3*3 grid.

In case of bond-funds, one axis is the term of the bonds involved (short, medium, long); and another is the credit worthiness of the bonds involved (low, medium, high).

In case of equity-funds, one axis classifies the companies into value/ balanced/ growth categories, while another axis classifies them based on the capitalization (small, medium, large).

\subsection{Allocation disclosure}
Many actively managed mutual funds disclose their allocation only quarterly, whereas index funds do it every few seconds - especially because of the ability to exchange them for the underlying securities.

\subsubsection{Net Asset value (NAV)}
The ideal value of a share in a fund can be deduced by the value of the underlying securities. The actual value of a share in the fund additionally depends on the faith of the market in potential value of that share - which may vary with changing allocation and the abilities of the fund managers.

NAV of a security with the symbol XYZ can often be tracked by looking at the charts corresponding to the symbol XYZ.IV.

\subsection{Exchange Traded Funds (ETF's)}
Ownership of some funds is shared, and these shares can often be traded in the stock exchange. The value of the ETF shares should reflect the value of the underlying securities in which it has invested, and the belief in the skill of the fund manager, in case of actively managed funds.

Because index-tracking funds' investment positions are more readily available, their prices on the stock market more closely reflects the evaluation of the underlying securities - ie. they have less tracking error.

\subsubsection{Inverse funds}
These ETF's systematically bet on decline in value of a certain set of businesses - perhaps using futures contracts. These involve active management.

Not suitable for buy and hold investors.

\subsubsection{Leveraged}
These ETF's use k-times (usually: 2x or 3x) leverage in an effort to increase profits k-fold. These involve active management, and interest needs to be paid. It is even possible for the underlying security to increase in value but for the leveraged ETF to fall in value.

Not suitable for buy and hold investors.

\subsubsection{USA examples}
One can use Morningstar's ETF screener to identify performance leaders.

Historically, the S\&P 500 index fund has grown at the rate of 17\% per annum. During the dot-com collapse, this fund grew at the rate of -1\% per annum. During the 1990's, it grew at the rate of 33\% per annum. 

The non-profit structure, Vanguard, issues ETF's.

VTI tracks the entire stock market, MGK: US Mega cap; VEA for Europe; VWO emerging markets; VNQ: high yielding real estate. LAG tracks the entire bond market.

Asia-Pacific: (GMF) gives China a 37\% weighting, with Taiwan and India making up another 46\%.

Latin america: ILF has had a two-year annualized return of 43.0\%.

Energy: XOP.

\subsubsection{Indian market}
CNX Nifty represents a market capitalization of about 59\%. CNX 100 represents about 66.61 \% of the total market capitalization as on April 10, 2007. CNX 500 represents about 90\% of the stock traded at NSE.

ETFs corresponding to CNX 500 are not available. ETFs corresponding to CNX NIFTY index are SUNDER and NIFTY BeES. NIFTY BeES seems to be better, in that it has fewer restrictions on purchase amount. It also seems to have a lower tracking error.

\subsection{Money market funds}
Money market funds are very safe funds which, among other things, focus on making low-risk short-term loans to businesses on the commercial paper market.

\subsection{Critique}
\subsubsection{Overhead}
In recent decades, financial services have become a more central and powerful force in the economy. This has resulted in drain of talent from production and invention to money management, which is fundamentally a overhead.

\subsubsection{Dumb allocation, individual greed}
As the 2011 real-estate booms in USA, Ireland and the role of investment banks in helping Greece defraud the EU about its financial condition show, the use of deliberately information-obfuscating financial instruments have been on the rise, leading to mis-allocation of funds by misled investors. All this has happened while employees of these institutions received high salaries, and while they were bailed out by governments gifting them taxpayer money to stay solvent (rather than nationalizing them).

\section{Personal/ Family finance}
\subsection{Importance of planning}
An individual earns money from employment when young and healthy; but eventually they retire or get sick- either early or after going old.

Many expenses are very predictable. Hence set up a retirement fund, children's education fund, health/ emergency fund. Many people fail at doing this.

\subsection{Goals}
\subsubsection{Lifestyle post-retirement}
Different living conditions will require different incomes. Projected cost of living should be calculated considering the cost of living, adjusted to the rate of inflation at the future abode.

To reduce need for money, one can lead a frugal lifestyle, live in a developing location/ country/ boat (at least part of the year) where cost of living is low and where, due to the exchange rate, savings in foreign currency can be exploited.

\subsection{Changes in portfolio over time}
As one ages, investment horizon reduces; so one's capability to take risk reduces.

\section{Tax on capital gains}
Tax is levied on one's gains realized using the sale of security holdings or from interest/ dividend income on such securities, with some quirky rules.

Long-term capital gains (Eg: selling stock after holding it for a year) are usually taxed at a lower rate.

Suppose one invests in a mutual fund, which during the financial year makes a profit, then one has to pay tax - even if one lost money by joining the mutual fund in the middle of the year when its value was higher than at the end of the financial year.

One can offset gains obtained by selling some securities by selling securities on which one has made a loss - but it is forbidden to buy the same security again immediately after doing so (wash sale).

Subscribing to a dividend reinvestment plan complicates calculation of capital gains.

\subsection{Retirement, medical, education funds}
To facilitate and encourage saving for large expenses in the future, governments often provide tax breaks for special investment accounts for retirement, medical expenses and children's education (college funds). 

See \url{https://docs.google.com/spreadsheets/d/1JHM7R5afB9CtE3_Vdun7kjUmbh6dQJEdDmkX9apOS4E/edit#gid=417813240} 

\subsubsection{Taxation-deferred funds}
Incomes earned using these special accounts are exempt from taxes while they stay within the fund. However, withdrawals, which are only under very restricted circumstances, are taxed. 

Eg: Individual Retirement Account [IRA] in USA, PPF in India (government managed with rate of interest determined by them). 

\subsubsection{Profit-tax free funds}
Money may be taxed at the time it is deposited in the fund, while withdrawals are tax-free Eg: Roth IRA. This may be relatively advantageous because higher incomes usually attract a higher tax rate.

\subsubsection{Tax-free funds}
See spreadsheet.

\chapter{Philanthropy}
This involves donating money for public welfare, without expectation of profit. Effective philanthropy is tightly focused on the outcome; though it often starts out as 'feel-good' philanthropy where one does not consciously try to maximize the good outcome.

\section{Donation}
Donations to charitable organizations are often tax exempt.

Some companies match employee donations up to a certain limit.

\subsection{USA}
It does not make sense to worry about tax deductions for charitable contributions, unless they exceed around 5500\$ in USA (when one can make itemized deductions). In any case, the organization you donate to must be registered in USA to qualify for tax deduction.


\section{Quality}
\subsection{Parameters}
The fraction of money donated going to organizational costs, rather than the poor, varies. 

Provably good charities meet certain standards for charity accountability, which includes disclosure of their Tax forms and expenditure details on their website.

\chapter{Copyrights, patents and trademarks}
Aka intellectual property, a much criticized term because it draws a analogy to physical property which fails in some ways.

\section{Patents}
The purpose of patents is to provide incentives for invention and discovery.

They work in some cases (eg: chemicals and pharmaceuticals) and fail in others (eg: software).


\tbc

\part{Particular businesses}
\chapter{Money lending}
\section{Operation}
\subsection{Capital}
Bank owners make an investment - this is called the capital. A part of the bank's capital goes towards increasing the size of the capital.

\subsection{Liabilities/ Borrowed money}
They also borrow money from investors - in other words, investors deposit money in a bank. The bank promises a return on investment to the depositor, but this return on investment is but a part of the profit the bank makes using that money.

The amount of money added to the deposit (principal) is called the interest. The amount of interest accrued per dollar per year may be specified using an interest rate/ fraction.

\subsubsection{Compounding}
Interest is often compounded ie added to the principal with a certain frequency- daily or monthly or quarterly or annually. For fluid accounts - where the deposited money varies daily, interest is calculated on the money available in the deposit account for each day and is added to the principal/ compounded usually at a lower frequency - eg: quarterly.

When interest is compounded, a number called 'annual percentage yield' (APY) may be used to specify compound interest accrued per dollar per year. Or, one may simply specify the simple interest which, together with the compounding frequency, determines the compound interest: In case it varies (and is applied) daily, this simple interest is often stated  as 7-day yield.

\subsection{Assets/ Lent money}
Banks lend money and get a greater amount of money, determined by a rate of interest, by some time - usually in installments. They sometimes tend to make other investments too.

The bank's assets may be liquid or illiquid (eg: mortgages for houses when the real estate prices are falling).

The assets may be good or toxic, depending on their risk/ projected/ current return on investment.

With risky lending (ie: where the probability of default is higher than usual), the bank charges a higher rate of interest in order to mitigate the risk by purchasing or acting as insurance against default.

\subsection{Balance sheet}
The bank should be capable of returning the money to the depositors whenever they demand it - as long as they do not do it all at once (a 'run on the bank'). So, always lent money <= borrowed money + capital. In order for the bank to be making the maximum possible safe profit, this should hold: lent money/ investments = borrowed money + capital. Banks for which this holds are said to be solvent.

\section{Defaulters}
Banks initially often may try to restructure the debt - allow later payment, increase installments, reduce interest etc..

\subsection{Recovery attempts}
Defaulters borrow money from banks but fail to pay back. THen, the bank employs various means - seizing assets of defaulters or better yet those who guaranteed payment, shaming and illegal coercion in some cases to recover the money. Defaulters with political power often get their debts written off.

\subsubsection{Collateral}
\tbc

\subsubsection{Insurance}
See section on credit default swaps.

\subsection{Reduced capital}
Bad loans usually reduce a bank's capital. But, sometimes, the amount of bad loans exceeds the bank's capital, and the bank becomes insolvent. The bank then is incapable of repaying the depositors.

\section{Insolvency consequences}
An insolvent bank, having made bad investments, is incapable of repaying depositors.

\subsection{Guarantees}
Small depositors are guaranteed to get their money back, up to a certain limit- usually by government agencies.  But, such guarantees do not protect big investors.

\subsection{Economic consequences}
Bad money lending leads to ruin of the lender; sometimes selling ownership of these loans on the stock market, they bring ruin to others: 2008 financial crisis resulted from bad housing loans in USA, for example.

Failure of banks also leads to a credit-crisis.

\subsection{Government rescue}
The government, is motivated to rescue banks - to maintain GDP growth. This has been true since Roman times. Means of rescue are described for businesses in general elsewhere.

\paragraph{Nationalization}
This involves stock injection by the government. When many banks are insolvent, the timing becomes important - all banks should be nationalized at the same time, in order to prevent panicked withdrawal of deposits from banks which are yet to be nationalized because nationalization indicates that the bank is insolvent, and depositors may suspect that the government may be incapable of rescuing it.

This nationalization is usually meant to be temporary - governments ideally find new buyers as soon as possible. But this is difficult when many banks are insolvent.

\section{Deposit account services}
\subsection{Fees}
Some banks may charge fees for their services, which are often waived if the customer has certain minimum deposits. Others may provide these services for free in order to beat competition, and to reduce workload of having to provide manual service at branches.

\subsubsection{Service quality trade-off}
Compared with Small/ local/ cooperative/ internet only banks/ credit unions, Large banks usually have the advantage of having a large ATM network and slicker online banking/ remote deposit systems.

But, large banks have the disadvantage of levying heavier fees, lower deposit interest rates, higher loan interest rates, worse customer service.

\paragraph{Advantageous combination}
An high yield savings or money market account (perhaps in a credit union) with a checking account in a large bank; and ACH transfers set up between them.

\subsection{Limited withdrawal accounts}
These include savings accounts and money market accounts - the latter, investing in bonds (the money market) usually yields a higher rate of interest. The number of withdrawals is limited to increase the probability of deposited funds being available to banks: It is 6 in USA. A certain minimum balance is usually required.

\subsection{Term deposits}
Funds from these accounts cannot be withdrawn for a certain period of time. Banks may allow withdrawal under certain penalties. In return for this guaranteed  deposit, these yield a relatively high rate of interest.

\subsubsection{Recurring/ periodic deposit}
These deposit accounts require small periodic deposits. This is often done at banks and even at the post office in India.

\subsection{Checking account}
Aka transactional/ current account. A checking account allows bank customers to write checks - so it can be used as money. It does yields much less interest than savings account - often nothing.

\subsubsection{Check redemption}
The person to whom the check is presented can redeem it for cash at his bank, which communicates with the issuers bank to get money from the issuer's account. If the issuer's account lacks sufficient funds, a penalty is imposed to discourage such irresponsibility.

\paragraph{Reversal/ availability}
The payer's bank also has the right to reverse the transfer within a few days (5 in USA) - so funds may not be immediately available for withdrawal.

Also, one can refuse/ stop payment in case of a dispute between the payee and the payer.

\subsubsection{Fee avoidance conditions}
Chase: avg daily balance of 1500, no interest. BoA campusEdge: free, no interest. BoA MyAccess: avg daily balance of 1500, no interest. UFCU: Free, no interest.

Ally checking: free checks, .5\% interest, no min balance, refunds ATM charges; bill pay reportedly unreliable. Morril Janes: 1.5\%, min 1500.

\subsection{Automatic/ remote tellers}
\subsubsection{ATM's}
Banks usually have a network of ATM machines whence cash may be drawn and deposits made. A debit card is issued for their use. They may be associated with an incentive to encourage its use: why?

Use of ATMs maintained by other banks results in fees, eg: 2\$ for BoA; which may be refunded by the bank with the account.

\subsubsection{Remote deposits}
Banks often offer smart-phone applications or websites from which deposits can be made. For example, deposits can be made by using the phone's camera on a check.

\subsection{Bill payment}
Banks offering this service send cashier's checks or electronic transfers on behalf of the customer.

\subsection{Electronic Fund transfer (EFT)}
\subsubsection{ACH}
With ACH (Automatic clearing house), money transfer between banks is done electronically within a given country. They are commonly used to process direct deposits etc.. To use ACH, one needs to know the routing numbers/ ABA codes and the account numbers for the source and destination accounts.

\paragraph{Reversal/ availability}
The payer's bank also has the right to reverse the transfer within a few days (5 in USA) - so funds may not be immediately available for withdrawal.

\paragraph{US charges}
Chase, BoA: Free pull, \$3 for a push (not for chase?). PayPal, ufcu: Free push/pull.

\paragraph{Between USA and India}
ACH to Indian bank branches and VISA cards is free if done through ICICI bank : they make money by using a favorable exchange rate.

With ICICI, "Outward remittances" used to transfer money from India to USA: While it is free for NRI's, for residents, ICICI charges the higher of 0.25\% and Rs 1200 for such transfers. They can be done online.

\subsubsection{Wiring money}
Suppose two banks have reciprocal accounts with each other: then, by sending a message with account and settlement information, they send money. Even if the banks do not hold reciprocal information, they act through 'corresponding'/ intermediate banks.

This is commonly used for international money transfers.

\section{Foreign deposit accounts in India}
Higher interest rates make opening savings account in foreign banks attractive. Repatriability and taxation should be considered.

\subsection{FCNR Deposits}
These accounts are held in foreign currency - usually \$. They yield a lower interest rate.

\subsection{NRE rupee Accounts}
Funds held in the account, interest earned on them and returns from investments using them are fully repatriable. NRE accounts are not taxable in India.

These may be term deposits or savings accounts (yielded 3.5\% in 2011). 

\subsection{NRO rupee Accounts}
The principal amount is not repatriable and can be used only for local payments. However, the interest earned is fully repatriable.  Income earned on investments done on non-repatriable basis is to be credited to NRO account.

Other current incomes such as pensions, dividends, rent, etc are also repatriable, subject to producing the appropriate certificate from a chartered accountant. Funds up to USD 1 million (or equivalent) per financial year can be repatriated out of the balance held in NRO accounts for the education of your children, for medical expenses for your family and you, etc. 

\subsection{Investment routing}
Portfolio Investment Scheme (PINS) is a scheme of Reserve Bank of India defined in Schedule 3 of Foreign Exchange Management Act 2000. Investments made from foreign accounts is repatriable only if it was made from a designated 'repatriable' account.

\subitem As per recent RBI guidelines, NRI/OCB should have a separate bank account exclusively for PINS purposes. Transactions relating to their personal banking as well as on account of transactions relating to shares acquired other then under PINS including IPOs should be routed in a separate bank account not linked to PINS. Account/s can be joint.

\subitem The sales proceeds are subject to Capital Gains tax. The rate of tax depends upon the period of holding. Currently the rate for short-term capital gains tax is 30.6\% and for long-term capital gains is 10.2\% inclusive of surcharge in both the cases. Such tax will be deducted at source by the Designated Bank.

\chapter{Manufacturing}
\section{Process design: Productivity and quality}
How to transform the shop-floor/ assembly process and the procurement/ design process to increase quantity and quality?

\subsection{Quality/ quantity trade-off}
When the focus is on quantity, quality often suffers, as workers are provided incentives and punishments which, focusing on quantity, largely ignore quality. Quality control is often postponed to a later stage.

The Toyota process, observing that correcting/ tolerating mistakes can be costly, tries to ensure that they get it right the first time - workers have the ability to stop the assembly-line in order to fix a mistake when they make/ detect it.

\subsection{Supplier coordination}
Workers on the shop floor, besides designers, often make valuable suggestions and observations about how the parts they assemble could be improved for their purposes. To implement these improvements, it is important to work closely with the suppliers.

\subsection{Teamwork}
Workers working in small teams, with minimal hierarchy; rotating and exchanging jobs often to resolve monotony; encouraging and helping each other to finish the job and to improve the process show excellent productivity.

But, workers stuck to individual jobs, antagonistic to management due to a lack of mutual trust and with a hierarchy, tend to have very low productivity. Eg: GM in the 70's.

\subsection{Continuous improvement}
Aka kaizen in japanese. Every movement/ procedure can be timed and measured, and skills can be improved. The processes (including worker comfort) and parts can be continually improved based on worker suggestions.

\section{Quality check}
Before the goods are deployed to the market-place, faulty goods must be detected.

\chapter{Food production}
\section{Agriculture}
\subsection{Resources}
Agriculture requires non-saline water, fertile land, seeds, labor. Of these, overall, water is often most scarce.

\section{Artificial production}
Research is on to produce meat in-vitro.

\section{Demand and supply}
\subsection{Past}
The discovery of agriculture enabled mass, reliable food production. For most of history, we have had a food surplus. When there was a famine in one country, one could rely on imports from another.

\subsection{Forecast}
Human population is growing.

Non vegetarian diet requires production of more grain, consumption of water, use of land. As people grow affluent in developing countries, the demand for meat grows higher. Eg: In 2010, India roughly consumed 400 units of grain per capita, while America consumed 1600 units of grain per capita.

\subsubsection{Water scarcity}
In many countries, people are pumping out water faster than nature can replenish the aquifers by precipitation. Also, people in many places have been inefficient in storing precipitated water - especially in modern times (unlike ancient rAjasthAna and kOlAra folk).

\subsubsection{Land scarcity}
Arable land is scarce. Some land is part of important nature reserves. Plus, there is competition from using it for growing ethanol producing crops.

\subsubsection{Climate change}
Because agricultural crops were domesticated when the climate was cooler, hotter climate leads to less yields. It is estimated that for wheat, every 1 degree increase corresponds to 10\% reduction in yield. Besides, climate change produces more extremes.

\section{Role of nations}
Food supply affects social stability. Eg: food riots in 18th century France, 2011 Tunisia etc.. So, governments are very concerned with preserving adequate food supply.

Food exporters, in times of inflation, ban exports. Considering the fact that the food market is a seller's market, they refuse to enter into long term agreements.

So, food importers have begun to purchase land and water supply abroad - especially Africa - to grow their own food. They have also begun to make deals directly with farmers.

\chapter{Insurance}
Insurance products, besides serving the main purpose of hedging to counter financial risk, often also function as investment in mutual funds.

\section{Functioning}
\subsection{As security}
This business takes advantage of the fact that risk of a particular kind (house burning down, theft, death), affects a small fraction of a population at any given time, with very high probability. So, it collects fees/ premiums from a large number of insured people, and compensates those on whom a bad event befalls.

When the total fees collected exceeds the money usually lost in providing compensation, there is profit.

\subsection{Risks to unowned entities}
Suppose that you wanted insurance to compensate you for a bad-event befalling someone else: like getting insurance to compensate you if some other company, with whom you have no direct dealing goes bankrupt! Such an interest would make sense if either a] you are making a bet, hoping to make a large profit, or b] the bad event befalling someone else would affect you, because it affects others you deal with.

This happened in the 'credit default swap' industry, for example in the first decade after 2000; and caused companies (eg: AIG insurance) unconnected (directly) with sub-prime mortgages to collapse.

\section{Health insurance}
Negative problems associated with using health insurance as a way of providing an essential public service are described elsewhere.

\subsection{Negotiation with care providers}
Hospitals often form regional or national conglomerates, so that they can negotiate with insurance providers from a stronger position. Insurance providers tend to be fragmented, and patients often side with care-providers, so they often are at a weaker position to reduce costs.

Also, care providers often give discounts to big insurers, by shifting costs to smaller insurers; leading to greater local monopoly.

\subsection{Unscrupulous practices}
\subsubsection{Refusal of insurance}
People showing any risk factor - being too thin

\subsubsection{Denial of treatment}
Often insurance companies deny (responsibility to pay for) essential care, claiming that the treatment is unnecessary, or that it is experimental, or claiming that the individual had a preexisting condition which was either not revealed, or was not properly reported or treated before insurance; often with death being a consequence.

Medical directors have the incentive to save money by denying care.

\subsubsection{Regulation}
If within USA, the state Insurance commissioner can look into complaints of wrongdoing on the part of the insurance companies.

\subsection{Generic vs branded medicine}
Insurance companies may demand that the patient bear a fraction of the cost (copayment), which often turns out to be significant. This is a way the insurer encourages patients to opt for generic medicine rather than much costlier branded medicine. Drug companies retaliate by paying the copayment themselves (through coupons) while getting the insurance company to bear most of the cost.

\subsection{Policy-types}
\subsubsection{Benefits: fixed vs comprehensive}
Fixed-benefit policies have sub-limits on how much they will pay for various events like hospitalization: it seems that this can be very inadequate - covering less than 1/3 of the cost according to some online accounts. Comprehensive benefit policies don't have such sub-limits.

\subsubsection{Deductibles and premiums}
Policies with high deductibles, often carry lower premium - as the insurance now only has to cover the cost of major health problems cropping up. They are often associated with a health-savings account where the employer and emplyoee contribute money. Such funds are described separately in the section on retirement/ health savings funds. 

By forcing providers and patients to deal with each other for common problems, competition to benefit the patients is said to thrive.

\subsubsection{Claims processing}
Some policies make sure that the hospital takes care of insurance claims, while others expect the patient to pay and then get reimbursed.

\subsubsection{Multiple policies}
In some cases where a person holds multiple policies, one of them acts as the primary policy while another acts as the secondary. In other cases, where there is ambuguity, there is some hassle with insurance companies trying to get each other to pay.

\subsubsection{Renewal restrictions}
Some policies require a minimum period of enrollment in order to be renewable. Ability to renew is important in order for being able to afford ongoing treatment when the policy expires, and in order to avoid having to tackle the problem of having a preexisting condition when buying a new policy.

\subsection{For visitors/ travellers}
These policies have no incentive to cover preventative care.

Also, even if the hospital charges the insurance company directly, the policy-holder still needs to file a claim in order to establish the authenticity of the claim : something which would not be required for local people.

\subsubsection{Buying choices}
Decent policies costing less than 50\$ a month are available at many places on the internet, including \\ \tiny{\url{http://www.americanvisitorinsurance.com/insurance/visitors-medical-summary.asp}} . \\Some costlier policies have limited coverage for acute onset of pre-existing conditions - kvrao.org (India foundation).

Travel insurance bought in the country of travel implies simpler service (due to an established network), and it is likely to be regulated by the government (so that complaints are possible).

\chapter{Arbitrage}
Goods/ services often have different values in different markets. So, often people can fulfill demand in the market where an item is costlier by buying it from a market where it is cheaper and getting it to the costlier market.

For securities traded in different stock exchanges, automatons do this effectively.

\chapter{Payment services}
\section{Credit cards}
Credit cards are a relatively safe way for lenders to encourage public borrowing while making purchases - that way, they can make a profit.

Stores - possibly getting a commission - often offer credit cards (with attractive terms) under their brand-name to their customers; but they are actually managed by banks.

The necessary infrastructure is managed by one of a handful of companies - like Visa, MasterCard, AmericanExpress etc..

\subsection{Advantages to customer}
You earn reward points, which often correspond to 1 cent per dollar spent; with spendings in certain categories earning greater reward points. Reward programs have minor differences: rewards could be up to 5 cents per dollar spent in certain (often rotating) categories.

You have some protection in case of dispute with merchant.

You get some travel/ vehicle rental/ purchase related insurance benefits.

\subsubsection{Official expenses}
Consider getting a Diner's club or an american express card, which usually allow more delay in credit repayment so as to accommodate dealys in reimbursement.

\section{Debit cards}
It is best to avoid using a debit card online - in case of fraud/ misuse due to details being leaked by a merchant, you can loose money in the bank account - if only temporarily.

\section{Intermediate services}
Services like Paypal ensure that credit card/ bank account information, which can potentially be misused, does not reach the merchant - instead, for a fee from the merchant, these services collect money from the customer and pass it over to the seller.

\chapter{Goods distribution}
\section{Retail vs wholesale}
Retail business focuses on distributing goods to  consumers, while making a profit. Wholesalers focus on selling goods in bulk at a lower price to other businesses and distributors. Eg in USA: Sam's club, costco. In India: Metro.

\section{Price discrimination}
Different people are willing to pay different prices for the same product, and retailers try to exploit this.

In general, retailers try to distinguish the two types of customers by offering lower prices only to those who participate in some undesirable process which costs extra time and attention- eg: waiting for a certain day to make a purchase, committing to purchase in advance, gathering coupons.

\subsection{Haggling}
In markets where there is a slower pace of life, prices were negotiated with each customer. But, this costs time and attention on both sides. Hence, in places with a fast pace of life, a base-price is fixed, and other techniques are used for price discrimination.

\subsection{Special discounts}
Discounts are offered to special categories of people known to be price sensitive - old people, students, people who shop in the same store repeatedly, members of clubs.

\subsection{Store sale}
To spur more purchases by price-sensitive customers- especially of items obsoleted by competition - retailers sometimes offer incentives/ price reductions. This often happens in special occasions Eg: Black Friday, Christmas in USA.

\subsubsection{Illusory discounts}
Often, there is only the appearance of price decrease: The label price is actually increased prior to offering a discount.

\subsection{Coupons}
Customers willing to find coupons in certain locations (including deal websites) are offered a lower price.

\section{Sales techniques}
\subsection{Salesmen}
Salesmen  are employed to answer questions, pitch various products, perhaps even bargain with potential customers. Traveling salespeople visit homes and try to sell products. Telemarketers call/ email to sell products.

They are provided incentives to do a good job  - eg: commissions which increase with the number of items sold.

\subsubsection{Pitching}
Demonstrations are often used, with the customer being allowed to handle the product.

\subsubsection{Sales-tricks}
Once the customer agrees that the product is desirable, salesfolk can use psychological tricks to encourage purchase at a good price. For example, they may appeal to the vanity of the customer by saying that ownership of the product leads to greater social/ moral status.

\subsection{Other incentives}
Music to encourage impulsive behavior/ spending is often used.

Physical interaction with products - touching, holding - increases probability of purchase.

Shopping experience/ comfort is elevated by clustering shops at an attractive location (malls),  offering food, rest-rooms, day-care etc..

\section{Internet retailers}
These have the advantage of incurring lower building/ sales labor costs because of not having to operate physical stores. Without space constraints, they are capable of maintaining a large inventory, which caters to quirky tastes of consumers, which constitutes a large fraction of consumer needs - aka the long tail of consumer needs distribution.

\section{Returns policy}
When returns are accepted, often a restocking fee is charged for electronic items in USA.

\section{Electronics retail}
Some stores, like Fry's guarantee meeting competitor's price within a few days of purchase.

\subsection{Pressure for low prices}
\subsubsection{Short life-span}
Due to technological progress, devices become obsolete fast; so there is a thriving secondary market. So, devices with newer technologies must compete with lower priced older devices.

\subsubsection{Competition}
Electronics industry is highly competitive; and deal-sharing websites ensure that customers are able to find the best deals/ prices online.

\chapter{Online reselling infrastructure}
Web-services like Ebay, Amazon etc.. try to ensure that there to be mutually satisfactory transactions between sellers and buyers.

For the sellers, they make it very convenient for to list their products on a popular platform on  the internet, perform much of the order-information processing for the seller. For the buyers, they ensure that they have access to the seller's rating/ reputation, have some guarantees about the quality of service.

\section{Fraud}
Buyers/ sellers sometimes try to avoid paying commission on items sold by finalizing the transaction outside the reselling website.

\section{Revenue sources}
Advertisements, commission from sellers.

\section{Set-time auctions}
Eg: EBay. In these transactions, the item is sold to the highest bidder at a certain point in time.

\subsection{Pricing assistance}
Auction websites may offer statistics about the price paid for an item of a certain type.

\subsection{Underselling avoidance}
Sellers have various ways to ensure that they don't sell cheaply: Auctioned items may have a starting bid revealed to bidders or reserve price not revealed to bidders - the latter option being used to not turn away bidders. Even for items without a reserve price (to attract bidders), sellers can ensure that they only loose the commission by using automatic bidding services acting on their behalf.

\subsection{Bidding services}
One can use services like esnipe to bid using a program in the last few moments of the auction. They charge commission for auctions won.

\section{Bid-price auction}
Aka penny, all-pay auctions.

Bids rise in small increments, and fees are charged for every bid.

The winner of such auctions often gets items at much lower prices: eg: 1/3 or 1/2. But, the auctioneer still recovers the cost of the item from the bid-fees. 

They often start out as set-time auctions, but auctioneers often extend the time by small increments (eg: 20 sec) to take advantage of peoples' bidding.

\chapter{Advertising}
\section{Fundamentals}
The basic problem in advertising is finding the audience who is likely to buy the product/ service (target audience).

\subsection{Placement and position}
There may be place for a limited number of advertisements to be displayed in a certain venue. Yet, there may be several advertisers competing for those places. Further they may be interested in gaining favorable placement in the ad-list.

\section{Coupon distribution}
Consider a seller wanting to advertise inventory to target customers, or wanting to spur business. In exchange for a share in the resulting sales revenue, Internet coupon distribution companies like groupon make a seller's discount-coupons available to customers. Customers are attracted by the prospect of getting goods/ services at a cheaper rate. The sellers are attracted by the prospect of undiscounted business the customer's visit may generate.

\section{Web content-driven advertising}
Many companies offer sellers the opportunity to advertise their products on webpages with related content. The problem then is to identify web-pages with visitors likely to buy the product. 

\subsection{Advertisement types}
These are either text based ads, which invite an immediate click or display ads, which result in a delayed action by the customer. Customer behavior is tracked and this is measurable.

Advertisers either pay per click (PPC) or pay per impression (based on the number of times the ad is viewed).

In case the advertiser pays per click, he can potentially fall victim to fraudulent clicks: clicks by competitors or bots employed by the displayer.

\subsection{Ad-Placement services}
Ad-placement services often facilitate placement of advertisements on content-pages meeting the criteria useful to the user. They may, as in the case of search engines or web portals, themselves have content-pages; or they may team up with other website owners to display advertisements on their pages in exchange for given them a cut in the advertising revenue earned by the advertising service. Eg: Google AdWords.

\subsubsection{Display auctions}
For every candidate content-page located by the ad-placement service, the service picks ads to display using a] an auction using bids preset by advertisers, b] the likelihood of the ad being clicked (thereby generating revenue for the placement service) - aka ad-quality, which can be measured using statistical experiments.

The ad-placement service may also consider the limited budget allotted by the advertiser in picking the auctions into which he is entered - for example to space advertisements in various parts of the day; and they may allow advertisers to set different bids at different times of the day etc..

\subsection{Search engine results}
In case of search engines, the visitor's intention is judged using the search phrase, the location of the device whence search is done, time of the day etc.. Advertisers are able to specify relevant result pages (on which they can advertise) using these parameters.

\subsubsection{Search phrase targeting}
Advertisers can specify positive and negative keywords. \tbc

\subsubsection{Case study}
Amrit's taxi service heavily depended on this advertising strategy. He made the transition from advertising in printed directories in 2008 when he observed that strategy failing.

\tbc

\chapter{Gambling}
\section{Bet contract}
\subsection{2 outcomes}
Consider an event E. A bet between two people (A and B) is a contract which says that if E comes to pass (usually by a specified time), person A will pay an amount x to B; otherwise if E doesn't happen, B will pay y to A.

y is called the bet liability.

\subsubsection{Viability of the bet}
Rational B (who is not risk-averse) will enter the contract if according to his evaluation of $a$, $ax - (1-a)y \geq 0$ or $a \geq \frac{y}{x+y}$. Rational A will similarly enter the contract only if $a \leq \frac{y}{x+y}$.

\subsubsection{Odds}
The probabilistic odds is $\frac{Pr(E)}{Pr(\lnot E)}$. But Pr(E) is not known beforehand and is only estimated by a certain party to be $a$. So, betting odds is $k = \frac{a}{1-a} = \frac{y}{x}$. It is often expressed as $y:x = k:1$ against. It can also be denoted by $k+1$, the decimal odds.

\subsection{Multiple outcomes}
Suppose that there is a universal set of $k$ mutually exclusive outcomes: \\$\set{E_1 .. E_k}$. Then, $k$ entities can enter into a betting contract, according to which side $i$ gets $x_i$ if $E_i$ happens and looses $y_i$ if it does not. So, from side $i$'s perspective, it is a binary bet on event $E_i$; for which calculations of viability and odds are explained elsewhere.

\subsection{Betting pool contracts}
Betting pools involve groups of people entering a bet contract, with every person specifying his share in the earnings/ liability. Winnings are distributed according to the shares specified while entering the contract.

Fixed odds contract has already been explained earlier.

\subsubsection{Unfixed odds contract}
Aka parimutuel betting. This contract does not specify the betting odds in advance, allowing people to enter or leave the betting pool independently at different points in time. Money from the loosing side is distributed to the winning side. Thus, the betting odds depends on the individual bets from groups of people on each side.

\subsection{Accounting Infrastructure}
Aka Book-keeping. In betting pools, people may enter or leave, and proper accounting is necessary. The book-keeper has to ensure that the people involved honor the bet contract and pay when they loose - so he often collects money owed in case of loss in advance.

\subsubsection{Fixed odds contract}
Quoting fixed betting odds for each outcome, the book-keeper admits peoples' bets for a certain outcome.

\paragraph{Anticipation of bet-placements}
$\forall i$ the fraction of \\bet-liabilities $b_i$ promised for a given outcome $E_i$ is such that $b_i = 1 - \sum_{j\neq i}b_j$ holds. If while deciding on the fixed odds, the book-keeper has perfectly anticipated $\set{b_i}$ and quoted odds correspondingly, the book-keeper does not loose any money irrespective of the outcome. In other words, the winning parties' earnings at settlement time are matched exactly by others' losses. The book-keeper does not stand to loose money: infact, promising slightly lower earnings than $1 - \sum_{j\neq i}b_j$, the book-keeper can extract a fee irrespective of the outcome.

But, if the odds quoted by the book-keeper do not match the eventual bet-liability allocations, it is possible in case of certain outcomes that, to honor the contract, the book-maker will loose money. Hence, he may need to maintain reserve capital.

\section{Binary bet exchange}
Such an exchange facilitates bets over an event $E$ happening by time $T$. Eg: Intrade.

\subsection{Contracts, final settlement}
Pairs of people enter into bet contracts at odds of the form $1-x: x$ against $E$. A betting exchange, modeled after the stock market, facilitates such contracts by displaying/ matching offers for entering into bets for and against $E$ at certain odds.

\subsubsection{Share trade representation}
In the betting exchange, these odds are specified in the form of 'share prices' of event $E$ which lie between [0, 1]. Consider a betting contract proposal with odds $1-x: x$ against $E$. An offer to enter such a contract on the side of $E$ is denoted by an offer to buy a share of $E$ at price $x$. An offer to enter such a contract against $E$ is denoted by an offer to sell a share of $E$ (which one need not possess at the time of selling) at price $x$. Thus, entering into a betting contract is denoted by a share trade.

At settlement time, of course, the odds of all logical agents are $0:1$ against $E$ if $E$ has occurred; and $1:0$ otherwise. These correspond to share price reaching 1 or 0 respectively.

The obligation of the loosing party to pay money to the winning party at settlement time is then represented by a] the obligation of the seller of the share to buy it at settlement time, and b] the rise or fall of the share price to 1 or 0.

Offers to enter a contract and the act of entering a contract corresponds to placing limit orders and market orders in the betting exchange.

\subsection{Probability evaluation}
\subsubsection{The problem}
One can use betting exchanges to answer the question: 'What is the probability that a certain event $E$ occurs by a certain 'settlement' time $T$?' The market tries to answer it using bets individuals place on that event.

\subsubsection{Diversity benefits}
The idea is that a large and diverse set of people sometimes evaluate the situation better than a panel of experts.

\subsubsection{Unfulfilled orders}
At any given point in time, and given the conditions of logical agents entering a bet examined elsewhere, $Pr(E)$ can be judged using the unmatched buy or sell offers on the betting exchange.

An unfulfilled offer to buy at price $x$ says that the market agrees that $Pr(E) \geq x$. An unfulfilled offer to sell at price $x$ says that the market agrees that $Pr(E) \leq x$. So, $Pr(E)$, according to the market, lies between the lowest unfulfilled sell order and the highest unfulfilled buy order.

\chapter{Air carriers}
\section{Factors affecting price}
\subsection{Travel date/ day}
Tickets cost more in certain months.

Tickets tend to cost more on Fridays, saturday, sunday, than on weekdays (especially monday).

\subsection{Airport demand/ competition}
Flights from Dallas or houston probably cost less than flights from austin.

\subsection{Operation costs}
Some airlines are cheaper than others.

\subsubsection{Fuel costs}
This constitutes a major part of the operating costs.

Airlines which have access to cheaper fuel - either due to a stockpile (Southwest Airlines in 2000-2009), or geographic location (Emirates airlines) can offer lower prices.

\subsubsection{Cheap international airlines}
Try Sri Lankan or Korean airlines. 

\section{Alliances}
The operating areas of any airline is limited. People often end up using multiple airlines to complete their journey. Alliances of airlines can offer tickets for such journeys - thus geographically diverse alliances are advantageous to the business of individual airlines.

\section{Customer retention programs}
\subsection{Rewards}
They offer incentives for customer loyalty with rewards which correspond to the number of miles traveled.

\section{Related businesses}
To make extra money, they sell products on flights, and offer lounges at various airports.

\chapter{Travel agencies}
Online and offline agencies sell tickets/ reservations on behalf of airlines, car rental agencies and hotels.

Clubbing these reservations together may result in lower costs for the consumer.

\section{Hotel reservations}
Expedia seemed to keep a 70\$ margin on reservations (which was discounted for its employes).

\section{Air-Ticket sales}
It is often advantageous to call an agent - especially for international travel - to look for the best deal: some airlines are not listed on travel search websites.

\subsection{For travel inside USA}
Look at Orbitz and expedia. Check Southwest website, as it is not listed in these.

\subsection{Agents}
EDESTINATIONS INC
1-800-949-FARE (rAjIv: works 7 days a week)

\section{Car rentals}
Car rentals seem to be cheaper on the internet - due to greater competition and clarity for the customer.

\chapter{Communication networks}
These networks provide infrastructure and services like internet access, (mobile) telephony, cable television etc..

\section{Mobile connections}
Mobile connection services are either prepaid or postpaid.

\subsection{Prepaid services}
With these services, customers buy a certain quanta of incoming and outgoing connections of various types.

\subsection{Postpaid services}
This is offered to credit-worthy customers.

\subsubsection{Contracts}
People are attracted to enter into a contract in which they agree to use a company's mobile connection services for a certain period of time, in exchange for incentives such as discounts on attractive mobile devices.

The cost of these devices and incentives provided to sales agents is, of course, recovered from the profits earned using the customer's business.

\paragraph{Group contracts}
Aka family plans. Companies try to attract the close contacts of a customer to use the same network by offering reduced prices for groups of people.

\paragraph{Early termination}
Early termination involves payment of certain fees. If a person is terminating the contract early because he is abroad, this fee may often be waived in USA.

\paragraph{Renewal}
Companies offer to renew the contract before the termination \tbc

\chapter{Crime}
This is considered in the 'society' survey.

\part{Economy/ sector outlooks}
\chapter{USA}
\section{General economy}
USA has high trade deficit: especially with oil producing countries and manufacturers like China.

GDP Growth is much lower than in emerging economies: 0-3\%.

\section{Industry}
USA's industry is highly innovative - one of the best in the world. Industries are highly mechanized.

\subsection{Manufacturing}
America is the biggest manufacturer of goods in the world. But, much of the manufacturing is done in highly automated factories which are run by a handful of skilled employees overseeing the automation; manufacturing involving low skilled labour has moved to countries (in the developing world) with cheaper labor.

\section{Public debt, consumption}
Public debt (not same as government debt) is high (as of 2007-2011): so the consuming power of the population is likely to decrease.

\subsection{Decreasing income for many}
The classic 'American dream' includes the hope that with each succeeding generation, the standard of living (esp household income) improves. But, household income has actually been decreasing since 1970's; but the decline was made less noticeable for a few decades due to easily available loans.

For a while, many people masked their declining income using easily granted debt.

\subsubsection{Low unskilled labor opportunities}
Historically, there were two ways in which one could attain a good standard of living - by entering the skilled-workforce (mainly in the service industry) after getting good education; and by entering the unskilled factory workforce. The latter option has ended due to the automation and off-shoring in the manufacturing industry. According to several studies, the most promising way of solving this seems to be to offer kindergarten training to ensure that kids acquire sufficient social skills!

\subsection{High expenses}
Another problem is the rising cost of medical care - the root of which problem probably lies with the fact that employers are incentivized to purchase and offer medical insurance to people.

\section{Government debt}
As of 2011, it is at 100\% of GDP: Historically, when deficit 90\%, nations' economies tend to decline.

\subsection{Deficit and response}
Tax rate is too low to pay for the local, state and federal government expenditures.

\subsubsection{Political deadlock}
A part of the population, particularly the Republican party supporters, favor a lower level of public service and lower taxation, further they don't even favor a higher taxation rate for the rich (a response which stems from socialism-phobia and the unrealistic hidden American dream of the non-rich that they will be rich one day).

Another part of the population, supporting the Democratic party, wants higher taxation and maintenance of good government services.

These factions are unable to reach a compromise as of 2011.

\subsection{Lenders}
Government debt is high: with many trade-partners (esp: China, Japan) lending back the excess dollars they get due to the trade surplus. For reasons why they do this, consider the case of China described elsewhere.

\section{Dollar}
Much currency is in foreign hands due to USAs' trade deficit. Also, in response to the credit crisis following the sub-prime mortgage crisis of 2007, the Fed devalued the dollar by printing trillions of dollars.

So, dollar is likely to fall in value relative to other currencies, especially Yuan.

\chapter{Japan}
Since 1990's the country has been in 'stagflation', with the economy shrinking in value due to the government allowing banks to make bad loans without repercussions by bailing them out.

\section{Deficit}
Faced with a declining work-force, strong anti-immigration sentiment and an aging population, government income is projected to decline compared to its expenditures.

\chapter{Europe}
\section{Economic union}
Many countries in Europe are part of a union which allows easy travel and relatively free trade with low taxation. Many are also use a common currency, Euro, issued by a central bank.

A big part of the motivation for doing this was to quell fears that the strong German economy might again give rise to an expansionist war-machine.

\subsection{Euro}
\tbc

\section{Strength of economies}
Economic viability of sovereigns is discussed elsewhere.

\subsection{Germany}
Germany is the economic and industrial powerhouse of Europe. Its products and demands are consumed in many countries.

\subsection{Switzerland}
The Swiss have their own currency. Their economy is strong, but is heavily dependent on foreign demand for their products. Euro has been getting weaker relative to the Swiss Franc due to financial problems described elsewhere. So, in order to keep demand for its products high, the Swiss have had to artificially weaken their currency by promising a fixed, lower exchange rate.

\section{Sovereigns' economic viability}
\subsection{Financially strong sovereigns}
German government is relatively efficient in collecting taxes and providing services.

\subsubsection{German income}
German citizens have a keen memory of the inter-bellum period when run-away inflation occurred. So, they are strongly resistant to irresponsible government spending which could lead to similar devaluation. Also, they are willing to pay relatively high taxes, and have of late even agreed to lower job security.

\subsubsection{France}
French banks have high exposure to Greek sovereign debt. So, they may need to be resuced by the French sovereign.

\subsection{Financially weak sovereigns}
As of 2011: Some countries, given relatively easy access to debt, mismanaged government economy.

\subsubsection{Greece, Portugal}
Greece lied to others about the state of their economy in order to join the Euro and pretend compliance with the associated demands. Greek politicians used money acquired from debt to pay unrealistically high wages to state employees. Its tax-collectors are deliberately inefficient.

The case of Portugal is similar.

\subsubsection{Ireland}
The Irish, poor for a millennium, became rich as population growth declined due to increased popularity of contraception in the traditionally Catholic nation. They then used easily granted debt to buy land and build (now wasted) buildings in a real-estate boom. The Irish banks then had to be rescued by the sovereign, which reduced the financial health of the Irish government.

\subsubsection{Iceland}
Icelanders, given easy access to debt, sought to use the money in buying goods and services abroad which they could not afford with the goods and services they exported. The anticipated domestic growth to pay this debt was unrealistic.

Some lenders were ordinary people: For example, banks promised high interest on foreign currency deposits. Icelandic government had to then make up for atleast the domestic commitments of its banks.

\chapter{China}
Chinese economy is growing at a fast pace. Politically, China seems to be stable in the short to medium term.

\section{Manufacturing}
China has a trade surplus relative to the USA. It is a manufacturing hub.

\section{Media}
China, due to tight censorship, does not seem to be a hot-spot for internet businesses. It blocks Facebook, twitter - anything which can be used to organize subversion of the government.

\section{Middle class}
1.4\% of urban households make more than \$15,000 a year, and only 11\% make \$5,000-15,000. Chinese people have high savings, low borrowing tendencies, because unless they work for the state they are unlikely to receive much of a pension. Consumer inflation is officially 5.3\% but probably higher.

\section{Yuan}
Yuan is set to appreciate in value relative to the dollar eventually - though it is being gamed in the reverse direction by the government restricting Yuan access to foreigners.

\subsection{Lending}
Chinese funds and government lend excess dollars to USA by buying US bonds. They do this in order to get rid of excess dollars and keep export profitable to their producers in the short-term, thereby encouraging further growth in the exports business.

\section{Culture}
\subsection{Politics}
1989 saw the Tianmen square crackdown.

\subsection{Religion}
The communist party tried to eradicate religion starting from the cultural revolution - they tortured monks, destroyed monasteries. But this ceased in 1982. After economic growth in the 1990's, recognizing a popular feeling of spiritual emptiness, the party promoted religion as an important part of developing a harmonious society, provided they support the party. Those that disagree are suppressed: Eg: Falun Gong, Catholic Church, some Tibetian Buddhism.

\chapter{India}
High political stability.

% \bibliographystyle{plain}
% \bibliography{gameTheory}
\end{document}
