\documentclass[oneside, article]{memoir}

\usepackage{amsmath, amssymb}
\usepackage{hyperref, graphicx, verbatim, listings, multirow, subfigure}
\usepackage{algorithm, algorithmic}
% \usepackage[bottom]{footmisc}
\lstset{breaklines=true}
\setcounter{tocdepth}{3}

% Lets verbatim and verb environments automatically break lines.
\makeatletter
\def\@xobeysp{ }
\makeatother
% \lstset{breaklines=true,basicstyle=\ttfamily}

% Configuration for the memoir class.
\renewcommand{\cleardoublepage}{}
% \renewcommand*{\partpageend}{}
\renewcommand{\afterpartskip}{}
\maxsecnumdepth{subsubsection} % number subsections
\maxtocdepth{subsubsection}

\addtolength{\parindent}{-5mm}
% Packages not included:
% For multiline comments, use caption package. But this conflicts with hyperref while making html files.
% subfigure conflicts with use with memoir style-sheet.

% Use something like:
% % Use something like:
% % Use something like:
% \input{../../macros}

% groupings of objects.
\newcommand{\set}[1]{\left\{ #1 \right\}}
\newcommand{\seq}[1]{\left(#1\right)}
\newcommand{\ang}[1]{\langle#1\rangle}
\newcommand{\tuple}[1]{\left(#1\right)}

% numerical shortcuts.
\newcommand{\abs}[1]{\left| #1\right|}
\newcommand{\floor}[1]{\left\lfloor #1 \right\rfloor}
\newcommand{\ceil}[1]{\left\lceil #1 \right\rceil}

% linear algebra shortcuts.
\newcommand{\change}{\Delta}
\newcommand{\norm}[1]{\left\| #1\right\|}
\newcommand{\dprod}[1]{\langle#1\rangle}
\newcommand{\linspan}[1]{\langle#1\rangle}
\newcommand{\conj}[1]{\overline{#1}}
\newcommand{\gradient}{\nabla}
\newcommand{\der}{\frac{d}{dx}}
\newcommand{\lap}{\Delta}
\newcommand{\kron}{\otimes}
\newcommand{\nperp}{\nvdash}

\newcommand{\mat}[1]{\left( \begin{smallmatrix}#1 \end{smallmatrix} \right)}

% derivatives and limits
\newcommand{\partder}[2]{\frac{\partial #1}{\partial #2}}
\newcommand{\partdern}[3]{\frac{\partial^{#3} #1}{\partial #2^{#3}}}

% Arrows
\newcommand{\diverge}{\nearrow}
\newcommand{\notto}{\nrightarrow}
\newcommand{\up}{\uparrow}
\newcommand{\down}{\downarrow}
% gets and gives are defined!

% ordering operators
\newcommand{\oleq}{\preceq}
\newcommand{\ogeq}{\succeq}

% programming and logic operators
\newcommand{\dfn}{:=}
\newcommand{\assign}{:=}
\newcommand{\co}{\ co\ }
\newcommand{\en}{\ en\ }


% logic operators
\newcommand{\xor}{\oplus}
\newcommand{\Land}{\bigwedge}
\newcommand{\Lor}{\bigvee}
\newcommand{\finish}{$\Box$}
\newcommand{\contra}{\Rightarrow \Leftarrow}
\newcommand{\iseq}{\stackrel{_?}{=}}


% Set theory
\newcommand{\symdiff}{\Delta}
\newcommand{\union}{\cup}
\newcommand{\inters}{\cap}
\newcommand{\Union}{\bigcup}
\newcommand{\Inters}{\bigcap}
\newcommand{\nullSet}{\phi}

% graph theory
\newcommand{\nbd}{\Gamma}

% Script alphabets
% For reals, use \Re

% greek letters
\newcommand{\eps}{\epsilon}
\newcommand{\del}{\delta}
\newcommand{\ga}{\alpha}
\newcommand{\gb}{\beta}
\newcommand{\gd}{\del}
\newcommand{\gf}{\phi}
\newcommand{\gF}{\Phi}
\newcommand{\gl}{\lambda}
\newcommand{\gm}{\mu}
\newcommand{\gn}{\nu}
\newcommand{\gr}{\rho}
\newcommand{\gs}{\sigma}
\newcommand{\gt}{\theta}
\newcommand{\gx}{\xi}

\newcommand{\sw}{\sigma}
\newcommand{\SW}{\Sigma}
\newcommand{\ew}{\lambda}
\newcommand{\EW}{\Lambda}

\newcommand{\Del}{\Delta}
\newcommand{\gD}{\Delta}
\newcommand{\gG}{\Gamma}
\newcommand{\gO}{\Omega}
\newcommand{\gL}{\Lambda}
\newcommand{\gS}{\Sigma}

% Formatting shortcuts
\newcommand{\red}[1]{\textcolor{red}{#1}}
\newcommand{\blue}[1]{\textcolor{blue}{#1}}
\newcommand{\htext}[2]{\texorpdfstring{#1}{#2}}

% Statistics
\newcommand{\distr}{\sim}
\newcommand{\stddev}{\sigma}
\newcommand{\covmatrix}{\Sigma}
\newcommand{\mean}{\mu}
\newcommand{\param}{\gt}
\newcommand{\ftr}{\phi}

% General utility
\newcommand{\todo}[1]{\footnote{TODO: #1}}
\newcommand{\exclaim}[1]{{\textbf{\textit{#1}}}}
\newcommand{\tbc}{[\textbf{Incomplete}]}
\newcommand{\chk}{[\textbf{Check}]}
\newcommand{\oprob}{[\textbf{OP}]:}
\newcommand{\core}[1]{\textbf{Core Idea:}}
\newcommand{\why}{[\textbf{Find proof}]}
\newcommand{\opt}[1]{\textit{#1}}


\DeclareMathOperator*{\argmin}{arg\,min}
\DeclareMathOperator{\rank}{rank}
\newcommand{\redcol}[1]{\textcolor{red}{#1}}
\newcommand{\bluecol}[1]{\textcolor{blue}{#1}}
\newcommand{\greencol}[1]{\textcolor{green}{#1}}


\renewcommand{\~}{\htext{$\sim$}{~}}


% groupings of objects.
\newcommand{\set}[1]{\left\{ #1 \right\}}
\newcommand{\seq}[1]{\left(#1\right)}
\newcommand{\ang}[1]{\langle#1\rangle}
\newcommand{\tuple}[1]{\left(#1\right)}

% numerical shortcuts.
\newcommand{\abs}[1]{\left| #1\right|}
\newcommand{\floor}[1]{\left\lfloor #1 \right\rfloor}
\newcommand{\ceil}[1]{\left\lceil #1 \right\rceil}

% linear algebra shortcuts.
\newcommand{\change}{\Delta}
\newcommand{\norm}[1]{\left\| #1\right\|}
\newcommand{\dprod}[1]{\langle#1\rangle}
\newcommand{\linspan}[1]{\langle#1\rangle}
\newcommand{\conj}[1]{\overline{#1}}
\newcommand{\gradient}{\nabla}
\newcommand{\der}{\frac{d}{dx}}
\newcommand{\lap}{\Delta}
\newcommand{\kron}{\otimes}
\newcommand{\nperp}{\nvdash}

\newcommand{\mat}[1]{\left( \begin{smallmatrix}#1 \end{smallmatrix} \right)}

% derivatives and limits
\newcommand{\partder}[2]{\frac{\partial #1}{\partial #2}}
\newcommand{\partdern}[3]{\frac{\partial^{#3} #1}{\partial #2^{#3}}}

% Arrows
\newcommand{\diverge}{\nearrow}
\newcommand{\notto}{\nrightarrow}
\newcommand{\up}{\uparrow}
\newcommand{\down}{\downarrow}
% gets and gives are defined!

% ordering operators
\newcommand{\oleq}{\preceq}
\newcommand{\ogeq}{\succeq}

% programming and logic operators
\newcommand{\dfn}{:=}
\newcommand{\assign}{:=}
\newcommand{\co}{\ co\ }
\newcommand{\en}{\ en\ }


% logic operators
\newcommand{\xor}{\oplus}
\newcommand{\Land}{\bigwedge}
\newcommand{\Lor}{\bigvee}
\newcommand{\finish}{$\Box$}
\newcommand{\contra}{\Rightarrow \Leftarrow}
\newcommand{\iseq}{\stackrel{_?}{=}}


% Set theory
\newcommand{\symdiff}{\Delta}
\newcommand{\union}{\cup}
\newcommand{\inters}{\cap}
\newcommand{\Union}{\bigcup}
\newcommand{\Inters}{\bigcap}
\newcommand{\nullSet}{\phi}

% graph theory
\newcommand{\nbd}{\Gamma}

% Script alphabets
% For reals, use \Re

% greek letters
\newcommand{\eps}{\epsilon}
\newcommand{\del}{\delta}
\newcommand{\ga}{\alpha}
\newcommand{\gb}{\beta}
\newcommand{\gd}{\del}
\newcommand{\gf}{\phi}
\newcommand{\gF}{\Phi}
\newcommand{\gl}{\lambda}
\newcommand{\gm}{\mu}
\newcommand{\gn}{\nu}
\newcommand{\gr}{\rho}
\newcommand{\gs}{\sigma}
\newcommand{\gt}{\theta}
\newcommand{\gx}{\xi}

\newcommand{\sw}{\sigma}
\newcommand{\SW}{\Sigma}
\newcommand{\ew}{\lambda}
\newcommand{\EW}{\Lambda}

\newcommand{\Del}{\Delta}
\newcommand{\gD}{\Delta}
\newcommand{\gG}{\Gamma}
\newcommand{\gO}{\Omega}
\newcommand{\gL}{\Lambda}
\newcommand{\gS}{\Sigma}

% Formatting shortcuts
\newcommand{\red}[1]{\textcolor{red}{#1}}
\newcommand{\blue}[1]{\textcolor{blue}{#1}}
\newcommand{\htext}[2]{\texorpdfstring{#1}{#2}}

% Statistics
\newcommand{\distr}{\sim}
\newcommand{\stddev}{\sigma}
\newcommand{\covmatrix}{\Sigma}
\newcommand{\mean}{\mu}
\newcommand{\param}{\gt}
\newcommand{\ftr}{\phi}

% General utility
\newcommand{\todo}[1]{\footnote{TODO: #1}}
\newcommand{\exclaim}[1]{{\textbf{\textit{#1}}}}
\newcommand{\tbc}{[\textbf{Incomplete}]}
\newcommand{\chk}{[\textbf{Check}]}
\newcommand{\oprob}{[\textbf{OP}]:}
\newcommand{\core}[1]{\textbf{Core Idea:}}
\newcommand{\why}{[\textbf{Find proof}]}
\newcommand{\opt}[1]{\textit{#1}}


\DeclareMathOperator*{\argmin}{arg\,min}
\DeclareMathOperator{\rank}{rank}
\newcommand{\redcol}[1]{\textcolor{red}{#1}}
\newcommand{\bluecol}[1]{\textcolor{blue}{#1}}
\newcommand{\greencol}[1]{\textcolor{green}{#1}}


\renewcommand{\~}{\htext{$\sim$}{~}}


% groupings of objects.
\newcommand{\set}[1]{\left\{ #1 \right\}}
\newcommand{\seq}[1]{\left(#1\right)}
\newcommand{\ang}[1]{\langle#1\rangle}
\newcommand{\tuple}[1]{\left(#1\right)}

% numerical shortcuts.
\newcommand{\abs}[1]{\left| #1\right|}
\newcommand{\floor}[1]{\left\lfloor #1 \right\rfloor}
\newcommand{\ceil}[1]{\left\lceil #1 \right\rceil}

% linear algebra shortcuts.
\newcommand{\change}{\Delta}
\newcommand{\norm}[1]{\left\| #1\right\|}
\newcommand{\dprod}[1]{\langle#1\rangle}
\newcommand{\linspan}[1]{\langle#1\rangle}
\newcommand{\conj}[1]{\overline{#1}}
\newcommand{\gradient}{\nabla}
\newcommand{\der}{\frac{d}{dx}}
\newcommand{\lap}{\Delta}
\newcommand{\kron}{\otimes}
\newcommand{\nperp}{\nvdash}

\newcommand{\mat}[1]{\left( \begin{smallmatrix}#1 \end{smallmatrix} \right)}

% derivatives and limits
\newcommand{\partder}[2]{\frac{\partial #1}{\partial #2}}
\newcommand{\partdern}[3]{\frac{\partial^{#3} #1}{\partial #2^{#3}}}

% Arrows
\newcommand{\diverge}{\nearrow}
\newcommand{\notto}{\nrightarrow}
\newcommand{\up}{\uparrow}
\newcommand{\down}{\downarrow}
% gets and gives are defined!

% ordering operators
\newcommand{\oleq}{\preceq}
\newcommand{\ogeq}{\succeq}

% programming and logic operators
\newcommand{\dfn}{:=}
\newcommand{\assign}{:=}
\newcommand{\co}{\ co\ }
\newcommand{\en}{\ en\ }


% logic operators
\newcommand{\xor}{\oplus}
\newcommand{\Land}{\bigwedge}
\newcommand{\Lor}{\bigvee}
\newcommand{\finish}{$\Box$}
\newcommand{\contra}{\Rightarrow \Leftarrow}
\newcommand{\iseq}{\stackrel{_?}{=}}


% Set theory
\newcommand{\symdiff}{\Delta}
\newcommand{\union}{\cup}
\newcommand{\inters}{\cap}
\newcommand{\Union}{\bigcup}
\newcommand{\Inters}{\bigcap}
\newcommand{\nullSet}{\phi}

% graph theory
\newcommand{\nbd}{\Gamma}

% Script alphabets
% For reals, use \Re

% greek letters
\newcommand{\eps}{\epsilon}
\newcommand{\del}{\delta}
\newcommand{\ga}{\alpha}
\newcommand{\gb}{\beta}
\newcommand{\gd}{\del}
\newcommand{\gf}{\phi}
\newcommand{\gF}{\Phi}
\newcommand{\gl}{\lambda}
\newcommand{\gm}{\mu}
\newcommand{\gn}{\nu}
\newcommand{\gr}{\rho}
\newcommand{\gs}{\sigma}
\newcommand{\gt}{\theta}
\newcommand{\gx}{\xi}

\newcommand{\sw}{\sigma}
\newcommand{\SW}{\Sigma}
\newcommand{\ew}{\lambda}
\newcommand{\EW}{\Lambda}

\newcommand{\Del}{\Delta}
\newcommand{\gD}{\Delta}
\newcommand{\gG}{\Gamma}
\newcommand{\gO}{\Omega}
\newcommand{\gL}{\Lambda}
\newcommand{\gS}{\Sigma}

% Formatting shortcuts
\newcommand{\red}[1]{\textcolor{red}{#1}}
\newcommand{\blue}[1]{\textcolor{blue}{#1}}
\newcommand{\htext}[2]{\texorpdfstring{#1}{#2}}

% Statistics
\newcommand{\distr}{\sim}
\newcommand{\stddev}{\sigma}
\newcommand{\covmatrix}{\Sigma}
\newcommand{\mean}{\mu}
\newcommand{\param}{\gt}
\newcommand{\ftr}{\phi}

% General utility
\newcommand{\todo}[1]{\footnote{TODO: #1}}
\newcommand{\exclaim}[1]{{\textbf{\textit{#1}}}}
\newcommand{\tbc}{[\textbf{Incomplete}]}
\newcommand{\chk}{[\textbf{Check}]}
\newcommand{\oprob}{[\textbf{OP}]:}
\newcommand{\core}[1]{\textbf{Core Idea:}}
\newcommand{\why}{[\textbf{Find proof}]}
\newcommand{\opt}[1]{\textit{#1}}


\DeclareMathOperator*{\argmin}{arg\,min}
\DeclareMathOperator{\rank}{rank}
\newcommand{\redcol}[1]{\textcolor{red}{#1}}
\newcommand{\bluecol}[1]{\textcolor{blue}{#1}}
\newcommand{\greencol}[1]{\textcolor{green}{#1}}


\renewcommand{\~}{\htext{$\sim$}{~}}


%opening
\title{Geometry and Topology: Quick reference}
\author{vishvAs vAsuki}

\begin{document}
\maketitle

\part{Notation}


\part{Themes}
Study of properties that describe how a space is assembled, such as connectedness and orientability.

Topology: Studies properties that are preserved through deformations, twistings, and stretchings of objects. Tearing not allowed.

\part{Euclidean and coordinate geometry}
\chapter{Euclidean geometry}
See \cite{hallStevens}.
Deals with $R^{k}$, the Euclidian $k$ space, which is described in the vector spaces survey.

Considers metric properties such as distances between points. Perpendiculars, parallels, projections.

Relationship between angles at the intersection of a line with parallels or in a triangle (and other polygons). Similarity and congruence of triangles (and other polygons). Angle bisectors, Medians in a triangle intersect at incenter, circumcenter.

Circle; lines, angles, triangles, chords, arcs in circles. Tangents, Intersection of circles. Ellipsoids, spheres.

Polytopes: an object in $R^n$ whose boundary surfaces are flat. Convex polytopes: polytopes which are also convex sets.

\section{Area/ volume}
The notion of area/ volume in case of euclidean spaces corresponds to the box (Lebesgue) measure over the euclidean space. This is described in the vector spaces survey.

Ellipse, circle. Surface area of n-ball: $\frac{dV_{n}}{dr}$.

The r-radius $n-1$ hypersphere $S^{n-1} = \set{x \in R^{n}: \norm{x} = r}$: a n-1 dim manifold. Encloses a n-ball with volume $V_{n} = \frac{\pi^{\frac{n}{2}}r^{n}}{\Gamma(\frac{n}{2}+1)}$ \why.

\section{Trigonometry}
Obvious with the right construction: sin (A+B) = sin A cos B + cos A sin B. cos (A+B) = cos A cos B - sin A sin B. Trigonometry in triangle calculations. $\sin^{-1} x, \cos^{-1} x$.

\chapter{Coordinate geometry}
\section{Vector spaces}
(See linear algebra ref). Vector and scalar product of 2 vectors; effect of Left vs right handedness of coordinate system.

\section{High dimensional objects}
Get vector equations from geometric properties. Use linear transformations like scaling, rotation, projection to describe effects.

\subsection{Hyperplane}
Hyperplane $\perp w$ through 0 : x such that $w^{T}x = 0$; shift c from 0: $w^{T}(x-c) = 0$. For halfspace, replace '=' with $\leq$.

\subsection{Polyhedron}
$\set{x: Ax \leq b}$: The intersection of halfspaces.

\subsection{Simplex/ hypertriangle}
n-d triagle. Construct from (d+1) hyperplanes with linearly independent w's.

\subsection{Hypersphere surface}
$x^{*}x = r$. Sift from 0: $(x-c)^{*}(x-c) = x^{*}x - 2c^{*}x = r'$.

\subsection{Hyper-ellipse surface E}
\subsubsection{Aligned with std basis}
\paragraph*{Skewed norm-ball form}
E aligned with the standard basis: diagonal $\SW \succeq 0$. $\set{x: x^{*}\SW x = r^{2}}$ is hyper-ellipse aligned with the axes, skewed as per $\SW$. After rescaling: $\set{x: x^{*}\SW x = 1}$.

\paragraph*{Matrix image form}
Take $\SW^{1/2} x = y$, assume $\SW \succ 0$. This is $\equiv$ $E = \set{M y: \norm{y} = 1}$, where $\SW^{-1/2} = M \succ 0$.

\paragraph*{Radii along major axes}
$\set{\sw_i^{-1/2}e_i} \subset E$. So, radii are:\\ $\set{\sw_i^{-1/2}} = \set{\sw_i(M)}$.

\subsubsection{Aligned with arbitrary basis}
Rotate previous ellipse. Take orthogonal rotator U and apply it to previous ellipses (do $y = U^{*}x$): major axes of E will then be aligned with U's columns.

Radii along major axes remains the same.

\paragraph*{Rotated Skewed norm-ball form}
$\set{x: x^{*}U\SW U^{*}x = 1}$.

\paragraph*{Matrix image form}
Take: $E = \set{M' y: \norm{y} = 1}$, rotate to get: $M = M' U^{*} \succ 0$. $E = \set{M y: \norm{y} = 1}$

\subsubsection{Shifted away from 0}
Just use y = x-c.

Radii along major axes remains the same.

\paragraph*{Shifted Rotated Skewed norm-ball form}
$\set{x: (x-c)^{*}U\SW U^{*}(x-c) = 1}$.

Using unscaled $M'=U\SW U^{*}$, the equation is : $(x-c)^{*}M'(x-c) - r^{2} = x^{*}M'x - r^{2} - 2x^{*}M'c = 0$.

\paragraph*{Shifted Matrix image form}
$\set{c + My | \norm{y} = 1}$ for $M \succ 0$. This can be reparametrized as: $\set{x | \norm{M^{-1}x - M^{-1}c} = 1} = \set{x: \norm{Ax + b } = 1}$.

\subsubsection{Volume}
Take the general expression for E: $\set{c + My | \norm{y} = 1}$. $vol(E) \propto \prod r_i$, where $r_i = \sw_i(M)$ are the radii along the major axes of the ellipsoid E. \why

So, $vol(E) \propto det(M^{-1}) = det(A^{-1})$.

For 2-D ellipsoids: $vol(E) = \pi\prod r_i$.

\section{Other coordinate systems}
Cylindrical and spherical coordinates: $x=r \cos \theta$.

\section{Graph drawing}
Find axis meeting points, maxima/ minima, \\
inflection points.

For visualization of functionals over vector spaces, their gradients: see linear algebra ref.

\section{Manifold}
Take any small enough area in a manifold: it resembles a euclidian space of a certain dimension, aka the manifold's dimension. 0 dim manifold: A point. 1 dim manifold: line, arc. 2 dim manifold: sphere surface.

\part{Metric spaces and topologies}
\chapter{Metric space S}
Set with a metric $d: S\times S \to R^{\geq 0}$. Metric obeys non negativity, positive definiteness, symmetry, $\triangle$ inequality. Eg: Euclidian k space: $R^{k}$: every point is a vector.

Absolute value: $|x| = d(x,0)$ for some 0.

\section{Open ball around p of radius r}
Aka r- neighborhood (nbd) of p: $N_{r}(p) = \set{x \in S: d(x,p) < r}$. Similarly define r-nbd of set of points S. A uniform nbd of S contains some r-nbd of S.

Set of open balls defines a topological space: topology from nbds. Similar topologies for vector spaces, manifolds.

Open ball in $R^{k}$ is convex.

\subsection{Interior point p of S}
If $\exists r: N_{r}(p) \subset S$. $(0,0) \in N_{2}(1,1) \subset R^{\geq 0} \times R^{\geq 0}$ is interior pt. All others are boundary points. Thence defined interior of S: int(S), and boundary of S: bd(S) = cl(S) - int(S). If S has a non-empty interior, it is \textbf{solid!}

[0, 1] has an interior wrt R, but not wrt $R^{2}$: then every pt is in boundary.

\subsection{Limit point p of set S}
$\forall r: N_{r}(p)$ contains a pt in S other than itself.

$\forall r : |N_{r}(p)| = \infty$: Else, can find small $r'$ with $N_{r'}(p) = \set{p}$. So, a finite set has no limit points.

p is the limit of some Cauchy sequence: Keep reducing r and pick $q \neq p \in N_{r}(p)$ in each step.

Every interior pt is a limit pt, but not vice versa. For $E\subset R$, sup(E) is a limit pt.

If p is a lt pt of E, $\exists$ convergent seq $(s_{i})$ in S with $s_{i} \to p$. Set with 1 limit pt: A convergent sequence in R.

\subsection{Closure of E}
cl(E): E with all its limit pts. Also: cl(E) = S - int(S - E).

\subsection{Diameter of E}
$diam(E) = \sup_{p, q} d(p, q)$. diam(E) = diam(cl(E)): by $\contra$, using triangle inequality.

\section{Sets in S: Topology}
\subsection{Nature of the boundary}
\subsubsection{Open set S}
Aka Open space. For every $p \in S$ is an interior point. Diagramatic representation: [] and () in R, dotted an undotted lines in $R^{2}$. Eg: dotted dumbbell in $R^{2}$. 

Open sets $S_{i}$: $\union S_{i}$ is open. $S = \inters_{i=1}^{k} S_{i}$ is open: for any $p\in S$, pick r small enough to ensure $\forall i: N_{r}(p) \in S_{i}$.

If $S \subset Y \subset X$: S open wrt $Y$ iff $\exists G \subset X$, G open wrt $X$ and $S = G \inters Y$: \pf if G open wrt Y, $G\inters Y$ open wrt Y; If S open wrt Y, take $\union_{p \in S} N_{r}^{X}(p)$ where r is radius which shows interiorness of p in S.

\subsubsection{Closed set S}
Set with all its limit points. So, finite sets closed. $[n, \infty )$ closed. Same as S with all its boundary points.

$S\subset X$ closed iff $S'$ open (good trick to show closedness). $\inters$ of closed sets $S_{i}$ is closed: $\union S_{i}'$ is open. Similarly, $\inters_{i=1}^{k} S_{i}$ is closed.

cl(E) is closed: as $(cl(E))'$ is open.

\subsubsection{Non-oppositeness of Openness and Closedness}
Eg: $\phi$ and R are both open and closed. (0, 1) open wrt $R$ but nor wrt $R^{2}$. Half open intervals in R are neither open nor closed.

\subsubsection{Boundedness of set S}
A is bounded if $\exists r, p: A \subset N_{r}(p)$.

\subsection{Compactness}
\subsubsection{Open cover of S}
Bounded Open sets $\set{G_{i}}$ with $\union G_{i} \supset S$. Subcovers: Subsets of open cover which also cover S.

\subsubsection{Definition}
Every open cover of S has a finite sub cover. In $R^{d}$, compactness $\equiv$ closed and bounded.

\subsubsection{Properties}
Finite S is compact. R is not compact: Take $G_{n} = (n-\frac{2}{3}, n+\frac{2}{3})$, $\set{G_{n \in Z}}$ is an open cover, but no finite or even proper subcover. Similarly, $[n, \infty]$ closed but not compact.

Any compact set S is closed: Any $p \in S'$ is interior pt in $S'$: $\union_{q \in S} N_{r}(q): r = \frac{d(p-q)}{2}$ is an open cover of S, within it is some finite subcover; so $\exists N_{r'}(p) \subset S'$.

Closed subset E of compact set S is compact: Take any open cover of E; add open set E' to it to get open cover of S; some finite subset of this without E' is also open cover of E.

Finite union of compact sets is compact.

If F closed and K compact, $F \inters K \subset K$ compact: $F \inters K$ is closed.

If $\set{K_{i}}$ is (possibly $\infty$) set of compact sets and if $\inters$ of every finite subclass $\neq \nullSet$, $\inters K_{i} \neq \nullSet$. Assume $\inters K_{i}=\nullSet$; Take $K_{1}$; every $p \in K_{1}$ is $\notin \inters_{i \neq 1} K_{i}$; so $p \in \union_{i \neq 1} K_{i}'$; so finite subset of $\set{K_{i}'}$ is an open cover of $K_{1}$; so some finite $\inters$ of $\set{K_{i}}$ is $\nullSet$: contradiciton.

So, if $\set{K_{i}}$ compact, $K_{n} \supset K_{n+1}$: $\inters_{i} K_{i} \neq \nullSet$. Does not hold for open sets: Take $G_{n} = (0, n^{-1})$.

If E is an $\infty$ subset of compact set K, E has a limit pt in K: Else every $p \in K$ would have some $N_{r}(p) = \set{p}$; $\union N_{r}(p)$ is an open cover of E without a finite subcover. Also, if every $E \subset K, |E| = \infty $ has a lt pt in K, K is compact. \why

If $\set{K_{i}}$ compact, $K_{n} \supset K_{n+1} \neq \nullSet$, $\lim_{n \to \infty} diam(K_{n}) = 0$, then $\inters K_{n}$ is 1 pt: else $\contra$.

\subsection{Connectedness and completeness}
\subsubsection{Connectedness}
A, B separated if $A \inters cl(B) = cl(B) \inters A = \nullSet$. Eg: (0, 1) and (1, 2) but not (0,1] and (1, 2). S is connected if it is not $\union$ of separated sets.

$E \subset R$ connected iff it is an interval.

\subsubsection{Dense set}
Contains points in the neighborhood of every point.

\subsubsection{Completeness of S}
Limit of every Cauchy sequence $(s_{n})$ wrt metric = some point $s \in S$.

Any closed set in complete metric space S is complete. Also, any compact space is complete.

\subsection{Sigma algebra of open sets}
Aka Borel Sigma algebra. This is the sigma algebra $(X, \bS)$ formed by the closure wrt $\union, \inters, \bar{X}$ of all open sets in $X$. All sets in $\bS$ are called Borel sets.

\section{Covering and packing Number}
Let the space have norm $\norm{}$, and let $C$ be a set in it.

\subsection{Covering number \htext{$N(\eps, C, \norm{})$}{..}}
$\eps$ covering $F_\eps$: Set of $\eps$ balls which contains $C$. Covering number $N(\eps, C, \norm{}) = \min |F_\eps|$.

\subsubsection{Covering entropy}
Aka metric entropy. $\log (N(\eps, C, \norm{}))$.

\subsubsection{Total boundedness}
If $N(\eps, C, \norm{})$ is finite for all $\eps$, $C$ is totally bounded. Else, $C$ is non totally bounded: for every $n$, there is some $\eps: N(\eps, C, \norm{}) > n$.

\subsubsection{For D dim sphere}
$\frac{Vol(sphere(r_1))}{Vol(sphere(r_2))} = (\frac{r_1}{r_2})^D$. Let $vol(B(f', \eps)) = k \eps^{D}$. Then, \\
$k(R+ \eps)^{N} \geq N(\eps, C, \norm{})k \eps^{D} \geq kR^{D}$. Thence, $\log (N(\eps, C, \norm{})) \approx D \log(\frac{R}{\eps})$.

\subsection{Packing number \htext{$M(\eps, C, \norm{})$}{..}}
$\eps$ packing is a set of points $\set{g_i}$ with $g_i\in C; \norm{g_i - g_j} \geq \eps$. The maximal $\eps$ packing: packing number.

\subsubsection{Relationship with N}
$M(2\eps, C, \norm{}) \leq N(\eps, C, \norm{}) \leq M(\eps, C, \norm{})$. 2nd ineq: For maximal packing $\set{g_i}$, $\forall h \in $C$: \norm{g_i-h} \leq \eps$. 1st ineq: For maximal $2 \eps$ packing: Any $\eps$ ball has $\leq 1$ $g_i$.

\subsubsection{Use}
Often easier to find than covering number; thence can bound covering number.

\section{\htext{$R^{k}$}{..}: Topological properties }
See complex analysis ref.

\section{Sequence \htext{$(s_{n})$}{..} in S}
For properties of sequences in fields and vector spaces, see complex analysis and linear algebra ref.

\subsection{Cauchy sequence}
After some point, elements get closer as sequence progresses: contraction or Cauchy criterion: $\forall m, n> N: d(p_{m}, p_{n}) < \eps$ or diameter of tail of seq tends to 0. Limit of sequence may not exist in S. Like convergence without needing a limit.

Any cauchy seq S in compact set $X$ converges: As $X$ compact, S has limit pt in X, also limit of S is unique.

\subsection{Bounded sequences}
Range is bounded.

\subsection{Convergent sequence}
Convergence to limit c: $\forall i>N: d(x_{i}, c) < \eps: x_{n} \to c$. Divergence. Limit is unique. If $x_{n} \to c$, every $N_{r}(c)$ has all but finitely many $x_{i}$.

Any convergent sequence is bounded. $1^{n}$ convergent but has finite range. If range not 1, it is $\infty$.

All convergent sequences are cauchy sequences.

Every subsequence of a convergent sequence converges to the same limit. If every subsequence of a sequence converges to the same limit, the sequence is convergent.

Sequence $(s_{n})$ in compact S has convergent subsequence: If S compact, every $\infty$ subset has limit pt p; make seq out of $s_{i}$ in decreasing $N_{r}(p)$.

\subsection{Subsequential limits}
Take seq $s_{n}$, subsequential limits form closed set E: Take any limit pt p of E, can find subseq limit e close to it, so can find $s_{n}$ close to it; so p is in E.

\section{Function across metric spaces: f:X to Y}
See algebra ref for general properties of functions. Also ref on analysis of functions over R and C.

\subsection{Limit of f}
$\lim_{x\to p}f(x) = q: \forall \eps, \exists \del: 0< d(x,p) < \del \implies d(f(x), q) < \eps$: f has a limit at p. q is unique. Visualize as balls in X, f(X).

$\forall (p_n), p_n \to p, f(p_n) \to q \equiv lt_{x \to p} f(x) = q$: show $\implies$ by $\contra$. So, can use properties of sequences. So, get $\lim f+g, f(x)g(x), f/g$.

\subsection{Continuity of f:X to Y}
f continuous at $p \in E$ if $\forall \eps \exists \del: d(x,p)<\del \implies d(f(x), f(p))< \eps$. If f has limit at p, continuity iff $\lim_{x \to p} f(x) = f(p)$: f defined only over p has no limit at p but is continuous. Continuity over $E \subseteq X$.

If f continuous at p, g continuous at f(p), then f(g(x)) continuous at p.

f continuous over $X$ iff $\forall$ open $V \subseteq Y$, $f^{-1}(V)$ open in X: Visualize interior pts, match $\del$ balls in $X$ with $\eps$ balls in Y.

If f continuous, $X$ compact, then f(X) compact: Take open cover $\set{V_{i}}$ of f(X); $\set{f^{-1}(V_{i})}$ is open, covers X; so take finite subcover; get \\
$f(X) \subseteq \union_{i=1}^{k} f(f^{-1}(V_{i})) \subseteq \union_{i=1}^{k} V_{i} $.

If f continuous, bijection, then $f^{-1}$ is cont: f(V) open iff V is open.

If f continuous, $E \subseteq X$ connected, then f(E) connected: else if f(E) separated into A, B but $f^{-1}(A) \union f^{-1}(B)$ not separated,  $cl(f^{-1}(A)) \inters f^{-1}(B) \neq \nullSet$ or $cl(f^{-1}(B)) \inters f^{-1}(A) \neq \nullSet$; then continuity of f violated, so $\contra$.

\subsection{Uniform continuity over X}
$\forall p, q \in X \forall \eps>0, \exists \del: \\
d_{x}(p,q) < \del \implies d_{y}(f(p), f(q)) < \eps$. $1/x$ continuous, but not uniformly cont over $R$: consider points near $0$; neither is $x^{2}$. A measure of whether gradient gets very big.

If f continuous, $X$ compact, then f uniformly cont: As $Y$ compact: Given $\eps$, take $\forall p \in X: g(p)$, radius which guarantees $\eps/2$ closedness to $f(p)$; make open cover $\set{N_{g(p)}}$; get finite subcover; take max $g(p)$; use $\triangle$ ineq to guarantee $\eps$ closedness anywhere.

Also see the more powerful notion of absolute continuity in the complex analysis survey.

\subsection{Bounding steepness}
Aka Lipschitz continuity/ smoothness. Lipschitz condition: $d(f(x), f(y)) \leq L d(x, y)$. $L$ is lipshcitz constant. Note that it implies the usual notion of continuity.

But, it does not imply differentiability! When differentiable, there is a relationship with the derivative, see complex analysis ref.

\subsubsection{A generalization}
Holder continuity: Holder condition of order a: $d(f(x), f(y)) \leq L d(x, y)^{a}$.

\section{Sequence of functions \htext{$(f_{n}: $X$ \to Y)$}{..}}
Consider the properties of sequence of functions from any set to a metric space, which is described in the survey on basic mathematical structures.

If $x$ is a limit pt of $E \subseteq X$, $lt_{t \to x}f_{n}(t) = A_{n}$, then $A_{n}$ converges, $lt_{t \to x} f(t) = li_{n \to \infty} A_{n}$. Pf: $d(f(t), A) \leq d(f(t), f_{n}(t)) + d(f_{n}(t), A_{n}) + d(A_{n}, A)$: make 1st and 3rd terms small by picking large $N$, make 2nd term small by picking large t.

So, if $(f_{n})$ continuous, f continuous: see $lt_{t \to x}f_{n}(t) = f_{n}(x)$, get $lt_{t \to x} f(t) = lt_{t \to x} f_{n}(x) = f(x)$.

\chapter{Point set topology}
\section{Motivation}
Coffee cup and donut are geometrically different, but topologically same: isotopes! Can deform one to the other. Generalize notions of convergence, connectedness, continuity.

\section{The topological space}
Set of points or Topological space X. Topology T: Class of some sets of points closed under $\union, \inters$.

Sets $S_{i}$ in T are said to be open. $S_{i}'$ are closed sets. Neighborhood of p is a set $V \supset$ open set $U \ni p$. Similarly define nbd of set S of points. $A \in X$ is dense if any nbd has some $a\in A$.

Spanning set (of sets); its linear span. Basis of topology.

\section{Topological Morphisms}
Every 'object' in $Y$ is a continuous function $f: $X$ \to Y$, where $X$ and $Y$ are topological spaces. A tea-cup is a function to $R^{3}$.

\subsection{Homotopy}
Take 2 objects/ cont functions $f, g: $X$ \to Y$. Homotopy is continuous function $H:X \times [0, 1] \to Y$, with $H(x, 0) = f; H(x, 1) = g$. Think of second parameter as time, and H as a continuous deformation.

If $H(x,t)$ is also 1:1, $H$ is an isotopy.

\subsection{Continuous morphism}
f(x) neighborhood corresponds to x neighborhood.

\subsection{Homeomorphism}
A bicontinuous fn: $X \to Y$. Respects topological properties.

\section{Knots}
A circular piece of thread. The simple ring or the unknot. The trefoil. Sketching knots. Strands: segments involved in a cross-over.

\section{The 3 Reidemeister moves}
See wikipedia article for figures. Sufficient and necessary to produce any valid deformation possible from a starting configuration.

\subsection{Knot invariants}
Property invariant to the Reidemesiter moves. 3 colorability of strands: Can assign 3 colors to strands such that all 3 colors are used; at each crossing, 3 or 1 colors seen.

\bibliographystyle{plain}
\bibliography{topology}

\end{document}
