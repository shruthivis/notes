\documentclass[10pt]{amsart}
\usepackage{amssymb, algorithm2e}

% Use something like:
% % Use something like:
% % Use something like:
% \input{../../macros}

% groupings of objects.
\newcommand{\set}[1]{\left\{ #1 \right\}}
\newcommand{\seq}[1]{\left(#1\right)}
\newcommand{\ang}[1]{\langle#1\rangle}
\newcommand{\tuple}[1]{\left(#1\right)}

% numerical shortcuts.
\newcommand{\abs}[1]{\left| #1\right|}
\newcommand{\floor}[1]{\left\lfloor #1 \right\rfloor}
\newcommand{\ceil}[1]{\left\lceil #1 \right\rceil}

% linear algebra shortcuts.
\newcommand{\change}{\Delta}
\newcommand{\norm}[1]{\left\| #1\right\|}
\newcommand{\dprod}[1]{\langle#1\rangle}
\newcommand{\linspan}[1]{\langle#1\rangle}
\newcommand{\conj}[1]{\overline{#1}}
\newcommand{\gradient}{\nabla}
\newcommand{\der}{\frac{d}{dx}}
\newcommand{\lap}{\Delta}
\newcommand{\kron}{\otimes}
\newcommand{\nperp}{\nvdash}

\newcommand{\mat}[1]{\left( \begin{smallmatrix}#1 \end{smallmatrix} \right)}

% derivatives and limits
\newcommand{\partder}[2]{\frac{\partial #1}{\partial #2}}
\newcommand{\partdern}[3]{\frac{\partial^{#3} #1}{\partial #2^{#3}}}

% Arrows
\newcommand{\diverge}{\nearrow}
\newcommand{\notto}{\nrightarrow}
\newcommand{\up}{\uparrow}
\newcommand{\down}{\downarrow}
% gets and gives are defined!

% ordering operators
\newcommand{\oleq}{\preceq}
\newcommand{\ogeq}{\succeq}

% programming and logic operators
\newcommand{\dfn}{:=}
\newcommand{\assign}{:=}
\newcommand{\co}{\ co\ }
\newcommand{\en}{\ en\ }


% logic operators
\newcommand{\xor}{\oplus}
\newcommand{\Land}{\bigwedge}
\newcommand{\Lor}{\bigvee}
\newcommand{\finish}{$\Box$}
\newcommand{\contra}{\Rightarrow \Leftarrow}
\newcommand{\iseq}{\stackrel{_?}{=}}


% Set theory
\newcommand{\symdiff}{\Delta}
\newcommand{\union}{\cup}
\newcommand{\inters}{\cap}
\newcommand{\Union}{\bigcup}
\newcommand{\Inters}{\bigcap}
\newcommand{\nullSet}{\phi}

% graph theory
\newcommand{\nbd}{\Gamma}

% Script alphabets
% For reals, use \Re

% greek letters
\newcommand{\eps}{\epsilon}
\newcommand{\del}{\delta}
\newcommand{\ga}{\alpha}
\newcommand{\gb}{\beta}
\newcommand{\gd}{\del}
\newcommand{\gf}{\phi}
\newcommand{\gF}{\Phi}
\newcommand{\gl}{\lambda}
\newcommand{\gm}{\mu}
\newcommand{\gn}{\nu}
\newcommand{\gr}{\rho}
\newcommand{\gs}{\sigma}
\newcommand{\gt}{\theta}
\newcommand{\gx}{\xi}

\newcommand{\sw}{\sigma}
\newcommand{\SW}{\Sigma}
\newcommand{\ew}{\lambda}
\newcommand{\EW}{\Lambda}

\newcommand{\Del}{\Delta}
\newcommand{\gD}{\Delta}
\newcommand{\gG}{\Gamma}
\newcommand{\gO}{\Omega}
\newcommand{\gL}{\Lambda}
\newcommand{\gS}{\Sigma}

% Formatting shortcuts
\newcommand{\red}[1]{\textcolor{red}{#1}}
\newcommand{\blue}[1]{\textcolor{blue}{#1}}
\newcommand{\htext}[2]{\texorpdfstring{#1}{#2}}

% Statistics
\newcommand{\distr}{\sim}
\newcommand{\stddev}{\sigma}
\newcommand{\covmatrix}{\Sigma}
\newcommand{\mean}{\mu}
\newcommand{\param}{\gt}
\newcommand{\ftr}{\phi}

% General utility
\newcommand{\todo}[1]{\footnote{TODO: #1}}
\newcommand{\exclaim}[1]{{\textbf{\textit{#1}}}}
\newcommand{\tbc}{[\textbf{Incomplete}]}
\newcommand{\chk}{[\textbf{Check}]}
\newcommand{\oprob}{[\textbf{OP}]:}
\newcommand{\core}[1]{\textbf{Core Idea:}}
\newcommand{\why}{[\textbf{Find proof}]}
\newcommand{\opt}[1]{\textit{#1}}


\DeclareMathOperator*{\argmin}{arg\,min}
\DeclareMathOperator{\rank}{rank}
\newcommand{\redcol}[1]{\textcolor{red}{#1}}
\newcommand{\bluecol}[1]{\textcolor{blue}{#1}}
\newcommand{\greencol}[1]{\textcolor{green}{#1}}


\renewcommand{\~}{\htext{$\sim$}{~}}


% groupings of objects.
\newcommand{\set}[1]{\left\{ #1 \right\}}
\newcommand{\seq}[1]{\left(#1\right)}
\newcommand{\ang}[1]{\langle#1\rangle}
\newcommand{\tuple}[1]{\left(#1\right)}

% numerical shortcuts.
\newcommand{\abs}[1]{\left| #1\right|}
\newcommand{\floor}[1]{\left\lfloor #1 \right\rfloor}
\newcommand{\ceil}[1]{\left\lceil #1 \right\rceil}

% linear algebra shortcuts.
\newcommand{\change}{\Delta}
\newcommand{\norm}[1]{\left\| #1\right\|}
\newcommand{\dprod}[1]{\langle#1\rangle}
\newcommand{\linspan}[1]{\langle#1\rangle}
\newcommand{\conj}[1]{\overline{#1}}
\newcommand{\gradient}{\nabla}
\newcommand{\der}{\frac{d}{dx}}
\newcommand{\lap}{\Delta}
\newcommand{\kron}{\otimes}
\newcommand{\nperp}{\nvdash}

\newcommand{\mat}[1]{\left( \begin{smallmatrix}#1 \end{smallmatrix} \right)}

% derivatives and limits
\newcommand{\partder}[2]{\frac{\partial #1}{\partial #2}}
\newcommand{\partdern}[3]{\frac{\partial^{#3} #1}{\partial #2^{#3}}}

% Arrows
\newcommand{\diverge}{\nearrow}
\newcommand{\notto}{\nrightarrow}
\newcommand{\up}{\uparrow}
\newcommand{\down}{\downarrow}
% gets and gives are defined!

% ordering operators
\newcommand{\oleq}{\preceq}
\newcommand{\ogeq}{\succeq}

% programming and logic operators
\newcommand{\dfn}{:=}
\newcommand{\assign}{:=}
\newcommand{\co}{\ co\ }
\newcommand{\en}{\ en\ }


% logic operators
\newcommand{\xor}{\oplus}
\newcommand{\Land}{\bigwedge}
\newcommand{\Lor}{\bigvee}
\newcommand{\finish}{$\Box$}
\newcommand{\contra}{\Rightarrow \Leftarrow}
\newcommand{\iseq}{\stackrel{_?}{=}}


% Set theory
\newcommand{\symdiff}{\Delta}
\newcommand{\union}{\cup}
\newcommand{\inters}{\cap}
\newcommand{\Union}{\bigcup}
\newcommand{\Inters}{\bigcap}
\newcommand{\nullSet}{\phi}

% graph theory
\newcommand{\nbd}{\Gamma}

% Script alphabets
% For reals, use \Re

% greek letters
\newcommand{\eps}{\epsilon}
\newcommand{\del}{\delta}
\newcommand{\ga}{\alpha}
\newcommand{\gb}{\beta}
\newcommand{\gd}{\del}
\newcommand{\gf}{\phi}
\newcommand{\gF}{\Phi}
\newcommand{\gl}{\lambda}
\newcommand{\gm}{\mu}
\newcommand{\gn}{\nu}
\newcommand{\gr}{\rho}
\newcommand{\gs}{\sigma}
\newcommand{\gt}{\theta}
\newcommand{\gx}{\xi}

\newcommand{\sw}{\sigma}
\newcommand{\SW}{\Sigma}
\newcommand{\ew}{\lambda}
\newcommand{\EW}{\Lambda}

\newcommand{\Del}{\Delta}
\newcommand{\gD}{\Delta}
\newcommand{\gG}{\Gamma}
\newcommand{\gO}{\Omega}
\newcommand{\gL}{\Lambda}
\newcommand{\gS}{\Sigma}

% Formatting shortcuts
\newcommand{\red}[1]{\textcolor{red}{#1}}
\newcommand{\blue}[1]{\textcolor{blue}{#1}}
\newcommand{\htext}[2]{\texorpdfstring{#1}{#2}}

% Statistics
\newcommand{\distr}{\sim}
\newcommand{\stddev}{\sigma}
\newcommand{\covmatrix}{\Sigma}
\newcommand{\mean}{\mu}
\newcommand{\param}{\gt}
\newcommand{\ftr}{\phi}

% General utility
\newcommand{\todo}[1]{\footnote{TODO: #1}}
\newcommand{\exclaim}[1]{{\textbf{\textit{#1}}}}
\newcommand{\tbc}{[\textbf{Incomplete}]}
\newcommand{\chk}{[\textbf{Check}]}
\newcommand{\oprob}{[\textbf{OP}]:}
\newcommand{\core}[1]{\textbf{Core Idea:}}
\newcommand{\why}{[\textbf{Find proof}]}
\newcommand{\opt}[1]{\textit{#1}}


\DeclareMathOperator*{\argmin}{arg\,min}
\DeclareMathOperator{\rank}{rank}
\newcommand{\redcol}[1]{\textcolor{red}{#1}}
\newcommand{\bluecol}[1]{\textcolor{blue}{#1}}
\newcommand{\greencol}[1]{\textcolor{green}{#1}}


\renewcommand{\~}{\htext{$\sim$}{~}}


% groupings of objects.
\newcommand{\set}[1]{\left\{ #1 \right\}}
\newcommand{\seq}[1]{\left(#1\right)}
\newcommand{\ang}[1]{\langle#1\rangle}
\newcommand{\tuple}[1]{\left(#1\right)}

% numerical shortcuts.
\newcommand{\abs}[1]{\left| #1\right|}
\newcommand{\floor}[1]{\left\lfloor #1 \right\rfloor}
\newcommand{\ceil}[1]{\left\lceil #1 \right\rceil}

% linear algebra shortcuts.
\newcommand{\change}{\Delta}
\newcommand{\norm}[1]{\left\| #1\right\|}
\newcommand{\dprod}[1]{\langle#1\rangle}
\newcommand{\linspan}[1]{\langle#1\rangle}
\newcommand{\conj}[1]{\overline{#1}}
\newcommand{\gradient}{\nabla}
\newcommand{\der}{\frac{d}{dx}}
\newcommand{\lap}{\Delta}
\newcommand{\kron}{\otimes}
\newcommand{\nperp}{\nvdash}

\newcommand{\mat}[1]{\left( \begin{smallmatrix}#1 \end{smallmatrix} \right)}

% derivatives and limits
\newcommand{\partder}[2]{\frac{\partial #1}{\partial #2}}
\newcommand{\partdern}[3]{\frac{\partial^{#3} #1}{\partial #2^{#3}}}

% Arrows
\newcommand{\diverge}{\nearrow}
\newcommand{\notto}{\nrightarrow}
\newcommand{\up}{\uparrow}
\newcommand{\down}{\downarrow}
% gets and gives are defined!

% ordering operators
\newcommand{\oleq}{\preceq}
\newcommand{\ogeq}{\succeq}

% programming and logic operators
\newcommand{\dfn}{:=}
\newcommand{\assign}{:=}
\newcommand{\co}{\ co\ }
\newcommand{\en}{\ en\ }


% logic operators
\newcommand{\xor}{\oplus}
\newcommand{\Land}{\bigwedge}
\newcommand{\Lor}{\bigvee}
\newcommand{\finish}{$\Box$}
\newcommand{\contra}{\Rightarrow \Leftarrow}
\newcommand{\iseq}{\stackrel{_?}{=}}


% Set theory
\newcommand{\symdiff}{\Delta}
\newcommand{\union}{\cup}
\newcommand{\inters}{\cap}
\newcommand{\Union}{\bigcup}
\newcommand{\Inters}{\bigcap}
\newcommand{\nullSet}{\phi}

% graph theory
\newcommand{\nbd}{\Gamma}

% Script alphabets
% For reals, use \Re

% greek letters
\newcommand{\eps}{\epsilon}
\newcommand{\del}{\delta}
\newcommand{\ga}{\alpha}
\newcommand{\gb}{\beta}
\newcommand{\gd}{\del}
\newcommand{\gf}{\phi}
\newcommand{\gF}{\Phi}
\newcommand{\gl}{\lambda}
\newcommand{\gm}{\mu}
\newcommand{\gn}{\nu}
\newcommand{\gr}{\rho}
\newcommand{\gs}{\sigma}
\newcommand{\gt}{\theta}
\newcommand{\gx}{\xi}

\newcommand{\sw}{\sigma}
\newcommand{\SW}{\Sigma}
\newcommand{\ew}{\lambda}
\newcommand{\EW}{\Lambda}

\newcommand{\Del}{\Delta}
\newcommand{\gD}{\Delta}
\newcommand{\gG}{\Gamma}
\newcommand{\gO}{\Omega}
\newcommand{\gL}{\Lambda}
\newcommand{\gS}{\Sigma}

% Formatting shortcuts
\newcommand{\red}[1]{\textcolor{red}{#1}}
\newcommand{\blue}[1]{\textcolor{blue}{#1}}
\newcommand{\htext}[2]{\texorpdfstring{#1}{#2}}

% Statistics
\newcommand{\distr}{\sim}
\newcommand{\stddev}{\sigma}
\newcommand{\covmatrix}{\Sigma}
\newcommand{\mean}{\mu}
\newcommand{\param}{\gt}
\newcommand{\ftr}{\phi}

% General utility
\newcommand{\todo}[1]{\footnote{TODO: #1}}
\newcommand{\exclaim}[1]{{\textbf{\textit{#1}}}}
\newcommand{\tbc}{[\textbf{Incomplete}]}
\newcommand{\chk}{[\textbf{Check}]}
\newcommand{\oprob}{[\textbf{OP}]:}
\newcommand{\core}[1]{\textbf{Core Idea:}}
\newcommand{\why}{[\textbf{Find proof}]}
\newcommand{\opt}[1]{\textit{#1}}


\DeclareMathOperator*{\argmin}{arg\,min}
\DeclareMathOperator{\rank}{rank}
\newcommand{\redcol}[1]{\textcolor{red}{#1}}
\newcommand{\bluecol}[1]{\textcolor{blue}{#1}}
\newcommand{\greencol}[1]{\textcolor{green}{#1}}


\renewcommand{\~}{\htext{$\sim$}{~}}


% Use something like:
% % Use something like:
% % Use something like:
% \input{../../amsartMacros}

\newtheorem{thm}{Theorem}[subsection]
\newtheorem{cor}[thm]{Corollary}
\newtheorem{lem}[thm]{Lemma}
\newtheorem{fact}[thm]{Fact}
\newtheorem{claim}[thm]{Claim}

\newtheorem{assumption}[thm]{Assumption}

\newtheorem{defn}[thm]{Definition}

\theoremstyle{remark}
\newtheorem{alg}[thm]{Algorithm}
\newtheorem*{notation}{Notation}
\newtheorem*{rem}{Remark}
\newtheorem*{hint}{Hint}
\newtheorem*{ack}{Acknowledgement}

\newtheorem{question}[thm]{Question}
\newtheorem{answer}[thm]{Answer}


\newtheorem{thm}{Theorem}[subsection]
\newtheorem{cor}[thm]{Corollary}
\newtheorem{lem}[thm]{Lemma}
\newtheorem{fact}[thm]{Fact}
\newtheorem{claim}[thm]{Claim}

\newtheorem{assumption}[thm]{Assumption}

\newtheorem{defn}[thm]{Definition}

\theoremstyle{remark}
\newtheorem{alg}[thm]{Algorithm}
\newtheorem*{notation}{Notation}
\newtheorem*{rem}{Remark}
\newtheorem*{hint}{Hint}
\newtheorem*{ack}{Acknowledgement}

\newtheorem{question}[thm]{Question}
\newtheorem{answer}[thm]{Answer}


\newtheorem{thm}{Theorem}[subsection]
\newtheorem{cor}[thm]{Corollary}
\newtheorem{lem}[thm]{Lemma}
\newtheorem{fact}[thm]{Fact}
\newtheorem{claim}[thm]{Claim}

\newtheorem{assumption}[thm]{Assumption}

\newtheorem{defn}[thm]{Definition}

\theoremstyle{remark}
\newtheorem{alg}[thm]{Algorithm}
\newtheorem*{notation}{Notation}
\newtheorem*{rem}{Remark}
\newtheorem*{hint}{Hint}
\newtheorem*{ack}{Acknowledgement}

\newtheorem{question}[thm]{Question}
\newtheorem{answer}[thm]{Answer}


%opening
\title{Cryptography: Answers to Homework 2}
\author{vishvAs vAsuki}

\begin{document}

\maketitle

\begin{rem}
 No Collaborators unless specifically acknowledged in individual solutions.
\end{rem}


\section{Answer to question 1}
One can build a secure hash and sign signature scheme from any secure IBE system.

Consider IBE system with algorithms: Setup, keygen, encrypt, decrypt.

\subsection{Signature scheme X}
\begin{itemize}
 \item Setup(l): Run $setup_{IBE}(l)$, get PP, MSK. Set verification key, VK = PP. Set signature secret key, SK = MSK.
 \item Sign(m): Signature s = Keygen(H(m), MSK).
 \item Verify(m, s): Pick some other message n. Get c = Encrypt(n, PP, H(m)), then get n' = decrypt(c, s) and return the result of the check: $n \iseq n'$.
\end{itemize}

\subsection{Proof of security}

We will show that, if X can be broken, then the underlying IBE system is not secure against CPA.

\subsubsection{Flaw}
This proof was later found to be flawed. It does not account for the fact that the scheme is not secure when l is small.

\subsubsection{The game}
Let A be the IBE challenger. Let B be the IBE attacker. Let C be the X forger.

A runs $setup_{IBE}(l)$ and generates PP, MSK. It sends PP to B.

B sends VK=PP to C.

Whenever C asks B to encrypt message m, B hashes m and asks A to generate a secret key for H(m). As a signature for m, B sends C the secret key SK returned by A.

Finally, C forges a signature for message m, and sends m, $SK_{H(m)}$ to B.

B tells A that it is attacking the ID=H(m).

B sends A messages $m_{1}$ and $m_{2}$.

A picks a random bit g and responds with $c = Encrypt(m_g, PP, ID)$.

B finds $m_g = decrypt(c, SK_{H(m)})$. By comparing this with $m_{1}$, it identifies the value of g. B then responds to A's challenge by sending it g.

\subsubsection{Analysis of success probability}
Suppose that C succeds with non negligible probability q in breaking scheme X. Whenever C succeeds, B succeeds in identifying g. Whenever C fails, B has no advantage over random guessing in identifying the random bit g. So, the overall success probability of B in this game is q + (1-q)/2 = 1/2 + q/2, which is a non negligible advantage over random guessing.

\section{Answer to question 2}
\begin{ack}
 This problem was solved in collaboration with Rashid Kaleem.
\end{ack}


\subsection{The signature system X}
We are given a weak signature scheme W and a one time signature system O. Consider the signature system defined by the following algorithms:

setup(l):
\begin{itemize}
 \item Run $setup_o(l)$ P(l) number of times, where P(l) is some polynomial.
 \item From this process, get a vector of one time signature key pairs: \\
$((OSK_{i}, OVK_{i}))$.
 \item Run $setup_w(l)$ and get $VK_w, SK_w$.
 \item Sign all $\set{OVK_{i}}$ with $SK_w$ to get the vector $(SOVK_i)$.
 \item Set the verification key of the system to be $VK = VK_w$.
\end{itemize}

sign(m):
\begin{itemize}
 \item Pick an $(OSK_{i}, OVK_{i}, SOVK_i)$ triple which has not been used in previous calls to sign().
 \item Get $s_1 = sign_o(m, OSK_{i})$. Then set $s = s_1, SOVK_i, OVK_{i}$ to be the signature.
\end{itemize}


verify(m, s):
\begin{itemize}
 \item Do the following checks: $verify_w(SOVK_i, OVK_{i}, VK_w) \iseq 1$ \\and $verify_o(s_1, m, OVK_{i}) \iseq 1$.
\end{itemize}

\subsection{Proof of security}
We show that, if a signature under X can be forged, then the signature scheme W is not a weak signature scheme as we assumed.

\subsubsection{Flaw}
This proof was later found to be flawed. It makes too strong an assumption about the nature of the attacker. It is possible that the attacker works by breaking only O, and not W.

\subsubsection{The security game}
Let A be the challenger for the W signature scheme. Let B be the attacker for the W signature scheme. Let C be the attacker for the X signature scheme.

B runs $setup_o(l)$ P(l) number of times, where P(l) is some polynomial which is greater than the number of signature queries ever made by C. Thus, a vector of key pairs under scheme O is generated: $((OSK_{i}, OVK_{i}))$.

B sends all $\set{OVK_{i}}$ to A and asks A to sign it.

A runs $setup_w(l)$ and sends $VK_w$ to B.

B passes this on to C.

A signs $\set{OVK_{i}}$ and returns signatures $(SOVK_i)$ to B.

Whenever C asks B to sign a message m, B responds with a signature $s = s_1, SOVK_i, OVK_{i}$ as described in the sign algorithm of the scheme X.

When C finally forges a message m, it sends B a valid signature of the form: $s = s_1, SOVK, OVK$, where $OVK \notin \set{OVK_{i}}$. As this is a valid signature, SOVK is a valid signature of OVK under the scheme W.

C sends the forgery (SOVK, OVK) to A and successfully answers A's challenge.

\subsubsection{Analysis of success probability}
Suppose that C succeds with non negligible probability p. Whenever C succeeds in breaking X, B succeds in breaking W. Thus, B succeeds with the same probability non negligible probability p. So, W is then broken.

One can similarly show, though this is not required for our purposes, that if you can forge a signature under scheme X, O is not a one-time signature scheme.

\section{Answer to question 3}
\subsection{The signing system X}
Setup: N = pq, where p, q are primes. Pick random $e<\phi(N)$,  and $u, h \in Z_{N}$, where e is relatively prime to $\phi(N)$. The bit-length of the message space is l, which is less than bit lengths of p and q. PK: N, e, u, h. $SK: e^{-1} \mod \phi(N)$.

Sign: $s = (u^{m}h)^{1/e}$.

Verify(M, VK): $s^{e} \iseq u^{m}h$.

\subsection{Forgeability}
We demonstrate forgeability using the following game. Let A be the X-chanllenger. Let B be the X attacker.

A tells B the public key PK.

B asks A to sign $m_{1}$ and thus deduces $u^{m_{1}/e}h^{1/e}$.

B asks A to sign $m_{2}$ and thus deduces $u^{m_{2}/e}h^{1/e}$.

Using these, B deduces $u^{\frac{m_{1}-m_{2}}{e}}$.

B asks A to sign $m_{1}-m_{2}$ and thus deduces $u^{(m_{1}-m_{2})/e}h^{1/e}$.

Using these, B deduces $h^{1/e}$.

B asks A to sign $m_{3}$ and thus deduces $u^{m_{3}/e}h^{1/e}$.

Using these, B deduces $u^{(m_{1}-m_{3})/e}$.

To this, B multiplies $h^{1/e}$ to get a valid signature for $(m_{1}-m_{3})$.

Thus, B has successfully forged a signature.

\section{Answer to question 4}
\subsection{The problem}
Consider two public key encryption schemes, scheme 1 and scheme 2, which are secure against CPA and which use the same message space. We encrypt the same message to user 1 under $PK_{1}$ and to user 2 under $PK_{2}$. We want to show that no attacker can break CPA security in this two scheme scenario.

\subsection{The proof}
We prove that any algorithm which can break CPA security in the two scheme scenario can be used to build an algorithm which breaks CPA security of the two encryption schemes. Below, we show a successful CPA attack against encryption scheme 1 using such an algorithm. An identical strategy can be used to break scheme 2 also.

\subsubsection{Flaw}
This proof was later found to be flawed. It makes too strong an assumption about the nature of the attacker. It is possible that the attacker works by breaking only scheme 2, and not scheme 1; or even by making a random choice between the two.

\subsubsection{The game}
Consider the following security game. Let $A$ be the challenger for encryption scheme 1. Let B be the attacker against encryption scheme 1. Let C be the attacker which is successful in the two scheme scenario.

A runs the setup algorithm for scheme 1.

A sends B $PK_{1}$.

B internally simulates scheme 2 during the game. It runs the setup algorithm for scheme 2, and gets $PK_{2}, SK_{2}$.

B sends C $PK_{1}, PK_{2}$.

C then (perhaps adaptively) asks for a polynomial number of messages $\set{m_{i}}$ to be encrypted.

B passes this on to A and gets cyphertexts $\set{c_{i}}$ under scheme 1. B also uses $SK_2$ and produces cyphertexts $\set{c_{i}'}$ under scheme 2.

B responds to C with $\set{c_{i}}$ and $\set{c_{i}'}$.

When C is ready to attack, it will choose plaintexts $m_{0}, m_{1}$ and send them to B.

B passes it on to A.

A picks a random bit g and returns to B the cyphertext for $m_g$: $c_g$.

B now *guesses* the value g' of the random bit g and creates the cyphertext $c_{g'}$ under encryption scheme 2.

B sends to C: $c_g, c_{g'}$.

C identifies g in case $g = g'$, and returns this to B.

B passes on this guess g to A.

\subsubsection{The analysis of success probability}
Take the case where $g = g'$. The probability that $g = g'$ is 1/2. In this case, suppose that C is successful with a non-negligible advantage t over random guessing. Then, B is successful with the same advantage over random guessing.

Take the case where $g \neq g'$. In this case, there is no guarantee about the success rate of C. So, B has no advantage over random guessing.

The over all success rate of B is: $1/2(1/2 + t) + 1/2(1/2) = 1/2 + t/2$. Thus, B breaks scheme 1 using an attacker which is successful in the 2-scheme scenario.

\section{Answer to question 5}
\tbc


% \bibliographystyle{plain}
% \bibliography{../colt}


\end{document}
