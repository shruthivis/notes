\documentclass[oneside, article]{memoir}
\usepackage{amsmath, amssymb}
\usepackage{hyperref, graphicx, verbatim, listings, multirow, subfigure}
\usepackage{algorithm, algorithmic}
% \usepackage[bottom]{footmisc}
\lstset{breaklines=true}
\setcounter{tocdepth}{3}

% Lets verbatim and verb environments automatically break lines.
\makeatletter
\def\@xobeysp{ }
\makeatother
% \lstset{breaklines=true,basicstyle=\ttfamily}

% Configuration for the memoir class.
\renewcommand{\cleardoublepage}{}
% \renewcommand*{\partpageend}{}
\renewcommand{\afterpartskip}{}
\maxsecnumdepth{subsubsection} % number subsections
\maxtocdepth{subsubsection}

\addtolength{\parindent}{-5mm}
% Packages not included:
% For multiline comments, use caption package. But this conflicts with hyperref while making html files.
% subfigure conflicts with use with memoir style-sheet.

\usepackage{fontspec, xunicode}
\setmainfont[Script=Devanagari]{Kalimati}

% Configuration for the memoir class.
\renewcommand{\cleardoublepage}{}
% \renewcommand*{\partpageend}{}
\renewcommand{\afterpartskip}{}
\maxsecnumdepth{subsubsection} % number subsections
\maxtocdepth{subsubsection}

\addtolength{\parindent}{-5mm}
% Packages not included:
% For multiline comments, use caption package. But this conflicts with hyperref while making html files.
% subfigure conflicts with use with memoir style-sheet.

% \setlength{\textwidth}{4.5in}

% Use something like:
% % Use something like:
% % Use something like:
% \input{../../macros}

% groupings of objects.
\newcommand{\set}[1]{\left\{ #1 \right\}}
\newcommand{\seq}[1]{\left(#1\right)}
\newcommand{\ang}[1]{\langle#1\rangle}
\newcommand{\tuple}[1]{\left(#1\right)}

% numerical shortcuts.
\newcommand{\abs}[1]{\left| #1\right|}
\newcommand{\floor}[1]{\left\lfloor #1 \right\rfloor}
\newcommand{\ceil}[1]{\left\lceil #1 \right\rceil}

% linear algebra shortcuts.
\newcommand{\change}{\Delta}
\newcommand{\norm}[1]{\left\| #1\right\|}
\newcommand{\dprod}[1]{\langle#1\rangle}
\newcommand{\linspan}[1]{\langle#1\rangle}
\newcommand{\conj}[1]{\overline{#1}}
\newcommand{\gradient}{\nabla}
\newcommand{\der}{\frac{d}{dx}}
\newcommand{\lap}{\Delta}
\newcommand{\kron}{\otimes}
\newcommand{\nperp}{\nvdash}

\newcommand{\mat}[1]{\left( \begin{smallmatrix}#1 \end{smallmatrix} \right)}

% derivatives and limits
\newcommand{\partder}[2]{\frac{\partial #1}{\partial #2}}
\newcommand{\partdern}[3]{\frac{\partial^{#3} #1}{\partial #2^{#3}}}

% Arrows
\newcommand{\diverge}{\nearrow}
\newcommand{\notto}{\nrightarrow}
\newcommand{\up}{\uparrow}
\newcommand{\down}{\downarrow}
% gets and gives are defined!

% ordering operators
\newcommand{\oleq}{\preceq}
\newcommand{\ogeq}{\succeq}

% programming and logic operators
\newcommand{\dfn}{:=}
\newcommand{\assign}{:=}
\newcommand{\co}{\ co\ }
\newcommand{\en}{\ en\ }


% logic operators
\newcommand{\xor}{\oplus}
\newcommand{\Land}{\bigwedge}
\newcommand{\Lor}{\bigvee}
\newcommand{\finish}{$\Box$}
\newcommand{\contra}{\Rightarrow \Leftarrow}
\newcommand{\iseq}{\stackrel{_?}{=}}


% Set theory
\newcommand{\symdiff}{\Delta}
\newcommand{\union}{\cup}
\newcommand{\inters}{\cap}
\newcommand{\Union}{\bigcup}
\newcommand{\Inters}{\bigcap}
\newcommand{\nullSet}{\phi}

% graph theory
\newcommand{\nbd}{\Gamma}

% Script alphabets
% For reals, use \Re

% greek letters
\newcommand{\eps}{\epsilon}
\newcommand{\del}{\delta}
\newcommand{\ga}{\alpha}
\newcommand{\gb}{\beta}
\newcommand{\gd}{\del}
\newcommand{\gf}{\phi}
\newcommand{\gF}{\Phi}
\newcommand{\gl}{\lambda}
\newcommand{\gm}{\mu}
\newcommand{\gn}{\nu}
\newcommand{\gr}{\rho}
\newcommand{\gs}{\sigma}
\newcommand{\gt}{\theta}
\newcommand{\gx}{\xi}

\newcommand{\sw}{\sigma}
\newcommand{\SW}{\Sigma}
\newcommand{\ew}{\lambda}
\newcommand{\EW}{\Lambda}

\newcommand{\Del}{\Delta}
\newcommand{\gD}{\Delta}
\newcommand{\gG}{\Gamma}
\newcommand{\gO}{\Omega}
\newcommand{\gL}{\Lambda}
\newcommand{\gS}{\Sigma}

% Formatting shortcuts
\newcommand{\red}[1]{\textcolor{red}{#1}}
\newcommand{\blue}[1]{\textcolor{blue}{#1}}
\newcommand{\htext}[2]{\texorpdfstring{#1}{#2}}

% Statistics
\newcommand{\distr}{\sim}
\newcommand{\stddev}{\sigma}
\newcommand{\covmatrix}{\Sigma}
\newcommand{\mean}{\mu}
\newcommand{\param}{\gt}
\newcommand{\ftr}{\phi}

% General utility
\newcommand{\todo}[1]{\footnote{TODO: #1}}
\newcommand{\exclaim}[1]{{\textbf{\textit{#1}}}}
\newcommand{\tbc}{[\textbf{Incomplete}]}
\newcommand{\chk}{[\textbf{Check}]}
\newcommand{\oprob}{[\textbf{OP}]:}
\newcommand{\core}[1]{\textbf{Core Idea:}}
\newcommand{\why}{[\textbf{Find proof}]}
\newcommand{\opt}[1]{\textit{#1}}


\DeclareMathOperator*{\argmin}{arg\,min}
\DeclareMathOperator{\rank}{rank}
\newcommand{\redcol}[1]{\textcolor{red}{#1}}
\newcommand{\bluecol}[1]{\textcolor{blue}{#1}}
\newcommand{\greencol}[1]{\textcolor{green}{#1}}


\renewcommand{\~}{\htext{$\sim$}{~}}


% groupings of objects.
\newcommand{\set}[1]{\left\{ #1 \right\}}
\newcommand{\seq}[1]{\left(#1\right)}
\newcommand{\ang}[1]{\langle#1\rangle}
\newcommand{\tuple}[1]{\left(#1\right)}

% numerical shortcuts.
\newcommand{\abs}[1]{\left| #1\right|}
\newcommand{\floor}[1]{\left\lfloor #1 \right\rfloor}
\newcommand{\ceil}[1]{\left\lceil #1 \right\rceil}

% linear algebra shortcuts.
\newcommand{\change}{\Delta}
\newcommand{\norm}[1]{\left\| #1\right\|}
\newcommand{\dprod}[1]{\langle#1\rangle}
\newcommand{\linspan}[1]{\langle#1\rangle}
\newcommand{\conj}[1]{\overline{#1}}
\newcommand{\gradient}{\nabla}
\newcommand{\der}{\frac{d}{dx}}
\newcommand{\lap}{\Delta}
\newcommand{\kron}{\otimes}
\newcommand{\nperp}{\nvdash}

\newcommand{\mat}[1]{\left( \begin{smallmatrix}#1 \end{smallmatrix} \right)}

% derivatives and limits
\newcommand{\partder}[2]{\frac{\partial #1}{\partial #2}}
\newcommand{\partdern}[3]{\frac{\partial^{#3} #1}{\partial #2^{#3}}}

% Arrows
\newcommand{\diverge}{\nearrow}
\newcommand{\notto}{\nrightarrow}
\newcommand{\up}{\uparrow}
\newcommand{\down}{\downarrow}
% gets and gives are defined!

% ordering operators
\newcommand{\oleq}{\preceq}
\newcommand{\ogeq}{\succeq}

% programming and logic operators
\newcommand{\dfn}{:=}
\newcommand{\assign}{:=}
\newcommand{\co}{\ co\ }
\newcommand{\en}{\ en\ }


% logic operators
\newcommand{\xor}{\oplus}
\newcommand{\Land}{\bigwedge}
\newcommand{\Lor}{\bigvee}
\newcommand{\finish}{$\Box$}
\newcommand{\contra}{\Rightarrow \Leftarrow}
\newcommand{\iseq}{\stackrel{_?}{=}}


% Set theory
\newcommand{\symdiff}{\Delta}
\newcommand{\union}{\cup}
\newcommand{\inters}{\cap}
\newcommand{\Union}{\bigcup}
\newcommand{\Inters}{\bigcap}
\newcommand{\nullSet}{\phi}

% graph theory
\newcommand{\nbd}{\Gamma}

% Script alphabets
% For reals, use \Re

% greek letters
\newcommand{\eps}{\epsilon}
\newcommand{\del}{\delta}
\newcommand{\ga}{\alpha}
\newcommand{\gb}{\beta}
\newcommand{\gd}{\del}
\newcommand{\gf}{\phi}
\newcommand{\gF}{\Phi}
\newcommand{\gl}{\lambda}
\newcommand{\gm}{\mu}
\newcommand{\gn}{\nu}
\newcommand{\gr}{\rho}
\newcommand{\gs}{\sigma}
\newcommand{\gt}{\theta}
\newcommand{\gx}{\xi}

\newcommand{\sw}{\sigma}
\newcommand{\SW}{\Sigma}
\newcommand{\ew}{\lambda}
\newcommand{\EW}{\Lambda}

\newcommand{\Del}{\Delta}
\newcommand{\gD}{\Delta}
\newcommand{\gG}{\Gamma}
\newcommand{\gO}{\Omega}
\newcommand{\gL}{\Lambda}
\newcommand{\gS}{\Sigma}

% Formatting shortcuts
\newcommand{\red}[1]{\textcolor{red}{#1}}
\newcommand{\blue}[1]{\textcolor{blue}{#1}}
\newcommand{\htext}[2]{\texorpdfstring{#1}{#2}}

% Statistics
\newcommand{\distr}{\sim}
\newcommand{\stddev}{\sigma}
\newcommand{\covmatrix}{\Sigma}
\newcommand{\mean}{\mu}
\newcommand{\param}{\gt}
\newcommand{\ftr}{\phi}

% General utility
\newcommand{\todo}[1]{\footnote{TODO: #1}}
\newcommand{\exclaim}[1]{{\textbf{\textit{#1}}}}
\newcommand{\tbc}{[\textbf{Incomplete}]}
\newcommand{\chk}{[\textbf{Check}]}
\newcommand{\oprob}{[\textbf{OP}]:}
\newcommand{\core}[1]{\textbf{Core Idea:}}
\newcommand{\why}{[\textbf{Find proof}]}
\newcommand{\opt}[1]{\textit{#1}}


\DeclareMathOperator*{\argmin}{arg\,min}
\DeclareMathOperator{\rank}{rank}
\newcommand{\redcol}[1]{\textcolor{red}{#1}}
\newcommand{\bluecol}[1]{\textcolor{blue}{#1}}
\newcommand{\greencol}[1]{\textcolor{green}{#1}}


\renewcommand{\~}{\htext{$\sim$}{~}}


% groupings of objects.
\newcommand{\set}[1]{\left\{ #1 \right\}}
\newcommand{\seq}[1]{\left(#1\right)}
\newcommand{\ang}[1]{\langle#1\rangle}
\newcommand{\tuple}[1]{\left(#1\right)}

% numerical shortcuts.
\newcommand{\abs}[1]{\left| #1\right|}
\newcommand{\floor}[1]{\left\lfloor #1 \right\rfloor}
\newcommand{\ceil}[1]{\left\lceil #1 \right\rceil}

% linear algebra shortcuts.
\newcommand{\change}{\Delta}
\newcommand{\norm}[1]{\left\| #1\right\|}
\newcommand{\dprod}[1]{\langle#1\rangle}
\newcommand{\linspan}[1]{\langle#1\rangle}
\newcommand{\conj}[1]{\overline{#1}}
\newcommand{\gradient}{\nabla}
\newcommand{\der}{\frac{d}{dx}}
\newcommand{\lap}{\Delta}
\newcommand{\kron}{\otimes}
\newcommand{\nperp}{\nvdash}

\newcommand{\mat}[1]{\left( \begin{smallmatrix}#1 \end{smallmatrix} \right)}

% derivatives and limits
\newcommand{\partder}[2]{\frac{\partial #1}{\partial #2}}
\newcommand{\partdern}[3]{\frac{\partial^{#3} #1}{\partial #2^{#3}}}

% Arrows
\newcommand{\diverge}{\nearrow}
\newcommand{\notto}{\nrightarrow}
\newcommand{\up}{\uparrow}
\newcommand{\down}{\downarrow}
% gets and gives are defined!

% ordering operators
\newcommand{\oleq}{\preceq}
\newcommand{\ogeq}{\succeq}

% programming and logic operators
\newcommand{\dfn}{:=}
\newcommand{\assign}{:=}
\newcommand{\co}{\ co\ }
\newcommand{\en}{\ en\ }


% logic operators
\newcommand{\xor}{\oplus}
\newcommand{\Land}{\bigwedge}
\newcommand{\Lor}{\bigvee}
\newcommand{\finish}{$\Box$}
\newcommand{\contra}{\Rightarrow \Leftarrow}
\newcommand{\iseq}{\stackrel{_?}{=}}


% Set theory
\newcommand{\symdiff}{\Delta}
\newcommand{\union}{\cup}
\newcommand{\inters}{\cap}
\newcommand{\Union}{\bigcup}
\newcommand{\Inters}{\bigcap}
\newcommand{\nullSet}{\phi}

% graph theory
\newcommand{\nbd}{\Gamma}

% Script alphabets
% For reals, use \Re

% greek letters
\newcommand{\eps}{\epsilon}
\newcommand{\del}{\delta}
\newcommand{\ga}{\alpha}
\newcommand{\gb}{\beta}
\newcommand{\gd}{\del}
\newcommand{\gf}{\phi}
\newcommand{\gF}{\Phi}
\newcommand{\gl}{\lambda}
\newcommand{\gm}{\mu}
\newcommand{\gn}{\nu}
\newcommand{\gr}{\rho}
\newcommand{\gs}{\sigma}
\newcommand{\gt}{\theta}
\newcommand{\gx}{\xi}

\newcommand{\sw}{\sigma}
\newcommand{\SW}{\Sigma}
\newcommand{\ew}{\lambda}
\newcommand{\EW}{\Lambda}

\newcommand{\Del}{\Delta}
\newcommand{\gD}{\Delta}
\newcommand{\gG}{\Gamma}
\newcommand{\gO}{\Omega}
\newcommand{\gL}{\Lambda}
\newcommand{\gS}{\Sigma}

% Formatting shortcuts
\newcommand{\red}[1]{\textcolor{red}{#1}}
\newcommand{\blue}[1]{\textcolor{blue}{#1}}
\newcommand{\htext}[2]{\texorpdfstring{#1}{#2}}

% Statistics
\newcommand{\distr}{\sim}
\newcommand{\stddev}{\sigma}
\newcommand{\covmatrix}{\Sigma}
\newcommand{\mean}{\mu}
\newcommand{\param}{\gt}
\newcommand{\ftr}{\phi}

% General utility
\newcommand{\todo}[1]{\footnote{TODO: #1}}
\newcommand{\exclaim}[1]{{\textbf{\textit{#1}}}}
\newcommand{\tbc}{[\textbf{Incomplete}]}
\newcommand{\chk}{[\textbf{Check}]}
\newcommand{\oprob}{[\textbf{OP}]:}
\newcommand{\core}[1]{\textbf{Core Idea:}}
\newcommand{\why}{[\textbf{Find proof}]}
\newcommand{\opt}[1]{\textit{#1}}


\DeclareMathOperator*{\argmin}{arg\,min}
\DeclareMathOperator{\rank}{rank}
\newcommand{\redcol}[1]{\textcolor{red}{#1}}
\newcommand{\bluecol}[1]{\textcolor{blue}{#1}}
\newcommand{\greencol}[1]{\textcolor{green}{#1}}


\renewcommand{\~}{\htext{$\sim$}{~}}

\title{॥संस्कृत-सूत्रं॥}
\author{विश्वासः वासुकेयः॥}

\begin{document}
\maketitle

\part{परिचयः॥}
श्रीमद्पाणिनि-वररुचि-पतञ्जलिभ्यः नमः॥ शब्दब्रह्मणे नमः॥ अक्षराणां, पदानां, वाक्यानां च शुद्धोच्चारणे च सुसंस्कारे पुण्यः।

\chapter{संस्कृते अद्भुतानि॥}
अक्षरोत्पत्तेः च उच्चारणस्य सूक्ष्म-अध्यायः (मन्त्र-शक्ति-मूलः एषः।)। उर्वरी शब्द-व्युत्पत्ति-व्यवस्था (तस्मात् व्युत्पत्तानां सूक्ष्मः विश्लेषणः च)। सूक्ष्माः पद-वाक्य-संस्कार-नियमाः (- तेषु अपि क्रियासु परस्मै-आत्मने-भेदः, द्वि-बहुवचनभेदः च)।

\chapter{व्याकरण-सूत्राणि॥}
व्याकरणशिक्षायै अनेकाः मार्गाः आसन् - सारस्वतं, पाणिनीयं इत्यादयः। पाणिनीयम् व्याकरणम् अधुना अधिकम् प्रसिद्धम्।

पाणिनीयान् सूत्रान् अवगन्तुं‌ काश्मीरे काशिकावृत्तिः, जैनेषु हैमचन्द्रं, वङ्गदेशे बौद्धलिखितम् चान्द्रं, महाराष्ट्रे सिद्धान्त-कौमुदी, उत्तरीय-वैष्णवेषु हरिनामव्याकरणम् इत्यादिनि रचितानि।

\section{परिभाषाः॥}
१३३ प्रख्याताः परिभाषाः पाणिनेः पूर्वमेव विरचिताः, तेन स्वीकृताः च। 

\section{सूत्रावगमनं॥}
\subsection{अनुवृत्तिः॥}
पाणिनीय-सूत्राणि सुलभ-स्मरणार्थं बहु सङ्क्षिप्ताणि सन्ति। 'अर्ध-मात्रा-लाघवेन पुत्रोत्सवं मन्यन्ते वय्याकरणाः'।

अतः बहूनाम् सूस्त्राणाम् अवगमनाय अनुवृत्तिः आवश्यकी।उदाहरणाय 'हलन्त्यं॥' इत्येतत् 'उपदेशे हल् अन्त्यं इत् स्यात्॥' इति अनुवर्त्यते। अनुवृत्यै पदग्रहणम् पाणिनीयक्रमे पूर्वतनेभ्यः समीपस्थेभ्यः स्वीक्रीयन्ते ।

वय्याकरण-पद्धति-अनुसारं 'सूत्रेषु अदृष्टं पदं सूत्रानन्तरात् अनुवर्तनीयं'। 

\subsection{क्रमः॥}
पाणिनीय-सूत्राणां क्रमः 'definition before use' शैल्यां नास्ति, प्रायः स्मरण-सौलभ्यार्थं। परन्तु अन्यैः वैयाकरणैः अवगमन-सौलभ्याय भिन्नाः क्रमाः दत्ताः।

\section{सूत्र-विभक्तिः॥}
\subsection{अधिकरण॥}
अधिकार-सूत्राणि आगामिषु केषुचित् सूत्रेषु अनुवृत्त्यां कश्चन पदं संयोक्तव्यं इति वदन्ति, उदाहरणाय '२.१.१ - प्रत्ययः'।

\subsection{संज्ञा॥}
संज्ञा-सूत्राणि सूत्र-उपयोगाय विशेषाः 'definitions' - उदाहरणाय 'वृद्धिरादैच्॥'। परन्तु कदाचित् अन्य्ऽन्य-आश्रयः इति दोषः दृश्यते संज्ञासूत्रेषु।

संज्ञाः प्रतीयने च विधाने उपयुज्यन्ते।

\subsection{परिभाषा॥}
परिभाषा-सूत्राणि सूत्र-अवगमनाय उपयोज्यानि नियमानि - उदाहरणाय 'शष्टि-विभ्क्त्यां प्रयुक्ते स्थाने इत्यर्थः' इत्येतत् 'इकः यण् अचि' अवगन्तुं आवश्यकं।

\subsection{विधिः॥}
विधि-सूत्राणि (generation rules) सर्व-मुख्य-सूत्राणि शब्द-परिवर्तन-साधनानि, उदाहरणाय - 'इको यणचि॥'। अतिदेश-सूत्राणि विधिसूत्र-विशेषाणि 'अन्य-शब्द-परिवर्तन-वत् अत्रापि भवति' इति वदन्ति।

विधिसूत्रेषु वर्णानां आदेशः (replacement) च आगमः (addition/ insertion) च लोपः (अदर्शणे) दृश्यन्ते - एतत् विधानं इति।

\subsection{नियम॥}
नियम-सूत्राणि विधिसूत्राणां कुत्र उपयोगः भवितव्यः, कुत्र न कर्तव्यः इति स्पष्टीकुर्वन्ति।

\section{संज्ञाः॥}
व्यकरणे शब्दानाम् कान्-चन विशेषः अर्थः - अन्यार्थाः अपि स्युः अन्यत्र।

\subsection{प्रत्ययाः।}
विधीयते उत प्रतीयते यत्  प्रत्ययः।

\subsection{इत्॥}
इत्-संज्ञाः कार्यं कृत्वा लुप्ताः भवन्ति - footnote सदृशं। तस्य लोपः (१.३.९) इति पाणिनिः। उदाहरणाय, माहेश्वरसूत्रात् प्रत्याहारे अकि नास्ति ण् च क्।

इताम् नामानि उक्-इत्, क्-इत्, श्-इत् इत्यादयः।

(उपदेशे) हलन्त्यम्॥

(उपदेशे) अजनुनासिकम्॥

(प्रत्ययादौ) लशक्वतद्धिते॥

(प्रत्ययादौ) चुटू॥

\subsection{वर्ण-संज्ञाः॥}
माहेश्वरसूत्रीयाः प्रत्याहाराः अन्यतः ज्ञातव्याः॥

व्यञ्जनानां निर्देशः एवं - 'क्' इत्येतस्य ककारः इति (अक्षरं + 'अकारः'), परन्तु 'रकार' इत्येतस्य स्थाने 'रेफः' इत्युपयुज्यते।

स्वराणां निर्देशः 'अकार'-वत्। तुल्यास्य-प्रयत्नम् सवर्णम् ।

कचतटपवर्गाणाम् कुचुतुटुपु इति।

तपरस्तत्कालस्य (१।१।७०)  - अस्मात् - अत् = अ, आत् = आ।

वृद्धिः आत् ऐच् ।

अदेङ् गुणः (१।१।२)

'अचोऽन्त्यादि टि' इत्येतस्मात् कस्यचिदपि शब्दस्य अन्तिमाचः प्रारभ्य प्राप्तम् उपशब्दम् टि इति। हविस् – टि-भाग: is इस्।

\chapter{अक्षराणि च वर्णाः॥}
दीर्घ-प्लुतादिषु उच्चर्याः ध्वनि-अणवः अक्षराणि (syllable)।

\section{माहेश्वरसूत्राणि॥}
१. अ इ उ ण् |

२. ऋ ऌ क् |

३. ए ओ ङ् |

४. ऐ औ च् |

५. ह य व र ट् |

६. ल ण् |

७. ञ म ङ ण न म् |

८. झ भ ञ् |

९. घ ढ ध ष् |

१०. ज ब ग ड द श् |

११. ख फ छ ठ थ च ट त व् |

१२. क प य् |

१३. श ष स र् |

१४. ह ल् |

\section{उच्चारणे काय-प्रयोगः॥}
अक्षराणां उच्चारणे जिह्वायाः, नासिकस्य च मुखस्य, ओष्टयोः च वक्ष-जटरयोः च कण्ठमणेः‌ (larynx) प्रयोगः। उच्चारण-विवरणाय वर्णोत्पत्ति-स्थानं च मुखे वायोः‌ गामयनस्य प्रयत्नः विव्रीयन्ते। पूर्वपण्डितैः उक्तं, श्री-धनंजयवैद्य-संवादेन ज्ञातं।

\subsection{वर्णाः॥}
अक्षरस्य विवरणं (characters) विविधानां वर्णानां संयोजनेन साध्यं। वर्णः उत्पत्ति-स्थान-प्रयत्नानां प्रतीकः।

\subsection{वर्णोत्पत्ति-स्थानानि॥}
स्थानं जिह्वायाः च नासिक-उपयोगस्य विवरणेन स्पष्टीकॄतं।

कवर्गस्य कण्ठ्यं guttural। चवर्गस्य तालव्यं palatal। टवर्गस्य मूर्धन्यं lingual। तवर्गस्य दन्त्यं dental। पवर्गस्य ओष्ट्यं labial।

तत्-विहाय जिह्वामूलीयः इति स्थानं गन्तुं जिह्वा वक्रं भूत्वा स्व-मूलं गच्छति - ऋ-कारे च 'इतः क' इत्येतस्मिन् विसर्गानन्तरं वीक्ष्यते।

अनुनासिकानां च अनुस्वारस्य नासिका।

एतेषां विविध-रीतिभिः (युगपदेव वा भिन्नेषु समयेषु) मिश्रणेन इतः अपि अधिकाः स्थानानि निर्दिष्टानि।

\subsection{प्रयत्नः॥}
\subsubsection{आभ्यन्तरः॥}
मुखप्रदेश-अभ्यन्तरे क्रीयमानः।

\paragraph{अस्पृष्टौ॥}
अत्र जिह्वा अस्पर्षात् वायु-गतिः अबद्धा परन्तु रुद्धा।

विवृत-प्रयत्ने ओष्ठौ open उद्घाटितौ। ह्रस्वं अकारं विहाय सर्वाणां स्वराणां एषः एव प्रयत्नः।

संवृत-प्रयत्ने ओष्ठौ ईषत् closed उद्घाटितौ। केवलं अकारस्य प्रयोगे एषः प्रयत्नः, तस्यापि पद-सिद्धि-प्रक्रियायां विवृतः एव।

\paragraph{ईषत्-स्पृष्टः॥}
अत्र जिह्वा स्पर्षात् वायु-गतिः अबद्धा परन्तु रुद्धा।

उदाह्अरणाय वण्।

\paragraph{ ईषद्विवृतः॥}
विवृत-प्रयत्नात् ईषत् अधिकं परन्तु असम्पूर्णतया रुद्धा वायुगतिः। विवृत-प्रयत्न-सदृशं ओष्टौ उद्घाटितौ।

उदाहरणाय ऊष्माणः।

\paragraph{स्पृष्टाः॥}
अत्र जिह्वायाः वा‌ ओष्टयोः स्पर्षात् वायु-गतिः बद्धा। उदाहरणाय वर्गीय-व्यञ्जनानि।

\subsubsection{बाह्यः॥}
मुखप्रदेशबाह्यः -  कण्ठमणौ (larynx) वक्ष-जठरयो: (chest-stomach)।

\paragraph{प्राणबलं॥}
("aspiration") अल्पप्राण-महाप्राणौ च स्तः - क्-ख्-भेदात् ज्ञेयौ। धनंजयोक्तं- ' यदा वायु: जठरात् महाप्रयत्नेन उद्गच्छति इति संवेदना स प्रयत्नः महाप्राणः , यत्र अल्पा सा संवेदना, स प्रयत्नः अल्पप्राणः ।'

\paragraph{नाद-घोषौ॥}
("voicing") विवार-संवारौ, श्वास-नादौ, अघोष-घोषौ इति सन्ति। यथा धनंजयेन विवृतं - 'संवारे कण्ठमणि: अवरुद्धः । विवारे अनावरुद्धः | नाद-घोषकरणे अवरोधः आवश्यकः एव ।' ' अभ्यन्तर-स्पर्शादि प्रयत्नात्पूर्वमपि कण्ठमणौ नादः भवति, नादो न भवति तत्र श्वासः । The humming "voicing" that precedes the articulation in the mouth for certain consonants is nAda.' 'अभ्यन्तर-स्पर्शादि प्रयत्नात् परमपि कण्ठमणौ घोषो भवति, न वा भवति तत्र अघोषः ।' तम्, vs दम् - द is voiced (- नाद or humming vibration felt in the nose), while त is not।

विवार-श्वास-अघोषाः (यथा ककारे) सर्वदा युगपत् दृश्यन्ते, संवार-नाद-घोष (यथा गकारे) युगपत् च।

\paragraph{tone}
उदात्तोच्चरणाय उर्ध्व-ओष्टः (विशेषतः तत्-अन्तौ) नासिकं प्रति कर्षयेत्।तेन सह अधरः अपि स्वाभाविकतया नासिकं प्रति कर्ष्यते। आसन्न-हसन्-मुखस्य आकारं अनुसरति उर्ध्व-ओष्टः। दूरे-स्थितस्य नाम-घोषणे तथैव अस्ति मुख-प्रयोजनः।

अनुदात्त-उच्चारणे मुखातर्गुहा बृहत्तरा करणीया, वा सामान्ये गात्रे रक्षितव्या। उदात्तात् अधिकं विश्रान्तः भवति उर्ध्व ओष्टः, तस्य अन्तौ नासिकात् दूरे (यथा स्वाभाविकं) स्थितौ।

\section{स्वराः॥}
येषां संवृतः वा विवृतः आभ्यन्तरप्रयत्नः ताः वर्णाः।

\subsection{दीर्घता।}
स्वराः ह्रस्वाः वा दीर्घाः वा प्लुताः। प्लुताक्षराः अ३ इ३ इत्यादयः। सर्वेषां स्वराणां‌ अनुनासिक-अननुनासिक-इति भेदः वर्तते।अनुनासिक-निर्देशाय ँ उपयुज्यते।

एचः दीर्घाः वा प्लुताः एव - तेषां ह्रस्व-उपयोगः न भवति। ऌ-कारस्य दीर्घ-रूपः न वर्तते।

\exclaim{ऌ-वर्णः संस्कृतवाङ्मये केवलं‌ कॢपि(कल्पते) दृश्यते। ॡ कुत्रापि न दृश्यते!}

\subsection{उच्चारणं॥}
श्रीमता धनञ्जयेन \htext{विवृतं}{http://samskrute.blogspot.com/2009/07/pronunciation-podcast-01.html}।


\subsubsection{स्थानानि।}
अ - कण्ठः। इ - तालु। उ- ओष्टौ। ऋ - मूर्धा वा जिह्वामूलीया। ऌ - दन्तः।

ऋ-अक्षरस्य उच्चारणे अस्ति buzzing रेफः। श्रीमता धनञ्जय-वैद्येन \htext{विवृतं}{http://samskrute.blogspot.com/2009/07/pronunciation-podcast-01.html}।

\subsubsection{आभ्यन्तर-प्रयत्नः॥}
ह्रस्व-अकारस्य संवृतः। अन्येषां विवृतः।

\subsubsection{बाह्यप्रयत्नः॥}
सर्वेषां अल्पप्राण-संवार-नाद-घोषः। Tone अन्यत्र विवृतः।

\subsection{Tone।}
वैदिक-भाषा साङ्गीतिका, लौकिकाः अपि।स्वराः मूलतः त्रिदा विभक्ताः। उदात्तानां स्वराणां उच्चारणे high pitch/ tone उपयुज्यते। अनुदात्तानां स्वराणां उच्चारणे low pitch/ tone/ grave accent उपयुज्यते। स्वरितानां स्वराणां उच्चारणे उदात्तात् अनुदात्तगमनं दृष्यते।

एते छन्दसं न परिवर्तन्ते। 

\subsubsection{आर्ग्वेद-विवरणम् ॥}
(१) उच्चैरुदात्तः, (२) नीचैरनुदात्तः । (३) समाहारः स्वरितः । (४) उदात्तस्वरितपरस्य सन्नतरः। (५) एकश्रुति दूरात् संबुद्धौ। (६) उच्चैस्तरां वा वषट्कारः। 

'कोशे एकविधानेन एव दातव्यं प्रातिपदिकम् । वाक्ये अन्ये द्वौ-त्रयः स्वरभेदा: जायन्ते, ते वाक्ये वाक्ये भिन्ना: ।' 'अदूरादुक्तेषु वाक्येषु अपि यत्र स्वरे उदात्तानुदात्तस्वरितत्वं सुस्पष्टं न कर्तुमिच्छसि, तत्र मध्यगम् उच्चारणं करोषि ।' इति धनञ्जयः।

\subsubsection{चिह्नानि॥}
प्रत्येकस्मिन् पदे स्थितः उदात्तः अभिज्ञातः चेत् उच्चरणं‌ स्प्ष्टं‌ भवति। भिन्नेषु शाखासु भिन्नानि चिह्नानि उपयुज्यन्ते। तत्तिरीय-शाखायां च ऋग्वेदे उदात्त-पूर्वं आगतस्य अनुदात्तस्य अ॒-वत् चिह्नः; स्वरितस्य हि॑-वत्। पद-प्रथमः स्वरः अनुदात्तः चेत् तदनन्तरं उदात्तं यावत् सर्वाः स्वराः अ॒-वत् चिह्नितः।

\subsubsection{ऋग्वेदचिह्नानि।}
ऋग्वेदोदहरणं - 'अ॒ग्निमी॑ळे पु॒रोहि॑तं य॒ज्ञस्य॑ दे॒वमृ॒त्विज॑म्।' 'के॒शिभि॑'

सन्नतराय अनुदात्ताय च द्वाभ्याम् " ॒ "चिह्नम् ।स्वरिताय "  ॓ " चिह्नम् । उदात्ताय च एकश्रुतिने च न किमपि चिह्नम् ।



\subsubsection{उपयोगः॥}
पद-छेदे, अर्थावगमने च उपयुज्यते उदात्तादयः।

\subsection{अनुनासिक-रूपाः।}
एतेषां अनुनासिक-रूपाः अपि भवन्ति।

\subsection{बलं॥}
मूलस्वराः अकः (दीर्घाः, प्लुताः अपि)॥ तेभ्यः अन्याः स्वराः उद्भवन्ति।

\subsubsection{गुणः॥}
मूलस्वरेभ्यः पूर्वं अ-संयोगेन जायन्ते मध्यम-बल-स्वराः - अ-ए-ओ-अर्। 'अत्-एङ्-गुणः'॥

\subsubsection{वृद्धिः॥}
मध्यमबलस्वरेभ्यः पूर्वं अ-संयोगेन जायन्ते बलवतः स्वराः - आ-ऐ-औ-आर्। 'वृद्धिः आत्-ऐच्'॥

वृद्धिः यस्य अचाम् आदिः तत् वृद्धं (शब्दं)॥ प्रथमस्वरे वृद्ध्यौ वृद्धसंज्ञा - गानं, मैत्रं।

\section{हलः॥}
Aka consonants. Produced by interrupting air-flow or redirecting it to the nose.

\subsection{उच्चारणं॥}
\subsubsection{स्थानं॥}
व्यञ्जनानां एका स्थानानुसारं एवं विभक्ताः। Aka point of pronunciation.

कवर्गस्य कण्ठ्यं guttural। चवर्गस्य तालव्यं palatal। टवर्गस्य मूर्धन्यं lingual। तवर्गस्य दन्त्यं dental। पवर्गस्य ओष्ट्यं labial।

अनुनासिकानां तेन सह नासिकं अपि उत्पत्ति-स्थानं। अनुनासिकाः - nasalize the sound of the previous letter in the varga।

कचटतप-वर्गाः वर्गीय-व्यञ्जनाः इति।

जिह्वामूलीययोः जिह्वामूलीयः। उपद्मानीययोः ओष्टौ।

ह्, ः - कण्ठः। ल्, स् - दन्ताः। र् ष् - मूर्धा। य्, श् - तालु।

\subsubsection{आभ्यन्तर-प्रयत्नः}
वर्गीयाः हलः 'स्पर्शाः' स्पृष्ट-प्रयत्न-जाताः।

श् ष् स् ह् ऊष्माणः sibilants ईषद्विवृताः।

अन्तस्थाणि semivowels य् र् व् ल् ईषद्स्पृष्टाः - दीर्घां वा अनुनासिकां -ध्वनिं जनयितुं साध्यं खलु।

\subsubsection{बाह्य-प्रयत्नः}
कचटतपवर्गाणां अन्तिम-अक्षर-त्रयः च ऊष्माणः अक्षराणि च हकारः‌ मृदुनी, तेषां संवार-नाद-घोषः(Voicing)। अन्येषां विवार-श्वास-अघोषः।

वर्गाणां २,४ वर्णाः च ऊष्माणः महाप्राणाः। अन्याः अल्पप्राणाः (aspiration)।

अननुनासिक-व्यञ्जनानतरं विरामे सति voicing-भेदः नास्ति - त्, द् सदृकेव श्रूयेते।

अननुनासिक-व्यञ्जनानतरं स्वराभावे (विराम-छिह्ने लिप्यां) सति aspiration-भेदः नास्ति - क्, ख् सदृकेव श्रूयेते।

\subsection{यरलव-अर्धस्वरत्वं। }
अन्तस्थाः (यरलव) voiced and aspirated - caused by suppressing air flow at the point of pronunciation। एतेभ्यः अपि स्वर-सदृशं‌ अनुनासिकत्वं साध्यं। तत्-निर्देशाय अपि ँ उपयुज्यते - स्वरं कृते उपयोक्तव्यं वा यरलव-कृते इत्येतत् केवलं प्रसङ्गात् ज्ञायते।

\subsection{विरलौ व्यञ्जनौ॥}
ळवर्णः केवलं वैदिके संस्कृते स्वरमध्ये आगतस्य ड्-वर्णस्य स्थाने ळ् इति च ढ वर्णस्य स्थाने ळ्ह इति दृश्यते।

\section{अयोगवाहौ।}
माहेश्वर-सूत्रेषु न दृष्येते। अनुस्वारः (उदाहरणाय अं) च विसर्गः (उदाहरणाय अः)।

\subsection{उपद्मानीयः च जिह्वामूलीयः च।}
जिह्वामूलीयाः (बाल\&करोति) च उपध्मानीयः (बाल\%पीबते) च कखभ्यां वा पफाभ्यां पूर्वार्धे विसर्गवत् रूपः। यथा उदाहरणयोः तौ सन्धि-करणेन जातौ।

\chapter{शब्दाः च पदानि॥}
धातु-प्रातिपतिपदिकाभ्याम् दद्मः शब्द-संज्ञाम्। शब्दाः (प्रकृतयः) सुप्-तिङ्-प्रत्यय-परिवर्तिताः भूत्वा‌ पदानि भवन्ति। एक-शब्दात् अन्यस्य शब्दस्य उत्पत्तेः नाम व्युत्पत्तिः।

\section{विभागाः।}
संस्कार-दृष्ट्या - 'सुप्तिङन्तं पदं।'

कारक-दृष्ट्या - नामपद-क्रियापद-तद्विशेषणानि च। सुबन्तानि अपि क्रियापदानि भवितुम् अर्हन्ति, उदाहरणाय 'गतम् धनम्', 'पुष्पैः विकसितम्'। कर्तृ-कर्मपदविशेषणैः अपि क्रियपदसंज्ञा योग्या  - 'गतम् धनम्'।

\section{स्वरः।}
"अनुदात्तं पदम् एकवर्जम्"।

\section{पदे वर्णपरिवर्तनम् ॥}
वृद्धिः च गुणः‌ यथा वृद्धि-गुणसध्योः विवृतौ। संप्रसारणम् धातुखण्डे च।

\section{सन्धिः।}
\subsection{संयोगः न सन्धिः।}
'हलोः अन्तरः संयोगः।' सम् + मतिः = सम्मतिः।

\subsection{परिचयः॥}
द्वौ शब्दौ (शब्दः + प्रत्ययः, वा पदम् + पदम्) वेगेन सहोच्चारिते चेत् ध्वनि-परिवर्तनम् दृश्यते। जिह्वा-आलस्यं एषां परिवर्तनानां मूलः। सन्धि-उपयोग-कारणेषु वेग-विवक्षा, लेखने मिताक्षरत्वस्य वा शब्दमाधुर्यस्य इच्छा।

१३ परः सन्निकर्षः संहिता (१।४।१०९) 

१८ स्थानेऽन्तरतमः (१।१।५०)

१४ हलोऽनन्तराः संयोगः (१।१।७) 

\subsubsection{करणाकरण-विकल्पः॥}
कदाचित् सन्धि-करण-अकरणयोः विकल्पः वर्तते। समस्त-पदानि नित्य-सन्धयः। उपसर्गैः सह शब्दस्य नित्य-सन्धिः।

\subsubsection{पदत्व-रक्षणम् ॥}
वि + आ + करण = व्याकरण । अत्र, 'इको यणचि' इत्यस्मात् व्य् आ कारण जातः चेत् अपि व्य् इत्येतत् पदम् एव।

सूत्रे विभक्तिभिः विवक्षा ज्ञातुम् 'इको यण् अचि' तथ 'शः छः अटि' ईक्षेत।

\subsection{स्वर-सन्धयः।}
२६ अकः सवर्णे दीर्घः (६।१।१०१)  - शकान्धु-गणे अपवादः - शक +‌अन्धु = शकन्धु।

२२ आद्गुणः (६।१।८७) [अचि परे]: अ + इ = ए।

२३ उरण् रपरः (१।१।५१) : अण् + ऋ/ ॠ = अण् + र्
 
'यण्-सन्धि'- इको (स्थाने) यण् अचि (परे)। इक् + अच् = यण् +‌अच् । हरि+अक्ष्यौ = हर्यक्ष्यौ।

१९ एचोऽयवायावः (६।१।७८) - हरि +‌ए = हरये ।

२५ वृद्धिरेचि (६।१।८८) - अ/आ + एच् = वृद्धिः।

२७ एङः पदान्तादति (६।१।१०९) - एङ् (पदान्ते) + अत् = एङ् : रामेऽत्र।

\subsubsection{प्लुत-प्रागृह्याः।}
२८ दूराद्धूते च (८।१।८४) - दूरात् वचने पदान्तस्वरस्य प्लुतः।

३० ईदूदेद् द्विवचनं प्रगृह्यम् (१।१।११)

२९ प्लुतप्रगृह्या अचि नित्यम् (६।१।१२५) - सन्धि-परिवर्तनम् नास्ति। आगच्छ कृष्ण३ अत्र गौश्चरति ।

\subsection{हल्-सन्धयः।}
३१ स्तोः श्चुना श्चुः (८।४।४०) - उच्चटनम्

३२ झलां जशोऽन्ते (८।२।३९) - ऋक् + मन्त्रः = ऋङ्मन्त्रः ।

३४ खरि च (८।४।५५) - झल् + खर् = चर् + खर् । ??

३९ छे च (६।१।७३)  - अच् + छ = अच् + त् + छ। शिव + छाया = शिवत् + छाया = शिवच्छाया।

ष्टुना ष्टु: (अग्रे) (8\.4\.41) - ष्/ टुँ + स् / तुँ = ष्/टुँ । इष् + त = इष्ट।

(झयः - अनुनासिक-वर्जित-वर्गीय-व्यञ्जनानेभ्यः अग्रे) शः छः अटि (अच्+ हयवर)। उत् + शिष्ट = उच् + शिष्ट = उच्छिष्ट ।

\subsection{अनुनासिक-सन्ध्यः।}
३८ ङमो ह्रस्वादचि ङमुण् नित्यम् (८।३।३२) (पदान्ते)- ङम् + ह्रस्वात् अच् : वयम् नास्ति।

३५ मोऽनुस्वारः (८।३।२३) - पदान्ते अनुस्वारस्य म् । मकारस्य हलि परे अनुस्वारः ।

३६ नश्च अपदान्तस्य झलि (८।३।२४)  - न्/म् (अपदान्तम्) + झल् = ं+ झल्। 

३७ अनुस्वारस्य ययि परसवर्णः (८।४।५८)
वर्गीय-व्यञ्जनेभ्यः पूर्वं अनुनासिकस्य (स्थाने) वर्गानुनासिकः। उदाहरणाय 'कक्ष्याङ्गच्छामि'॥ 

उपसर्ग-योगे क्वचित् विकल्पः वर्तते, उदाहरणाय संकेतः च सङ्केतः उभयौ अपि शुद्धौ।

\subsection{विसर्ग-सन्धिः॥}
४० ससजुषो रुः (८।२।६६) (पदान्ते)- स्/ष् = र् । अग्निरत्र।

खरवसानयोर्विसर्जनीय॥ = र् + खर्/ वचने अवसानम् = ः + खर्/ अवसानम्। उदाहरणाय - अन्तःकरणम् अपेक्षया अन्तर्गतम्।



\chapter{अलङ्काराः।}
\section{छन्दस्।}
गायत्री। अनुष्टुप्। त्रिष्टुप्।

\part{पदमूलाः॥}
\chapter{प्रकृति-प्रत्ययौ॥}
संस्कृतम् प्रत्ययमयी। प्रकृति-प्रत्यय-विवेकः शुद्धभाषायै आवश्यकम्। प्रकृतिः कश्चन शब्दांशः (स्वयम् प्रायः
प्रत्यययुक्तः) येन सह अन्यः प्रत्ययः युज्यते।

प्रत्ययाः विभज्यन्ते इत्-अक्षर-अनुसारम्, च प्रयोजनानुसारम्।

\chapter{क्रिया-पदमूलाः धातवः॥}
\section{गणविभक्तिः॥}
कर्मगणन-अनुसारम् विभक्तिः अन्यत्र विवृता। क्रियाफलदिशा-वाचक-प्रत्ययानाम् च सार्वधातुक-लकार-विकरण-प्रत्ययानाम् योग्यतानुसारा विभक्तिः अन्यत्र विवृता।

\subsection{उत्पत्ति-विभक्तिः॥}
'भ्वादयः धातवः॥ सनादयः धातवः॥' इति पाणिनिः।

२२०० सन्ति भ्वादयः, परन्तु वाङ्मये केवलम् ८०० उपयुज्यन्ते। ते अनेकान् नमपदान् जनयन्ति, अतः स्मरणं लाभाय (यथा GRE परिक्षायै Latin धातवः स्म्रीयन्ते)!

\section{सनादिभिः व्युत्पत्तिः॥}
सनादयः सन्ति १२ - सन्, क्यच्, काम्यच्, क्यङ्, क्यष्, क्विप्, णिच्, यङ्, यक्, आय, ईयङ्, णिङ्।

\subsection{इच्छायां सन्॥}
अत्र धातोः द्वित्त्वं (reduplication) भवति।

धातोः प्रथम-वर्णस्य द्वित्त्वं + ष-आगमः अन्ते।

कर्तुं इच्छति - चिकीर्षति।पातुं इच्छति पिपसति। वक्तुं इच्छति विवक्षति।

\subsection{प्रेरणार्थे णिच्।}
'हेतुमति च॥' इति पाणिनिः। 'आर्धधातुकं शेषः॥' इति पाणिनिः। 'णिचश्च॥' इति पाणिनिः, अतः उभयपदी।

अनुबन्ध-लोपात् इ-आगमः।

(causative) उदाहरणं 'रामः वृक्षं पातयति'। पातुं प्रेरयति = पातयति।

धातुः + अय, पूर्वस्वरस्य गुणः वा वृद्धिः। पदसंस्कारः परस्मै-पदवत्।

\subsubsection{चुरादिगणीयानाम् अप्रेरणे अपि साम्यम् ॥}
१०(चुरादि)-गणस्य धातुषु प्रेरणार्थे न सति (स्वार्थे) अपि सार्वधातुकेषु लकारेषु णिच्-प्रत्यय-आगमः दृश्यते।

\subsubsection{दिवादिगणीयानाम् आत्मनेपदिषु कर्तरि साम्यम् ॥}
दिव्यते। परस्मैपदिषु तावत् क्लेषः नास्ति।

\subsection{अतिशये यङ्॥}
यङन्ताः नित्य-आत्मनेपदिनः। भू - बोभूयते। नृत् - नरिनृत्यते।

धातोः प्रथम-वर्णस्य द्वित्त्वं + अन्ते य-आदेशः।

\subsubsection{यङ्-लुक्॥}
भू - बोभवीति। धातोः प्रथम-वर्णस्य द्वित्त्वं + अन्ते ई-आदेशः।

\subsection{नामधातवः॥}
(Denominatives)

\subsubsection{भवने॥}
णिच्- नटयति रूपयति।

क्यप् - क्यत् वत् - लोहितः भवति - लोहितायते, लोहितायति।

यक् - सख्यति।

\subsubsection{इच्छायां॥}
कामच् - पुत्रकाम्यति। क्यच् - इय/य-आगमः - पुत्रीयति, यशस्यति।

\subsubsection{आचरणे॥}
क्यच् - यथा इच्छायां। क्विप् - हंसति, कृष्णति।

क्यत् - अय-आगमः- हंसायते, कृष्णायते।

\section{सामान्य-धातु-परिवर्तनानि।}
उपधायाश्च॥ - उपधः इति धातोः अन्तिमपूर्वाक्षरम्, तत्र ऋकारस्य इकारः। 'उरण् रपरः' इत्येतस्मात् र्-आदेशः तत्-सन्दर्भे। \tbc

\subsection{संप्रसारणं।}
\subsubsection{प्रभावः॥}
पदसंस्कारे यरलवानां स्थाने इऋऌउ-आदेशः - इक्-यणः संप्रसारणं॥

\subsubsection{योग्यता॥}
'ग्रह्‌, ज्या, वय्‌, व्यध्‌, वश्‌, व्यच्‌, व्रश्च्‌, प्रच्छ्‌, भ्रस्ज्‌
इत्येषां धातूनां ङिति च किति सम्प्रसारणम्‌।' इति पाणिनिः (६.१.१६)। तथा सार्वधातुकेषु अपिति अपि (सार्वधातुक-विषये विवृतम्)। अतः - गृह्णन्, न तु ग्रह्णन् ।


पुक्-अन्त लघु-उपधस्य॥

\section{सार्वधातुकाः प्रत्ययाः।}

\subsection{सार्वधातुकत्वम् ॥}
तिङ्‌-शित्-सार्वधातुकम्‌ (३.४.११३)॥ परन्तु लृट्/ष्य-परे नास्ति सार्वधातुकत्वम्।

\subsection{प्रभावः॥}
'सार्वधातुकार्धधातुकयोः' इक्-अन्त-अङ्गषु इकः गुणः भवति।

'रुदादिभ्यः सार्वधातुके (हलादौ) इट्-आगमः।' इति पाणिनिः। अतः रोदिति इति।

\subsubsection{अपिति किति च शिति।}
"सार्वधातुकम् अपित्‌ (ङित्-वत्)" (१.२.४)। तस्मात् संप्रसारणम् केभ्यः चित् ।


\chapter{प्रातिपदिकानि च तदुत्पत्तिः॥}
\section{सुप्-तद्धितादिषु योग्यता॥}
'ङ्याप्‌प्रातिपदिकात्‌' इति अधिकारसूत्रम् ।

अनुवृत्तिः - "(अस्मिन् अधिकारे यः) प्रत्ययः (सः) ङी-आप्-प्रातिपदिकात्, परश्च, आद्युदात्तश्च ।"

प्रातिपदिकं + सुप्-प्रत्ययः = सुबन्तानि पदानि।

[किमर्थम् अत्र ङ्याप्प्रातिपदिकात् इत्यत्र ङ्याप् संयुक्तम्? पतञ्जलेः उत्तरम् । ङ्याबोः उपस्थितिः सूत्रे नास्ति चेत्‌, "स्वार्थ-द्रव्य-लिङ्ग-संख्या-कारकाणां क्रमिकत्वम्‌" इति नियमेन तद्धितः प्रत्ययः प्रथमम्‌ आगच्छति, तदनन्तरमेव स्त्री प्रत्ययः आगच्छेत्‌ । आर्य-शब्दे कन्‌ प्रत्ययः स्वार्थे प्रथमम्‌ आगच्छति, अनन्तरमेव टाप्‌ इति। प्रक्रियायाम्‌ आर्यिका इति रूपमेव निष्पन्नं, न तु आर्यका इति।]

\section{प्रातिपदिकस्य लिङ्गः, न तु वस्तोः॥}
(Grammatical gender.) लिङ्गः प्रातिपदिकं अनुसरति, न तु वस्तुं। वस्तोः तु लिङ्गः विविध-दृश्टिसु विविधः। उदाहरणानि 'मित्र' नपुंसकलिङ्गि, 'आप' पुंलिङ्गि।

कानिचन प्रातिपदिकानि अनियत-लिङ्गिनः - बहूनि विशेषणानि तथा- गतः, गतम् - वत्।

\section{उत्पत्तिः॥}
'अर्थवत् अधातुः अप्रत्ययः प्रातिपदिकं॥ कृत्-तद्धितसमासाः च॥' इति पाणिनिः। "प्रातिपदिकग्रहणे लिङ्गविशिष्टस्यापि ग्रहणम्‌।" इति परिभाषा।

प्रातिपदिकं मूलभूतं वा व्युत्पत्तं। अमूलभूत-प्रातिपदिक-व्युत्पत्त्यां अन्ते उपयुक्त-प्रक्रिया-प्रत्ययानुसारं समासान्त-तद्धितान्त-कृदन्त-इति विभक्तानि।

\section{लिङ्गविशिष्ट-शब्दाः॥}
परिभाषा अस्ति-- "प्रातिपदिकग्रहणे लिङ्गविशिष्टस्यापि ग्रहणम्‌।"

ङी च आप् च इति स्त्री-प्रत्यय-गण-द्वयम् ।

प्रत्यय-ग्रहण-परिभाषया तदन्त-ग्रहणम्।'

\subsection{ङी॥}
'ङी इति ङीप्-ङीष्ङीनां सामान्येन ग्रहणम्। 

'उक्-इतः च।' इत्यस्मात् उक्-इत्-युक्त-प्रत्ययान्तेभ्यः प्रातिपदिकेभ्यः ङीप्-योय्जते। तथा शतृँ-क्तवतु-मतुप्-प्रत्ययान्तेभ्यः भवत्-चलत्-हनुमत्-आदिभ्यः।

ङीप् तु एषां योज्यते - त्-न्-अन्तानां, अन-अन्तानां (शोधन-वत्), वृद्धि-जातानां (आवश्यक-वत्), मय-तन-अन्तानां, सङ्ख्यानां। विवरणं सिद्धान्तकौमुध्यां।

\subsection{आप् ॥}
आबिति टाप् डाप् चापां ग्रहणम्। 

चालक -> चालिका।

कदाचित् आक्षेपाः उच्यन्ते। उदाहरणाय - नर्तकः - Although the ण्वुल्  counterpart is नर्तिका, its meaning is
different from नर्तकी ।


\section{समासान्तानि।}
\subsection{पदसन्धानम् ।}
समास-लक्षणम् अनेक-पद-सन्धानम् । यत्र उत्पत्तेः अन्ते एक-पदमेव वर्तते, तत् समासः न।

अत्र साधारणतः सन्धि-नियमाः उपयुज्यन्ते । परन्तु पूर्वपदे नकारान्ते सति, न्-लोपः - यथा आत्मबलम्।

\subsubsection{स्वरः।}
समासान्तस्वरः उदात्तः। यथा ब्रह्मवर्च॒सेन॑।

\subsubsection{पुंवत्-भावः॥}
एकार्थ-भोधकत्वे सामानाधिकरण्यम् समासजनक-पदयोः। तत्र समान-विभक्तिकत्वम्‌ अपेक्षितम् ।

सामानाधिकरण्ये स्त्रीशब्दस्य पुंवत्-भावः। उदाहरणाय - शुद्धभाषा, रूपवत्पत्नी । एवम् कर्मधार्य-समासे बहुव्रीहि-समासे च, नान्यत्र ।

\subsubsection{विग्रहवाक्यम् ।}
लौकिकः विग्रह: - मयुरः नर्तकः यस्य सः = मयुरनर्तकः।

अलौकिकः विग्रहः - मयुर + सु + नायक + सु = मयूरनायक। अत्र 'यस्य सः' इत्येतौ वर्जितौ।

\subsection{प्रक्रिया॥}
"सह सुपोः २-१-४" इति सूत्रस्य विवरणे "समासत्वात् प्रातिपदिकत्वेन सुपोः लुक्"। अस्मात् अलौकिकः विग्रहः स्वीकरणीयः, तदनन्तरम् सुपोः लोपः भवति।

\subsection{तत्पुरुषाः।}
तत्पुरुष-समासेषु उत्तर-पद-प्राधान्यं।

उदाहरणं सप्तमी-तत्पुरुषस्य - रामस्य धनुः राम-धनुः, तस्य पुरुषः तत्पुरुषः।

\subsubsection{प्रथमा तत्पुरुषः कर्मधारी।}
मयुरः नर्तकः = मयुरनर्तकः।

उपमाने अपि - पुरुषः व्याघ्रः इव = पुरुषव्याघ्रः।

अवधारणे अपि - युक्तयः एव रत्नानि = युक्तिरत्नानि।

\subsubsection{नञ्॥}
नासत्य, असत्य। शब्दस्य आदौ स्वरे सति तु अन्-योगः - यथा अनिच्छा।

\subsubsection{सङ्ख्या-वैशिष्ट्यम् ॥}
षष्टमः बालः षष्बालः। शततमः आब्दः शताब्दः।

\subsection{बहुव्रीहिः।}
'शेषो बहुव्रीहिः॥' इति पाणिनिः।

'अनेकम् अन्यपदार्थे' इति अस्मात् अन्य-पद-प्राधान्यं।

उदाहरणार्थं बहवः व्रीहिः यस्य सः बहुव्रीहिः - एतत् बहुव्रीहिन् इति तद्धितान्तात् भिन्नम्।

सह-वर्तने - सकलाः, सहसन्तान, सहपुत्र, सपुत्र।

दिगन्तराळलक्षणे - उत्तरपूर्वं।

युद्ध-विवरणे - बाहूबाहु, मुष्टीमुष्टि।

\subsubsection{सङ्ख्या-वचने।}
द्वौ वा त्रयो वा = द्वित्राः। दशानां समीपे ये ते = उपदशाः।


\subsection{समाहारे॥}
एकवत्-भावी द्विगुः - पञ्चावटी, त्रिलोकी।

द्वन्द्वं - शीतोष्णं, सुखदुःखं - नपुंसकलिङ्गि, एकवचने।

इतरेतरयोगद्वन्द्व- रामः च कॄष्णः च - रामकृष्णौ - लिङ्ग-संख्या-यथोचितं।



\subsection{अव्ययीभावः॥}
प्रथम-पद-प्राधान्यं, च अव्यय-जननं - यथा इच्छा = यथेच्छा।

सूपस्य लेशः - सूपप्रति।



\section{तद्धितान्तानि॥}
प्रातपदिकेन वा तिङन्तेन सह तद्धित-प्रत्यय-योगेन जायन्ते तद्धितान्त-प्रातिपदिकानि।

\subsection{मात्रायाम् तरप्, तमप्॥}
तरप्, तमप् - हरितर, हरितम (पुं, न)। हरितरा, हरितमा (स्त्री)।

उच्छतर, उच्छतम॥

\subsubsection{तिङन्तेभ्यः॥}
पचते - पचतेतराम्, पचतेतमाम् = अधिकेन पचते॥

\subsection{भावे॥}
दृढादिभ्यः - ष्यञ्-इमनिच्- शुक्लिमा = शैक्ल्य।

ताल् - सुन्दरता ।

\subsubsection{नित्य-नपुंसकानि।}
ईश्वरस्य गुणं ऐश्वर्यं। प्रीति-गुणं प्रीतित्वं।

ष्यञ् - सौन्दर्य, चातुर्य, शैत्य॥

\subsection{युक्तत्वे॥}
अलुच् - शङ्कालु, दयालु।

इष्ठन् - बलिष्ठ श्रेष्ठ प्रेष्ठ।

इतच् - आधिक्येन युक्तत्वे - कुसुमित।

इनि - दण्डिन्, धनिन्। एतत् बहुव्रीहेः भिन्नम् ।

ईयसुन् - ईय-आदेशः - बलीयस् श्रेयस् प्रेयस् ॥

द्ह - ईन-आदेशः - कुलीन।

छ - ईय-आदेशः - त्वदीयः।

मतुप् - हनुमत्, हरिमत्, अजवत्, वीर्यवत् (पुं, न)। हरिमती (स्त्री)।

मयट् - आधिक्येन युक्तत्वे - हरिमय (पुं, न)। हरिमयी (स्त्री)।

र - मधुर।

विनि - विन्-आदेशः - यशस्विन्।

\subsection{वरणे तरप्-तमप्॥}
डतर - द्वयोः वरणे - कतर।

डतम - बहुषु वरणे - कतम।

\subsection{सादृश्य-वचनाय॥}
वत् अव्यय-प्रत्ययं - हरिवत्।

\subsection{प्रजा-वेदत्वादि-वचनाय॥}
अञ् - प्रथम-स्वर-वृद्धिः - औत्स, वैद।

इङ् - प्रथमस्वरवृद्धिः, अन्ते इकार-आदेशः - कार्णि, दाशरथि॥

ढक् - एय-आदेशः, प्रथमस्वरवृद्धिः - कौन्तेय, रामेय।

\subsubsection{अण् ॥}
तदधीते तद्वेद (अण्) । ४.२.५९
अण् - प्रथम-स्वर-वृद्धिः - शैव, पार्वती।

न य्वाभ्यां पदान्ताभ्यां पूर्वौ ताभ्यामैच् । ७.३.३ इत्यस्मात्, पदान्ते य्/व् सति, तत्-परस्य स्वरस्य वृद्धिः नास्ति, ताभ्याम् ऐच् आगमः।

अतः - वैयाकरण, वैयक्तिक, नैयून्य, वैयाघ्र ।


\subsection{रीतिवाचकं॥}
था-थल् - सर्वथा, पूर्वथा।

\subsection{कारक-वाचकं अव्ययं॥}
\subsubsection{करणे।}
ष्ठ्रन् - त्र-आदेशः - वक्त्रम् - नित्यनपुंसकं।

\subsubsection{अपादाने।}
तसिल् अव्ययं - हरितः।

\subsubsection{अधिकरणे॥}
त्यक्, त्यप् - दक्षिणात्य, अत्रत्य।

त्रल् अव्ययजनकं - अत्र, तत्र, सर्वत्र।

फक्, फञ् - आयन-आदेशः - दक्षिणायन।


\subsection{संख्या-वचने अव्ययानि॥}
अमुक-वारं, अमुक-कृत्वः।

अमुक-शः।

बृहत्-शः।

\subsection{प्राय।}
गत-प्रायः। 

\section{कृदन्तानि॥}
धातुना सह कृत्-प्रत्यय-संयोगात् कृदन्ताः जायन्ते। कीदृशीम् चित् क्रियाम् सूचयति एव। 

सार्वधातुकत्व-आर्धधातुकत्व-प्रभावाः अन्यत्र विवृताः।

\subsection{क्त॥}
\subsubsection{शब्द-योगः॥}
क्त - पतित m. n. पतिता f। वह्-वोढम्। इष्- इष्टम् ()। 

\subsubsection{उपयोगः॥}
कर्तृ-कर्म-विशेषणम् वा/ तथा क्रियासूचकम् उत्पत्तम् पदम् ।

वाक्ये कर्मणि-भावे-प्रयोगयोः कर्मपद-विशेषणे उपयुज्यते - घटेन ग्रामः गतः (अस्ति), घटेन गतम् (ergative - सकर्मकधातोः अपि भाव-वचनम्)। 

\subsubsection{कर्तृ-विशेषणे उपयोगः॥}
कर्तरि-प्रयोगे कर्तृ-विशेषणाय उपयोगः कदाचित् शुद्धः। अकर्मकधातोः तु क्त-कृदन्त-उपयोगः शुद्धः - पुष्पाणि विकसितानि। सकर्मधातुनि गत्यर्थे च शयादिभ्यः धातुभ्यः एव शुद्धः। 

3\.4\.71 AdikarmaNi [ktaH kartari cha] .

3\.4\.72
gatyartha-akarmaka-shliShashI~NsthA.a.asavasajanaruhajIryatibhyashcha

भागवतपुराण-व्यवहारात् नास्ति पूर्णतः प्रामादिकम् 'देवम् प्रणतः अस्मि' इति प्रयोगः।

\subsection{वर्तमति क्रियायाम् कर्तृ-विशेषणम् ॥}
शतृँ - इत्-लोपानन्तरम् त्-आगमः- पतत् m. n. पतन्ती f।  लट्-साधर्म्यम् अस्य प्रत्ययस्य। अतः सार्वधातुकत्वम् अपि।

शानच् - (म्)आन-आदेशः - पतमान m. n. पतमाना f, कुर्वाण।

क्रु - रु-आगमः - भीरु।

घिनुण् - घञ् + इनि - योगिन्, भोगिन्।

तृच् - कर्तृ m. n., कत्री f॥

णिनि - इन्-आगमः - चारिन्।

ण्वुल्, वुञ्, वुन् - चल् -> चालक।

वरच् - ईश्वर।

वन् - ह्रस्व

\subsubsection{साधारणतः पुंसः॥}
'टु-इतः अथुच्' - वेपथुः।

ट - चरः॥

ड - अन्तिम-हलः लोपः - दूरग, मध्यग।

\subsection{समाप्तायाम् क्रियायाम् ॥}
क्तवत् - पतितवत् m. n. पतितवती f। वाक्ये कर्तृपद-विशेषणम् ।

वेदे वनिप् - यज्वन्, धीवन्।

\subsubsection{परोक्षभूते क्रियायाम्॥}
लिडादेशः परस्मै-पदेषु - पेतिवः m. n. पेतुषी f.। लिडादेशः आत्मने-पदेषु - पेतान m. n. पेताना f.।

\subsection{भविष्यति क्रियायाम्॥}
लुटादेश परस्मै-पदेषु - पत्स्यत् m. n. पत्स्यन्ती f.।

लुटादेश आत्मने-पदेषु - पत्स्यमान m. n. पत्स्यमाना f.।

\subsection{कर्म-विशेषणम्॥}
घ - गोचरः।

नङ् - प्रश्नः, यत्नः।


\subsubsection{घुञ्-प्रत्ययः॥}
घुञन्तः पुंलिङ्गे एव साधारणतः प्रयुज्यते। पूर्वस्वरस्य अकारस्य वृद्धिः च  अन्यह्रस्वस्य गुणः भवति।

पठ्तः - पाठः। वद्तः वादः। ऊह्तः ऊहः। पद्तः पादः।

\subsection{करणे ॥}
ष्ठ्रङ् नपुंसकानि - योक्त्र सूत्र नेत्र पत्त्र ।

\subsection{भावे एव॥}
\subsubsection{भाववचने नपुंसकानि॥}
भावे ल्युट् - नित्य-नपुंसकलिङ्गि - अन-आगमः - पठनं, स्नानं। ल्युट् इत्यत्र लकारस्य टकास्य च लोपः भवति (हलन्त्यम् (१.३.३), लशक्वतद्धिते (१.३.८), (तस्य लोपः १.३.९) । युवोरनाकौ (७.१.१) एतदनुसारं ’यु’ इत्यस्य स्थाने ’अन’ इत्यागमः।

णमुल् अव्ययं - कारं, भावं।

\subsubsection{युच्॥}
अनपुंसक-शब्दान् जनयति - तस्मिन्नपि साधारणतः स्त्री-लिङ्गे आप्-प्रत्ययेन सह उपयुज्यन्ते। उदाहरणाय - प्रार्थना, भावना।

सर्वेभ्यः णिजन्तेभ्यः योग्यः। तथा परिगणितेभ्यः अणिजन्त-धातुभ्यः अपि।

युवोरनाकौ (७.१.१) एतदनुसारं ’यु’ इत्यस्य स्थाने ’अन’ इत्यागमः।

\subsubsection{भाववाचने स्त्र्यः॥}
क्तिच्-न्- निय्त-स्त्री-लिङ्गि - कृतिः, मतिः ।

कि - शुच् + कि = शुचि

\subsubsection{योग्यतायां॥}
तव्य - पत्तव्य m. n. पत्तव्या f.। 

अनीयत् - पतनीय m. n. पतनीया f.।

ण्यत्, णिच्-यत् -पत्य m. n. पत्या f.।

उन् - साधु।

\subsubsection{उद्देश-वचनं॥}
तुमुँन् अव्यय-प्रत्ययः - पत्तुम्। वोढुम् । 

\subsubsection{क्रिया-उत्तर-काल-वाचकं अव्ययं॥}
क्त्वा - पतित्वा। उपसर्गः अस्ति चेत् ल्यप् - आगम्य।

\subsection{प्रवृत्ति-वचने॥}
इष्णुच् - सहिष्णु।

उकञ् - पातुक, स्थायुक।

कु - नु-आदेशः - क्षिप्नु।

\subsection{उण्-आदि-प्रत्ययाः॥}
एतेषां सूचिः उणादिसूत्रे। तेषां उपयोगः बहुसीमितः - सूचिकृतधातुभ्यः एव।

उण् - कृ, वा, स्वाद् - कारु, वायु, स्वादु।

\subsection{अन्याः।}
श्रि + क्विप् = श्रीः।

\section{अन्तिमाक्षर-विभजनम् अनव्यायानाम्॥}
\subsection{अन्तिमाक्षरत्व-सङ्ख्या॥}
अमरकोषे ८०\% नामशब्दानि अकारान्ते पुँल्लिङ्गे वा‌ नपुंसकलिङ्गे वा आ-ई-कारान्ते स्त्रीलिङ्गे सन्ति!



\part{पद-वाक्य संस्कारः॥}
'सुप्-तिङन्तम् पदम्॥' इति पाणिनिः।

\chapter{वाक्य-संस्कारः॥}
\section{कारकाणि॥}
(gramatical case)

\subsection{क्रियापदानि प्रधानानि।}
क्रियायाः‌ प्राधान्यम् वर्तते वाक्यावगमने। अनेकानि क्रियासूचकपदानि सन्ति चेत्, ते एव वाक्यकेन्द्राणि - 'रामः पत्रं लिखित्वा क्रीडते'।

\subsection{नामपदानि च तत्-पात्राणि।}
नाम-पदानि ते ये क्रियायां पात्रं वहन्ति। एतानि पात्रानि कारकाणि इति उच्छ्यन्ते। क्रियायाः कर्ता वर्तते एव, कर्मपदानि अपि स्युः।

६ कारकाणि - कर्तृ (nominative), कर्मन् (related to accusative), करण (instrumental), संप्रदान (dative), अपादान (recieving/ ablative), अधिकरण (locative)। तद्विहाय संबन्ध-कारकेति उच्यन्ते शष्टीविभक्तियुक्तानाम् पदानाम् पात्रः।

\subsection{कारक-क्रिया-सूचनम् ॥}
मुख्यक्रियासूचकम् क्रियापदम् तिङन्तम् वा सुबन्तम्। क्रियापदान् विहाय अन्यानि सुबन्तानि।  कारकानुसारम् विभिन्नेषु विभक्तिषु (declension) वा भिन्न-तद्धित-प्रत्यय-युक्तानि भवन्ति नामपदानि।

\subsection{कर्तृ-सूचनम् ॥}
'प्रातिपदिकार्थ-लिङ्ग-परिमाण-वचन-मात्रे प्रथमा॥'

कर्तृकारक-युक्तेन पदेन प्रथमायाम् उत तृतीयायाम् सुबन्त-विभक्त्याम् भवितुम् साध्यम्।

\subsection{कर्म-सूचनम् ॥}
कर्मणि-प्रयोगे प्रथमा। कर्तरि-प्रयोगे द्वितीया। विना-योगे अपि कर्म-सूचनम् च द्वितीया।

भाव-वाचक-तद्धितान्ते कर्म-पदस्य षष्ठी - पुस्तकस्य अध्ययनम्।

\subsection{करणम्॥}
करणे तृतीया। सह-सदृश-योगे अपि - रामेण सह। कारणवचने अपि। अलम्-योगे अपि तृतीया।

\subsection{संप्रदान॥}
संप्रदाने चतुर्थी। कारणवचने अपि। नमः-कारे अपि। केषां-चन धातूनाम् प्रयोगे अपि - कृष्णाय रोचते, क्रुध्यति, असूयति, द्रुह्यति, ईर्ष्यति, कल्पते/ भवति।

\subsection{अपादान॥}
वियोगे/ विभागे पञ्चमी। यस्मात् भयम् वा रक्षणम्, तस्यापि। ऋते-प्रयोगे अपि। प्रभृति-आरभ्य-पूर्व-पर-योगे अपि।

\subsection{संबन्धः॥}
संबन्धे षष्ठी। पृथक्-करणे अपि।

\subsection{अधिकरणम् ॥}
अधिकरणे सप्तमी। विषये सप्तमी। पृथक्-करणे अपि - देशेषु भारतम्। विश्वसिति, स्निह्यति-योगे अपि।

परितः-उभयतः-योगे द्वितीया।

\subsubsection{सन्दर्भ-वचनम्॥}
सतः षष्टी - 'पश्यतः दशरथस्य कौसल्या गतवती'। (उपेक्षा-सूचकम् ।)

सति सप्तमी - 'गच्छति रामे भरतो आगतः'।


\section{कर्म-गणनं।}
क्रिया-पदानि (तथा धातवः) सकर्मकाः (तस्मिन्नपि द्विकर्मकाः) वा अकर्मकाः - इति कर्मपद-गणनात् वक्तुं शक्यते।

\subsection{द्विकर्मकत्वम् ॥}
उदाहरणाय - 'रामः गुरुं सन्देशं ददाति' इति प्रयोगात् दा द्विकर्मकः धातुः इति ज्ञायते। प्रेरणार्थे णिच्-अन्त-धातवः तु द्विकर्मकाः एव भवन्ति। 'रामः गुरुम् कार्यम् कारयति।' णिच्-योगे द्विकर्मकधातवः तु त्रिकर्मक-धातून् जनेयुः।

द्विकर्मक-धातु-युक्त-वाक्ये एकम् प्रधानम् कर्मपदम् वर्तते, अपरम् च 'गौण्यम्' इति। गौण्यकर्मपदम् तु अन्येषु विभक्तिषु भवितुम् अर्हति।

\section{कर्तृ-प्राधान्यम्}
\subsection{कर्तरि-प्रयोगः॥}
कर्तरिप्रयोगे कर्तृपदम् प्रधानम्। क्रियापदम् लिङ्ग-सङ्ख्या-पुरुषेषु कर्तृपदम् अनुसरति। 'रामः ग्रामं गच्छति।, रामः ग्रामम् गतवान्। रामः ग्रामम् गातः' - इत्यत्र 'ग्रामं' कर्मपदम्।

अकर्मक-धातुभ्यः अपि 'पुष्पाणि विकसन्ति।'।

\subsection{कर्मणि-प्रयोगः॥}
कर्मणिप्रयोगे क्रियापदम् प्रधानम्। कर्मणि प्रयोगे तु क्रियापदम् लिङ्ग-सङ्ख्या-पुरुषयोः कर्मपदम् अनुसरति, विशेषम् प्रत्ययम् प्राप्त्वा। 'रामेण ग्रामः गम्यते। रामेण ग्रामम् गतम्' यक्-प्रत्ययः अन्यत्र विवृतः।

\subsection{भावे-प्रयोगे क्रियापद-स्वातन्त्र्यम्॥}
'पुष्पैः विकस्यते। पुष्पैः विकसितम् ।' -इत्यत्र कर्म नास्ति, अतः एषः भावे-प्रयोगः। अकर्मक-धातुभ्यः साधारणं।

सर्वदा प्रथम-पुरुषे एक-वचने तिङन्ताः। क्तान्ताः च नपुंसकलिङ्ग-एकवचने।

\subsection{प्रयोग-तोलनम् ॥}
कर्तरि प्रयोगे कर्म-पदस्य तृतिया-विभक्ति-स्थापने सौलभ्यं, परन्तु धातोः सम्यगि लकार-रूप-दाने कष्टः (धातु-गण-ज्ञानं आवश्यकं)। कर्मणि प्रयोगे धातोः कर्मणि-प्रत्ययेन सह प्रथम-पुरुष-स्थितिः सुलभं (गण-ज्ञानं अनावश्यकं), परन्तु कर्म-पदस्य प्रथम-पुरुषरूपदानाय लिङ्गज्ञानं आवश्यकं।


\section{नामपद-विशेषण-प्रयोगे लिङ्गवचनौ॥}
अनेकदा नामपद-विशेषणानि प्रयोगे विशिष्ठीकृतं नामपदं अनुसरन्ति लिङ्गे च सङ्ख्यायां। कदाचित् तु विशेषण-प्रयोगे तथा न, उदाहरणाय - 'मम पत्नी मम द्रव्यं'॥

\chapter{सुप्-अन्त्यं॥}
\section{स्वरव्यवस्था॥}
'सुप्-प्रत्ययास्तु अनुदात्ता एव । तस्मात् प्रत्ययगता स्वरा: अनुदात्ता: । प्रकृति-शब्दे यत्र उदात्तः स शिष्यते । यत्र तस्य गुणवृद्ध्यादि-एकादेशः स एकादेशः स्वरः उदात्तः ।' इति धनञ्जयः।

\section{नाम-पद-रूपं॥}
सुप् इति विभक्ति-प्रत्ययः (Declension)। सुबन्तं प्रत्येक-नामपदं। सुबन्त-पदं = प्रातिपदिकं + सङ्ख्या-क्रियासंबन्ध-वचकः विभक्तिः। सुबन्तं नाम-पद-विशेषणं वा विशेष्यं वा सर्व-नाम वा सङ्ख्या।

\section{वाक्यसंस्कारे कारकाणि, विभक्तयः च।}
\subsection{विभक्तयः।}
(declensions) नाम-पदं कारकानुसारं विभक्तं । विभिन्न-नाम-पद-वर्गाणां विभिन्न-लिङ्गानां विभिन्न-अन्तिम-अक्षर-युक्त-पदानां विभिन्न-विभक्ति-रूपाणि सन्ति। तेष्वपि विशेष-शब्दानां विभिन्न-शब्दानि सन्ति।

\subsubsection{नपुंसक-पुंलिङ्ग-समानता।}
३-७ विभक्तयः साधारणतया पुंवत् भवन्ति।

\section{सुप्प्रत्ययाः॥}
स्वौजसमौट्छष्टाभ्यांभिस्ङेभ्याम्भ्यस्ङसिभ्यांभ्यस्ङसोसाम्ङ्योस्सुप्॥ (४।१।१)

१। सु, औ, जस्, 

२। अम्, औट्, शस्, 

३। टा, भ्याम्, भिस्, 

४। ङे, भ्याम्, भ्यस्, 

५। ङसि, भ्याम्, भ्यस्, 

६। ङस्, ओस्, आम्, 

७। ङि, ओस्, सुप्, 

(स्वौजस्॰सुप्, समाहारः द्वन्द्वः।)

'न विभक्तौ तुस्माः' इत्येतस्मात् तु-स्-म् एते इत् न।

'रोः सुपि' = 'रोरेव विसर्गः सुपि।'

\section{सु-औ-जस्॥}
\subsection{नपुंसके वैशिष्ट्यम् ।}
सु-अमोः नपुंसकात् (लुक्)॥ इत्यस्मात् सु-लोपः।

\subsection{अ-आ-कारान्तेषु॥}
सोः विसर्गः।

पुं- रामः रामौ रामाः।

नपुंसक - गृहं गृहे गृहाणि।

पुं - विश्वपा -- विश्वपा: विश्वपौ विश्वपः ... इत्यादीनि रूपाणि ।

स्त्रीलिङ्गे रमा-शब्दः - आप्-अन्तेषु 'सु' लोपः। अतः गायिका। रमा रमे रमाः।

\subsection{इ-ई-कारान्तेषु॥}
पुंलिङ्गे हरि-शब्दः - हरिः हरी हरयः।

पुं - सुधी -- सुधी: सुधियौ सुधियः ... इत्यादीनि रूपाणि ।

स्त्रीलिङ्गे तथैव।

नपुंसकलिङ्गे वारि - वारि वारिणी वारीणि । (सु-अमोः नपुंसकात् (लुक्)॥ इत्यस्मात् सु-लोपः। )

स्त्रीलिङ्गे ईकारान्ते ङि-अन्ते सति 'सु' इत्यस्य लोपः - नदी नद्यौ नद्यः।

स्त्रीलिङ्गे ईकारान्ते कृदन्ते सति - श्री, लक्ष्मी-शब्दः - लक्ष्मीः लक्ष्म्यौ लक्ष्म्यः। 

\subsection{उ-ऊ-कारान्तेषु॥}
पुंलिङ्गे गुरु-शब्दः - गुरुः --गुरू-- गुरवः।

नपुंसकलिङ्गे तनु-शब्दः - तनु --तनुनी-- तनूनी।

स्त्रीलिङ्गे पुंवत्।

स्त्रीलिङ्गे ऊकारान्ते ङि-प्रत्यय-अन्ते सति अपि सु-लोपः नास्ति- वधूः, चञ्चूः। 

\subsection{ऋकारान्तेषु॥}
पुंलिङ्गे पितृ- पिता पितरौ पितरः।

परन्तु संबन्ध-असूचके - कर्ता कर्ताारौ कर्तारः। 

स्त्रीलिङ्गे पुंवत्।

\subsection{ऐ-औ-कारान्तेषु॥}
एकः एव ऐ-अन्त-शब्दः वर्तते - पुंलिङ्गे रै - रैः रायौ रायः।

एकः एव ओकारान्त-शब्दः वर्तते। पुंलिङ्गे गो - गौः गावौ गावः।

\subsection{पुंलिङ्गे हलन्त्यम्॥}
पुंलिङ्गे बलवत्-शब्दः - बलवान् बलवन्तौ बलवन्तः।

पुंलिङ्गे आत्मन्- आत्मा आत्मानौ आत्मानः।


श्रेयस् - श्रेयान् श्रेयांसौ श्रेयांसः।

\subsection{नपुंसके हलन्त्यम् ॥}
श्रेयस् - श्रेयः श्रेयसी श्रेयांसी।

\section{अम्-औट्-शस्॥}
\subsection{नपुंसके वैशिष्ट्यम् ।}
सु-अमोः नपुंसकात् (लुक्)॥ इत्यस्मात् अम्-लोपः।

\subsection{प्रथमा-विभक्ति-सामानताः।}
पुंलिङ्गे द्वितीय-सङ्ख्यायां प्रथमा-विभक्ति-वत् रूपं।

स्त्रीलिङ्गे अनेक-सङ्ख्यारूपयोः प्रथमा-विभक्ति-वत् रूपं।

नपुंसकलिङ्गशब्देषु प्रथम-द्वितीय-विभक्त्यौ समानौ। 

\subsection{अम्॥}
अ-आ-इ-ई-उ-अन्त-शब्देषु प्रथमा-विभक्ति-रूपाय मकार-संयोगः - उदाहरणाय रामं, रमां।

अन्यशब्देभ्यः अम्कार-संयोगः। उदाहरणाय रायं गावं आत्मानं।

\subsection{शस् पुंलिङ्गे}
अ-आ-इ-ई-उ-अन्त-शब्देषु प्रथमा-विभक्ति-रूपाय अन्कार-संयोगः - उदाहरणाय रामान्, गुरून्।

अन्यशब्देषु प्रथमा-विभक्ति-वत् रूपं।

\section{टा-भ्याम्-भिस्॥}
\subsection{भ्याम्॥}
शब्दाय भ्यां-कारस्य संयोगः।

\subsection{भिस्॥}
अकारान्ते राम- रामैः इति भवति।

अन्यशब्देषु शब्दायाय भिः-कार-संयोगः।

\subsection{टा॥}
अकारान्ते राम - रामेन इति भवति।

आकारान्ते रमा - रमया इति भवति।

इकारान्ते पुंलिङ्गे हरि - हरिणा इति भवति।

उकारान्तेषु गुरु - गुरुणा इति भवति, तरु - तरुना इति च।

अन्य-शब्देषु अन्ते आकार-संयोगः।

\section{ङे-भ्याम्-भ्यस्॥}
\subsection{भ्यस्॥}
अन्यशब्देषु शब्दायाय भ्यः-कार-संयोगः।

\subsection{ङे॥}
अकारान्ते राम - रामाय इति भवति।

आकारान्ते रमा - रमायै इति भवति।

इकारान्ते पुंलिङ्गे हरि - हरये इति भवति।

इ-ई-कारान्ते स्त्रीलिङ्गे ऐ-कार-संयोगः।

उकारान्तेषु गुरु - गुरवे इति भवति।

ऐकारान्तेषु रै - राये इति भवति।

अन्य-शब्देषु अन्ते एकार-संयोगः।

\subsection{ङसि-भ्याम्-भ्यस्॥}
\subsection{ङसि॥}
रामात्।

\section{ङस् ओस् आम्}
रामस्य रामयोः रामेषां।

\tbc

\section{७। ङि, ओस्, सुप्॥}
रामे रामयोः रामेषु।

\tbc

\section{संभोदने॥}
(vocative)

अनेक-वचनयोः प्रथमाविभक्तिवत्।

ई-कारान्त-स्त्र्याम् - ह्रस्वः - यथा भवती-शब्दस्य भवति इति - यथा भवति भिक्षाम् देहि।

\tbc

\section{सर्वनाम-पदानि॥}
\subsection{तत्॥}
केवलम् प्रथमाविभक्तौ विशेषः - अन्यत्र त-ता-प्रातिपदिकवत्।

पुम् - सः तौ ते। स्त्री - सा ते ताः। नपुम्सक- तत् ते तानि।

\subsection{इदं-विभक्तयः॥}
इयं इमे इमाः। इमां इमे इमाः। अनया आभ्यां आभिः। अस्यै आभ्यां आभ्यः। अस्याः आभ्यां आभ्यः। अस्याः अनयोः आसां। अस्यां अनयोः आसु।

पुँल्लिङ्गे- अयं इमौ इमे। शेषं सुलभं - स्त्रीलिङ्ग-रूपं पश्यतु।

नपुंसकलिङ्गे- इदं इमे इमानि। इदं इमे इमानि। शेषं पुंवत्।

\subsection{अदस् ॥}
पुं - असौ अमू अमी । 

नपुं - अदः अमू अमूनि।

\subsection{अहं त्रिलिङ्गेषु समानः॥}
अहं आवां वयं। मां अवां अस्मान्। मया आवाभ्यां अस्माभिः। मह्यम्/ मे आवाभ्यां अस्मभ्यं। मत् आवाभ्यां अस्मत्। मम आवयोः अस्माकं। मयि आवयोः अस्मासु।

\subsection{सङ्ख्याः॥}
एक, एका । द्वौ, द्वे, द्वे। त्रयः, त्रिणि, तिस्रः। चत्वारः चत्वारि चतस्रः। तत्-अधिक-सङ्ख्यानाम् त्रिलिङ्गेषु समानरूपत्वम्, प्रातिपदिकानि पञ्चन्, षष्, सप्तन्, अष्टन्, नवन् इत्यादिनि ।

\chapter{तिङ्-अन्त्यं॥}
\section{तिङ्-गणः॥}
तिङ् इति पुरुषप्रत्ययः। क्रियापदं तिङन्तं। क्रियापद = क्रियादिशा-सङ्ख्या-पुरुष-काल-प्रकार-भोदनाय विविधैः प्रत्ययैः विकृतः धातुः।

वर्तमान-काल-वाचकानाम् १८ प्रत्ययानाम् श्रेण्याः सुप्वत् प्रथम-अन्तिम-अक्षराभ्याम् तिङ् इति नाम कृतम्।

\section{सार्वधातुक-विकरण-प्रत्यय-गणाः॥}
\subsection{प्रभावः॥}
शपः सार्वधातुकत्वात् जायमानाः प्रभावाः अन्यत्र विवृताः।

'(शपि) श्रवः शृ च।' इति पाणिनिः। तस्मात् श्रु धातोः श्रुति, परन्तु सार्वधातुके प्रत्यये परे - शृणोति वत् ।

\subsection{आदेशाः।}
धातु-विकरणेषु दश-गणाः सन्ति, एकैकस्य अपि गणस्य भिन्नः विकरण-प्रत्ययः। अधिकाः धातूनाम् विकरणानि तु १,२,३,४ गणेषु एव वर्तन्ते। काश्चन धातवः अनेकेषु गणेषु भवन्ति।

१ भ्वादि - शप् - अ-आगमः, उकारान्तानां संप्रसारणं। 'कर्तरि शप्॥' तथा अनुबन्ध-लोपयद्भ्यः (' हलन्त्यम्॥ लशक्वतद्धिते॥ तस्य लोपः॥') पाणिनि-यसूत्रेभ्यः।

२ अदादि -लुक्- ''। 'अदिप्रभृतिभ्यः शपः (लुक्)॥' इत्येतस्मात् शपः लोपः।

३ जुहोत्यादि - श्लु - '' - धातोः द्वित्त्वं। 'जुहोत्यादिभ्यः (शपः) श्लुः॥' इत्येतस्मात्   शपः लोपः। 'श्लौ॥' इत्येतस्मात् धातोः द्वित्त्वम्।

४ दिवादि - श्यन् - य-आगमः - अनुबन्धलोपात्।

५ स्वादि - श्नुः - नो-नु-आगमः।

६ तुदादि - शः - भ्वादि-वत्।

७ रुधादि - श्नम् - न-न्-आगमः - अनुबन्धलोपात्।

८ तनादि - उः -ओ-उ-आगमः ।

९ क्र्यादि - श्ना - ना-नी-आगमः।

१० चुरादि  - णिच् - अय-आगमः (णिच्-प्रत्ययः), पूर्व-स्वरस्य-गुणः।

\section{उपगणाः॥}
\subsection{क्रिया-फल-दिशा।}
क्रियायाः प्रभाव-दिशा भवति - परस्मै वा आत्मने। भेदोदाहरणं - 'कार्यं करिष्यामि।' च 'कार्यं करिष्ये।' च 'वृक्षः वर्धते।'। एतत् सकर्मक-अकर्मकत्वं न अवलम्ब्यते। तदनुसारम् परस्मै/आत्मने-पदिनः इति विभक्ताः।

क्वचन धातूनाम् फलः उभयाभ्याम् वस्तुभ्याम् अपि विकल्पेन भवति - ते उभयपदिनः।

क्रिया-फल-दिशा-अनुसारम् भवति तिङ्प्रत्यय-चयनम्।

'1-3-72 स्वरितञितः कर्त्रभिप्राये क्रियाफले॥ शेषात् कर्तरि परस्मैपदम् ॥' इति पाणिनिः।

\subsection{लकार-विभक्तिः व्याकरणसङ्ज्ञानुसारम्॥}
एतेषां प्रत्ययानां नामानां प्रारम्भे लकारे सति, ते लकाराः इति उच्छ्यन्ते। तेषां अन्तिमाक्षरः (ट् वा ङ्) इत्-संज्ञाम् आप्नोति। तदनुसारम् लकाराः ङितः वा टितः। मध्यमः स्वरः अड्-प्रत्याहृतेभ्यः स्वीकृताः।

\subsection{लकार-विभक्तिः भोदनानुसारम्॥}
लकाराः ६ कालवाचकाः (time/ tense), ४ प्रकारभोदकाः (mood)।

\subsection{लकार-विभक्तिः विकरणानुसारम्॥}
अन्या विभक्तिः - ४ सार्वधातुकाः च ६ आर्ध-धातुकाः इति।
सार्वधातुकेषु लकारेषु रूपः - धातुः + विकरण-प्रत्ययः + पुरुष-सङ्ख्या-वाचक-प्रत्यय।

आर्धधातुकेषु कदाचित् (सेट्-धातुषु, न अनिट्-धातुषु) इकार-संयोगम् विहाय विकरण-प्रत्ययः न वर्तते।

\section{कर्मणि वा भावे यक्-प्रत्ययः।}
कर्तरि/ कर्मणि/ भावे इत्यादिषु प्रयोगेषु वाक्यसंस्कारः अन्यत्र विवृतः। एतत् सनादिगणे अन्तर्भूतात् यक्-तः भिन्नः। एतस्य योगेन जातः शब्दः धातु-सञ्ज्ञाम् न प्राप्नोति - समान-अर्थ-वचनात्।

\subsection{शब्द-जनन-प्रत्ययः॥}
यक्-प्रत्ययः - धातु + य्। पदसंस्कारे आत्मने-पदत्वं।

तिङन्तजनने ङे-प्रत्ययम् स्वीकरोति। लिङ्लोटौ अपि साधारणौ - उदाहरणाय - भवता पठ्यताम् पुस्तकम् । (भवान् पठ्यताम् इत्येतत् असाधुः।) त्वया पठ्यस्व पुस्तकम् ।

\section{धातु-द्वित्त्वम् अभ्यासः॥}
'पूर्वोऽभ्यासः॥' इति पाणिनिः।

उदाहरणाय बिभेति।

'ह्रस्वः॥' इति पाणिनिः - अतः द्वित्त्व-स्वरः ह्रस्वः।

'अभ्यासे चर्च॥' इत्येतस्मात् झलः स्थाने चरः वा जसः 'स्थानेऽन्तरतमः॥' अनुसारम् चित्वा।

\section{स्वरव्यवस्था॥}
"यदि तिङन्तं पदम् अतिङन्तात् पदाद् उत्तरं तर्हि तस्य सर्वानुदात्तः।" इति धनञ्जयः। 'प्रचो॒दया॑त्' इत्यादिषु आक्षेपः।

\chapter{काल-वाचकाः लकाराः॥}
\section{काल-विवरणं॥}
वर्तमान-आह्नस्य कालः अद्यतनः, तद्भिन्नोऽनद्यतनः। अद्यतनः इत्युक्ते इदानीन्तन इति न!

बहु-पूर्वकालिकं वा कुड्य-कुट-अन्तरितं वृत्तं परोक्षम्। कृतस्य स्मरणे कर्तुरत्यन्तापह्नवेऽपि च, कर्मकर्त्रोरदृश्यत्वे त्रिषु विद्यात् परोक्षताम्।

\section{वर्तमान-कालवाचक-लकाराः लँट्।}
एषः सार्वधातुकः लँकारः। 'वर्तमाने लट्॥ तिङ्शित्सार्वधातुकम्॥' इति पाणिनिः॥

परस्मैपदिनः॥

तिप् तस् झि – भवति भवतः भवन्ति

सिप् थस् थ – भवसि भवथः भवथ

मिप् वस् मस् – भवामि भवावः भवामः  

आत्मनेपदिनः॥ 'तङानावात्मनेपदम्॥' इति पाणिनिः।

त् आतम् झ – कम्पते कम्पेते कम्पन्ते

थास् आथाम् ध्वम् – कम्पसे कम्पेथाम् कम्पध्वम्

इट् वहि महिङ् – कम्पे कम्पावहे कम्पामहे

\subsection{तिङन्त-सिद्धिः॥}
भू + लँट्  (अजनुनासिकमित्, हलन्त्यम्, तस्य लोपः) = भू + ल् (तिङ्-सूत्रम्) = भू + शब् + तिप् (?)= भवति।  

\subsection{जुहोत्यादि-गण-वैशिष्ट्यम् ।}
ददाति ददतः ददति (ददन्ति न)। बिभेति बिभ्यतः बिभ्यति। (शि) शेते शयाते शेरते ।

\subsection{भूतकाल-वाचनं।}
लट्-उपयुज्य भूतकाल-वचनं : "सः तदा हसति स्म" वाक्य-इव। 

\section{भूत-काल-वाचकाः।}
\subsection{शुद्धे भूते लुँङ्॥}
(पाश्चात्यानुसारं Past aortist/ perfect, noncommittal)।
सदा धातोः पूर्वं 'अ' संयुजति - उदाहरणौ - तिष्ट् अतिष्ट् भवति। परन्तु, उपसर्गस्य पूर्वं न स्थितः अकारः- उदाहरणार्थं उत्तिष्ट् उततिष्ट् भवति।

एषः आर्धधातुकः लँकारः।

अभूत् अभूतां अबूवन्, अभूः अभूतं अभूत, अभूवं अभूव अभूम॥

अट्- सेट्-धातुः। आटीत् आटिष्टाम् आटिषुः, आटीः  आटिष्टम्  आटिष्ट, आटिषम्  आटिष्व  आटिष्म॥

\subsection{लँङ् अनद्यतनभूतः}
एषः सार्वधातुकः लँकारः। (पाश्चात्यानुसारं past imperfect: action may be incomplete)।

'तस्-थस्-थ-मिपां तां-तं-त-अमः॥' इति पाणिनिः 'ङित्'-लकारेषु। 

\subsubsection{परस्मै-पदे।}
अभवत् अभवतां अभवन्, अभवस् अभवतं अभवत, \\अभवं अभवाव अभवाम।

\subsubsection{आत्मने-पदे।}
अवर्धत अवर्धेतां अवर्धन्त, अवर्धथाः अवर्धेथां अवर्धध्वं, अवर्धे अवर्धावहि अवर्धामहि।

\subsection{लिँट् परोक्षभूतः}
एषः आर्धधातुकः लँकारः।

\subsubsection{परस्मै-पदे।}
बभूव बभूवतुः बबूवुः, बभूविथ बबूवथुः बभूव, बभूव बभूविव बभूविम।

\section{भविष्यत्-काल-वाचकाः।}
\subsection{लुँट् अनद्यतन-भविष्यति।}
एषः आर्धधातुकः लँकारः - 'स्यतासीँ लृँलुँटोः॥' इति 'कर्तरि शप्‌' नियम-अपवादात्।

\subsubsection{परस्मै-पदे।}
'लुटः प्रथमस्य डारौरसः॥'

'स्यतासीँ लृँलुँटोः॥' इति पाणिनिः। अतः‌ धातोः अन्ते प्रथमं तासिँ प्रत्ययः स्थापितः। 

भविता भवितारौ भवितारः, भवितासि भवितास्थः भवितास्थ, \\
भवितास्मि भवितास्वः भवितास्मः।

\subsection{लृँट् शुद्धे भविष्यति।}
एषः आर्धधातुकः लँकारः - 'स्यतासीँ लृँलुँटोः॥' इति 'कर्तरि शप्‌' नियम-अपवादात्।

'स्यतासीँ लृँलुँटोः॥' इति पाणिनिः। अतः‌ धातोः अन्ते प्रथमं स्य('इष्य्') प्रत्ययः स्थापितः। उदाहरणं 'भवति' स्थले 'भविष्यति' उपयुक्तः।

\chapter{प्रकार-भोदकाः लकाराः॥}
\section{विधि-प्रार्थनादिषु॥}
\subsection{सूत्रः॥}
'विधि-निमन्त्रण-आमन्त्रण-अधीष्ट-संप्रश्न-प्रार्थनेषु लिँङ्॥ लोँट् च॥ आशिषि लिँङ्लोँटौ ॥ लिँङ्-अर्थे लेँट्॥'  इति पाणिनिः। सन्देहे अपि लिँङ् ।

विधिः प्रेरणम्। निमन्त्रणम् नियोगकरणम्। आमन्त्रणम् कामचारकरणम्। अधीष्टः सत्कारपूर्वको व्यापारः। सम्प्रश्नः सम्प्रधारणम्। प्रार्थनम् याञ्चा। विध्यादिष्वर्थेषु धातोः लिङ् प्रत्ययो भवति।


\subsection{लोँट्।}
एषः सार्वधातुकः लँकारः।(पाश्चात्यैः imperative इत्युच्च्यते।)

\subsubsection{परस्मै-पदे।}
भवतु भवतां भवन्तु, भव भवतं भवत, भवानि भवाव भवाम।

'किं करवाणि इदानीं?' इत्येवम् उत्तमपुरुषे संप्रश्नार्थे प्रयोगः।

\subsubsection{आत्मनेपदे}
सेर्ह्यपिच्च॥

\subsection{विधिलिँङ् सन्देहे।}
एषः सार्वधातुकः लँकारः।

'लिङः सलोपोऽनन्‍त्‍यस्‍य' - इति पाणिनिः॥

'तस्-थस्-थ-मिपां तां-तं-त-अमः॥' इति पाणिनिः 'ङित्'-लकारेषु। 


\subsubsection{परस्मै-पदे।}
भवेत् भवेतां भवेयुः, भवेः भवेतं भवेत, भवेयं भवेव भवेम।

\subsubsection{आत्मने-पदे।}
वर्धेत . .

\subsection{सन्देहं प्रकटितुं 'स्यात्' उपयोगः।}
'एषः उत्तित्वा सन्ध्यां वन्दितवान् स्यात्।'

\subsection{आशिर्लिँङ् आशीः।}
एषः आर्धधातुकः लँकारः।

\subsubsection{परस्मै-पदे।}
भूयात् भूयास्तां भूतासुः, भूयाः भूयास्तं भूयास्त, भूयासं भूयास्व भूयास्म।

\section{लृँङ् भूते भाविनि च क्रियायां अनिष्पत्तौ}
(Conditional)। 'वर्षा अपतिष्यत् चेत् धान्यजननं अभविष्यत्!'

एषः आर्धधातुकः लँकारः - 'स्यतासीँ लृँलुँटोः॥' इति 'कर्तरि शप्‌' नियम-अपवादात्।

'स्यतासीँ लृँलुँटोः॥' इति पाणिनिः। अतः‌ धातोः अन्ते प्रथमं स्य('इष्य्') प्रत्ययः स्थापितः। 

'तस्-थस्-थ-मिपां तां-तं-त-अमः॥' इति पाणिनिः 'ङित्'-लकारेषु। 


अनद्य-तन-भूत-लँकारः उपयुज्यते। उदाहरणाय - अभविष्यत् अभविष्येतां अभविष्यन्, ? ? ?, अभविष्यम् अभविष्याव अभविष्याम।


\chapter{पदे अक्षर-परिवर्तनम् ॥}
\section{नो णः॥}
'रषाम् नोणः॥' इति पाणिनिः। 'ऋ-कारस्यापि' इति कात्यायनः।

नकारः पदान्ते न सति Regular expression भाषायाम् - .*(र्,ष्,ऋ)[हयवरट् लण् कु पु अच्]*न्[अच्] इत्यत्र नोः णः। उदाहरणाय, 'रामेण' च 'रमन्', न तु 'रामेन' वा 'रमण्'।

\part{शब्द-ज्ञानम् ॥}
\chapter{प्रातिपदिकानि।}
आप् जल नीर। उदधि सागर नदी सर ताल।

पृथिवि तेजस् अनल जातवेदः। वायुः अनिल। आकाशः व्योमन्।

\section{जीविनः॥}
स्त्री-पात्राणि- वन्ध्या ।

मृगेषु - मत्स्य मीन तिमिङ्गल। मकर। वराह सुकर। कच्छप कूर्म। श्वान शुनक। हरिण। गौ धेनु वृषभ। गज करिन् सिन्धुर हस्तिन्। अश्व वाजि गदर्भ। सिंह हरि व्याघ्र चित्रक।

पक्षिषु - खग विहङ्ग। पीक काक वायस कुक्कुट मयूर पारिजात गृध्र हंस कालहंस उलूक।

वृक्षः पादपः तरु वनस्पति सस्य लता वेणिः । पद्म कमल सरोज। नारिकेल कदली आम्र।


\subsection{अङ्गानि।}
मूर्धा कर्णम् मुखम् जिह्वा दन्त। हनु कपोल गण्ड। चिबुक। अक्षिण् ।

कण्ठ ग्रीवा कुञ्जा वक्षस् उरस् नाभि प्रजननम् लिङ्ग योनि ।

बाहु ऊरु जानु जङ्ग पाद।

स्कन्ध 

जटर उदर हृदय।

अस्ति स्नायु पीन।

\section{आकृति-विशेषणानि।}
उन्नत वामन। गभीर गाध। स्थूल कृष। दीर्घ ह्रस्व/ लघु।

\section{उपकरणानि}
परुष खड्ग चाप धनुस् (त्रि)शूल

सङ्गीते डक्का डमरू वंशी वीणा 

\section{हलन्त-नपुंसकानि।}
वयस् मनस् तमस् रजस् तेजस्

जन्मन् कर्मन् नामन् प्रेमन् अक्षिन्


\chapter{धातुपाठः॥}
\section{भोगे ॥}
खादति भक्षति पिबति भुङ्क्ते ग्रसति अशति॥

\subsection{प्रयत्ने च शक्त्याम्॥}
अर्हति शक्नोति

यतते श्रमति।

\section{चिन्तने च भावनासु॥}
मृशति चिन्तयति जानति वेत्ति स्मरति ऊहते कल्पते शपते॥

नन्दति (वि)रमति जुषते मोदते॥

विन्दति इच्छति एषते/ इष्यति काङ्क्षते कण्ठति कामयते आशते

विश्वसिति।

वृणोति चयति चुनोति शपते ।

शिक्षते

लिखति पठति अङ्कति

(शुच्) शोचति खिद्नाति

स्वपिति निद्राति जागरति

लोलति लालति लसति क्रीडते (अकर्मकम्) स्पर्धते

सोढति सहति

\subsection{संबन्धेषु}
कुप्यते कृध्यते

रोचते (अकर्मकम्) स्निह्यति दयते॥

चोदति हिनोति (विहितम्) (प्र)शंसति कलयति॥

सहति क्षमते सेवति।

बिभेति।

\section{आङ्गिकम् ॥}
\subsection{इन्द्रियाणि च मानम्॥}
ईक्षते पश्यति

(अव)लोक्यते दृष्यते॥

श्रुणोति 

तोलयति मेयति 

श्वसिति। जिघ्राति।

उत्/नि-मिषति

\subsection{नाद-भाषणे}
वदति वचति बणति भाषते ब्रवीति कथयति चर्चयति क्रोषति

निन्दति तर्जयति स्तोति श्लाघते

गायति गर्जति नादति घोषति क्रोशति रवति

पृच्छति याचति/ते वाञ्चति

मन्त्रयति भोदति दधाति दिशति ।

आह्वयति 

हसति रोदिति 

भजति॥

\subsection{करणे}
करोति/कुरुते कृषति नुदति स्पृषति खनति॥

क्षिपति

\subsection{गतौ च स्थाने}
स्थाति स्तभति। गाधति तिष्टति सीदति। सदति विषदति।

चलति झम्पति सरति गच्छति याति सर्पति (उत्)अयति धावति पलायति एति।

विशते शयति/शेते श्लथति

अटति भ्रमति 

रोहति

कम्पते कुर्दते खञ्जति दोलयति॥

पतति गलति।

तरति

खण्डते युङ्क्ते

\section{प्रकृत्याम् च तत्-प्रकटने॥}
वर्षति स्रुवति।

स्फुरति ज्वलति भाति दीप्यते दिव्यते ।

जृम्भते (वि)कसति (उत्)घटति स्फरति।

\subsection{सत्तायाम् ॥}
अस्ति वर्तते भवति भासते भाजते

मरति।

वसति श्रयति

\subsection{सृजने ॥}
जायते वर्धते एधते बृंह (प्र)ईर्ते।

सूते (प्र)सवति।

वपति सिञ्चति प्रोक्षति क्षालयति॥

लयति व्ययति

\section{अन्योपयोगे सेवायाम् च}

शोधति पावयति।

\subsubsection{विकृतौ ॥}
श्यायते/ शीत (श्यैङ्) तपति।

मथति/ मन्थनम् ।

\subsection{(वि)योजने}
रचति

युङ्क्ते? लिम्पति सीवति॥ घटते ।

भञ्जति तक्षति रुजति विपुलति

(कॄ) किरति/ अकरीत्

\subsection{अन्यस्य बाधादिषु}
पीडति बर्हते बाधते रोधति यमति बध्नाति॥

ताडयति व्रणति विधति हन्ति ध्वंसति कशति सूदते।

वर्जति (नि)वारयति 

जयति।

\subsection{अन्यस्य सेवायाम् ॥}
सेवते।

रक्षति गोपयति अवति गुहति छादयति कुचति कुशति।

\subsection{अन्यात् सहाय्ये}
लम्बते

\subsection{द्रव्यस्य}
मार्जयति क्षालति अञ्जति (नि)मज्जति (अभि)षेक लिम्पति॥

लुडति

\section{संप्रदान-अपादानयोः॥}
ददाति यच्छति

लभते आप्नोति॥

क्रेणोति चोरयति हरति 

नयति वहति भर्ति 

पूरयति (वि)अयते क्रेति।

गृहति भर्ति।

मुञ्चति ।

त्यजति जुहोति हिनोति

नमति वन्दते यजति पूजयति (परि)वेषयति।

\chapter{अव्ययं।}
2-4-82 अव्ययादाप्सुपः॥ - अव्ययादुत्तरस्य आपः सुपश्च लुग् (लोपः) भवति। 'pratyayalope pratyayalakShaNam' इत्यस्मात् सुबन्तत्वम् ।

प्रयोग-सौलभ्यं! मादरयः यदा तदा अद्य इत्यादयः।

तद्धित-कृदन्त-समासान्त-अव्ययानि अन्यत्र विवृतानि।

\section{क्रिया-विशेषणानि॥}
शनैः।



\section{निपाताः॥}
(Exclamations/ utterences)

प्रादयः निपाताः॥

\subsection{उपसर्गाः॥}
क्रियापदेन सह योगे क्वचन निपाताः भवन्ति उपसर्गाः। उपसर्गात् धातोः अर्थः अन्यत्र नीयते प्रहार-विहार-संहार-वत्। परन्तु उपसर्गाः सुबन्ताः अव्ययानि एव।

\tbc

वि-संयोगे ङे-प्रत्ययः तिङ्-गणीयः उपयुज्यते साधारणतः - अतः विजयते (यद्यपि जयति साधुः)।

\section{प्रश्न-उत्तराभ्यां।}
किल, खलु, नु ननु।

क्व = kutra।

कदा।

नाम।

\subsection{चित्-चन-प्रयोगः अस्पष्टतायां।}
कश्चन, काचित्, केनचित्, कस्मैचित्, क्वचित् इत्यादयः।

\section{काल/क्रम-वचने॥}
अद्य श्वः परश्वः ह्यः परह्यः।

पश्चात्। प्राक्।

सद्यः संप्रतिः इदानीं।

पूर्वेद्युः परेद्युः परेद्यवि अन्येद्युः उभयेद्युः।

शीघ्रं, अचिरात्, अचिरेण, अचिरम्।

युगपत्।

अथ।

पुनः।

\section{संख्या-मात्र-वचने॥}
तन्निमित्तानि अव्ययानि अन्यत्र लिखितानि।

मनाक् ईषत् ।

सकृत्, असकृत्।

पृथक्।

पुनः मूहुः भूयः॥

\section{(वि)केन्द्रयणे॥}
अपि एव

\part{पाठनम्, पठनम् च ॥}
\chapter{भावनम् ॥}
कक्ष्यायाम् आनन्दस्य, आत्मीयतायाः, विकासस्य अनुभवः। संस्कृतभारत्या सह समरसत्वस्य अनुभवः। अतः भाषाशिक्षणम्, कार्यशिक्षणम् चापि। संस्कृतवाचकेषु स्नेहवर्धनम् शिबिरैः च दूरवाणी-संपर्केण।

कार्यशिक्षणे शिक्षकेण सरल-आदर्शव्यवहारः, अल्पोपदेशः, सन्निवेशवर्णनम् च, तस्मात् विचार-मन्थनम् , भाव-तरङ्गोत्पादनम्। शिक्षार्थिने प्रति-पक्षम् लघु-कार्यस्य (उदाहरणाय प्रचारस्य) दानम् । अनौपचारिकम् मेलनम् क्रीडा-हास्यादि-संयुक्तम्।

संभाषण-सन्देशात् लघुकथा-पाठनान्तरम् ग्राहकत्वे वचनम् ।

\chapter{क्रमः।}
श्रवण-वचन-पठन-लेखन-युक्तम् सोपानम् । श्रवण-वचनानतरम् व्याकरण परिचयः क्रमेण। शिक्षार्थीनाम् रुचीन् अनुसरेत् ।

\section{प्रथमे स्तरे प्रयोग-कौशलम् ।}
संस्कृतेनैव वचनस्य अनुरोधः। संस्कृतेनैव चिन्तनस्य अपि (न्यूनातिन्यूनम् भारतीयभाषासु चिन्तनस्य) अनुरोधः।

दैनिकस्य अभ्यासस्य, संभाषणे उपयोगस्य च अनुरोधः - दूषितः एव स्यात् प्रयोगः प्ररम्भे, क्रमेण प्राकृतम् संस्कृतत्वम् आप्नुयात्। 


\end{document}
