\documentclass[oneside, article]{memoir}

\usepackage{amsmath, amssymb}
\usepackage{hyperref, graphicx, verbatim, listings, multirow, subfigure}
\usepackage{algorithm, algorithmic}
% \usepackage[bottom]{footmisc}
\lstset{breaklines=true}
\setcounter{tocdepth}{3}

% Lets verbatim and verb environments automatically break lines.
\makeatletter
\def\@xobeysp{ }
\makeatother
% \lstset{breaklines=true,basicstyle=\ttfamily}

% Configuration for the memoir class.
\renewcommand{\cleardoublepage}{}
% \renewcommand*{\partpageend}{}
\renewcommand{\afterpartskip}{}
\maxsecnumdepth{subsubsection} % number subsections
\maxtocdepth{subsubsection}

\addtolength{\parindent}{-5mm}
% Packages not included:
% For multiline comments, use caption package. But this conflicts with hyperref while making html files.
% subfigure conflicts with use with memoir style-sheet.

% Use something like:
% % Use something like:
% % Use something like:
% \input{../../macros}

% groupings of objects.
\newcommand{\set}[1]{\left\{ #1 \right\}}
\newcommand{\seq}[1]{\left(#1\right)}
\newcommand{\ang}[1]{\langle#1\rangle}
\newcommand{\tuple}[1]{\left(#1\right)}

% numerical shortcuts.
\newcommand{\abs}[1]{\left| #1\right|}
\newcommand{\floor}[1]{\left\lfloor #1 \right\rfloor}
\newcommand{\ceil}[1]{\left\lceil #1 \right\rceil}

% linear algebra shortcuts.
\newcommand{\change}{\Delta}
\newcommand{\norm}[1]{\left\| #1\right\|}
\newcommand{\dprod}[1]{\langle#1\rangle}
\newcommand{\linspan}[1]{\langle#1\rangle}
\newcommand{\conj}[1]{\overline{#1}}
\newcommand{\gradient}{\nabla}
\newcommand{\der}{\frac{d}{dx}}
\newcommand{\lap}{\Delta}
\newcommand{\kron}{\otimes}
\newcommand{\nperp}{\nvdash}

\newcommand{\mat}[1]{\left( \begin{smallmatrix}#1 \end{smallmatrix} \right)}

% derivatives and limits
\newcommand{\partder}[2]{\frac{\partial #1}{\partial #2}}
\newcommand{\partdern}[3]{\frac{\partial^{#3} #1}{\partial #2^{#3}}}

% Arrows
\newcommand{\diverge}{\nearrow}
\newcommand{\notto}{\nrightarrow}
\newcommand{\up}{\uparrow}
\newcommand{\down}{\downarrow}
% gets and gives are defined!

% ordering operators
\newcommand{\oleq}{\preceq}
\newcommand{\ogeq}{\succeq}

% programming and logic operators
\newcommand{\dfn}{:=}
\newcommand{\assign}{:=}
\newcommand{\co}{\ co\ }
\newcommand{\en}{\ en\ }


% logic operators
\newcommand{\xor}{\oplus}
\newcommand{\Land}{\bigwedge}
\newcommand{\Lor}{\bigvee}
\newcommand{\finish}{$\Box$}
\newcommand{\contra}{\Rightarrow \Leftarrow}
\newcommand{\iseq}{\stackrel{_?}{=}}


% Set theory
\newcommand{\symdiff}{\Delta}
\newcommand{\union}{\cup}
\newcommand{\inters}{\cap}
\newcommand{\Union}{\bigcup}
\newcommand{\Inters}{\bigcap}
\newcommand{\nullSet}{\phi}

% graph theory
\newcommand{\nbd}{\Gamma}

% Script alphabets
% For reals, use \Re

% greek letters
\newcommand{\eps}{\epsilon}
\newcommand{\del}{\delta}
\newcommand{\ga}{\alpha}
\newcommand{\gb}{\beta}
\newcommand{\gd}{\del}
\newcommand{\gf}{\phi}
\newcommand{\gF}{\Phi}
\newcommand{\gl}{\lambda}
\newcommand{\gm}{\mu}
\newcommand{\gn}{\nu}
\newcommand{\gr}{\rho}
\newcommand{\gs}{\sigma}
\newcommand{\gt}{\theta}
\newcommand{\gx}{\xi}

\newcommand{\sw}{\sigma}
\newcommand{\SW}{\Sigma}
\newcommand{\ew}{\lambda}
\newcommand{\EW}{\Lambda}

\newcommand{\Del}{\Delta}
\newcommand{\gD}{\Delta}
\newcommand{\gG}{\Gamma}
\newcommand{\gO}{\Omega}
\newcommand{\gL}{\Lambda}
\newcommand{\gS}{\Sigma}

% Formatting shortcuts
\newcommand{\red}[1]{\textcolor{red}{#1}}
\newcommand{\blue}[1]{\textcolor{blue}{#1}}
\newcommand{\htext}[2]{\texorpdfstring{#1}{#2}}

% Statistics
\newcommand{\distr}{\sim}
\newcommand{\stddev}{\sigma}
\newcommand{\covmatrix}{\Sigma}
\newcommand{\mean}{\mu}
\newcommand{\param}{\gt}
\newcommand{\ftr}{\phi}

% General utility
\newcommand{\todo}[1]{\footnote{TODO: #1}}
\newcommand{\exclaim}[1]{{\textbf{\textit{#1}}}}
\newcommand{\tbc}{[\textbf{Incomplete}]}
\newcommand{\chk}{[\textbf{Check}]}
\newcommand{\oprob}{[\textbf{OP}]:}
\newcommand{\core}[1]{\textbf{Core Idea:}}
\newcommand{\why}{[\textbf{Find proof}]}
\newcommand{\opt}[1]{\textit{#1}}


\DeclareMathOperator*{\argmin}{arg\,min}
\DeclareMathOperator{\rank}{rank}
\newcommand{\redcol}[1]{\textcolor{red}{#1}}
\newcommand{\bluecol}[1]{\textcolor{blue}{#1}}
\newcommand{\greencol}[1]{\textcolor{green}{#1}}


\renewcommand{\~}{\htext{$\sim$}{~}}


% groupings of objects.
\newcommand{\set}[1]{\left\{ #1 \right\}}
\newcommand{\seq}[1]{\left(#1\right)}
\newcommand{\ang}[1]{\langle#1\rangle}
\newcommand{\tuple}[1]{\left(#1\right)}

% numerical shortcuts.
\newcommand{\abs}[1]{\left| #1\right|}
\newcommand{\floor}[1]{\left\lfloor #1 \right\rfloor}
\newcommand{\ceil}[1]{\left\lceil #1 \right\rceil}

% linear algebra shortcuts.
\newcommand{\change}{\Delta}
\newcommand{\norm}[1]{\left\| #1\right\|}
\newcommand{\dprod}[1]{\langle#1\rangle}
\newcommand{\linspan}[1]{\langle#1\rangle}
\newcommand{\conj}[1]{\overline{#1}}
\newcommand{\gradient}{\nabla}
\newcommand{\der}{\frac{d}{dx}}
\newcommand{\lap}{\Delta}
\newcommand{\kron}{\otimes}
\newcommand{\nperp}{\nvdash}

\newcommand{\mat}[1]{\left( \begin{smallmatrix}#1 \end{smallmatrix} \right)}

% derivatives and limits
\newcommand{\partder}[2]{\frac{\partial #1}{\partial #2}}
\newcommand{\partdern}[3]{\frac{\partial^{#3} #1}{\partial #2^{#3}}}

% Arrows
\newcommand{\diverge}{\nearrow}
\newcommand{\notto}{\nrightarrow}
\newcommand{\up}{\uparrow}
\newcommand{\down}{\downarrow}
% gets and gives are defined!

% ordering operators
\newcommand{\oleq}{\preceq}
\newcommand{\ogeq}{\succeq}

% programming and logic operators
\newcommand{\dfn}{:=}
\newcommand{\assign}{:=}
\newcommand{\co}{\ co\ }
\newcommand{\en}{\ en\ }


% logic operators
\newcommand{\xor}{\oplus}
\newcommand{\Land}{\bigwedge}
\newcommand{\Lor}{\bigvee}
\newcommand{\finish}{$\Box$}
\newcommand{\contra}{\Rightarrow \Leftarrow}
\newcommand{\iseq}{\stackrel{_?}{=}}


% Set theory
\newcommand{\symdiff}{\Delta}
\newcommand{\union}{\cup}
\newcommand{\inters}{\cap}
\newcommand{\Union}{\bigcup}
\newcommand{\Inters}{\bigcap}
\newcommand{\nullSet}{\phi}

% graph theory
\newcommand{\nbd}{\Gamma}

% Script alphabets
% For reals, use \Re

% greek letters
\newcommand{\eps}{\epsilon}
\newcommand{\del}{\delta}
\newcommand{\ga}{\alpha}
\newcommand{\gb}{\beta}
\newcommand{\gd}{\del}
\newcommand{\gf}{\phi}
\newcommand{\gF}{\Phi}
\newcommand{\gl}{\lambda}
\newcommand{\gm}{\mu}
\newcommand{\gn}{\nu}
\newcommand{\gr}{\rho}
\newcommand{\gs}{\sigma}
\newcommand{\gt}{\theta}
\newcommand{\gx}{\xi}

\newcommand{\sw}{\sigma}
\newcommand{\SW}{\Sigma}
\newcommand{\ew}{\lambda}
\newcommand{\EW}{\Lambda}

\newcommand{\Del}{\Delta}
\newcommand{\gD}{\Delta}
\newcommand{\gG}{\Gamma}
\newcommand{\gO}{\Omega}
\newcommand{\gL}{\Lambda}
\newcommand{\gS}{\Sigma}

% Formatting shortcuts
\newcommand{\red}[1]{\textcolor{red}{#1}}
\newcommand{\blue}[1]{\textcolor{blue}{#1}}
\newcommand{\htext}[2]{\texorpdfstring{#1}{#2}}

% Statistics
\newcommand{\distr}{\sim}
\newcommand{\stddev}{\sigma}
\newcommand{\covmatrix}{\Sigma}
\newcommand{\mean}{\mu}
\newcommand{\param}{\gt}
\newcommand{\ftr}{\phi}

% General utility
\newcommand{\todo}[1]{\footnote{TODO: #1}}
\newcommand{\exclaim}[1]{{\textbf{\textit{#1}}}}
\newcommand{\tbc}{[\textbf{Incomplete}]}
\newcommand{\chk}{[\textbf{Check}]}
\newcommand{\oprob}{[\textbf{OP}]:}
\newcommand{\core}[1]{\textbf{Core Idea:}}
\newcommand{\why}{[\textbf{Find proof}]}
\newcommand{\opt}[1]{\textit{#1}}


\DeclareMathOperator*{\argmin}{arg\,min}
\DeclareMathOperator{\rank}{rank}
\newcommand{\redcol}[1]{\textcolor{red}{#1}}
\newcommand{\bluecol}[1]{\textcolor{blue}{#1}}
\newcommand{\greencol}[1]{\textcolor{green}{#1}}


\renewcommand{\~}{\htext{$\sim$}{~}}


% groupings of objects.
\newcommand{\set}[1]{\left\{ #1 \right\}}
\newcommand{\seq}[1]{\left(#1\right)}
\newcommand{\ang}[1]{\langle#1\rangle}
\newcommand{\tuple}[1]{\left(#1\right)}

% numerical shortcuts.
\newcommand{\abs}[1]{\left| #1\right|}
\newcommand{\floor}[1]{\left\lfloor #1 \right\rfloor}
\newcommand{\ceil}[1]{\left\lceil #1 \right\rceil}

% linear algebra shortcuts.
\newcommand{\change}{\Delta}
\newcommand{\norm}[1]{\left\| #1\right\|}
\newcommand{\dprod}[1]{\langle#1\rangle}
\newcommand{\linspan}[1]{\langle#1\rangle}
\newcommand{\conj}[1]{\overline{#1}}
\newcommand{\gradient}{\nabla}
\newcommand{\der}{\frac{d}{dx}}
\newcommand{\lap}{\Delta}
\newcommand{\kron}{\otimes}
\newcommand{\nperp}{\nvdash}

\newcommand{\mat}[1]{\left( \begin{smallmatrix}#1 \end{smallmatrix} \right)}

% derivatives and limits
\newcommand{\partder}[2]{\frac{\partial #1}{\partial #2}}
\newcommand{\partdern}[3]{\frac{\partial^{#3} #1}{\partial #2^{#3}}}

% Arrows
\newcommand{\diverge}{\nearrow}
\newcommand{\notto}{\nrightarrow}
\newcommand{\up}{\uparrow}
\newcommand{\down}{\downarrow}
% gets and gives are defined!

% ordering operators
\newcommand{\oleq}{\preceq}
\newcommand{\ogeq}{\succeq}

% programming and logic operators
\newcommand{\dfn}{:=}
\newcommand{\assign}{:=}
\newcommand{\co}{\ co\ }
\newcommand{\en}{\ en\ }


% logic operators
\newcommand{\xor}{\oplus}
\newcommand{\Land}{\bigwedge}
\newcommand{\Lor}{\bigvee}
\newcommand{\finish}{$\Box$}
\newcommand{\contra}{\Rightarrow \Leftarrow}
\newcommand{\iseq}{\stackrel{_?}{=}}


% Set theory
\newcommand{\symdiff}{\Delta}
\newcommand{\union}{\cup}
\newcommand{\inters}{\cap}
\newcommand{\Union}{\bigcup}
\newcommand{\Inters}{\bigcap}
\newcommand{\nullSet}{\phi}

% graph theory
\newcommand{\nbd}{\Gamma}

% Script alphabets
% For reals, use \Re

% greek letters
\newcommand{\eps}{\epsilon}
\newcommand{\del}{\delta}
\newcommand{\ga}{\alpha}
\newcommand{\gb}{\beta}
\newcommand{\gd}{\del}
\newcommand{\gf}{\phi}
\newcommand{\gF}{\Phi}
\newcommand{\gl}{\lambda}
\newcommand{\gm}{\mu}
\newcommand{\gn}{\nu}
\newcommand{\gr}{\rho}
\newcommand{\gs}{\sigma}
\newcommand{\gt}{\theta}
\newcommand{\gx}{\xi}

\newcommand{\sw}{\sigma}
\newcommand{\SW}{\Sigma}
\newcommand{\ew}{\lambda}
\newcommand{\EW}{\Lambda}

\newcommand{\Del}{\Delta}
\newcommand{\gD}{\Delta}
\newcommand{\gG}{\Gamma}
\newcommand{\gO}{\Omega}
\newcommand{\gL}{\Lambda}
\newcommand{\gS}{\Sigma}

% Formatting shortcuts
\newcommand{\red}[1]{\textcolor{red}{#1}}
\newcommand{\blue}[1]{\textcolor{blue}{#1}}
\newcommand{\htext}[2]{\texorpdfstring{#1}{#2}}

% Statistics
\newcommand{\distr}{\sim}
\newcommand{\stddev}{\sigma}
\newcommand{\covmatrix}{\Sigma}
\newcommand{\mean}{\mu}
\newcommand{\param}{\gt}
\newcommand{\ftr}{\phi}

% General utility
\newcommand{\todo}[1]{\footnote{TODO: #1}}
\newcommand{\exclaim}[1]{{\textbf{\textit{#1}}}}
\newcommand{\tbc}{[\textbf{Incomplete}]}
\newcommand{\chk}{[\textbf{Check}]}
\newcommand{\oprob}{[\textbf{OP}]:}
\newcommand{\core}[1]{\textbf{Core Idea:}}
\newcommand{\why}{[\textbf{Find proof}]}
\newcommand{\opt}[1]{\textit{#1}}


\DeclareMathOperator*{\argmin}{arg\,min}
\DeclareMathOperator{\rank}{rank}
\newcommand{\redcol}[1]{\textcolor{red}{#1}}
\newcommand{\bluecol}[1]{\textcolor{blue}{#1}}
\newcommand{\greencol}[1]{\textcolor{green}{#1}}


\renewcommand{\~}{\htext{$\sim$}{~}}


\title{Language: survey}
\author{vishvAs vAsuki}

\begin{document}
\maketitle
\tableofcontents

\part{General structure}
\chapter{Themes}
Computational linguistics tries to understand human language generation and comprehension using computer models.

\section{Human vs programming languages}
Natural language processing considers how human language can be generated or understood by a computer.

Design of language which can be translated to instructions understood by a compiler is considered elsewhere.

\section{Descriptive vs prescriptive}
Descriptive linguistics studies the language as it is actually used by a certain group of people. Prescriptive linguistics considers the language as it 'ought' to be used.

\section{Example applications/ problems}
Authorship attribution.

Interaction with humans: Eg: IBM built 'Watson', a program which participates in the propular TV trivia show jeopardy, where it responds to phrases with questions to which they are an answer.

Human language acquisition.

\section{Ambiguity: main challenge}
Comprehension and generation of text is just translation between natural language and the language of logic. The ambiguity in natural language is the main problem in this process.

\section{Following research}
ACL> NACL (Held only when ACL is outside north america)> COLING.


\chapter{Language structure}
\section{Sounds words sentences}
There are sounds. Certain ordered sounds make up words. Words make up phrases (Eg: noun phrases, verb phrases), which in-turn make up sentences.

\section{Word morphology}
\subsection{Morpheme}
Parts of a word which are not further divisible are called morphemes. Eg: In 'sunism', 'sun' is the root/ stem word, while 'ism' is an affix; both are morphemes.

\subsection{Inflection}
Inflectional morphemes alter the words gender, tense, number etc.: 's' in suns. 'ism' (as in 'sunism') is a derivational morpheme as it is used to derive a new separate word.



\section{Word/ phrase roles/ categories}
The general role a word/ phrase plays (as against its specific meaning) in a sentence is called a 'part of speech'.

Depending on the number of words that can play a certain role, the word role/ class is closed or open. Eg: determiners in English is a closed class.

\subsection{Content vs function words}
Words are either content words which have lexical meaning, or are function words (aka grammatical / structure-class words) which serve grammatical purpose while carrying little or no independent meaning.

\subsection{Parts of speech}
\begin{itemize}
\item noun (communicates objects/ subjects)
\item verb (communicates action), article (a, the .. )
\item adjective (qualifies the noun)
\item adverb
\item preposition and postposition (Aka adposition, clarifies a noun's grammatical case: of, on, in etc..)
\item pronoun (a generic noun, like 'he' ..)

\item conjunction (and, but, or)
\subitem A subordinating conjunction, which joins a subordinate clause to the main clause (Eg: if).

\item particle: A function word which does not belong to other classes.
\subitem interjection and filler words('oh!').
\end{itemize}



Other important parts of speech include wh-word (question word), Modal (could, would, must, can, might ..), list item markers. 


\subsection{Noun and verb sub-roles}
A noun word can have further generic sub-roles based on what gender, number and grammatical case (especially possessive in English) it communicates.

A verb word can also have sub-roles based on whether it communicates grammatical/ subject's gender, subject's number, person (1st, 2nd or 3rd person), aspect, mood (imperative, wishfulness), time and completeness of the action (present/ past/ future) etc.. This is called 'tense'. If it calls attention to the completeness of an action it is called 'perfect'.


\section{Semantics and ambiguity}
\subsection{Word meaning ambiguity}
Meanings of words themselves are ambiguous. Even their part-of-speech can be ambiguous - this ambiguity can lead to different parse trees; for example 'dogs' in: 'The sailor dogs the barmaid'.

\subsection{Sentence ambiguity}
Meanings of sentences, even after the meanings of words have been fixed, is still ambiguous due to the presence of differing possible parse trees. Then, there is analogy, sarcasm etc..

\section{Grammar}
Language structure is described by a grammar. A grammar attempts to model how a sentence of the language is formed.

A grammar can be descriptive or generative, the distinction being that in the latter case the decision made by the grammar of sentence correctness/ membership is ultimate. Generative grammars include computer languages and Sanskrit.

\chapter{Language and thought}
\section{Complex thought and language}
Complex thought about a certain topic requires suitable vocabulary; complex ideas become simple in science when the right theory/ notation/ framework is in place.

\subsection{Examples/ evidence}
The first generation of deaf-mutes in a central American country were unable to put themselves in others' place: they failed the test where they are asked to guess where a person would look for a thing whose position has been changed in their absence. This is seem among children below a certain age.

A certain Hispanic deaf-mute man did not know that objects had names or symbols. So, to communicate or comprehend simple ideas: such as reliving the memory of a bull-fight, took a long time: 45 minutes of miming.

\section{World-view and language}
“Languages differ essentially in what they must convey and not in what they may convey.” Your language changes the way you think and feel about many things - by making them more or less important in your world-view.

\subsection{Examples}
In saMskRRita, you are forced to convey the sex of the person you interact with, in English, you are often forced to specify the time of a meal; and inanimate objects have gender: this changes the way you feel about them.

In some languages, you specify geographic direction 'N/S/E/W', whereas in others you specify front/ back etc..: memories (even gestures) in such languages are tagged with geographic direction: the sense of direction in speakers of such languages is extremely keen. They have perhaps trained their brains to maintain an accurate bird's eye view of their location.

Our brains are trained to exaggerate the distance between shades of color if these have different names in our language.

Some languages, like Matses in Peru, oblige their speakers, like the finickiest of lawyers, to specify exactly how they came to know about the facts they are reporting. 

\section{Social interactions}
Vagueness and indirectness in language is used in communicating an idea while leaving a slight window of deniability. Eg: Idioms such as 'will you come up to see my etchings?'.

\chapter{Models of communication}
\section{Speech generation}
Speech generation involves the following tasks:
\begin{itemize}
 \item Intention: Generating the thought to be spoken in the internal language of logic.
 \item Generation: Translating the logical language into natural language.
 \item Synthesis: Speaking the generated words with appropriate stresses, accents etc..
\end{itemize}

\section{Speech comprehension}
\subsection{Tasks}
Speech comprehension involves the following tasks:
\begin{itemize}
 \item Perception: Translating the sounds heard or symbols seen to words or tokens; this is akin to lexical analysis by compilers.  
 
\item Analysis: Inferring a logical sentence equivalent to the spoken words. This involves the following tasks:

  \subitem Parsing/ syntactic tasks: Understanding the phrasal structure, for example using parse trees. As part of this process, the following tasks may be done:
  \subsubitem Part of speech tagging may be done.
  \subsubitem Phrase chunking: In this task, words are chunked to form entities such as a noun phrase and a verb phrase, by collecting together qualifiers (adjectives/ adverbs) and prepositional phrases.

  \subitem Semantic interpretation: Translating the words spoken to the language of logic. As part of this, the following tasks show up:
  \subsubitem Word sense disambiguation.
  \subsubitem Semantic role labeling: Here, a noun phrase's relation to the verb is decided. In case of Indic languages (esp Sanskrit), doing this is simplified due to the kAraka system.

  \subitem Pragmatic interpretation: Understanding what was meant as opposed to what was said. Eg: Detecting sarcasm.

 \item Incorporation: Relating the inferred logical statement to the knowledge base, and judging whether should be incorporated into the knowledge base and doing so if appropriate.

  \subitem A common task is resolving what entity a given common noun or pronoun refers to. Depending on whether the referred entity is within the corpus or whether it should be judged from deep understanding of the context, one classifies these into endophora and exophora.

\end{itemize}

There is ambiguity in every subtask above.

\subsection{Comprehenders: Manual vs automatic building}
\subsubsection{Manual design}
In this approach, a group of people use their knowledge of the language to build an expert system to comprehend a language. But, it is very hard for a group of people to encode all the rules of a language, for which there are a huge number of exceptions. Eg: 'is' in different tenses. An exception to this is probably pANini's grammar for saMskRRita.

Also, even people, in the face of ambiguity, make use of statistics/ probability in comprehension, eg: 'What do people commonly mean by this sentence?'

\subsubsection{Automatic learning}
Here, the computer learns the rules for comprehending a language by looking at statistics learned from annotated corpora.

\subsection{Non-Modularity}
Consider the various tasks involved in speech comprehension. These tasks may be viewed as happening in a sequence. In reality, however, they are intertwined: For example, one uses the syntactic structure to determine that the words in a sound correspond to 'I ate a rabbit', rather than 'Eye eight arab it.'.

\section{Part of speech tagging}
\subsection{Problem}
The task is to label all words in a sentence with the most appropriate part of speech. The set of appropriate parts of speech is usually fixed. It either taken to be the set defined by classical grammars of the language or produced more dynamically (supertagging?).

\subsection{Purpose}
Once part of speech tags are produced, they can be used as a feature while building parse-trees.

It is also useful in word sense disambiguation.

\subsection{Approaches}
This is an instance of the sequence label prediction problem. So, the approaches described for that general problem in the probabilistic models survey apply here as well.

\subsubsection{Performance ideal}
Given the same text, due to ambiguity in natural language, even taggings produced by human taggers don't agree with each other fully. For example, in english, the level of agreement is around 97\%. So, this forms an ideal for tagging algorithms.

\subsection{Using a tagged sample}
\subsubsection{Results}
Consider the following for the Portuguese-boscque corpus from CoNLL 2006.

Simply labeling every word with the most frequent tag (noun) yields around 60\% accuracy. Labeling each word with the tag most likely to correspond to that word yields around 92\% accuracy. Using HMM yields around 95\% accuracy.

OpenNLP's implementation of MAXENT model with event threshold 5 and 100 iterations yields around 96.5\% accuracy.

\chapter{Language models}
\section{Definition}
A language model models the probability of every possible string being a sentence spoken by a human.

\section{Applications}
Language models are useful in speech recognition, machine translation, handwriting recognition, machine translation, speech generation, (context sensitive) spelling correction etc..

\section{Multi-set of words model}
This very simple model ignores word order - a very important information. It models $Pr(w_{1:m}) = \prod Pr(w_i)$; and the parameters of this model - the word occurrence probabilities - are easily estimated; and it is known to follow the Zipf's law, which is a heavy tailed distribution. This is actually the 1-gram model, a member of the n-gram model family described elsewhere.

\section{n-gram model}
See Probabilistic models survey.

\subsection{Performance}
A trigram model is very commonly used. Google, as of 2011, seems to have built a good 5-gram model.

\section{Finite state transducer (FST)}
\subsection{Definition}
These are Finite State Automata (FSA described in the boolean functions survey) extended to include production of outputs corresponding to each state transition.

It is specified by a state set $S$, a subset of initial and final states, a state transition relation: $S \times I \to S \times O$ - where $I, O$ are the input and output alphabet.

Because of the non-deterministic nature of the transitions, multiple output strings may be produced for a single input string. A probabilistic version of an FST produces probability scores along with the output strings.

Depending on whether a non-empty output string is produced by the operation of the FST on a given input string $s$, it may be declared to be accepted or rejected.

\subsection{Transition graph}
As in the case of a FSA, FST can be written as a transition graph - except that each edge is labelled by both an input and an output character.


\tbc

\part{Languages}
\chapter{Activity language}
A person's motion, as he goes about various tasks, can be modeled as a language. There is a hierarchy in motions, which is akin to the sound - word - sentence - hierarchy.

\chapter{Japanese}
\section{Sounds}
l is pronounced/ heard as r. T is heard as t.

\section{Words}
Pronouns: watashi anata.

\subsection{Noun Inflections}
watashi no: my. Tokyo kara: from Tokyo. Tokyo made: To tokyo. Tokyo ni: In Tokyo.

\subsection{Verb}
kire - kill. desu - is. There are no plural forms.

\subsubsection{Adverbs}
shite (hurry). gosaimas (very much). not - ni.


\section{Sentences}
\subsection{Greetings/ politeness}
konnichiwa (pre-dusk), oyasuminasai (night). Jamen (I go/ bye). 
sayonAra (good bye). arigato (thanks).

\subsection{Questions and answers}
Questioning appendage: des-ka?

Answering: hai. iye.

\chapter{References}
Ray Mooney's course notes.

%\bibliographystyle{plain}
%\bibliography{colt}

\end{document}
