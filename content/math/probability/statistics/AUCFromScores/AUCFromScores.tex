\documentclass{article}

% Configuration for the memoir class.
\renewcommand{\cleardoublepage}{}
% \renewcommand*{\partpageend}{}
\renewcommand{\afterpartskip}{}
\maxsecnumdepth{subsubsection} % number subsections
\maxtocdepth{subsubsection}

\addtolength{\parindent}{-5mm}
% Packages not included:
% For multiline comments, use caption package. But this conflicts with hyperref while making html files.
% subfigure conflicts with use with memoir style-sheet.

% Use something like:
% % Use something like:
% % Use something like:
% \input{../../macros}

% groupings of objects.
\newcommand{\set}[1]{\left\{ #1 \right\}}
\newcommand{\seq}[1]{\left(#1\right)}
\newcommand{\ang}[1]{\langle#1\rangle}
\newcommand{\tuple}[1]{\left(#1\right)}

% numerical shortcuts.
\newcommand{\abs}[1]{\left| #1\right|}
\newcommand{\floor}[1]{\left\lfloor #1 \right\rfloor}
\newcommand{\ceil}[1]{\left\lceil #1 \right\rceil}

% linear algebra shortcuts.
\newcommand{\change}{\Delta}
\newcommand{\norm}[1]{\left\| #1\right\|}
\newcommand{\dprod}[1]{\langle#1\rangle}
\newcommand{\linspan}[1]{\langle#1\rangle}
\newcommand{\conj}[1]{\overline{#1}}
\newcommand{\gradient}{\nabla}
\newcommand{\der}{\frac{d}{dx}}
\newcommand{\lap}{\Delta}
\newcommand{\kron}{\otimes}
\newcommand{\nperp}{\nvdash}

\newcommand{\mat}[1]{\left( \begin{smallmatrix}#1 \end{smallmatrix} \right)}

% derivatives and limits
\newcommand{\partder}[2]{\frac{\partial #1}{\partial #2}}
\newcommand{\partdern}[3]{\frac{\partial^{#3} #1}{\partial #2^{#3}}}

% Arrows
\newcommand{\diverge}{\nearrow}
\newcommand{\notto}{\nrightarrow}
\newcommand{\up}{\uparrow}
\newcommand{\down}{\downarrow}
% gets and gives are defined!

% ordering operators
\newcommand{\oleq}{\preceq}
\newcommand{\ogeq}{\succeq}

% programming and logic operators
\newcommand{\dfn}{:=}
\newcommand{\assign}{:=}
\newcommand{\co}{\ co\ }
\newcommand{\en}{\ en\ }


% logic operators
\newcommand{\xor}{\oplus}
\newcommand{\Land}{\bigwedge}
\newcommand{\Lor}{\bigvee}
\newcommand{\finish}{$\Box$}
\newcommand{\contra}{\Rightarrow \Leftarrow}
\newcommand{\iseq}{\stackrel{_?}{=}}


% Set theory
\newcommand{\symdiff}{\Delta}
\newcommand{\union}{\cup}
\newcommand{\inters}{\cap}
\newcommand{\Union}{\bigcup}
\newcommand{\Inters}{\bigcap}
\newcommand{\nullSet}{\phi}

% graph theory
\newcommand{\nbd}{\Gamma}

% Script alphabets
% For reals, use \Re

% greek letters
\newcommand{\eps}{\epsilon}
\newcommand{\del}{\delta}
\newcommand{\ga}{\alpha}
\newcommand{\gb}{\beta}
\newcommand{\gd}{\del}
\newcommand{\gf}{\phi}
\newcommand{\gF}{\Phi}
\newcommand{\gl}{\lambda}
\newcommand{\gm}{\mu}
\newcommand{\gn}{\nu}
\newcommand{\gr}{\rho}
\newcommand{\gs}{\sigma}
\newcommand{\gt}{\theta}
\newcommand{\gx}{\xi}

\newcommand{\sw}{\sigma}
\newcommand{\SW}{\Sigma}
\newcommand{\ew}{\lambda}
\newcommand{\EW}{\Lambda}

\newcommand{\Del}{\Delta}
\newcommand{\gD}{\Delta}
\newcommand{\gG}{\Gamma}
\newcommand{\gO}{\Omega}
\newcommand{\gL}{\Lambda}
\newcommand{\gS}{\Sigma}

% Formatting shortcuts
\newcommand{\red}[1]{\textcolor{red}{#1}}
\newcommand{\blue}[1]{\textcolor{blue}{#1}}
\newcommand{\htext}[2]{\texorpdfstring{#1}{#2}}

% Statistics
\newcommand{\distr}{\sim}
\newcommand{\stddev}{\sigma}
\newcommand{\covmatrix}{\Sigma}
\newcommand{\mean}{\mu}
\newcommand{\param}{\gt}
\newcommand{\ftr}{\phi}

% General utility
\newcommand{\todo}[1]{\footnote{TODO: #1}}
\newcommand{\exclaim}[1]{{\textbf{\textit{#1}}}}
\newcommand{\tbc}{[\textbf{Incomplete}]}
\newcommand{\chk}{[\textbf{Check}]}
\newcommand{\oprob}{[\textbf{OP}]:}
\newcommand{\core}[1]{\textbf{Core Idea:}}
\newcommand{\why}{[\textbf{Find proof}]}
\newcommand{\opt}[1]{\textit{#1}}


\DeclareMathOperator*{\argmin}{arg\,min}
\DeclareMathOperator{\rank}{rank}
\newcommand{\redcol}[1]{\textcolor{red}{#1}}
\newcommand{\bluecol}[1]{\textcolor{blue}{#1}}
\newcommand{\greencol}[1]{\textcolor{green}{#1}}


\renewcommand{\~}{\htext{$\sim$}{~}}


% groupings of objects.
\newcommand{\set}[1]{\left\{ #1 \right\}}
\newcommand{\seq}[1]{\left(#1\right)}
\newcommand{\ang}[1]{\langle#1\rangle}
\newcommand{\tuple}[1]{\left(#1\right)}

% numerical shortcuts.
\newcommand{\abs}[1]{\left| #1\right|}
\newcommand{\floor}[1]{\left\lfloor #1 \right\rfloor}
\newcommand{\ceil}[1]{\left\lceil #1 \right\rceil}

% linear algebra shortcuts.
\newcommand{\change}{\Delta}
\newcommand{\norm}[1]{\left\| #1\right\|}
\newcommand{\dprod}[1]{\langle#1\rangle}
\newcommand{\linspan}[1]{\langle#1\rangle}
\newcommand{\conj}[1]{\overline{#1}}
\newcommand{\gradient}{\nabla}
\newcommand{\der}{\frac{d}{dx}}
\newcommand{\lap}{\Delta}
\newcommand{\kron}{\otimes}
\newcommand{\nperp}{\nvdash}

\newcommand{\mat}[1]{\left( \begin{smallmatrix}#1 \end{smallmatrix} \right)}

% derivatives and limits
\newcommand{\partder}[2]{\frac{\partial #1}{\partial #2}}
\newcommand{\partdern}[3]{\frac{\partial^{#3} #1}{\partial #2^{#3}}}

% Arrows
\newcommand{\diverge}{\nearrow}
\newcommand{\notto}{\nrightarrow}
\newcommand{\up}{\uparrow}
\newcommand{\down}{\downarrow}
% gets and gives are defined!

% ordering operators
\newcommand{\oleq}{\preceq}
\newcommand{\ogeq}{\succeq}

% programming and logic operators
\newcommand{\dfn}{:=}
\newcommand{\assign}{:=}
\newcommand{\co}{\ co\ }
\newcommand{\en}{\ en\ }


% logic operators
\newcommand{\xor}{\oplus}
\newcommand{\Land}{\bigwedge}
\newcommand{\Lor}{\bigvee}
\newcommand{\finish}{$\Box$}
\newcommand{\contra}{\Rightarrow \Leftarrow}
\newcommand{\iseq}{\stackrel{_?}{=}}


% Set theory
\newcommand{\symdiff}{\Delta}
\newcommand{\union}{\cup}
\newcommand{\inters}{\cap}
\newcommand{\Union}{\bigcup}
\newcommand{\Inters}{\bigcap}
\newcommand{\nullSet}{\phi}

% graph theory
\newcommand{\nbd}{\Gamma}

% Script alphabets
% For reals, use \Re

% greek letters
\newcommand{\eps}{\epsilon}
\newcommand{\del}{\delta}
\newcommand{\ga}{\alpha}
\newcommand{\gb}{\beta}
\newcommand{\gd}{\del}
\newcommand{\gf}{\phi}
\newcommand{\gF}{\Phi}
\newcommand{\gl}{\lambda}
\newcommand{\gm}{\mu}
\newcommand{\gn}{\nu}
\newcommand{\gr}{\rho}
\newcommand{\gs}{\sigma}
\newcommand{\gt}{\theta}
\newcommand{\gx}{\xi}

\newcommand{\sw}{\sigma}
\newcommand{\SW}{\Sigma}
\newcommand{\ew}{\lambda}
\newcommand{\EW}{\Lambda}

\newcommand{\Del}{\Delta}
\newcommand{\gD}{\Delta}
\newcommand{\gG}{\Gamma}
\newcommand{\gO}{\Omega}
\newcommand{\gL}{\Lambda}
\newcommand{\gS}{\Sigma}

% Formatting shortcuts
\newcommand{\red}[1]{\textcolor{red}{#1}}
\newcommand{\blue}[1]{\textcolor{blue}{#1}}
\newcommand{\htext}[2]{\texorpdfstring{#1}{#2}}

% Statistics
\newcommand{\distr}{\sim}
\newcommand{\stddev}{\sigma}
\newcommand{\covmatrix}{\Sigma}
\newcommand{\mean}{\mu}
\newcommand{\param}{\gt}
\newcommand{\ftr}{\phi}

% General utility
\newcommand{\todo}[1]{\footnote{TODO: #1}}
\newcommand{\exclaim}[1]{{\textbf{\textit{#1}}}}
\newcommand{\tbc}{[\textbf{Incomplete}]}
\newcommand{\chk}{[\textbf{Check}]}
\newcommand{\oprob}{[\textbf{OP}]:}
\newcommand{\core}[1]{\textbf{Core Idea:}}
\newcommand{\why}{[\textbf{Find proof}]}
\newcommand{\opt}[1]{\textit{#1}}


\DeclareMathOperator*{\argmin}{arg\,min}
\DeclareMathOperator{\rank}{rank}
\newcommand{\redcol}[1]{\textcolor{red}{#1}}
\newcommand{\bluecol}[1]{\textcolor{blue}{#1}}
\newcommand{\greencol}[1]{\textcolor{green}{#1}}


\renewcommand{\~}{\htext{$\sim$}{~}}


% groupings of objects.
\newcommand{\set}[1]{\left\{ #1 \right\}}
\newcommand{\seq}[1]{\left(#1\right)}
\newcommand{\ang}[1]{\langle#1\rangle}
\newcommand{\tuple}[1]{\left(#1\right)}

% numerical shortcuts.
\newcommand{\abs}[1]{\left| #1\right|}
\newcommand{\floor}[1]{\left\lfloor #1 \right\rfloor}
\newcommand{\ceil}[1]{\left\lceil #1 \right\rceil}

% linear algebra shortcuts.
\newcommand{\change}{\Delta}
\newcommand{\norm}[1]{\left\| #1\right\|}
\newcommand{\dprod}[1]{\langle#1\rangle}
\newcommand{\linspan}[1]{\langle#1\rangle}
\newcommand{\conj}[1]{\overline{#1}}
\newcommand{\gradient}{\nabla}
\newcommand{\der}{\frac{d}{dx}}
\newcommand{\lap}{\Delta}
\newcommand{\kron}{\otimes}
\newcommand{\nperp}{\nvdash}

\newcommand{\mat}[1]{\left( \begin{smallmatrix}#1 \end{smallmatrix} \right)}

% derivatives and limits
\newcommand{\partder}[2]{\frac{\partial #1}{\partial #2}}
\newcommand{\partdern}[3]{\frac{\partial^{#3} #1}{\partial #2^{#3}}}

% Arrows
\newcommand{\diverge}{\nearrow}
\newcommand{\notto}{\nrightarrow}
\newcommand{\up}{\uparrow}
\newcommand{\down}{\downarrow}
% gets and gives are defined!

% ordering operators
\newcommand{\oleq}{\preceq}
\newcommand{\ogeq}{\succeq}

% programming and logic operators
\newcommand{\dfn}{:=}
\newcommand{\assign}{:=}
\newcommand{\co}{\ co\ }
\newcommand{\en}{\ en\ }


% logic operators
\newcommand{\xor}{\oplus}
\newcommand{\Land}{\bigwedge}
\newcommand{\Lor}{\bigvee}
\newcommand{\finish}{$\Box$}
\newcommand{\contra}{\Rightarrow \Leftarrow}
\newcommand{\iseq}{\stackrel{_?}{=}}


% Set theory
\newcommand{\symdiff}{\Delta}
\newcommand{\union}{\cup}
\newcommand{\inters}{\cap}
\newcommand{\Union}{\bigcup}
\newcommand{\Inters}{\bigcap}
\newcommand{\nullSet}{\phi}

% graph theory
\newcommand{\nbd}{\Gamma}

% Script alphabets
% For reals, use \Re

% greek letters
\newcommand{\eps}{\epsilon}
\newcommand{\del}{\delta}
\newcommand{\ga}{\alpha}
\newcommand{\gb}{\beta}
\newcommand{\gd}{\del}
\newcommand{\gf}{\phi}
\newcommand{\gF}{\Phi}
\newcommand{\gl}{\lambda}
\newcommand{\gm}{\mu}
\newcommand{\gn}{\nu}
\newcommand{\gr}{\rho}
\newcommand{\gs}{\sigma}
\newcommand{\gt}{\theta}
\newcommand{\gx}{\xi}

\newcommand{\sw}{\sigma}
\newcommand{\SW}{\Sigma}
\newcommand{\ew}{\lambda}
\newcommand{\EW}{\Lambda}

\newcommand{\Del}{\Delta}
\newcommand{\gD}{\Delta}
\newcommand{\gG}{\Gamma}
\newcommand{\gO}{\Omega}
\newcommand{\gL}{\Lambda}
\newcommand{\gS}{\Sigma}

% Formatting shortcuts
\newcommand{\red}[1]{\textcolor{red}{#1}}
\newcommand{\blue}[1]{\textcolor{blue}{#1}}
\newcommand{\htext}[2]{\texorpdfstring{#1}{#2}}

% Statistics
\newcommand{\distr}{\sim}
\newcommand{\stddev}{\sigma}
\newcommand{\covmatrix}{\Sigma}
\newcommand{\mean}{\mu}
\newcommand{\param}{\gt}
\newcommand{\ftr}{\phi}

% General utility
\newcommand{\todo}[1]{\footnote{TODO: #1}}
\newcommand{\exclaim}[1]{{\textbf{\textit{#1}}}}
\newcommand{\tbc}{[\textbf{Incomplete}]}
\newcommand{\chk}{[\textbf{Check}]}
\newcommand{\oprob}{[\textbf{OP}]:}
\newcommand{\core}[1]{\textbf{Core Idea:}}
\newcommand{\why}{[\textbf{Find proof}]}
\newcommand{\opt}[1]{\textit{#1}}


\DeclareMathOperator*{\argmin}{arg\,min}
\DeclareMathOperator{\rank}{rank}
\newcommand{\redcol}[1]{\textcolor{red}{#1}}
\newcommand{\bluecol}[1]{\textcolor{blue}{#1}}
\newcommand{\greencol}[1]{\textcolor{green}{#1}}


\renewcommand{\~}{\htext{$\sim$}{~}}


%opening
\title{Determining AUC from a score vector}
\author{nAgarAjan naTarAjan, vishvAs vAsuki}

\begin{document}
\maketitle
\section{Problem}
The problem is to identify the set of items with a certain property from given set of items.

\section{Notation and Terminology}
Let $U$ be the universe of items, such that $|U| = u$. Let $T \subseteq U$ be the set of items having the property we are interested in.

\subsection{Score generator}
The task of a ``score generator'' is to output a score function $score(i)$, which can be used in generating a partial ordering of the items such that an item with a higher score is perceived to be more likely in $T$. 

\subsection{Predictor}
Using the scores produced by a score generator, the associated predictor, parameterized by $n$ identifies a set of $n$ candidates called the \emph{prediction}, $P_n$. The predictor works as follows.

Consider the bag of scores produced by the action of $score()$ on $U$. Let $sortedScores$ be a vector of these scores arranged in non-increasing order. The cutoff score is $c = sortedScores_n$.

Then, let $GT = \set{i: score(i)>c}$. Let $EQ = \set{i: score(i)=c}$. Let $C \subseteq EQ$ be a set of $n - |GT|$ distinct items selected uniformly at random from EQ. Then, $P_n = GT \union C$.

Another way of looking at the action of a predictior by the following algorithm \ref{alg:predictor}.
\begin{algorithm}[h!]
   \caption{Predictor}
   \label{alg:predictor}
\begin{algorithmic}
\STATE P = $\emptyset$
\FOR{i = 1 to n}
\STATE Select an element $k$ at random from $\set{j : score(j) = sortedScores_i}$.
\STATE P = P $\union \set{k}$.
\ENDFOR
\end{algorithmic}
\end{algorithm}


\section{Evaluating a score generator at n}
Suppose that the scores from a score generator are used in producing the prediction $P_n$. 

\subsection{Sensitivity and Specificity}
Sensitivity $X = \frac{|P_n \inters T|}{|T|}$, measures the ability of the predictor to identify items in $T$. Specificity $Y = \frac{|(U - P_n) \inters (U - T)|}{|U - T|}$, measures the ability of the predictor to exclude items not in $T$.

The problem is to find $E[X]$ and $E[Y]$, given $score()$.

\subsection{Expected Sensitivity}
For every $i \in T$, let $X_i$ be a binary random variable, which is 1 if $i \in P_n$ and 0 otherwise. Then, $X = (1/|T|)\sum_i{X_i}$. By linearity of expection, $E[X] = (1/|T|)\sum_i{E[X_i]}$.
\begin{eqnarray*}
 Pr(X_i = 1|score(i)=c) &=& \frac{n-|GT|}{|EQ|} \\
 Pr(X_i = 1|score(i)>c) &=& 1 \\
 Pr(X_i = 1|score(i)<c) &=& 0 \\
 \forall i \in T \inters GT: E[X_i] &=& 1 \\
 \forall i \in T \inters EQ: E[X_i] &=& \frac{n-|GT|}{|EQ|} \\
 E[X] &=& (1/|T|)( |GT \inters T| + |EQ \inters T|\frac{n-|GT|}{|EQ|})\\
\end{eqnarray*}

Note that $EQ \geq n - |GT|$.

\subsubsection{Sanity check}
Consider what happens when $\forall i: score(i) = 0$. Then, $|GT| = 0, |EQ \inters T| = |T|, E[X] = \frac{n}{|EQ|} = \frac{n}{|U|}$, as expected.

\subsection{Expected Specificity}
\begin{eqnarray*}
Y &=& \frac{|(U - P_n) \inters (U - T)|}{|U - T|}\\
&=& \frac{|(U - T) - P_n \inters (U - T)|}{|U - T|}\\
&=& \frac{|(U - T)| - |P_n \inters (U - T)|}{|U - T|}\\
&=& 1 - \frac{n - |T|X}{|U| - |T|}\\
&=& 1 - \frac{n - |T|X}{|U| - |T|}\\
E[Y] &=& 1 - \frac{n - |T|E[X]}{|U| - |T|}\\
\end{eqnarray*}

\subsubsection{Sanity check}
Consider what happens when $\forall i: score(i) = 0$. Then, $E[X] = \frac{n}{|U|}$, $E[Y] = 1 - \frac{n}{|U|} = \frac{|U| - n}{|U|}$, as expected.

\subsection{Summary}
In summary, the performance of a score generator at $n$ is evaluated as follows. $|U|$ and $|T|$ will already be known. The cutoff score $c$ is determined. The score vector is used to determine the sets $GT, EQ$. These are used to evaluate expected sensitivity and expected specificity.

\section{Evaluating a score generator over the entire range of n}
\subsection{ROC and AUC}
The sensitivity vs (1-specificity) plot for varying 'number of prediction parameters n is called ROC curve. Note that, both sensitivity and (1-specificity) are monotonically non-decreasing functions of n. The expected ROC curve, or E[ROC], can be produced by determining $E[X]$ and $E[Y]$ for various values of $n \in [1, |U|]$. The area under the ROC curve, AUC, is a measure of the overall performance of the score generator.

\subsubsection{Evaluating expected AUC}
E[AUC] can be approximated analytically with the area under a piecewise-linear function to approximate ROC. Alternatively, one can calculate E[AUC] exactly using score() and T as explained below.

% \subsection{Exactly computing E[SEN]}
% Consider the E[ROC] curve. As (1-specificity) is a monotonically non-decreasing function of n, one can equivalently consider a E[X] vs n curve.
% 
% \subsubsection{Intervals of n}
% Intervals are defined to be subsequences in range(n) (ie: $1:|U|$). Eg: (1, 2, 3) is an interval, but (2, 4) is not.
% 
% For a given predictor, the ROC changes exactly when some $j\in T$ has a non-zero probability of being included as the nth item in the prediction. Type A intervals, or R(A), are defined to be maximal subsequences in n where certain elements in T have a non zero but equal probability of being included. Type B intervals, or R(B), are defined to be maximal intervals where this is not true.
% 
% Then, $E[SEN] = \sum_{I \in R(A) \union R(B)} E[\change SEN_I]$.
% 
% \paragraph*{Some properties of intervals}
% The width w(I) of an interval I is defined to be the number of elements in I. $w(I) = max(I) - min(I) + 1$.
% 
% The inequality $I<J$ is said to hold between intervals I and J if $max(I) < min(J)$.
% 
% Let $B = \set{i \in T: score(i)> sortedScores_{min(I)}}$. $sensitivity_I \dfn |T|^{-1} \size{B}$
% 
% \subsubsection{Change in SEN for I in R(B)}
% 
% $E[\change SEN_I] = \size{U}^{-1}w(I)sensitivity_I$.
% 
% \subsubsection{Change in SEN for I in R(A)}
% Consider a region I in R(A). It has the following properties.
% 
% $I_U = \set{i: score(i) = sortedScores_{min(I)}}, \size{I_U} = w(I)$. Let $I_T = I_U \inters T$.
% 
% Let $X_j$ be a binary random variable, which is 1 exactly when some element $k \in T$ is the last item to be included in a prediction when n=j.
% \begin{eqnarray*}
% E[X_j] &=& \frac{\size{I_T}}{w(I)}\\
% Y &\dfn& \size{U}^{-1}w(I)sensitivity_I\\
% E[\change SEN_I] &=&  Y + (|U||T|)^{-1}\sum_{i \in I} (E[X_i] + \sum_{j < i, j\in I}E[X_j])\\
% &=& Y + (|U||T|)^{-1}\sum_{i \in I} (i - min(I)+1)E[X_i]\\
% &=& Y + (|U||T|)^{-1}\sum_{i =1}^{w(I)} i\frac{\size{I_T}}{w(I)}\\
% &=& Y + (|U||T|)^{-1}\size{I_T}(w(I)+1)/2\\
% \end{eqnarray*}
% 
% 
% \subsubsection{Summary}
% One can calculate E[SEN] by first partitioning the range of n (which is $1:\size{U}$) in intervals of type A and B, finding the change in SEN, $\change SEN_I$ for each interval I, and finally using $E[SEN] = \sum_{I \in R(A) \union R(B)} E[\change SEN_I]$.
% 
% \section{Evaluating a score generator over given sub-range of n}
% \subsection{Motivation}
% Suppose you were doing link prediction for a social network website, like orkut or facebook; or that you were a biologist working with a biological network; or someone like netflix selling movies. You would be interested in making, let us say, 5 pages of recommendations, but not more than that: certainly not 1000.
% 
% In all such cases, one would want to evaluate the performance of a score generator over a small initial portion of the range of n. Let the range we are interested in be $1:N$.
% 
% \subsection{Truncated SEN (tSEN)}
% One natural way to evaluate the performance of a score generator over $1:N$ is to consider a small slice of the ROC curve, whose horizontal axis may be considered to range over n, rather than over (1-specificity), as the latter is a monotonically non-decreasing function of the former. Just as SEN was a way of measuring the goodness of an ROC curve, one can consider the area under this small slice of the ROC curve. We call this quantity, truncated SEN or tSEN.
% 
% Below we describe a way to exactly calculate E[tSEN], given N.
% 
% \subsubsection{Exactly calculating E[tSEN]}
% $I_N$ be the interval where $N$ lies. Then,\\
% $$E[tSEN] = \sum_{I \leq I_N, I \in R(A) \union R(B)} E[\change tSEN_I]$$
% 
% For all $I < I_N$, note that $\change tSEN_I = \change SEN_I$, which has been described in a previous section. Now we consider how the calculation of $E[\change tSEN_{I_N}]$ differs from that of $E[\change SEN_{I_N}]$.
% 
% Let $w_N = N-\min(I_N)+1$.
% 
% If $I_N \in R(B)$: $E[\change tSEN_{I_N}] = w_N w(I_N)^{-1} E[\change SEN_{I_N}]$.
% 
% If $I_N \in R(A)$: 
% \begin{eqnarray*}
% E[X_j] &=& \frac{\size{I_T}}{w(I)}\\
% Y &\dfn& \size{U}^{-1}w_N sensitivity_I\\
% E[\change SEN_I] &=&  Y + (|U||T|)^{-1}\sum_{i \in [min(I_N), N]} (E[X_i] + \sum_{j < i, j\in I_N}E[X_j])\\
% &=& Y + (|U||T|)^{-1}\sum_{i \in [min(I_N), N]} (i - min(I)+1)E[X_i]\\
% &=& Y + (|U||T|)^{-1}\sum_{i =1}^{w_N} i\frac{\size{I_T}}{w(I)}\\
% &=& Y + (|U||T|)^{-1}\size{I_T}\frac{w_N(w_N+1)}{2w(I)}\\
% \end{eqnarray*}

% \bibliographystyle{plain}
% \bibliography{statistics}

\end{document}
