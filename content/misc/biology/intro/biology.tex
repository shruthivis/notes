\documentclass[oneside, article]{memoir}

\usepackage{amsmath, amssymb}
\usepackage{hyperref, graphicx, verbatim, listings, multirow, subfigure}
\usepackage{algorithm, algorithmic}
% \usepackage[bottom]{footmisc}
\lstset{breaklines=true}
\setcounter{tocdepth}{3}

% Lets verbatim and verb environments automatically break lines.
\makeatletter
\def\@xobeysp{ }
\makeatother
% \lstset{breaklines=true,basicstyle=\ttfamily}

% Configuration for the memoir class.
\renewcommand{\cleardoublepage}{}
% \renewcommand*{\partpageend}{}
\renewcommand{\afterpartskip}{}
\maxsecnumdepth{subsubsection} % number subsections
\maxtocdepth{subsubsection}

\addtolength{\parindent}{-5mm}
% Packages not included:
% For multiline comments, use caption package. But this conflicts with hyperref while making html files.
% subfigure conflicts with use with memoir style-sheet.

% Use something like:
% % Use something like:
% % Use something like:
% \input{../../macros}

% groupings of objects.
\newcommand{\set}[1]{\left\{ #1 \right\}}
\newcommand{\seq}[1]{\left(#1\right)}
\newcommand{\ang}[1]{\langle#1\rangle}
\newcommand{\tuple}[1]{\left(#1\right)}

% numerical shortcuts.
\newcommand{\abs}[1]{\left| #1\right|}
\newcommand{\floor}[1]{\left\lfloor #1 \right\rfloor}
\newcommand{\ceil}[1]{\left\lceil #1 \right\rceil}

% linear algebra shortcuts.
\newcommand{\change}{\Delta}
\newcommand{\norm}[1]{\left\| #1\right\|}
\newcommand{\dprod}[1]{\langle#1\rangle}
\newcommand{\linspan}[1]{\langle#1\rangle}
\newcommand{\conj}[1]{\overline{#1}}
\newcommand{\gradient}{\nabla}
\newcommand{\der}{\frac{d}{dx}}
\newcommand{\lap}{\Delta}
\newcommand{\kron}{\otimes}
\newcommand{\nperp}{\nvdash}

\newcommand{\mat}[1]{\left( \begin{smallmatrix}#1 \end{smallmatrix} \right)}

% derivatives and limits
\newcommand{\partder}[2]{\frac{\partial #1}{\partial #2}}
\newcommand{\partdern}[3]{\frac{\partial^{#3} #1}{\partial #2^{#3}}}

% Arrows
\newcommand{\diverge}{\nearrow}
\newcommand{\notto}{\nrightarrow}
\newcommand{\up}{\uparrow}
\newcommand{\down}{\downarrow}
% gets and gives are defined!

% ordering operators
\newcommand{\oleq}{\preceq}
\newcommand{\ogeq}{\succeq}

% programming and logic operators
\newcommand{\dfn}{:=}
\newcommand{\assign}{:=}
\newcommand{\co}{\ co\ }
\newcommand{\en}{\ en\ }


% logic operators
\newcommand{\xor}{\oplus}
\newcommand{\Land}{\bigwedge}
\newcommand{\Lor}{\bigvee}
\newcommand{\finish}{$\Box$}
\newcommand{\contra}{\Rightarrow \Leftarrow}
\newcommand{\iseq}{\stackrel{_?}{=}}


% Set theory
\newcommand{\symdiff}{\Delta}
\newcommand{\union}{\cup}
\newcommand{\inters}{\cap}
\newcommand{\Union}{\bigcup}
\newcommand{\Inters}{\bigcap}
\newcommand{\nullSet}{\phi}

% graph theory
\newcommand{\nbd}{\Gamma}

% Script alphabets
% For reals, use \Re

% greek letters
\newcommand{\eps}{\epsilon}
\newcommand{\del}{\delta}
\newcommand{\ga}{\alpha}
\newcommand{\gb}{\beta}
\newcommand{\gd}{\del}
\newcommand{\gf}{\phi}
\newcommand{\gF}{\Phi}
\newcommand{\gl}{\lambda}
\newcommand{\gm}{\mu}
\newcommand{\gn}{\nu}
\newcommand{\gr}{\rho}
\newcommand{\gs}{\sigma}
\newcommand{\gt}{\theta}
\newcommand{\gx}{\xi}

\newcommand{\sw}{\sigma}
\newcommand{\SW}{\Sigma}
\newcommand{\ew}{\lambda}
\newcommand{\EW}{\Lambda}

\newcommand{\Del}{\Delta}
\newcommand{\gD}{\Delta}
\newcommand{\gG}{\Gamma}
\newcommand{\gO}{\Omega}
\newcommand{\gL}{\Lambda}
\newcommand{\gS}{\Sigma}

% Formatting shortcuts
\newcommand{\red}[1]{\textcolor{red}{#1}}
\newcommand{\blue}[1]{\textcolor{blue}{#1}}
\newcommand{\htext}[2]{\texorpdfstring{#1}{#2}}

% Statistics
\newcommand{\distr}{\sim}
\newcommand{\stddev}{\sigma}
\newcommand{\covmatrix}{\Sigma}
\newcommand{\mean}{\mu}
\newcommand{\param}{\gt}
\newcommand{\ftr}{\phi}

% General utility
\newcommand{\todo}[1]{\footnote{TODO: #1}}
\newcommand{\exclaim}[1]{{\textbf{\textit{#1}}}}
\newcommand{\tbc}{[\textbf{Incomplete}]}
\newcommand{\chk}{[\textbf{Check}]}
\newcommand{\oprob}{[\textbf{OP}]:}
\newcommand{\core}[1]{\textbf{Core Idea:}}
\newcommand{\why}{[\textbf{Find proof}]}
\newcommand{\opt}[1]{\textit{#1}}


\DeclareMathOperator*{\argmin}{arg\,min}
\DeclareMathOperator{\rank}{rank}
\newcommand{\redcol}[1]{\textcolor{red}{#1}}
\newcommand{\bluecol}[1]{\textcolor{blue}{#1}}
\newcommand{\greencol}[1]{\textcolor{green}{#1}}


\renewcommand{\~}{\htext{$\sim$}{~}}


% groupings of objects.
\newcommand{\set}[1]{\left\{ #1 \right\}}
\newcommand{\seq}[1]{\left(#1\right)}
\newcommand{\ang}[1]{\langle#1\rangle}
\newcommand{\tuple}[1]{\left(#1\right)}

% numerical shortcuts.
\newcommand{\abs}[1]{\left| #1\right|}
\newcommand{\floor}[1]{\left\lfloor #1 \right\rfloor}
\newcommand{\ceil}[1]{\left\lceil #1 \right\rceil}

% linear algebra shortcuts.
\newcommand{\change}{\Delta}
\newcommand{\norm}[1]{\left\| #1\right\|}
\newcommand{\dprod}[1]{\langle#1\rangle}
\newcommand{\linspan}[1]{\langle#1\rangle}
\newcommand{\conj}[1]{\overline{#1}}
\newcommand{\gradient}{\nabla}
\newcommand{\der}{\frac{d}{dx}}
\newcommand{\lap}{\Delta}
\newcommand{\kron}{\otimes}
\newcommand{\nperp}{\nvdash}

\newcommand{\mat}[1]{\left( \begin{smallmatrix}#1 \end{smallmatrix} \right)}

% derivatives and limits
\newcommand{\partder}[2]{\frac{\partial #1}{\partial #2}}
\newcommand{\partdern}[3]{\frac{\partial^{#3} #1}{\partial #2^{#3}}}

% Arrows
\newcommand{\diverge}{\nearrow}
\newcommand{\notto}{\nrightarrow}
\newcommand{\up}{\uparrow}
\newcommand{\down}{\downarrow}
% gets and gives are defined!

% ordering operators
\newcommand{\oleq}{\preceq}
\newcommand{\ogeq}{\succeq}

% programming and logic operators
\newcommand{\dfn}{:=}
\newcommand{\assign}{:=}
\newcommand{\co}{\ co\ }
\newcommand{\en}{\ en\ }


% logic operators
\newcommand{\xor}{\oplus}
\newcommand{\Land}{\bigwedge}
\newcommand{\Lor}{\bigvee}
\newcommand{\finish}{$\Box$}
\newcommand{\contra}{\Rightarrow \Leftarrow}
\newcommand{\iseq}{\stackrel{_?}{=}}


% Set theory
\newcommand{\symdiff}{\Delta}
\newcommand{\union}{\cup}
\newcommand{\inters}{\cap}
\newcommand{\Union}{\bigcup}
\newcommand{\Inters}{\bigcap}
\newcommand{\nullSet}{\phi}

% graph theory
\newcommand{\nbd}{\Gamma}

% Script alphabets
% For reals, use \Re

% greek letters
\newcommand{\eps}{\epsilon}
\newcommand{\del}{\delta}
\newcommand{\ga}{\alpha}
\newcommand{\gb}{\beta}
\newcommand{\gd}{\del}
\newcommand{\gf}{\phi}
\newcommand{\gF}{\Phi}
\newcommand{\gl}{\lambda}
\newcommand{\gm}{\mu}
\newcommand{\gn}{\nu}
\newcommand{\gr}{\rho}
\newcommand{\gs}{\sigma}
\newcommand{\gt}{\theta}
\newcommand{\gx}{\xi}

\newcommand{\sw}{\sigma}
\newcommand{\SW}{\Sigma}
\newcommand{\ew}{\lambda}
\newcommand{\EW}{\Lambda}

\newcommand{\Del}{\Delta}
\newcommand{\gD}{\Delta}
\newcommand{\gG}{\Gamma}
\newcommand{\gO}{\Omega}
\newcommand{\gL}{\Lambda}
\newcommand{\gS}{\Sigma}

% Formatting shortcuts
\newcommand{\red}[1]{\textcolor{red}{#1}}
\newcommand{\blue}[1]{\textcolor{blue}{#1}}
\newcommand{\htext}[2]{\texorpdfstring{#1}{#2}}

% Statistics
\newcommand{\distr}{\sim}
\newcommand{\stddev}{\sigma}
\newcommand{\covmatrix}{\Sigma}
\newcommand{\mean}{\mu}
\newcommand{\param}{\gt}
\newcommand{\ftr}{\phi}

% General utility
\newcommand{\todo}[1]{\footnote{TODO: #1}}
\newcommand{\exclaim}[1]{{\textbf{\textit{#1}}}}
\newcommand{\tbc}{[\textbf{Incomplete}]}
\newcommand{\chk}{[\textbf{Check}]}
\newcommand{\oprob}{[\textbf{OP}]:}
\newcommand{\core}[1]{\textbf{Core Idea:}}
\newcommand{\why}{[\textbf{Find proof}]}
\newcommand{\opt}[1]{\textit{#1}}


\DeclareMathOperator*{\argmin}{arg\,min}
\DeclareMathOperator{\rank}{rank}
\newcommand{\redcol}[1]{\textcolor{red}{#1}}
\newcommand{\bluecol}[1]{\textcolor{blue}{#1}}
\newcommand{\greencol}[1]{\textcolor{green}{#1}}


\renewcommand{\~}{\htext{$\sim$}{~}}


% groupings of objects.
\newcommand{\set}[1]{\left\{ #1 \right\}}
\newcommand{\seq}[1]{\left(#1\right)}
\newcommand{\ang}[1]{\langle#1\rangle}
\newcommand{\tuple}[1]{\left(#1\right)}

% numerical shortcuts.
\newcommand{\abs}[1]{\left| #1\right|}
\newcommand{\floor}[1]{\left\lfloor #1 \right\rfloor}
\newcommand{\ceil}[1]{\left\lceil #1 \right\rceil}

% linear algebra shortcuts.
\newcommand{\change}{\Delta}
\newcommand{\norm}[1]{\left\| #1\right\|}
\newcommand{\dprod}[1]{\langle#1\rangle}
\newcommand{\linspan}[1]{\langle#1\rangle}
\newcommand{\conj}[1]{\overline{#1}}
\newcommand{\gradient}{\nabla}
\newcommand{\der}{\frac{d}{dx}}
\newcommand{\lap}{\Delta}
\newcommand{\kron}{\otimes}
\newcommand{\nperp}{\nvdash}

\newcommand{\mat}[1]{\left( \begin{smallmatrix}#1 \end{smallmatrix} \right)}

% derivatives and limits
\newcommand{\partder}[2]{\frac{\partial #1}{\partial #2}}
\newcommand{\partdern}[3]{\frac{\partial^{#3} #1}{\partial #2^{#3}}}

% Arrows
\newcommand{\diverge}{\nearrow}
\newcommand{\notto}{\nrightarrow}
\newcommand{\up}{\uparrow}
\newcommand{\down}{\downarrow}
% gets and gives are defined!

% ordering operators
\newcommand{\oleq}{\preceq}
\newcommand{\ogeq}{\succeq}

% programming and logic operators
\newcommand{\dfn}{:=}
\newcommand{\assign}{:=}
\newcommand{\co}{\ co\ }
\newcommand{\en}{\ en\ }


% logic operators
\newcommand{\xor}{\oplus}
\newcommand{\Land}{\bigwedge}
\newcommand{\Lor}{\bigvee}
\newcommand{\finish}{$\Box$}
\newcommand{\contra}{\Rightarrow \Leftarrow}
\newcommand{\iseq}{\stackrel{_?}{=}}


% Set theory
\newcommand{\symdiff}{\Delta}
\newcommand{\union}{\cup}
\newcommand{\inters}{\cap}
\newcommand{\Union}{\bigcup}
\newcommand{\Inters}{\bigcap}
\newcommand{\nullSet}{\phi}

% graph theory
\newcommand{\nbd}{\Gamma}

% Script alphabets
% For reals, use \Re

% greek letters
\newcommand{\eps}{\epsilon}
\newcommand{\del}{\delta}
\newcommand{\ga}{\alpha}
\newcommand{\gb}{\beta}
\newcommand{\gd}{\del}
\newcommand{\gf}{\phi}
\newcommand{\gF}{\Phi}
\newcommand{\gl}{\lambda}
\newcommand{\gm}{\mu}
\newcommand{\gn}{\nu}
\newcommand{\gr}{\rho}
\newcommand{\gs}{\sigma}
\newcommand{\gt}{\theta}
\newcommand{\gx}{\xi}

\newcommand{\sw}{\sigma}
\newcommand{\SW}{\Sigma}
\newcommand{\ew}{\lambda}
\newcommand{\EW}{\Lambda}

\newcommand{\Del}{\Delta}
\newcommand{\gD}{\Delta}
\newcommand{\gG}{\Gamma}
\newcommand{\gO}{\Omega}
\newcommand{\gL}{\Lambda}
\newcommand{\gS}{\Sigma}

% Formatting shortcuts
\newcommand{\red}[1]{\textcolor{red}{#1}}
\newcommand{\blue}[1]{\textcolor{blue}{#1}}
\newcommand{\htext}[2]{\texorpdfstring{#1}{#2}}

% Statistics
\newcommand{\distr}{\sim}
\newcommand{\stddev}{\sigma}
\newcommand{\covmatrix}{\Sigma}
\newcommand{\mean}{\mu}
\newcommand{\param}{\gt}
\newcommand{\ftr}{\phi}

% General utility
\newcommand{\todo}[1]{\footnote{TODO: #1}}
\newcommand{\exclaim}[1]{{\textbf{\textit{#1}}}}
\newcommand{\tbc}{[\textbf{Incomplete}]}
\newcommand{\chk}{[\textbf{Check}]}
\newcommand{\oprob}{[\textbf{OP}]:}
\newcommand{\core}[1]{\textbf{Core Idea:}}
\newcommand{\why}{[\textbf{Find proof}]}
\newcommand{\opt}[1]{\textit{#1}}


\DeclareMathOperator*{\argmin}{arg\,min}
\DeclareMathOperator{\rank}{rank}
\newcommand{\redcol}[1]{\textcolor{red}{#1}}
\newcommand{\bluecol}[1]{\textcolor{blue}{#1}}
\newcommand{\greencol}[1]{\textcolor{green}{#1}}


\renewcommand{\~}{\htext{$\sim$}{~}}


%opening
\title{Biology: Quick reference}
\author{vishvAs vAsuki}

\begin{document}
\maketitle
\tableofcontents

\chapter{Prelude}
\section{Themes}
Empirical study of different organisms. How does the machinery of life work? How do animals interact with their environment?

Computational biology is considered elsewhere.

\section{Research approach}
It is mainly an observational discipline. Predictive aspects are gaining prominance: what things (Eg: genes) to observe to learn more about something?

Based on observation, scientists model the mechanism behind the\\ phenomenon. This can involve sophisticated mathematics, chemistry, physics etc..

\part{Organisms}
\chapter{Introduction}
\section{Defining life}
Life reproduces. Living organisms consume and transform energy. An organism will regulate its internal environment to maintain a stable and constant condition.

\section{Phenes}
Organisms have traits, or phenes. The way phenes are expressed is influenced by the organism's hereditary information and by the environment. Phenome and phenotype are defined similar to genes and genotypes (described elsewhere).



\section{Origin and equilibria}
New species of living organisms and the inherited traits which distinguish them are usually the product of the process of the arising of variation in traits, followed by selection of what (groups of) traits survive in competition with others for resources necessary for the organism's life. The arising of variation and the selection process may be natural or artificial.

Traits are passed on across generations by the means of programs which are recorded when an offspring is created.

\subsection{Natural selection}
Natural selection favors the survival of the fittest: a tautology, since 'fittest' is defined in terms of ability to survive and transmit the high-fitness characteristics.

\chapter{Hereditary information}
\section{Heridity}
Information which is responsible for the phenes of an organism is mostly hereditary. But, environmental factors do significantly influence the expression of this hereditary information.

\section{Information type}
There are protein-coding regions, non protein coding RNA regions, regulatory sequences, introns, and noncoding DNA.


\section{Chemical encoding, expression}
The most basic stage of expression of hereditary information is its chemical encoding. This happens in either the language of the DNA or the RNA.

\subsection{Genes}
Information-segments encoded in the form of a DNA are sometimes copied into an RNA sequence, which are sometimes inturn copied into amino-acid sequences (proteins).

Protein-coding genes are arguably the most important part of the hereditary information - because changes here can easily lead to markedly varied phenes.

\subsubsection{Genome, genotype}
The set of all genes in an organism, or a species is a genome. A genotype is a form of a particular gene which distinguishes an individual.

\subsubsection{Gene expression}
This is affected by many environmental factors (Epigenetics).

\subsection{Variation}
\subsubsection{Types}
Types of variations include point mutations, recombinations, insertions, deletions in copy stored using the DNA.

\subsubsection{Source}
These variations could be the result of copying errors during reproduction, intermixing of DNA during reproduction, damage to DNA due to radiation or chemicals, gene insertions/ transfers via viruses.

\section{Variation}
The genes and phenes of organisms vary. These variations arise due to environmental processes such as radiation, chemicals, infection etc..

\subsection{Homology}
Homologous genes/ phenes are similar genes/ phenes occurring in different species. If the homology arose due to a speciation event, it is called orthology.

\section{Human genome}
There are 23 pairs of chromosomes in a diploid cell, with over $3*10^9$ base pairs and $\approx 23000$ protein coding genes.

\chapter{Cells, capsules}
\section{Organelles}
\subsection{Nucleus}
Cell may be Eukaryote or prokaryote. In the former case, it has a well defined neucleus, with chromosomes, in which DNA is bundled tightly.


\section{Cell}
Cells are the basic unit of life: they consist of genetic material together with apparatus to create and expend energy in order for the genes to survive and proliferate. 

\subsection{Differentiation}
Cell take on different functions based on their circumstance.

\subsection{Reproduction}
\subsubsection{Asexual reproduction}
Cells divide to form other cells. 'Mitosis' is asexual reproduction: copies of the DNA are made.

\subsubsection{Gamete formation}
Gamete is a cell with only one set of genes: it is haploid. It is generated during the first stage of 'meiosis', a type of cell division. During this process, gene exchange occurs among the chromosomes, so that the gamete formed is a unique combination of genes.

\subsubsection{Gamete combination}
The gametes combine, and a new organism with distinct gene-combination is created.

\section{Chromosome Structure}
Genes are contained in clusters called chromosomes.

\subsection{Chromosome number}
Cells may be diploid or haploid, depending on the number of genes it has. In diploid cells, each copy comes from one parent.

Ordinary human cells have 23 pairs of chromosomes. 

\subsection{Special chromosomes}
Sex is usually determined by the types of sex chromosomes an entity has.

\chapter{Evolutionary tree}
Based on similarity and dissimilarity of genomes, one can organize various species / classes of organisms into a tree.

\section{Species of cellular life}
The top branches (superregna) are: archaea, bacteria and eukaryota (true nucleus). The prior two are called prokaryota (primitive - neucleus).

The top branches of Eukaryota (regna) include plants, animals, fungi and protista (amoeba and some other primitive classes of organisms).

A notable branch of animals include the phylum chordata - subphylum vertebrata. The subphylum vertebrata includes classes named mammals, fish, reptiles, birds, amphibians.


\part{Adaptations}
\chapter{Locomotion}
Whales, having to move a heavy body through the ocean, have high musculature.

Hitching a ride: Parasites do this.

\chapter{Nutrient absorption}
Here, we consider adaptations for absorbing nutrients from the surrounding media. Overcoming other organisms for this purpose is considered elsewhere.

Breathing \tbc

Temperature control adaptations \tbc

\chapter{Feeding}
Plant cells have organelles called platelets which enable photosynthesis of food.

\section{Predation}
\subsection{Catching food}
Predators may be mostly static, lying in wait  for prey to come to them. Or, they may actively move and entrap or chase down food.

\subsubsection{Baiting}
Predators may attract food organisms to a certain location for entrapment. Eg: brightly colored flowers of the venus fly trap, the attractive tail of the rattle-snake.

\subsubsection{Overcoming a single prey}
A prey is usually overcome using asphyxiation, bleeding, cutting, crushing or poison. Sometimes, this is done by extending cell boundaries around the prey, as in the case of amoeba and protozoa.
Powerful instruments like claws, jaws and teeth - even special tools - overcome resistance from the prey organism and kill it.

\subsubsection{Catching multiple individuals}
Intelligent strategies like corralling with bubbles or sand from a sea bottom are observed among marine mammal populations.

Filter feeding is a popular strategy, in which media containing prey organisms is channelled through a special organ. Eg: Baleen whales filter-feeding on krill.

Or food may be grown. Eg: agriculture by humans, cultivation of fungi by leaf-cutter ants.
Tools such as webs are used for trapping food.

\subsection{Protecting the food}
Food captured may be taken away by another predator. So, some predators are motivated to drag/ hide their food away. Eg: Cheetas drag their food up tall trees.

\subsection{Digesting the food}
Sometimes food is ingested into a digestive organ. Before doing so, the large organisms may be broken into small pieces (chewed), or it may be forced inside, as in snakes.

In ingesting the food, it may either be drawn to the mouth using appendages like hands, feet or tentacles, or the mouth may be drawn to it.

Otherwise, the food may be digested externally and the nutrients later sucked in. Eg: sea stars, some snakes and spiders.

\subsection{Camoflage adaptations}
It is advantageous for both predator and prey to blend into the environment so as to not fall into each others’ notice.

\section{Anti-predation adaptations}
Camouflage is considered elsewhere.

Toxicity: Many animals and plants develop toxicity/ poison to defend themselves against predators. 

Standing-out adaptations: To advertise their toxicity to potential predators, poisonous animals may be brightly colored.

\subsection{Behaviors}
Safety in numbers strategy can confound predators’ ability to track down a single individual. Eg: schools of coral fish, herds of herbivores in grasslands.

More complex social adaptations are described in a separate chapter.

\section{Symbiosis}
As genes evolve interacting with one another, organisms coevolve to form mutually beneficial relationships. Example: A huge fraction (80\%) of the cells in the human body are microbes in the gut, which help digest food efficiently.

\chapter{Social adaptations}
\section{Kindness}
For close relatives, there is a strong altruistic instinct because of shared genes. For others, reciprocal altruism has evolved.

\subsection{Reciprocal altruism}
In the prisoner’s dilemma game, every player has an option of being nice or nasty. For repeated games of the prisoner’s dilemma, in a famous tournament conducted by Axelrod, ‘tit for tat’ won. ‘Tit for tat’ says: ‘Be nice first. In round i do what the opponent did to you in round i-1.’ Observed in the period of truce between the British and the Germans in WW1 trenches. Also seen in vampire bats which vomit some blood for bats who did not get a meal, but who have not refused them in the past. But it is not perfect: there is the problem of endless retaliation when two quasi-tit-for-tat players play each other and one of them happens to be nasty initially. This problem is solved by randomness.

Smarter models would generalize past experiences and learn how to react to various kinds of situations.

\subsection{Social reputation}
Thus cultures of man have developed to praise kindness and heroism, especially if it is conspicuous. They have also developed means of spreading along reputation: awards are an example.

\chapter{Reproduction}
Here, rather than considering reproduction of cells in the body of a multicellular animal, we are concerned with production of entire new generations of organisms.

\section{Sexual vs asexual}
Reproduction can be sexual or asexual. Some organisms are capable of both - they are aka hermaphrodites.

\subsection{Sexual reproduction}
\subsubsection{Combination of genes}
Sexual reproduction involves combination of genetic material from multiple (usually 2) organisms to produce genetic material for the next generation. Usually these parent organisms are of the same species.

The advantage of sexual reproduction is that it provides a reliable way to generate variation in traits. Providing material for natural selection, this increases the overall chances of survival of the species.

\subsubsection{Static parts of the genome}
Some parts of the DNA (eg: the Y chromosome in Human males) undergo no change during sexual reproduction. So, variation in this chromosome occurs due to other causes - ie far more infrequently.

This then provides a way to study/ trace the ancestry of organisms. By this technique, it has been hypothesized that Gengheis Khan is the most (reproductively) successful human male ever.

\subsection{Asexual reproduction}
Among bacteria, which usually undergo asexual reproduction, viruses play an important part in transferring genes from one organism to another - thereby generating variation in the population.



\section{Fertilization organs}
Males and females have organs to get the sperm and the egg to unite. In case of mammals, males have a specialized organ penis (separate from the anal tract) to deliver sperms to the female; and females have a specialized organ separate from both the anal and urinary tracts to receive sperms from the male. In case of birds, reptiles and fish, mostly chloaca are used, which is a common egress for sprems, eggs, feces and urine; though in exceptional cases like water-fowl, males possess a separate phallus and females may have a separate subchannel in their chloaca to receive the phallus.

\chapter{Signalling chemicals}
Aka hormones. Hormone levels are not only influenced by external stimulii, but also by the developmental history and previous internal state of the organism.

\section{Tracking time}
In animals, there is the darkness-hormone, melatonin which then acts as a trigger for sleep. Increasing melatonin secretion frequency may trigger seasonal behavior like mating, change in coat etc..

\section{Learning, attention, reward}
The dopamine system identifies patterns in the environment. Dopamine is most  secreted in anticipation of reward, rather than due to the obtainment of reward itself. Eg: If a bell following a flash of light signifies arrival of food, the dopamine neurons fire when there is a flash of light, and not upon the ring of the bell or the arrival of food.

Sometimes the dopamine system makes mistakes. Patients prescribed dopamines tend to fall to gambling addiction. Even pegions can develop superstitions (studied in behaviorist experiments).

\subsection{Well-being, analgesia}
Endogenous morphine (endorphine) is associated with this. It is produced in response to exercise, post-crisis stress-release, laughter.

\section{Empathy, social trust, reciprocation}
Oxytocin is deeply involved in the proximal mechanism for generating empathy, trust and reciprocation.

This is observed in the famous experiment to measure trust: A is given \$10 and is given an option of transferring part of it to B, with whom it will be tripled, with the expectation that B might return some money if he feels like it.

\subsection{Inhibition}
Inhibitory factors include Stress, history of abuse, testosterone.

\section{Fight or flight}
Adrenaline increases blood pressure, increases blood flow to the muscles and brain in crisis situations.



\part{Physiology}
Physical and chemical functions of the tissues, organs, and organ systems.

\chapter{Diseases}
\section{Cancer}
Sometimes, cells in an organism go rogue; they don't listen to signals from other cells, and start reproducing and form a tumor; they hog the nutrient supply coming to that region. As they reproduce, they make inexact copies of themselves, and they fight with each other for the limited nutrient supply: the 'survival of the fittest' game is seen within the tumor, and repeatedly one type of cell comes to dominate others.

Often, after many generations some of these cells evolve the ability to swim in blood vessels and deposit themselves in other parts of the organism (or even in other organisms in rare cases such as the Tasmanian devil mouth tumor epidemic). THe cancer is then said to be malignant.

\chapter{Human nervous system}
\section{Brain construction}
\subsection{Wiring and insulation}
Brain is has a lot of many armed cells called neurons (grey matter). Using these arms (axons) they communicate with each other using electrical signals. Assisting this signaling but decreasing branching flexibility of grey matter, there exists an insulatory 'myelin' sheath around axons formed by fatty, cholesterol-built cells (white matter). 

\subsection{Plumbing}
The brain is fed by a lot of capillaries.  Brain consumes a huge fraction of oxygen obtained by breathing. Ruptured vessels lead to oxygen and food deprivation, and cells begin to die within a few minutes.

So it is sensitive to high blood pressure and concussions (considered elsewhere). 

\subsection{Protection}
For protection from impact, there is the skull. For immunological protection, there is the famed blood-brain barrier.

\section{Brain parts and regions}
\subsection{Connection to function}
Different areas have cells focused on different functions. Ultimately, however, different firing patterns correspond to different computations.

\subsection{Areas}
It is roughly divided into two hemishperes. 
Corpus collussum is a bunch of wires incharge of inter-hemisphere communication.

Forebrain is associated with higher cognition.

\subsubsection{Close to brain stem}
Areas close to the brain stem (in the rear of the brain) are in-charge of behaviorally basic functions, such as vision, movement, and fundamental processing. This is the brain of our evolutionary ancestors.

\subsubsection{Hippocampus}
Hippocampus is a sort of memory directory, strongly associated with emotion. It is not where the memories themselves are stored, but where the 'index' to the memories are stored. Damage to hippocampus is associated with amnesia.

\section{Development stages}
\subsection{Teenage}
The teenage brain is in a great state of flux. Plus, different hormones begin to act during teenage years.

There is much synaptic pruning (removal of unused circuits), increase of myelin sheaths/ white matter (making circuits more efficient and less flexible). These changes move in a slow wave from the brain's rear (in charge of basic functions) to its front (in charge of complicated thinking). Corpus collussum thickens. Stronger links also develop between the hippocampus and the forebrain.

Effect of these changes on behavior is considered elsewhere.

\section{Learning}
Chemicals along with firing patterns trigger changes in brain wiring.

\subsection{Fire together, wire together}
Acquisition of new skills and habits involves the growth, and the reinforcement, with insulation, of neural pathways activated together frequently: neurons which fire together wire together. This is in keeping with the associative nature of neurons in the brain. This is true irrespective of whether an instinctive skill (eg: tennis, predicting complex sequence of events) or conceptual knowledge is being learned.

\subsection{Sequence prediction}
The role of the Dopamine system is considered elsewhere.


\part{Ecology}
How do various organisms interrelate with their environment?


% \bibliographystyle{plain}
% \bibliography{chemistry}

\end{document}
