In this section, we examine our first attempts at the problem of identifying the influence of genes in human diseases. In particular, we try to predict novel links or associations in the human gene-phenotype association matrix. We extend some of the methods that were used for link prediction in social networks in recent research work conducted in the PI's lab. We will see that the results of even our preliminary experiments are promising, yielding results comparable to the significant results by McGary et al \cite{McGaryOrthologousPhenotypes}. We will also see that there is more scope for effectively exploiting the multiple sources of information. 

We use $P_i$ to refer to the gene-phenotype matrix of species $i$, $i = 1, 2, \dots 6$ and $G$ to denote the Human gene-gene matrix. In particular, $P_1$ refers to the human gene-phenotype matrix in which we make predictions.

\subsection{PI's Recent research}
PI Dhillon's research is in the areas of large-scale data analysis, data mining and machine learning, where mathematical and computational tools are used for various predictive tasks. Over the past year, Dhillon has started a concerted research program in large-scale network analysis as evidenced by several recent publications\cite{savasQiuClusteredEmbedding, vasukiNatarajan, berkantSupervised, PrateekSVP, mekaMatrixCompletion,zhengdongAgarwal}. The tasks undertaken in this research can be broadly categorized as scalable inference/prediction in massive networks, for example, predicting friendship relationships in online social networks\cite{savasQiuClusteredEmbedding, vasukiNatarajan}, matrix completion\cite{PrateekSVP, mekaMatrixCompletion}, and mining multiple networks\cite{berkantSupervised}. This recent research involves mathematical analysis, for example, generalizing the compressed sensing work of Candes and Tao, as well as scalable implementations. The test beds for the developed methods so far have been online social networks, collaboration networks and commercial networks (for example, call phone graphs).

In particular, two models have been proposed by the group, for the problem of link prediction in affiliation networks, namely the Graph Proximity model and the Latent Factor model\cite{vasukiNatarajan}. Here, the problem of recommending affiliations to users of a social network is posed as one of link prediction in an affiliation network, which can be solved using methods based on graph proximity or by using matrix completion methods. The problem of predicting affiliations of genes with phenotypes can be posed similarly as one of link prediction in the gene-phenotype network.

\subsection{Methods based on graph proximity}
We introduced graph proximity approaches in Section \ref{section:graphProximityIntro}. In such methods, the identification of new links is based on the proximity of nodes in the graph. Recall that the relationships between genes and phenotypes may be modelled as edges in a graph. Proximity between genes can then be calculated as the weighted sum of the number of paths connecting two genes with varying lengths. The \emph{Katz} measure is one of the commonly used measures that is based on graph proximity. In our experiments, we use a truncated \emph{Katz}-measure defined as $tKatz(A, \beta, k) = \sum_{i=1}^{k}\beta^i A^i$. We will see some interesting choices for $A$, and how the performance of link prediction is improved with certain choices, in our preliminary experiments. 

We now discuss a few predictors used in our preliminary experiments, based on this similarity measure. These early experiments are based on computing a measure of similarity between all human genes, expressed by a $|genes(Hs) \times genes(Hs)|$ matrix S. The association of genes with phenotypes is then determined by computing the $|genes(Hs) \times phenotypes(Hs)|$ score matrix $S P_1$. For a given phenotype, these scores are used to identify genes which may play a role in a disease. These preliminary methods differ only in the way they compute S. We describe these different ways of computing S below.

\begin{itemize}
 \item One basic way of computing S is to use the human gene-gene interaction matrix, computing $S = tKatz(G, \gb, k)$.
 \item Another way of obtaining the relationships between two human genes is to find the number of common diseases which they both influence. The resulting similarity graph among genes is given by the matrix $P_1P_1^T$. Then, one can use $S = tKatz(P_1P_1^T, \beta, k)$.
 \item Another related way of obtaining the relationships between the human genes is to consider the network formed by the genes and phenotypes of all species other than humans. Here, \\$S = tKatz(\sum_{i>1} P_i P_i^T, \beta, k)$ is a measure of similarity between the genes.
\end{itemize}

\subsection{Methods based on matrix completion}
We now discuss some early experiments exploring the use of matrix completion approaches in identifying gene-disease connections. Recall that such approaches were introduced in Section \ref{section:matrixComp}. Suppose that we had a gene-gene affinity matrix S, which could also be viewed as an affinity network among genes. We then consider the combined graph, whose nodes are human genes and human diseases. The adjacency matrix of this graph is given by: 
$$C(S) = \begin{bmatrix}\lambda S & P_1 \\P_1^T & 0 \end{bmatrix}. \label{eqn:C}$$
Here, $\lambda$ controls the ratio of the weight assigned to gene-gene interaction network to that of the gene-phenotype network.

We look upon the zeros, particularly those in the $P_1$ portion of C as missing entries. The gene-disease link identification problem is then one of matrix completion. We tackle this problem by approximating $P_1$ with a low rank matrix: $$P_1 \approx Q_g^T R_p,$$ where $Q_g$ is the matrix of gene factors and $R_p$ is the matrix of phenotype factors. We want to find the gene and phenotype factors such that the reconstruction error on the observed entries in $C(S)$ is minimized. This is similar to the formulation of a similar model proposed in \cite{vasukiNatarajan}. We compute the $d$-rank approximation of $C$ using Singular Value Decomposition (SVD), and use the resulting $|genes(Hs) \times phenotypes(Hs)|$ matrix $U_d \Sigma_d V_d^T$ as the score matrix. These scores are used to identify genes which may play a role in a given disease. The parameters $\lambda, d$ are learned through cross-validation and the performance achieved with the best parameters is reported in Table \ref{tab:results}.

\subsection{Discussion of preliminary results}
We now present the preliminary experimental results. The prediction is made on the human gene-disease association matrix. Given a disease, a predictor produces a score for each gene, with higher scores assigned to genes with higher likelihood of influencing the disease as determined by the predictor. Recall the analytical evaluation methodology introduced in section \ref{section:evaluation}. We use average AUC as the measure of performance of predictors in our experiments. In all the experiments, we considered only those human diseases in which at least 2 genes were observed.

Table \ref{tab:results} provides a summary of the preliminary experimental results. Two AUC plots, showing the performance of two graphical proximity based methods are presented in Figure \ref{fig:katzAUC}.
\begin{table} [ht]
\centering
\begin{tabular}{| p{7 cm} | c | c |} \hline
Predictor & Avg AUC & Success fraction\\ \hline
$tKatz(P_1P_1^T, 10^{-6}, 1)$ & 0.604 & 101 \\ \hline
$tKatz(\sum_{i>1} P_iP_i^T, 10^{-6}, 1)$& 0.773 & 228 \\ \hline
$tKatz(G, 10^{-6}, 3)$& 0.776 & 236 \\ \hline
$tKatz(G, 10^{-6}, 4)$& 0.781 & 233 \\ \hline
SVD($C(G)$) & 0.596 & 175 \\ \hline
SVD($C(\sum_{i>1} P_iP_i^T,P_1$)) & 0.748 & 233 \\ \hline
\end{tabular}
\caption{Comparison of the performance of various predictors on the biological dataset. The last column 'Success fraction' denotes the number of diseases for which AUC was observed to be better than random, which has AUC equal to 0.5.}
\label{tab:results}
\end{table}

\begin{figure}[ht]
\begin{center}
\subfigure[$tKatz(G, \beta, 4)$]{\includegraphics[height=4cm, width=7cm]{figures/katzRandom.jpg}}
\subfigure[$tKatz(\sum_{i>1} P_iP_i^T, \beta, 1)$]{\includegraphics[height=4cm, width=7cm]{figures/katzOtherSpecies.jpg}}
\end{center}
\vspace{-5ex}
\caption{AUC curve for some of the predictors. Observe that the predictors perform better than random (0.5 AUC) for more than 85\% of the diseases.}
\label{fig:katzAUC}
\end{figure}

We observe that methods based on both the graph proximity approach and matrix completion approach yield promising results. All the methods used are simple in that they did not use information from all the networks in identifying gene-disease links. Even these simple methods have yielded results (see Figure \ref{fig:katzAUC}), comparable to those achieved in \cite{McGaryOrthologousPhenotypes} using a neighborhood-based probabilistic model. The success of early methods that rely only on gene-phenotype networks from non-human species and the human gene-interaction network indicate that a wealth of information is available besides prior observations of gene-disease interactions, and that this information can be fruitfully exploited in identifying the genes involved in human genetic diseases.