\documentclass[11pt]{article}
% \usepackage[left=3cm,top=1.5cm,right=3cm,bottom=3cm]{geometry}
\usepackage{caption}
\usepackage{amsmath, amssymb}
\usepackage{hyperref, graphicx, verbatim, listings, multirow, subfigure}
\usepackage{algorithm, algorithmic}
% \usepackage[bottom]{footmisc}
\lstset{breaklines=true}
\setcounter{tocdepth}{3}

% Lets verbatim and verb environments automatically break lines.
\makeatletter
\def\@xobeysp{ }
\makeatother
% \lstset{breaklines=true,basicstyle=\ttfamily}

\textheight 8.8in
\textwidth 6.4in
\oddsidemargin +0.05in
\evensidemargin +0.05in
%\textheight 9.0in
%\textwidth 6.5in
%\oddsidemargin 0in
%\evensidemargin 0in
\topmargin 0in
\headheight 0in
\headsep 0in

\thispagestyle{empty}

% Use something like:
% % Use something like:
% % Use something like:
% \input{../../macros}

% groupings of objects.
\newcommand{\set}[1]{\left\{ #1 \right\}}
\newcommand{\seq}[1]{\left(#1\right)}
\newcommand{\ang}[1]{\langle#1\rangle}
\newcommand{\tuple}[1]{\left(#1\right)}

% numerical shortcuts.
\newcommand{\abs}[1]{\left| #1\right|}
\newcommand{\floor}[1]{\left\lfloor #1 \right\rfloor}
\newcommand{\ceil}[1]{\left\lceil #1 \right\rceil}

% linear algebra shortcuts.
\newcommand{\change}{\Delta}
\newcommand{\norm}[1]{\left\| #1\right\|}
\newcommand{\dprod}[1]{\langle#1\rangle}
\newcommand{\linspan}[1]{\langle#1\rangle}
\newcommand{\conj}[1]{\overline{#1}}
\newcommand{\gradient}{\nabla}
\newcommand{\der}{\frac{d}{dx}}
\newcommand{\lap}{\Delta}
\newcommand{\kron}{\otimes}
\newcommand{\nperp}{\nvdash}

\newcommand{\mat}[1]{\left( \begin{smallmatrix}#1 \end{smallmatrix} \right)}

% derivatives and limits
\newcommand{\partder}[2]{\frac{\partial #1}{\partial #2}}
\newcommand{\partdern}[3]{\frac{\partial^{#3} #1}{\partial #2^{#3}}}

% Arrows
\newcommand{\diverge}{\nearrow}
\newcommand{\notto}{\nrightarrow}
\newcommand{\up}{\uparrow}
\newcommand{\down}{\downarrow}
% gets and gives are defined!

% ordering operators
\newcommand{\oleq}{\preceq}
\newcommand{\ogeq}{\succeq}

% programming and logic operators
\newcommand{\dfn}{:=}
\newcommand{\assign}{:=}
\newcommand{\co}{\ co\ }
\newcommand{\en}{\ en\ }


% logic operators
\newcommand{\xor}{\oplus}
\newcommand{\Land}{\bigwedge}
\newcommand{\Lor}{\bigvee}
\newcommand{\finish}{$\Box$}
\newcommand{\contra}{\Rightarrow \Leftarrow}
\newcommand{\iseq}{\stackrel{_?}{=}}


% Set theory
\newcommand{\symdiff}{\Delta}
\newcommand{\union}{\cup}
\newcommand{\inters}{\cap}
\newcommand{\Union}{\bigcup}
\newcommand{\Inters}{\bigcap}
\newcommand{\nullSet}{\phi}

% graph theory
\newcommand{\nbd}{\Gamma}

% Script alphabets
% For reals, use \Re

% greek letters
\newcommand{\eps}{\epsilon}
\newcommand{\del}{\delta}
\newcommand{\ga}{\alpha}
\newcommand{\gb}{\beta}
\newcommand{\gd}{\del}
\newcommand{\gf}{\phi}
\newcommand{\gF}{\Phi}
\newcommand{\gl}{\lambda}
\newcommand{\gm}{\mu}
\newcommand{\gn}{\nu}
\newcommand{\gr}{\rho}
\newcommand{\gs}{\sigma}
\newcommand{\gt}{\theta}
\newcommand{\gx}{\xi}

\newcommand{\sw}{\sigma}
\newcommand{\SW}{\Sigma}
\newcommand{\ew}{\lambda}
\newcommand{\EW}{\Lambda}

\newcommand{\Del}{\Delta}
\newcommand{\gD}{\Delta}
\newcommand{\gG}{\Gamma}
\newcommand{\gO}{\Omega}
\newcommand{\gL}{\Lambda}
\newcommand{\gS}{\Sigma}

% Formatting shortcuts
\newcommand{\red}[1]{\textcolor{red}{#1}}
\newcommand{\blue}[1]{\textcolor{blue}{#1}}
\newcommand{\htext}[2]{\texorpdfstring{#1}{#2}}

% Statistics
\newcommand{\distr}{\sim}
\newcommand{\stddev}{\sigma}
\newcommand{\covmatrix}{\Sigma}
\newcommand{\mean}{\mu}
\newcommand{\param}{\gt}
\newcommand{\ftr}{\phi}

% General utility
\newcommand{\todo}[1]{\footnote{TODO: #1}}
\newcommand{\exclaim}[1]{{\textbf{\textit{#1}}}}
\newcommand{\tbc}{[\textbf{Incomplete}]}
\newcommand{\chk}{[\textbf{Check}]}
\newcommand{\oprob}{[\textbf{OP}]:}
\newcommand{\core}[1]{\textbf{Core Idea:}}
\newcommand{\why}{[\textbf{Find proof}]}
\newcommand{\opt}[1]{\textit{#1}}


\DeclareMathOperator*{\argmin}{arg\,min}
\DeclareMathOperator{\rank}{rank}
\newcommand{\redcol}[1]{\textcolor{red}{#1}}
\newcommand{\bluecol}[1]{\textcolor{blue}{#1}}
\newcommand{\greencol}[1]{\textcolor{green}{#1}}


\renewcommand{\~}{\htext{$\sim$}{~}}


% groupings of objects.
\newcommand{\set}[1]{\left\{ #1 \right\}}
\newcommand{\seq}[1]{\left(#1\right)}
\newcommand{\ang}[1]{\langle#1\rangle}
\newcommand{\tuple}[1]{\left(#1\right)}

% numerical shortcuts.
\newcommand{\abs}[1]{\left| #1\right|}
\newcommand{\floor}[1]{\left\lfloor #1 \right\rfloor}
\newcommand{\ceil}[1]{\left\lceil #1 \right\rceil}

% linear algebra shortcuts.
\newcommand{\change}{\Delta}
\newcommand{\norm}[1]{\left\| #1\right\|}
\newcommand{\dprod}[1]{\langle#1\rangle}
\newcommand{\linspan}[1]{\langle#1\rangle}
\newcommand{\conj}[1]{\overline{#1}}
\newcommand{\gradient}{\nabla}
\newcommand{\der}{\frac{d}{dx}}
\newcommand{\lap}{\Delta}
\newcommand{\kron}{\otimes}
\newcommand{\nperp}{\nvdash}

\newcommand{\mat}[1]{\left( \begin{smallmatrix}#1 \end{smallmatrix} \right)}

% derivatives and limits
\newcommand{\partder}[2]{\frac{\partial #1}{\partial #2}}
\newcommand{\partdern}[3]{\frac{\partial^{#3} #1}{\partial #2^{#3}}}

% Arrows
\newcommand{\diverge}{\nearrow}
\newcommand{\notto}{\nrightarrow}
\newcommand{\up}{\uparrow}
\newcommand{\down}{\downarrow}
% gets and gives are defined!

% ordering operators
\newcommand{\oleq}{\preceq}
\newcommand{\ogeq}{\succeq}

% programming and logic operators
\newcommand{\dfn}{:=}
\newcommand{\assign}{:=}
\newcommand{\co}{\ co\ }
\newcommand{\en}{\ en\ }


% logic operators
\newcommand{\xor}{\oplus}
\newcommand{\Land}{\bigwedge}
\newcommand{\Lor}{\bigvee}
\newcommand{\finish}{$\Box$}
\newcommand{\contra}{\Rightarrow \Leftarrow}
\newcommand{\iseq}{\stackrel{_?}{=}}


% Set theory
\newcommand{\symdiff}{\Delta}
\newcommand{\union}{\cup}
\newcommand{\inters}{\cap}
\newcommand{\Union}{\bigcup}
\newcommand{\Inters}{\bigcap}
\newcommand{\nullSet}{\phi}

% graph theory
\newcommand{\nbd}{\Gamma}

% Script alphabets
% For reals, use \Re

% greek letters
\newcommand{\eps}{\epsilon}
\newcommand{\del}{\delta}
\newcommand{\ga}{\alpha}
\newcommand{\gb}{\beta}
\newcommand{\gd}{\del}
\newcommand{\gf}{\phi}
\newcommand{\gF}{\Phi}
\newcommand{\gl}{\lambda}
\newcommand{\gm}{\mu}
\newcommand{\gn}{\nu}
\newcommand{\gr}{\rho}
\newcommand{\gs}{\sigma}
\newcommand{\gt}{\theta}
\newcommand{\gx}{\xi}

\newcommand{\sw}{\sigma}
\newcommand{\SW}{\Sigma}
\newcommand{\ew}{\lambda}
\newcommand{\EW}{\Lambda}

\newcommand{\Del}{\Delta}
\newcommand{\gD}{\Delta}
\newcommand{\gG}{\Gamma}
\newcommand{\gO}{\Omega}
\newcommand{\gL}{\Lambda}
\newcommand{\gS}{\Sigma}

% Formatting shortcuts
\newcommand{\red}[1]{\textcolor{red}{#1}}
\newcommand{\blue}[1]{\textcolor{blue}{#1}}
\newcommand{\htext}[2]{\texorpdfstring{#1}{#2}}

% Statistics
\newcommand{\distr}{\sim}
\newcommand{\stddev}{\sigma}
\newcommand{\covmatrix}{\Sigma}
\newcommand{\mean}{\mu}
\newcommand{\param}{\gt}
\newcommand{\ftr}{\phi}

% General utility
\newcommand{\todo}[1]{\footnote{TODO: #1}}
\newcommand{\exclaim}[1]{{\textbf{\textit{#1}}}}
\newcommand{\tbc}{[\textbf{Incomplete}]}
\newcommand{\chk}{[\textbf{Check}]}
\newcommand{\oprob}{[\textbf{OP}]:}
\newcommand{\core}[1]{\textbf{Core Idea:}}
\newcommand{\why}{[\textbf{Find proof}]}
\newcommand{\opt}[1]{\textit{#1}}


\DeclareMathOperator*{\argmin}{arg\,min}
\DeclareMathOperator{\rank}{rank}
\newcommand{\redcol}[1]{\textcolor{red}{#1}}
\newcommand{\bluecol}[1]{\textcolor{blue}{#1}}
\newcommand{\greencol}[1]{\textcolor{green}{#1}}


\renewcommand{\~}{\htext{$\sim$}{~}}


% groupings of objects.
\newcommand{\set}[1]{\left\{ #1 \right\}}
\newcommand{\seq}[1]{\left(#1\right)}
\newcommand{\ang}[1]{\langle#1\rangle}
\newcommand{\tuple}[1]{\left(#1\right)}

% numerical shortcuts.
\newcommand{\abs}[1]{\left| #1\right|}
\newcommand{\floor}[1]{\left\lfloor #1 \right\rfloor}
\newcommand{\ceil}[1]{\left\lceil #1 \right\rceil}

% linear algebra shortcuts.
\newcommand{\change}{\Delta}
\newcommand{\norm}[1]{\left\| #1\right\|}
\newcommand{\dprod}[1]{\langle#1\rangle}
\newcommand{\linspan}[1]{\langle#1\rangle}
\newcommand{\conj}[1]{\overline{#1}}
\newcommand{\gradient}{\nabla}
\newcommand{\der}{\frac{d}{dx}}
\newcommand{\lap}{\Delta}
\newcommand{\kron}{\otimes}
\newcommand{\nperp}{\nvdash}

\newcommand{\mat}[1]{\left( \begin{smallmatrix}#1 \end{smallmatrix} \right)}

% derivatives and limits
\newcommand{\partder}[2]{\frac{\partial #1}{\partial #2}}
\newcommand{\partdern}[3]{\frac{\partial^{#3} #1}{\partial #2^{#3}}}

% Arrows
\newcommand{\diverge}{\nearrow}
\newcommand{\notto}{\nrightarrow}
\newcommand{\up}{\uparrow}
\newcommand{\down}{\downarrow}
% gets and gives are defined!

% ordering operators
\newcommand{\oleq}{\preceq}
\newcommand{\ogeq}{\succeq}

% programming and logic operators
\newcommand{\dfn}{:=}
\newcommand{\assign}{:=}
\newcommand{\co}{\ co\ }
\newcommand{\en}{\ en\ }


% logic operators
\newcommand{\xor}{\oplus}
\newcommand{\Land}{\bigwedge}
\newcommand{\Lor}{\bigvee}
\newcommand{\finish}{$\Box$}
\newcommand{\contra}{\Rightarrow \Leftarrow}
\newcommand{\iseq}{\stackrel{_?}{=}}


% Set theory
\newcommand{\symdiff}{\Delta}
\newcommand{\union}{\cup}
\newcommand{\inters}{\cap}
\newcommand{\Union}{\bigcup}
\newcommand{\Inters}{\bigcap}
\newcommand{\nullSet}{\phi}

% graph theory
\newcommand{\nbd}{\Gamma}

% Script alphabets
% For reals, use \Re

% greek letters
\newcommand{\eps}{\epsilon}
\newcommand{\del}{\delta}
\newcommand{\ga}{\alpha}
\newcommand{\gb}{\beta}
\newcommand{\gd}{\del}
\newcommand{\gf}{\phi}
\newcommand{\gF}{\Phi}
\newcommand{\gl}{\lambda}
\newcommand{\gm}{\mu}
\newcommand{\gn}{\nu}
\newcommand{\gr}{\rho}
\newcommand{\gs}{\sigma}
\newcommand{\gt}{\theta}
\newcommand{\gx}{\xi}

\newcommand{\sw}{\sigma}
\newcommand{\SW}{\Sigma}
\newcommand{\ew}{\lambda}
\newcommand{\EW}{\Lambda}

\newcommand{\Del}{\Delta}
\newcommand{\gD}{\Delta}
\newcommand{\gG}{\Gamma}
\newcommand{\gO}{\Omega}
\newcommand{\gL}{\Lambda}
\newcommand{\gS}{\Sigma}

% Formatting shortcuts
\newcommand{\red}[1]{\textcolor{red}{#1}}
\newcommand{\blue}[1]{\textcolor{blue}{#1}}
\newcommand{\htext}[2]{\texorpdfstring{#1}{#2}}

% Statistics
\newcommand{\distr}{\sim}
\newcommand{\stddev}{\sigma}
\newcommand{\covmatrix}{\Sigma}
\newcommand{\mean}{\mu}
\newcommand{\param}{\gt}
\newcommand{\ftr}{\phi}

% General utility
\newcommand{\todo}[1]{\footnote{TODO: #1}}
\newcommand{\exclaim}[1]{{\textbf{\textit{#1}}}}
\newcommand{\tbc}{[\textbf{Incomplete}]}
\newcommand{\chk}{[\textbf{Check}]}
\newcommand{\oprob}{[\textbf{OP}]:}
\newcommand{\core}[1]{\textbf{Core Idea:}}
\newcommand{\why}{[\textbf{Find proof}]}
\newcommand{\opt}[1]{\textit{#1}}


\DeclareMathOperator*{\argmin}{arg\,min}
\DeclareMathOperator{\rank}{rank}
\newcommand{\redcol}[1]{\textcolor{red}{#1}}
\newcommand{\bluecol}[1]{\textcolor{blue}{#1}}
\newcommand{\greencol}[1]{\textcolor{green}{#1}}


\renewcommand{\~}{\htext{$\sim$}{~}}

%opening
% \title{Social and biological network analysis}
% \author{}
% \date{}

\begin{document}
% \maketitle
% \tableofcontents

\newpage
\section{Introduction and problem description}
There has been an explosion in the number of online social networks and their active members. This wealth of information in the social networks has driven prolific work on the analysis of the networks, understanding the processes that explain the evolution of the networks, modeling the spread of behavior through the networks, predicting their future state and so on.

Users of a social network tend to affiliate with communities. In some social networks, the groups are identified more by the preferences of the members of the social network than by direct declaration: e.g. the genre of movies that a set of customers tend to patronize more often in Netflix. Online social networks like Facebook, Orkut and Live Journal are more interesting examples because the affiliation networks here are explicitly established by the members of the network. Thus, two networks exist simultaneously: the friendship network among users, and the affiliation network between users and groups.

\paragraph*{The problem}
Group formation and evolution in social networks\cite{GroupFormation}, and co-evolution of social and affiliation networks\cite{Coevolution} have been recently studied. One of the interesting challenges in social network analysis is the affiliation recommendation problem, where the task is to recommend communities to users. The fact that the social and affiliation networks ``co-evolve'' suggests that a better solution to the affiliation recommendation problem can be obtained if the friendship network is considered along with the affiliation network. This problem setting has applications beyond community recommendation. Affiliations, for example, can be interpreted in general as a user's taste for an item. Neither is it limited to social networks. For example, in biology, the friendship network can correspond to a network among genes, whereas the affiliation network can correspond to a network between genes and traits/ diseases, and the affiliation recommendation problem can be viewed as one of identifying genes affecting the expression of a disease.

\paragraph*{Contributions}
We consider how one can model the interplay between users and communities in both the networks simultaneously. An ideal unifying model would not only explain the current state of the networks, but also help in predicting future relationships among the nodes. Using a simple way of combining these networks, we suggest and explore two ways of modeling the networks for the purpose of making affiliation recommendations. The graph proximity model is based on estimating the affinity between a user and a community by considering their proximity as nodes in a combined graph, while the latent factors model is based on the proposition that community affiliations arise from interactions of \textit{user} and \textit{group factors}. Each of these network models suggests affiliation recommendation algorithms.

We evaluate these algorithms on social networks from Orkut consisting of 9,123 users and 75,546 communities, and Youtube consisting of 16,575 users and 21,326 communities. We propose a way of evaluating affiliation recommendations, by measuring how good the top 50 recommendations per user are, and demonstrate the importance of designing the right evaluation strategy. Of the algorithms proposed, those suggested by the graph proximity model turn out to be the most effective and efficient. This use of link prediction techniques for the purpose of affiliation recommendation is, to our knowledge, novel. We show that information in the friendship network can be used effectively for affiliation recommendation. We also observe that the benefit we derive from the social network in affiliation recommendation is strongly contingent on how the problem is modeled and what algorithms are used.

\paragraph*{Overview}
We now provide a brief overview of the organization of the paper. In Section \ref{Models}, we consider a network formed by merging the friendship and affiliation networks and introduce two models of the behaviour of nodes in this network: the graph proximity model, and the latent factors model, and explore recommendation algorithms that arise from these models. In Section \ref{Related work}, we consider how the proposed models and algorithms relate to prior work. Then, in Section \ref{Experimental Evaluation}, we describe our chosen evaluation strategy, and using experiments on real world networks, we evaluate the effectiveness of various algorithms in affiliation recommendation in practical scenarios. Finally, in Section \ref{Conclusion and Future Work}, we conclude with a summary of our findings, and a discussion of future lines of research in this direction.




\section{Biological data from different organisms}
\label{sec:dataDescription}
% Data description section for the grant proposal.
Substantial efforts have been spent over the years by many biologists and fellow researchers to collect data from biological systems. This includes information on how human genes interact with each other, relationships between genes in one organism with that of the genes in other organisms, what genes play a role in a certain human disease, how genes influence characteristics in other organisms and so on. The collected data is by no means complete. It could be highly unlikely to observe some of the natural phenomena in biological systems(like genes influencing a disease) even with state-of-the-art technology. Furthermore, time, money and other resource constraints may play great barriers that render such experiments impossible. The collected biological data not only give insights into understanding the fundamental processes in biological systems, but also form the basis for application of computational techniques to make interesting and useful inferences. In this section, we describe in detail the biological networks that form the basis of our proposed research in network analysis. Before we delve into details, let us introduce some basic terminology used in the document.\\
\exclaim{Gene.} A portion of DNA of an organism (species) responsible for hereditary transfer of ``traits''.\\
\exclaim{Phenotype.} A trait of an organism: morphological, physiological or behavioral. Phenotypes result from the expression of specific genes, among other factors. Examples of phenotypes are \emph{lung cancer, color blindness, abnormal sleep pattern}, etc.\\
\exclaim{Ortholog.} A gene in a different species that evolved from an ancestral gene when the two species diverged.  A number of human genes have orthologs in mouse, yeast and other species.

We will represent networks as graphs with the network entities as nodes and the ``communication'' between the entities as edges or links between the nodes. In social networks that we consider, e.g. a social networking site like \emph{Orkut} or movie rental site like Netflix, the entities are the people who are members of the network and the communities or groups they affiliate with. In biological networks, the entities are genes and phenotypes. Different types of networks can co-exist among a set of entities depending on the nature of relationships. To continue with our examples in social networks, we can conceive of a ``friendship'' network among people or an ``affiliation'' network where edges are from people to communities (or affiliations). Interactions between genes in a species and their expression in characteristic phenotypes or traits of the species can be conceptualized as biological networks for purposes of study and analysis. We now discuss in detail about the origin, representation, and characteristics of various biological networks.

\subsection{Biological networks: Description and Sources}
\label{section:dataset}
Biological networks arise naturally due to the interaction of genes in biological processes and influence of genes in morphological or physiological traits of different species. In biological systems, we are interested in the network of interaction of genes, i.e., \emph{gene-gene} networks and the expression of genes in phenotypes, i.e., \emph{gene-phenotype} networks. The edges may be weighted in order to indicate the strength of the interaction. Our biological networks comprise 6 different species namely human, plant(\emph{Arabidopsis}), worm, fly, mouse and yeast as shown in Table \ref{tab:data}. Note that we can conceive of a single biological network with the genes and phenotypes of all the species as entities. Figure \ref{fig:yeast} shows one set of interactions among genes in yeast species\cite{pmid15567862}.
\begin{figure}[ht]
  \begin{center}
  \subfigure[Yeast species]{\includegraphics[height=5cm, width=7cm]{figures/yeast.jpg}}
  \subfigure[Internet]{\includegraphics[height=5cm, width=7cm]{figures/wired.png}}
  \end{center}
\vspace{-5ex}
\caption{Networks from different domains}
\label{fig:yeast}
\end{figure}

\begin{table}[ht]
\centering
\begin{tabular}{| c | c | r | r | r |} \hline
\textsc{Species} & \textsc{Acronym} & \textsc{Genes} & \textsc{Phenotypes} & \textsc{\# Known Relations} \\ \hline
Human & Hs & 1,488 & 292 & 1,921\\ \hline
Plant & At & 4,074 & 890 & 13,968\\ \hline
Worm & Ce & 3,148 & 440 & 19,161\\ \hline
Fly  & Dm & 4,079 & 2,609 & 53,985\\ \hline
Mouse & Mm & 4,795 & 4,056 & 72,601\\ \hline
Yeast & Sc & 3,805 & 1,150 & 66,755\\ \hline
\end{tabular}
\caption{Species which can be used for disease modeling, and sizes of the biological networks for different species considering only orthologs of human genes.}
\label{tab:data}
\end{table}

\paragraph{Bipartite graphs perspective.}
A bipartite graph $G = (V, E)$ is a graph whose vertex set $V$ can be partitioned into two disjoint independent sets $V_{1}$ and $V_{2}$ such that for any edge $(u, v) \in E$, $u \in V_{1}$ and $v \in V_{2}$. The gene-phenotype network for a particular species can be viewed a bipartite graph where $V_{1}$ is the set of orthologs of human genes in that species and $V_{2}$ is the set of phenotypes of that species\cite{pmid19010805, pmid20215462}. Furthermore, the network of the genes and phenotypes of all species can be viewed a bigger bipartite graph. 

The biological data was collected from multiple sources including \cite{pmid18223650, pmid17912365, pmid18613949, pmid20118918}; additional data for flies have since been assembled from publicly available datasets. Detailed description on the extraction of the data sets can be found in \cite{McGarySI}. In particular, when a human gene is known to occur as multiple orthologs in a species, a single row represents the entire set of orthologs in the species-specific gene-phenotype association matrix. Note that the human gene-gene network is obtained from sources independent of gene-phenotype networks\cite{pmid19767740, pmid19728866, pmid19246570}. Networks are represented using adjacency matrices. Recall that the problem of interest is to predict the association of genes in human diseases (phenotypes are diseases in this case). Consequently, the genes in the network are only the human genes. In case of other species, we are interested in the orthologs of human genes in the species. The size of the preprocessed data sets is shown in Table \ref{tab:data}.

\subsection{Biological networks and social networks: an analogy}
In a social network like \emph{Orkut}, there exists a friendship network among its users and affiliation network among users and affiliations, which is a bipartite graph. It is interesting to note that this is analogous to the biological network with genes playing the role of users and phenotypes playing the role of affiliations. Hence, models for evolution and link formation in social networks could be adapted to explain biological networks. The study and analysis of social networks could be applied to biological networks. We will see in Section \ref{sec:preliminaryResearch} that our preliminary experiments indicate that our previous work on social network link prediction does well in predicting the association of genes in phenotypes.

It is interesting to observe some of the statistics of the data set. For example, Figure \ref{fig:genePhenotypeDist} shows the frequency distributions of genes observed in a human and mouse phenotype, and diseases or phenotypes in which a human gene (or its ortholog) is observed. Observe that only a very few diseases have more than 10 genes known to be involved. Importantly, we see that the number of genes associated with phenotypes in mouse is significantly greater than the number of genes associated with diseases in humans. In fact, gene-phenotype connections are relatively better studied in many other species as well. The observations imply that there is much more information in other species that can potentially be used for identifying human gene-disease connections.

\begin{figure}[ht]
  \begin{center}
%     \subfigure[Distribution of genes in human diseases]{\includegraphics[scale=0.3]{figures/NumGenesInDiseases1.jpg}}
%     \subfigure[Distribution of diseases of human genes]{\includegraphics[scale=0.3]{figures/NumDiseasesOfGenes1.jpg}}
%     \subfigure[Distribution of orthologs in mouse phenotypes]{\includegraphics[scale=0.3]{figures/NumGenesInPhenotypes5.jpg}}
%     \subfigure[Distribution of phenotypes of orthologs in mouse]{\includegraphics[scale=0.3]{figures/NumPhenotypesOfGenes5.jpg}}
    \subfigure[]{\includegraphics[height=4cm, width=7cm]{figures/NumGenesInDiseases1.jpg}}
    \subfigure[]{\includegraphics[height=4cm, width=7cm]{figures/NumDiseasesOfGenes1.jpg}}
    \subfigure[]{\includegraphics[height=4cm, width=7cm]{figures/NumGenesInPhenes5.jpg}}
    \subfigure[]{\includegraphics[height=4cm, width=7cm]{figures/NumPhenesOfGenes5.jpg}}
  \end{center}
\vspace{-5ex}
\caption{Statistics of the gene-phenotype network in humans(top) and mouse(bottom). (a) Only a few diseases have more than 10 genes known to be involved. (b) The number of genes associated with human diseases is rather small. (c) Many phenotypes in mouse have larger number of genes known to be associated as compared to human diseases. (d) The number of genes associated with mouse phenotypes is significantly higher than humans. }
\label{fig:genePhenotypeDist}
\end{figure}


\section{Related research}
A promising approach to disease modeling is through data analysis, in particular, analysis of disease-gene, phenotype-gene as well as gene-gene networks. The proposed research will involve the development of novel mathematical techniques that have their roots in diverse areas such as graph theory, network analysis, graphical models and compressive sensing. In this section, we describe prior and related work. 

Traditionally, gene-disease link identification has been done using Genome-Wide Association studies (GWA studies), which became truly feasible only after the completion of the Human Genome Project\cite{HumanGenomeProjectBook} in 2003 and the International HapMap Project\cite{HaplotypeLiu} in 2005. To carry out a genome-wide association study\cite{GWAurl, GWASZhang, GWASLi}, researchers use two sets of people: people with the disease being studied, and similar people without the disease. Researchers obtain DNA from each participant, usually by drawing a blood sample. Each person's complete set of DNA, or genome, is then purified from the blood, and specially designed machines then survey each person's genome for single nucleotide polymorphisms, or SNPs. If certain genetic variations are found to be significantly more frequent in people with the disease compared to people without disease, the variations are said to be "associated" with the disease.

GWA studies tend to be relatively slow and laborious. They search the entire genome for associations rather than focusing on small candidate areas. Besides this, there are other weaknesses of this approach. GWA studies often identify common variants of genes which tag along with rare variants of other genes which actually play a role in the disease as risk factors. The common variants identified by GWAS contribute little value to individual disease risk predictions over existing clinical markers for most common diseases. Also, knowledge of the rare variants of genes which actually play a role in the disease is essential for designing drugs. The gene-disease link identification approach proposed in this proposal attempts to rectify these drawbacks.

An approach to model human diseases by identifying  the ``closest'' orthologous phenotypes\footnote{phenotypes which are formed from orthologous genes in two given species} from various other species has been pioneered by Prof. Ed Marcotte of UT Austin \cite{McGaryOrthologousPhenotypes}. The method developed in \cite{McGaryOrthologousPhenotypes, McGarySI} uses a naive Bayes classifier to determine the probability of a gene influencing a given disease, given the evidence of expression of the gene in orthologous phenotypes of other organisms. The ``closest'' phenotypes are determined using a distance function based on the hypergeometric probability of observing the overlap of genes between two given phenotypes by chance. The preliminary results in \cite{McGaryOrthologousPhenotypes} indicate great promise in predicting genes that are responsible for human diseases. Marcotte's lab has already been able to predict novel genes associated with diseases, for example, a yeast model for angiogenesis effects, a worm model for breast cancer, mouse models of autism, and a plant model for the neural crest defects associated with Waardenberg syndrome, among others\cite{McGaryOrthologousPhenotypes}. However, they have barely scratched the surface of this promising line of research - better computational models are expected to have a significant impact on the accuracy of the predictions. As preliminary experiments described in Section \ref{sec:preliminaryResearch} indicate, application of ideas from social network analysis, recommender systems and matrix completion hold great promise in significantly furthering the state of the art in gene-disease link identification.

Research on large, complex networks and their properties has attracted attention from physicists, computer scientists, biologists, and social scientists \cite{internet, smallworldPNAS, mat1,mat2, nn, collab}. Networks are highly dynamic objects; they grow and change quickly over time through the addition of new edges, signifying the appearance of new interactions in the underlying structure. Understanding the mechanisms by which they evolve and discovering the underlying structure are  fundamental questions that are still not well understood. Particularly relevant to this proposal is the network analysis task of \emph{link prediction}. Indeed, the task of identifying links between genes and diseases may be cast as one of link prediction, when the interactions between genes and phenotypes are modelled as a graph.

The link prediction model tries to predict the presence or absence of a link between a certain pair of nodes, based on observed links in other parts of the networks. Typically this prediction is performed using features intrinsic to the network itself. Commonly used models include the Katz measure~\cite{Katz}, rooted PageRank~\cite{KleinbergLinkPred}, escape probability~\cite{1281272}, see~\cite{KleinbergLinkPred} for a comprehensive empirical comparison. In practice, in addition to the links between nodes, we also have extra information, including the properties of nodes or auxiliary links between nodes from different sources. Preliminary research in using such side information has been conducted in Prof. Dhillon's lab over the past year \cite{vasukiNatarajan, berkantSupervised}. As part of the goal of this proposal, we plan to investigate algorithms that make use of various forms of such side information to boost the prediction accuracy in the gene-phenotype network.

The gene-disease link identification problem can also be viewed as an item recommendation problem, where the task is to identify or recommend genes relevant to a particular phenotype. Recommendation systems\cite{ResnickRecommender,BurkeRecommender,SandvigRecommender,DemirizRecommender,GunawardanaRecommender,TikkRecommender,DBLP:conf/recsys/ViappianiB09,DBLP:conf/recsys/ParkT08,itemRecommendation} have attracted great deal of attention in recent years, due to their important commercial applications. Consider, for example, the famous example of the Netflix problem \cite{yehudaMillionDollar}, where the problem is to identify movies of interest to users of the movie rental service, Netflix, given some of the movies a user has watched in the past. The process of making such item-recommendations in the absence of side information, is called collaborative filtering. Even though there is a large body of literature addressing this problem \cite{yehudaMillionDollar, GoogleCFLatent, BillsusCollaborative,GeorgeCollaborative,LindenCollaborative,SarwarCollaborative,oneClassCollaborative}, the problem of using side information, such as relationships between various users and relationships among items, is just beginning to be tackled \cite{GoogleCCF}.

While combinations of recommendation algorithms have proven to be more powerful than using a single recommendation algorithm, identifying good recommendation algorithms which can be used in such predictor-ensembles is an important task. Particularly successful algorithms are based on posing the recommendation problem as a matrix completion problem or as a low rank matrix approximation problem. Matrix completion has been an active area of research and several impressive theoretical results guaranteeing exact recovery have recently been obtained, that generalize results from the area of compressive sensing\cite{DBLP:journals/tit/CandesT05,candesRPCA09,candesRecht08,Tsaig06compressedsensing, PrateekSVP, Baraniuk07compressivesensing, mekaMatrixCompletion,KeshavanMatrixCompletion,RechtMatrixCompletion}.

\section{Proposed research}
\label{sec:proposedResearch}
% Proposed research Section.
We now briefly discuss the technical details of our plan to develop algorithms to identify gene-disease connections. In contrast to most previous research, we propose to leverage the power of informatics to exploit all the information available in solving this problem. Our approaches, by and large, focus on using multiple sources of information in novel ways that not only better explain the current observations but also help in making better predictions. We begin by describing the problem from the network link prediction perspective. We then provide varied formulations for the problem in Section \ref{section:problemFormulation}. In Section \ref{section:proposedApproaches}, we detail the various approaches that we propose to tackle the problem. In the following subsections, we discuss the evaluation strategies which are critical to measure the success of our proposed approaches. The strategies include both analytical and practical (wetlab experiments) evaluation. We conclude with a proposition of engineering tools for bioinformatics.

\paragraph*{The network link identification perspective.}
Given a phenotype (thyroid disease, for example), the problem is to identify genes affecting the expression of the phenotype. To solve this problem, one can use a variety of information sources, some of which were described in Section \ref{sec:dataDescription}.

One can view the above problem from the perspective of link identification on bipartite graphs. If one were to consider the bipartite graph composed of nodes corresponding to genes and phenotypes, the problem is to predict new links between genes and a given phenotype. Alternate sources of information, such as the gene-phenotype networks corresponding to other species, together with the information of homology relationships between genes of various species, may be similarly modelled as graphs. The question then is one of exploiting information from these graphs to predict new links between human genes and human diseases.

Such network link identification problems have been explored in the field of social network analysis \cite{KleinbergLinkPred}. This field has seen active research in recent years due to the explosion of online social networks, such as Facebook, LinkedIn and Orkut. In the social network link prediction problem, given the current state of a network among users, the task is to predict new links among these users. The community recommendation problem in social networks with explicitly defined communities of users is to accomplish a similar task in the prediction of user-community links \cite{GoogleCCF, GoogleCFLatent}.

The problem of how best to use information from secondary networks in order to make predictions or identify links in the network of interest (the human gene-phenotype network, in our case) is of critical importance in this project. PI Dhillon's research group has conducted concerted research in this direction, in the context of link prediction in social networks\cite{berkantSupervised} and affiliation networks \cite{vasukiNatarajan}.

\subsection{Varied problem formulations}
\label{section:problemFormulation}
We have described and modeled the problem using network analysis methodologies. In this section, we present a few different problem formulations.

\subsubsection{Formulation as a ranking problem}
\label{forumulationRankingProblem}
The problem of identifying genes associated with a certain disease may be viewed as one of ranking genes. For a given phenotype, the task of a gene-phenotype link predictor can be viewed as one of ranking genes in the order of their likelihood of influencing the phenotype.

Ranking a set of items in the order of relevance is a central problem in the field of recommender systems. Consider the famous example of the Netflix competition \cite{yehudaMillionDollar}. In this problem, ratings given to some movies by some users is known. Based on this information, for a particular user, the task is to sort all movies according to how much a user will like these movies. This information can then be used to recommend movies to users, which plays an important role in the movie rental company's revenue. For our purposes, phenotypes or diseases may be considered to play the role of users, whereas genes may be considered to play the role of movies.

This perspective is very convenient because most approaches to identifying \textit{relevant} items function using scores, a concept we describe next. A relevance function $relevant:\set{items} \to \set{0, 1}$ is a binary function which is 1 for any \textit{relevant} item, and 0 otherwise. In practice, for many problems including the problem of identifying genes relevant to a certain phenotype, this relevance function is unknown. Many algorithms designed for identifying relevant items function by assigning \textit{relevance scores} to every candidate item. Then, for a threshold score, one can identify all items with scores higher than the threshold to be 'relevant'.

\subsubsection{Probabilistic formulations}
\label{section:outlier}
One can also attempt to solve the problem of identifying genes relevant to a particular phenotype by modeling $Pr(relevant(g|p))$, the probability of gene-phenotype associations. These probabilities may then be used to play the role of the \textit{relevance scores} described earlier. This approach has been taken by Marcotte et al \cite{McGaryOrthologousPhenotypes} with some reasonable early success.

A related formulation is one of one-class classification \cite{Scholkopf99, ProteinOneClass, OutlierDetection, VillalbaOneClass}. The problem here, given a set of examples belonging to a certain class, is to identify whether a previously unseen item belongs to that class. This problem may also be viewed as one of determining the support of a distribution (which is $Pr(relevant(g|p))$ in our case). The problem can also be formulated as one of learning the graphical model G describing a probability distribution over the vector space spanned by all human genes. Various human diseases are then viewed as partially observed samples from this distribution. The task then would be to learn the graphical model G and model the gene-phenotype prediction problem as one of inferring conditional probabilities in G.

These different ways of modeling the problem suggest various approaches to solving the gene-disease link identification problem, which are described next.

\subsection{Proposed Approaches}
\label{section:proposedApproaches}
We now describe different methods for tackling the varied problem formulations discussed in Section \ref{section:problemFormulation}. In the remainder of the section, we will use $genes(Hs)$ to represent the set of human genes, and $phenotypes(Hs)$ to represent the set of human diseases under consideration.

\subsubsection{Matrix completion approaches}
\label{section:matrixComp}
Consider the ranking formulation of the gene-phenotype link identification problem described in Section \ref{forumulationRankingProblem}. Recall that the task, for a given phenotype $p$, may be viewed as one of ranking genes in the order of their likelihood of influencing $p$. As mentioned earlier, many algorithms designed for identifying relevant items function by assigning \textit{relevance scores} to every candidate item. In other words, every pair $(g, p) \in genes(Hs) \times phenotypes(Hs)$ is assigned a score $score(g,p) \in \Re$.

For gene-disease connections that are already known, denoted by the set $\Omega$, suppose that $score(g, p) = 1$. The scoring problem is then to assign scores to $\set{(g, p) \in genes(Hs) \times phenotypes(Hs) - \Omega}$. The closer score(g, p) is to exceeding or equalling 1, the more confidence one has about there being a connection between g and $p$.

The problem can be viewed as a \textbf{matrix completion problem} in the following manner. Form a \\
$\size{genes(Hs)} \times \size{phenotypes(Hs)}$ matrix S, in which only entries $\set{S_{g, p} = 1 : \forall (g, p) \in \Omega}$ are known. Note that one can think of S as a part of the adjacency matrix corresponding to the network formed by gene-disease interactions. The task is to estimate the remaining entries of S, so that the estimated values are meaningful as scores which can be used for gene-phenotype link prediction.

If one were to further assume that all genes and phenotypes have a low rank representation, one can then model the problem as one of \textbf{low rank matrix approximation}. Here, one imposes the model $\bS \approx \bG*\bP^{T}$, where G and P have dimensions $\size{genes(Hs)} \times k, \size{phenotypes(Hs)} \times k$ for some small rank $k$. So, one tries to find a low rank approximation for $\bS$.

Here, the modeling assumption is that, for a gene-phenotype pair $(g, p)$, $\dprod{v_g, v_p} \approx S_{g, p}$, where $v_g$ and $v_p$ are low dimensional representations of the gene g and the phenotype $p$ respectively. This assumption is appealing to the intuition. For example, in the Netflix movie recommendation example, the low dimensional representation of movies and users may be interpreted as having meaning related to various features of a movie or a person's taste, like the theme, the music and so on. This approach has been enormously influential in recommender systems research, and the PI has developed fast algorithms for such problems \cite{PrateekSVP}.

Alternate sources of information such as the gene-phenotype networks from other species, and the gene-gene orthologous relationships can also be viewed as incomplete matrices. One may then extend matrix completion techniques to discern low dimensional representations of genes and phenotypes which are good at accounting for these secondary observations, besides accounting for observed entries of S. The idea here is that such low dimensional representations of genes and phenotypes better reflect reality, and have greater predictive ability. Thus the matrix completion problem itself is a novel extension of the classical problem: infer the missing values given secondary data sources. Preliminary results are reported in Section \ref{sec:preliminaryResearch}.

We now describe other mathematical challenges that arise in the matrix completion formulation. Let $S$ be the adjacency matrix corresponding to a network, and suppose that only some entries $\Omega$ of $S$ are observed. Also, suppose that we want to derive a low rank approximation $X$ of $S$. Firstly, $S$ is binary matrix. Secondly, most of the unknown entries in $S$ are expected to be zero: so, a good predictor must have a heavy bias towards predicting that the unknown entries are zero. Thirdly, all the known entries in $S$ are 1. Matrix completion for binary matrices, especially for the purpose of binary classification presents challenges that have not been adequately addressed in the literature. For example, while using mathematical programming to find X to approximate S, one classical way to penalize deviation from the known entries in S is to use the quadratic penalty, $\sum_{(i, j) \in \Omega} (S_{i, j} - X_{i,j})^{2}$. Considering the fact that we are dealing with a binary matrix, using a hinge loss or exponential loss may be more appropriate. Furthermore, the requirement that most inferred entries should be zero implies a sparsity structure on $X$, which may be enforced using an L1-norm penalty function. Combining the L1-norm penalty function with the trace norm regularization would be a challenging optimization problem (which is different from the low-rank + outlier problems considered in \cite{SanghaviChandrasekaran,candesRPCA09}). The gene-gene network would additionally enforce smoothness in the low-rank, sparse approximation to be computed. The proposed research, besides tackling the biological problem of gene-disease link prediction will also involve the development of general techniques to solve such novel mathematical problems.

\subsubsection{Graph proximity approaches}
\label{section:graphProximityIntro}
An alternative way of modeling connections between genes and phenotypes can be obtained from looking at the problem from a more graph-theoretic perspective. Consider the nodes in a graph --- various ways of measuring similarity between nodes have been proposed and tested in the context of social networks analysis\cite{KleinbergLinkPred}. The basic idea is that, the more \textit{similar} two nodes are, the greater the probability of a link arising between those nodes.

Consider the adjacency matrix A of a graph. One well known similarity measure between nodes in a graph is the \textit{common neighbors} measure, where the number of neighbors shared by a pair of nodes is used as a similarity measure. This similarity measure between nodes may be concisely expressed using the matrix $S = A^{2}$, where $S_{i, j}$ denotes the similarity between nodes $i$ and $j$. $S_{i,j}$ can be interpreted as counting the number of paths of length two between nodes $i$ and $j$. Similarly, similarity measures can be defined to take into account paths of arbitrary length. For example, the Katz measure, defined by $S(\gb) = \sum_{i=1}^{\infty} \gb^{i}A^{i}$. The utility of such similarity measures in link prediction on social networks has been an active area of research. The PI has recently worked on producing better similarity measures which assign weights that can be learned using supervised approaches\cite{berkantSupervised}, which are described later in this section.

Such notions of node similarity can be extended to the case of bipartite graphs of genes and phenotypes. A challenge of using this approach is to find a way of exploiting secondary sources of information, such as the gene-phenotype interaction networks in other species, and the homology information between genes of various species. Preliminary work in this direction is reported in Section \ref{sec:preliminaryResearch}.

\paragraph*{Supervised graph proximity methods.}
Another way of estimating new proximities between genes and phenes is to extend the above graph proximity methods into a supervised framework. For instance, if we adopt a linear regression model, a linear function by combining various path counting features (extracted from gene-phene bipartite graphs based on other species) can be used to approximate the target gene-phene bipartite graph. The model parameters, i.e., weights for different path counting features, can be obtained adaptively through linear regression by fitting the data. More sophisticated models such as generalized linear models can be applied by extending the linear function to the nonlinear case. For example, we can transform the output of the linear model through a logistic function. This will lead to the logistic regression model, which can capture more complex intrinsic properties of the data than the linear regression model. 

Multiple gene-phene interactions in species other than humans can be naturally treated as multiple sources of information characterizing various connections between genes and phenes. It would be interesting to find out whether gene-phene interactions in multiple species can collectively predict potential gene-phene connections in human diseases. With multiple proximity graphs, we have a much richer choice of graph topological features (hybrid path counting features) by allowing cross routes between graphs. Let us take an example for better understanding. Note that multiple gene-gene connections can be formed based on gene-phene networks from multiple species. In the context of \emph{multigraphs}\cite{harary94}, between any two genes there can be an edge from source $A$ and an edge from source $B$. Both $A$ and $B$ are in the form of bipartite graphs. By traversing edges in both $A$ and $B$, we can construct hybrid path counting features across multiple sources in addition to pure path counting features within just one source. To control the model complexity due to the increase of the number of path counting features, we can enforce some sparsity constraints on the feature weights through L1 regularization\cite{lasso}. This will lead to the Lasso or the grouped Lasso problem (following the hierarchical structure formed by multiple sources)\cite{jenatton09,bach08,zhao09}. In the spirit of selectively combining hybrid path counting features from multiple sources, the PI's group has sucessfully applied the idea to the link prediction problem in social network analysis, which yields promising results on real world application data\cite{berkantSupervised}.  


\subsubsection{Other approaches}
We also propose to explore various other novel approaches to solving the problem. In particular, we plan to explore the use of probabilistic modeling of the relationships between genes and phenotypes. Consider the matrix completion approaches described in Section \ref{section:matrixComp}, for example, where the objective was to compute a low rank approximation $\bA$ to an incomplete matrix $\bS$. One can design a statistical model parameterized by a low rank matrix $\bA$ for the process generating $\bS$ and then learn the model parameters which are most likely to have generated the observations in $\bS$.

In Section \ref{section:outlier}, we formulated the gene-phenotype link identification problem as one of one-class classification. Here, previously known gene-phenotype connections serve as examples coming from the distribution D of ``positive examples''. But, there are no ``negative examples'', or cases where connections between genes and phenotypes are certainly known not to exist. The application of one-class classification to the problem of link prediction would also be explored as part of the proposed research.

In Section \ref{section:outlier}, we also formulated the problem as one of learning a graphical model describing a probability distribution $P$ over a binary vector space spanned by the set of human genes. Prior gene-disease links, for a given disease, are viewed as samples drawn from this distribution. This task presents a new challenge in research about probabilistic graphical models --- generate a model so that most conditional probabilities are close to zero, reflecting the belief that most genes do not contribute to a human disease. The proposed research will involve tackling this theoretical problem and applying it to the gene-disease link prediction.

In the context of link prediction and recommender systems, research has shown that a combination of many predictors is capable of better performance than any single predictor. For example, the winners in the recent Netflix competition used a non-trivial combination of many predictors. The task of identifying connections between genes and diseases is one of great importance. In proposed research, we plan to employ such a combination of predictors in identifying gene-disease relationships.

\subsection{Identifying new information sources}
A predictor is only as good as the information it gets, and the more information that is available, the greater our freedom to develop powerful predictors which exploit this information. As part of the proposed research, we plan to explore the use of new sources of information in addition to the data described in Section \ref{section:dataset}. There are other interesting sources of information like the gene-gene ortholog information. An orthologous relationship is a mapping from the subset of human genes to a subset of genes of other species. The mapping is many-to-many, i.e., one or more human genes can be mapped to the same ortholog or a human gene can be mapped to more than one ortholog. This relationship is represented as a binary matrix. This information can potentially be used for link prediction in biological network, and has not been exploited by any previous work on this problem. Also, there is the interesting aspect of determining association between phenotypes. These associations can be derived from from the medical literature or from medical records of patients, e.g., if some diseases are known to co-occur, it is suggestive of a relationship between them.


\subsection{Evaluation Strategies}
\label{section:evaluation}
Evaluation is critical and complementary to the development of novel methods. Quantifying the success of the proposed techniques presents a challenge by itself. Theoretical evaluation of the methods is one way of measuring the goodness of the results. However, the nature of the problem at hand calls for direct confirmation of the results from biologists. Our goal is to use analytical evaluation strategies to measure the performance of the individual methods proposed, while ultimately enabling wetlab experiments to confirm the role of predicted genes in human diseases of concern. Considering the costly and time consuming nature of wet-lab experiments, development of good evaluation techniques is essential for identification of good predictors. This is also important in combining predictions from multiple predictors: one needs to assign greater weight to predictions from good predictors, and lesser weight to predictions from mediocre predictors.

\subsubsection{Analytical evaluation}
Consider a predictor whose task is to identify the set of genes $T$ that are known to be associated with certain diseases. The predictor comes up with a ranking of the genes with higher scores assigned to genes, it believes to have higher likelihood of influencing the diseases. We need to measure how good the prediction is. Our evaluation methodology uses area under ROC curve as a measure of performance of the predictor. To understand the evaluation methodology, we need to define two terms: sensitivity and specificity. Let $P_n$ denote a set of $n$ genes predicted by a predictor for a given phenotype.

\paragraph*{Sensitivity and Specificity.}
Sensitivity measures the ability of the predictor to identify genes in $T$:

$$\emph{Sensitivity} = \frac{|P_n \cap T|}{|T|}. $$

Let $U$ refer to the set of human genes not in T, i.e., $U = genes(Hs) - T$. Specificity measures the ability of the predictor to exclude genes not in $T$:

$$\emph{Specificity} = \frac{|(U - P_n) \cap (U - T)|}{|U-T|}.$$

\paragraph*{ROC, AUC and average AUC.}
The Receiver Operating Characteristics (ROC) curve is the plot of sensitivity vs (1-specificity) for varying $n$. The area under this ROC curve denoted by AUC is a measure of overall performance of the predictor. The performance of a predictor is evaluated using AUC for different human diseases and the average performance of the predictor is determined by the average of all the AUC values. This average AUC $\in [0,1]$ can be used as a performance measure used to compare the performance of different predictors.

For the purpose of wetlab experimentation, however, only the top, say fifty, recommendations produced by a predictor matter. So, ideally we would want to evaluate predictors based on how good their top fifty recommendations are, rather than evaluating them over the entire range of $n$. We can do this by examining a slice of the ROC curve formed by measuring the sensitivity and specificity the predictor achieves for an average disease at regular intervals between $n = 1$ and $n = 50$, an evaluation strategy we propose in \cite{vasukiNatarajan}.

\subsubsection{Enabling wetlab experiments}
After all analytical validation, the ultimate test of the predictions produced by the techniques, of course, will be done by biologists. Considering the recent success of analytical modeling in identifying gene-disease links \cite{McGaryOrthologousPhenotypes}, we expect to discover many new connections between genes and diseases.

Bioinformatics has proven to be of enormous importance in helping biologists cope with the deluge of data they now face during their explorations. Given predictions made with different levels of confidence, a natural question to ask is one of how many resources one can allocate in investigating a predicted gene-disease link in wet-lab experiments. Wet-lab experiments generally tend to be costly and time consuming, making this a question of great importance. This question, which comes under the ambit of Decision Theory, is one we plan to explore in the final phase of this project.

\subsection{Development of bioinformatics tools}
A long term objective of this project is to impact the art of discovering gene-phenotype associations in cases beyond the dataset described in Section \ref{section:dataset}. Our knowledge of gene-phenotype interactions is constantly expanding; this yields new information which can be exploited for making predictions about gene-disease associations many years down the line. For this kind of broad applicability and impact of the methods we develop as part of this project, the development of general software tools is essential.

Accordingly, we plan to develop and release software packages for use by biologists, with algorithms which will identify gene-phenotype associations using the input they are fed. Besides gene-phenotype link prediction software, we also plan to release ancillary tools we develop as part of this research, for example, software to visualize the various gene-phenotype networks.



\vspace{-3ex}
\section{Preliminary research and results}
\label{sec:preliminaryResearch}
In this section, we examine our first attempts at the problem of identifying the influence of genes in human diseases. In particular, we try to predict novel links or associations in the human gene-phenotype association matrix. We extend some of the methods that were used for link prediction in social networks in recent research work conducted in the PI's lab. We will see that the results of even our preliminary experiments are promising, yielding results comparable to the significant results by McGary et al \cite{McGaryOrthologousPhenotypes}. We will also see that there is more scope for effectively exploiting the multiple sources of information. 

We use $P_i$ to refer to the gene-phenotype matrix of species $i$, $i = 1, 2, \dots 6$ and $G$ to denote the Human gene-gene matrix. In particular, $P_1$ refers to the human gene-phenotype matrix in which we make predictions.

\subsection{PI's Recent research}
PI Dhillon's research is in the areas of large-scale data analysis, data mining and machine learning, where mathematical and computational tools are used for various predictive tasks. Over the past year, Dhillon has started a concerted research program in large-scale network analysis as evidenced by several recent publications\cite{savasQiuClusteredEmbedding, vasukiNatarajan, berkantSupervised, PrateekSVP, mekaMatrixCompletion,zhengdongAgarwal}. The tasks undertaken in this research can be broadly categorized as scalable inference/prediction in massive networks, for example, predicting friendship relationships in online social networks\cite{savasQiuClusteredEmbedding, vasukiNatarajan}, matrix completion\cite{PrateekSVP, mekaMatrixCompletion}, and mining multiple networks\cite{berkantSupervised}. This recent research involves mathematical analysis, for example, generalizing the compressed sensing work of Candes and Tao, as well as scalable implementations. The test beds for the developed methods so far have been online social networks, collaboration networks and commercial networks (for example, call phone graphs).

In particular, two models have been proposed by the group, for the problem of link prediction in affiliation networks, namely the Graph Proximity model and the Latent Factor model\cite{vasukiNatarajan}. Here, the problem of recommending affiliations to users of a social network is posed as one of link prediction in an affiliation network, which can be solved using methods based on graph proximity or by using matrix completion methods. The problem of predicting affiliations of genes with phenotypes can be posed similarly as one of link prediction in the gene-phenotype network.

\subsection{Methods based on graph proximity}
We introduced graph proximity approaches in Section \ref{section:graphProximityIntro}. In such methods, the identification of new links is based on the proximity of nodes in the graph. Recall that the relationships between genes and phenotypes may be modelled as edges in a graph. Proximity between genes can then be calculated as the weighted sum of the number of paths connecting two genes with varying lengths. The \emph{Katz} measure is one of the commonly used measures that is based on graph proximity. In our experiments, we use a truncated \emph{Katz}-measure defined as $tKatz(A, \beta, k) = \sum_{i=1}^{k}\beta^i A^i$. We will see some interesting choices for $A$, and how the performance of link prediction is improved with certain choices, in our preliminary experiments. 

We now discuss a few predictors used in our preliminary experiments, based on this similarity measure. These early experiments are based on computing a measure of similarity between all human genes, expressed by a $|genes(Hs) \times genes(Hs)|$ matrix S. The association of genes with phenotypes is then determined by computing the $|genes(Hs) \times phenotypes(Hs)|$ score matrix $S P_1$. For a given phenotype, these scores are used to identify genes which may play a role in a disease. These preliminary methods differ only in the way they compute S. We describe these different ways of computing S below.

\begin{itemize}
 \item One basic way of computing S is to use the human gene-gene interaction matrix, computing $S = tKatz(G, \gb, k)$.
 \item Another way of obtaining the relationships between two human genes is to find the number of common diseases which they both influence. The resulting similarity graph among genes is given by the matrix $P_1P_1^T$. Then, one can use $S = tKatz(P_1P_1^T, \beta, k)$.
 \item Another related way of obtaining the relationships between the human genes is to consider the network formed by the genes and phenotypes of all species other than humans. Here, \\$S = tKatz(\sum_{i>1} P_i P_i^T, \beta, k)$ is a measure of similarity between the genes.
\end{itemize}

\subsection{Methods based on matrix completion}
We now discuss some early experiments exploring the use of matrix completion approaches in identifying gene-disease connections. Recall that such approaches were introduced in Section \ref{section:matrixComp}. Suppose that we had a gene-gene affinity matrix S, which could also be viewed as an affinity network among genes. We then consider the combined graph, whose nodes are human genes and human diseases. The adjacency matrix of this graph is given by: 
$$C(S) = \begin{bmatrix}\lambda S & P_1 \\P_1^T & 0 \end{bmatrix}. \label{eqn:C}$$
Here, $\lambda$ controls the ratio of the weight assigned to gene-gene interaction network to that of the gene-phenotype network.

We look upon the zeros, particularly those in the $P_1$ portion of C as missing entries. The gene-disease link identification problem is then one of matrix completion. We tackle this problem by approximating $P_1$ with a low rank matrix: $$P_1 \approx Q_g^T R_p,$$ where $Q_g$ is the matrix of gene factors and $R_p$ is the matrix of phenotype factors. We want to find the gene and phenotype factors such that the reconstruction error on the observed entries in $C(S)$ is minimized. This is similar to the formulation of a similar model proposed in \cite{vasukiNatarajan}. We compute the $d$-rank approximation of $C$ using Singular Value Decomposition (SVD), and use the resulting $|genes(Hs) \times phenotypes(Hs)|$ matrix $U_d \Sigma_d V_d^T$ as the score matrix. These scores are used to identify genes which may play a role in a given disease. The parameters $\lambda, d$ are learned through cross-validation and the performance achieved with the best parameters is reported in Table \ref{tab:results}.

\subsection{Discussion of preliminary results}
We now present the preliminary experimental results. The prediction is made on the human gene-disease association matrix. Given a disease, a predictor produces a score for each gene, with higher scores assigned to genes with higher likelihood of influencing the disease as determined by the predictor. Recall the analytical evaluation methodology introduced in section \ref{section:evaluation}. We use average AUC as the measure of performance of predictors in our experiments. In all the experiments, we considered only those human diseases in which at least 2 genes were observed.

Table \ref{tab:results} provides a summary of the preliminary experimental results. Two AUC plots, showing the performance of two graphical proximity based methods are presented in Figure \ref{fig:katzAUC}.
\begin{table} [ht]
\centering
\begin{tabular}{| p{7 cm} | c | c |} \hline
Predictor & Avg AUC & Success fraction\\ \hline
$tKatz(P_1P_1^T, 10^{-6}, 1)$ & 0.604 & 101 \\ \hline
$tKatz(\sum_{i>1} P_iP_i^T, 10^{-6}, 1)$& 0.773 & 228 \\ \hline
$tKatz(G, 10^{-6}, 3)$& 0.776 & 236 \\ \hline
$tKatz(G, 10^{-6}, 4)$& 0.781 & 233 \\ \hline
SVD($C(G)$) & 0.596 & 175 \\ \hline
SVD($C(\sum_{i>1} P_iP_i^T,P_1$)) & 0.748 & 233 \\ \hline
\end{tabular}
\caption{Comparison of the performance of various predictors on the biological dataset. The last column 'Success fraction' denotes the number of diseases for which AUC was observed to be better than random, which has AUC equal to 0.5.}
\label{tab:results}
\end{table}

\begin{figure}[ht]
\begin{center}
\subfigure[$tKatz(G, \beta, 4)$]{\includegraphics[height=4cm, width=7cm]{figures/katzRandom.jpg}}
\subfigure[$tKatz(\sum_{i>1} P_iP_i^T, \beta, 1)$]{\includegraphics[height=4cm, width=7cm]{figures/katzOtherSpecies.jpg}}
\end{center}
\vspace{-5ex}
\caption{AUC curve for some of the predictors. Observe that the predictors perform better than random (0.5 AUC) for more than 85\% of the diseases.}
\label{fig:katzAUC}
\end{figure}

We observe that methods based on both the graph proximity approach and matrix completion approach yield promising results. All the methods used are simple in that they did not use information from all the networks in identifying gene-disease links. Even these simple methods have yielded results (see Figure \ref{fig:katzAUC}), comparable to those achieved in \cite{McGaryOrthologousPhenotypes} using a neighborhood-based probabilistic model. The success of early methods that rely only on gene-phenotype networks from non-human species and the human gene-interaction network indicate that a wealth of information is available besides prior observations of gene-disease interactions, and that this information can be fruitfully exploited in identifying the genes involved in human genetic diseases.

\vspace{-2ex}
\section{Project timeline}
For proper coordination, one needs to establish a priority and sequencing among the proposed research topics. We outline key priorities on a yearly basis.

\vspace*{-3ex}
\paragraph*{Year 1}
\begin{itemize}
 \item Theoretical work on the development of algorithms for network analysis and matrix completion.
 \item Develop visualization toolbox to explore gene-phenotype data sets and gene-gene networks.
 \item Implementations in Matlab of predictors based on graph proximity and latent factor models.
 \item Generating gene-phenotype connection predictions using the implemented models.
\end{itemize}

\vspace*{-3ex}
\paragraph*{Year 2}
\begin{itemize}
 \item Exploration of the use of probabilistic modelling and one-class classification to solve the problem.
 \item Explore and develop graphical model-based approaches.
 \item Further development of network analysis algorithms and recommender systems.
 \item Application to other problems in social network analysis.
\end{itemize}

\vspace*{-3ex}
\paragraph*{Year 3}
\begin{itemize}
 \item Wet-lab experiments to check gene-phenotype predictions.
 \item Explore various ways of combining multiple predictors to produce better predictions.
 \item Release software tools for use by biologists.
 \item Explore the use of so-far unused sources of information for gene-phenotype link prediction.
\end{itemize}

\section{Research Team}
The research will be conducted at UT Austin in the Dhillon lab as well as the Marcotte lab. PI Dhillon is an expert in data mining and machine learning, while Co-PI Marcotte is an expert in bioinformatics and is pioneering the proposed new approach in disease modeling\cite{newsMarcotteNY}. PI Dhillon and Co-PI Marcotte have collaborated previously on research papers \cite{isd:emm:ur:2002a} and an NSF ITR project.

\pagebreak
\bibliographystyle{plain}
\bibliography{zhengdong,references}

\end{document}
