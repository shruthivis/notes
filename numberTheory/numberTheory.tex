\documentclass[10pt]{amsart}

\usepackage{amsmath, amssymb}
\usepackage{hyperref, graphicx, verbatim, listings, multirow, subfigure}
\usepackage{algorithm, algorithmic}
% \usepackage[bottom]{footmisc}
\lstset{breaklines=true}
\setcounter{tocdepth}{3}

% Lets verbatim and verb environments automatically break lines.
\makeatletter
\def\@xobeysp{ }
\makeatother
% \lstset{breaklines=true,basicstyle=\ttfamily}

% Configuration for the memoir class.
\renewcommand{\cleardoublepage}{}
% \renewcommand*{\partpageend}{}
\renewcommand{\afterpartskip}{}
\maxsecnumdepth{subsubsection} % number subsections
\maxtocdepth{subsubsection}

\addtolength{\parindent}{-5mm}
% Packages not included:
% For multiline comments, use caption package. But this conflicts with hyperref while making html files.
% subfigure conflicts with use with memoir style-sheet.

% Use something like:
% % Use something like:
% % Use something like:
% \input{../../macros}

% groupings of objects.
\newcommand{\set}[1]{\left\{ #1 \right\}}
\newcommand{\seq}[1]{\left(#1\right)}
\newcommand{\ang}[1]{\langle#1\rangle}
\newcommand{\tuple}[1]{\left(#1\right)}

% numerical shortcuts.
\newcommand{\abs}[1]{\left| #1\right|}
\newcommand{\floor}[1]{\left\lfloor #1 \right\rfloor}
\newcommand{\ceil}[1]{\left\lceil #1 \right\rceil}

% linear algebra shortcuts.
\newcommand{\change}{\Delta}
\newcommand{\norm}[1]{\left\| #1\right\|}
\newcommand{\dprod}[1]{\langle#1\rangle}
\newcommand{\linspan}[1]{\langle#1\rangle}
\newcommand{\conj}[1]{\overline{#1}}
\newcommand{\gradient}{\nabla}
\newcommand{\der}{\frac{d}{dx}}
\newcommand{\lap}{\Delta}
\newcommand{\kron}{\otimes}
\newcommand{\nperp}{\nvdash}

\newcommand{\mat}[1]{\left( \begin{smallmatrix}#1 \end{smallmatrix} \right)}

% derivatives and limits
\newcommand{\partder}[2]{\frac{\partial #1}{\partial #2}}
\newcommand{\partdern}[3]{\frac{\partial^{#3} #1}{\partial #2^{#3}}}

% Arrows
\newcommand{\diverge}{\nearrow}
\newcommand{\notto}{\nrightarrow}
\newcommand{\up}{\uparrow}
\newcommand{\down}{\downarrow}
% gets and gives are defined!

% ordering operators
\newcommand{\oleq}{\preceq}
\newcommand{\ogeq}{\succeq}

% programming and logic operators
\newcommand{\dfn}{:=}
\newcommand{\assign}{:=}
\newcommand{\co}{\ co\ }
\newcommand{\en}{\ en\ }


% logic operators
\newcommand{\xor}{\oplus}
\newcommand{\Land}{\bigwedge}
\newcommand{\Lor}{\bigvee}
\newcommand{\finish}{$\Box$}
\newcommand{\contra}{\Rightarrow \Leftarrow}
\newcommand{\iseq}{\stackrel{_?}{=}}


% Set theory
\newcommand{\symdiff}{\Delta}
\newcommand{\union}{\cup}
\newcommand{\inters}{\cap}
\newcommand{\Union}{\bigcup}
\newcommand{\Inters}{\bigcap}
\newcommand{\nullSet}{\phi}

% graph theory
\newcommand{\nbd}{\Gamma}

% Script alphabets
% For reals, use \Re

% greek letters
\newcommand{\eps}{\epsilon}
\newcommand{\del}{\delta}
\newcommand{\ga}{\alpha}
\newcommand{\gb}{\beta}
\newcommand{\gd}{\del}
\newcommand{\gf}{\phi}
\newcommand{\gF}{\Phi}
\newcommand{\gl}{\lambda}
\newcommand{\gm}{\mu}
\newcommand{\gn}{\nu}
\newcommand{\gr}{\rho}
\newcommand{\gs}{\sigma}
\newcommand{\gt}{\theta}
\newcommand{\gx}{\xi}

\newcommand{\sw}{\sigma}
\newcommand{\SW}{\Sigma}
\newcommand{\ew}{\lambda}
\newcommand{\EW}{\Lambda}

\newcommand{\Del}{\Delta}
\newcommand{\gD}{\Delta}
\newcommand{\gG}{\Gamma}
\newcommand{\gO}{\Omega}
\newcommand{\gL}{\Lambda}
\newcommand{\gS}{\Sigma}

% Formatting shortcuts
\newcommand{\red}[1]{\textcolor{red}{#1}}
\newcommand{\blue}[1]{\textcolor{blue}{#1}}
\newcommand{\htext}[2]{\texorpdfstring{#1}{#2}}

% Statistics
\newcommand{\distr}{\sim}
\newcommand{\stddev}{\sigma}
\newcommand{\covmatrix}{\Sigma}
\newcommand{\mean}{\mu}
\newcommand{\param}{\gt}
\newcommand{\ftr}{\phi}

% General utility
\newcommand{\todo}[1]{\footnote{TODO: #1}}
\newcommand{\exclaim}[1]{{\textbf{\textit{#1}}}}
\newcommand{\tbc}{[\textbf{Incomplete}]}
\newcommand{\chk}{[\textbf{Check}]}
\newcommand{\oprob}{[\textbf{OP}]:}
\newcommand{\core}[1]{\textbf{Core Idea:}}
\newcommand{\why}{[\textbf{Find proof}]}
\newcommand{\opt}[1]{\textit{#1}}


\DeclareMathOperator*{\argmin}{arg\,min}
\DeclareMathOperator{\rank}{rank}
\newcommand{\redcol}[1]{\textcolor{red}{#1}}
\newcommand{\bluecol}[1]{\textcolor{blue}{#1}}
\newcommand{\greencol}[1]{\textcolor{green}{#1}}


\renewcommand{\~}{\htext{$\sim$}{~}}


% groupings of objects.
\newcommand{\set}[1]{\left\{ #1 \right\}}
\newcommand{\seq}[1]{\left(#1\right)}
\newcommand{\ang}[1]{\langle#1\rangle}
\newcommand{\tuple}[1]{\left(#1\right)}

% numerical shortcuts.
\newcommand{\abs}[1]{\left| #1\right|}
\newcommand{\floor}[1]{\left\lfloor #1 \right\rfloor}
\newcommand{\ceil}[1]{\left\lceil #1 \right\rceil}

% linear algebra shortcuts.
\newcommand{\change}{\Delta}
\newcommand{\norm}[1]{\left\| #1\right\|}
\newcommand{\dprod}[1]{\langle#1\rangle}
\newcommand{\linspan}[1]{\langle#1\rangle}
\newcommand{\conj}[1]{\overline{#1}}
\newcommand{\gradient}{\nabla}
\newcommand{\der}{\frac{d}{dx}}
\newcommand{\lap}{\Delta}
\newcommand{\kron}{\otimes}
\newcommand{\nperp}{\nvdash}

\newcommand{\mat}[1]{\left( \begin{smallmatrix}#1 \end{smallmatrix} \right)}

% derivatives and limits
\newcommand{\partder}[2]{\frac{\partial #1}{\partial #2}}
\newcommand{\partdern}[3]{\frac{\partial^{#3} #1}{\partial #2^{#3}}}

% Arrows
\newcommand{\diverge}{\nearrow}
\newcommand{\notto}{\nrightarrow}
\newcommand{\up}{\uparrow}
\newcommand{\down}{\downarrow}
% gets and gives are defined!

% ordering operators
\newcommand{\oleq}{\preceq}
\newcommand{\ogeq}{\succeq}

% programming and logic operators
\newcommand{\dfn}{:=}
\newcommand{\assign}{:=}
\newcommand{\co}{\ co\ }
\newcommand{\en}{\ en\ }


% logic operators
\newcommand{\xor}{\oplus}
\newcommand{\Land}{\bigwedge}
\newcommand{\Lor}{\bigvee}
\newcommand{\finish}{$\Box$}
\newcommand{\contra}{\Rightarrow \Leftarrow}
\newcommand{\iseq}{\stackrel{_?}{=}}


% Set theory
\newcommand{\symdiff}{\Delta}
\newcommand{\union}{\cup}
\newcommand{\inters}{\cap}
\newcommand{\Union}{\bigcup}
\newcommand{\Inters}{\bigcap}
\newcommand{\nullSet}{\phi}

% graph theory
\newcommand{\nbd}{\Gamma}

% Script alphabets
% For reals, use \Re

% greek letters
\newcommand{\eps}{\epsilon}
\newcommand{\del}{\delta}
\newcommand{\ga}{\alpha}
\newcommand{\gb}{\beta}
\newcommand{\gd}{\del}
\newcommand{\gf}{\phi}
\newcommand{\gF}{\Phi}
\newcommand{\gl}{\lambda}
\newcommand{\gm}{\mu}
\newcommand{\gn}{\nu}
\newcommand{\gr}{\rho}
\newcommand{\gs}{\sigma}
\newcommand{\gt}{\theta}
\newcommand{\gx}{\xi}

\newcommand{\sw}{\sigma}
\newcommand{\SW}{\Sigma}
\newcommand{\ew}{\lambda}
\newcommand{\EW}{\Lambda}

\newcommand{\Del}{\Delta}
\newcommand{\gD}{\Delta}
\newcommand{\gG}{\Gamma}
\newcommand{\gO}{\Omega}
\newcommand{\gL}{\Lambda}
\newcommand{\gS}{\Sigma}

% Formatting shortcuts
\newcommand{\red}[1]{\textcolor{red}{#1}}
\newcommand{\blue}[1]{\textcolor{blue}{#1}}
\newcommand{\htext}[2]{\texorpdfstring{#1}{#2}}

% Statistics
\newcommand{\distr}{\sim}
\newcommand{\stddev}{\sigma}
\newcommand{\covmatrix}{\Sigma}
\newcommand{\mean}{\mu}
\newcommand{\param}{\gt}
\newcommand{\ftr}{\phi}

% General utility
\newcommand{\todo}[1]{\footnote{TODO: #1}}
\newcommand{\exclaim}[1]{{\textbf{\textit{#1}}}}
\newcommand{\tbc}{[\textbf{Incomplete}]}
\newcommand{\chk}{[\textbf{Check}]}
\newcommand{\oprob}{[\textbf{OP}]:}
\newcommand{\core}[1]{\textbf{Core Idea:}}
\newcommand{\why}{[\textbf{Find proof}]}
\newcommand{\opt}[1]{\textit{#1}}


\DeclareMathOperator*{\argmin}{arg\,min}
\DeclareMathOperator{\rank}{rank}
\newcommand{\redcol}[1]{\textcolor{red}{#1}}
\newcommand{\bluecol}[1]{\textcolor{blue}{#1}}
\newcommand{\greencol}[1]{\textcolor{green}{#1}}


\renewcommand{\~}{\htext{$\sim$}{~}}


% groupings of objects.
\newcommand{\set}[1]{\left\{ #1 \right\}}
\newcommand{\seq}[1]{\left(#1\right)}
\newcommand{\ang}[1]{\langle#1\rangle}
\newcommand{\tuple}[1]{\left(#1\right)}

% numerical shortcuts.
\newcommand{\abs}[1]{\left| #1\right|}
\newcommand{\floor}[1]{\left\lfloor #1 \right\rfloor}
\newcommand{\ceil}[1]{\left\lceil #1 \right\rceil}

% linear algebra shortcuts.
\newcommand{\change}{\Delta}
\newcommand{\norm}[1]{\left\| #1\right\|}
\newcommand{\dprod}[1]{\langle#1\rangle}
\newcommand{\linspan}[1]{\langle#1\rangle}
\newcommand{\conj}[1]{\overline{#1}}
\newcommand{\gradient}{\nabla}
\newcommand{\der}{\frac{d}{dx}}
\newcommand{\lap}{\Delta}
\newcommand{\kron}{\otimes}
\newcommand{\nperp}{\nvdash}

\newcommand{\mat}[1]{\left( \begin{smallmatrix}#1 \end{smallmatrix} \right)}

% derivatives and limits
\newcommand{\partder}[2]{\frac{\partial #1}{\partial #2}}
\newcommand{\partdern}[3]{\frac{\partial^{#3} #1}{\partial #2^{#3}}}

% Arrows
\newcommand{\diverge}{\nearrow}
\newcommand{\notto}{\nrightarrow}
\newcommand{\up}{\uparrow}
\newcommand{\down}{\downarrow}
% gets and gives are defined!

% ordering operators
\newcommand{\oleq}{\preceq}
\newcommand{\ogeq}{\succeq}

% programming and logic operators
\newcommand{\dfn}{:=}
\newcommand{\assign}{:=}
\newcommand{\co}{\ co\ }
\newcommand{\en}{\ en\ }


% logic operators
\newcommand{\xor}{\oplus}
\newcommand{\Land}{\bigwedge}
\newcommand{\Lor}{\bigvee}
\newcommand{\finish}{$\Box$}
\newcommand{\contra}{\Rightarrow \Leftarrow}
\newcommand{\iseq}{\stackrel{_?}{=}}


% Set theory
\newcommand{\symdiff}{\Delta}
\newcommand{\union}{\cup}
\newcommand{\inters}{\cap}
\newcommand{\Union}{\bigcup}
\newcommand{\Inters}{\bigcap}
\newcommand{\nullSet}{\phi}

% graph theory
\newcommand{\nbd}{\Gamma}

% Script alphabets
% For reals, use \Re

% greek letters
\newcommand{\eps}{\epsilon}
\newcommand{\del}{\delta}
\newcommand{\ga}{\alpha}
\newcommand{\gb}{\beta}
\newcommand{\gd}{\del}
\newcommand{\gf}{\phi}
\newcommand{\gF}{\Phi}
\newcommand{\gl}{\lambda}
\newcommand{\gm}{\mu}
\newcommand{\gn}{\nu}
\newcommand{\gr}{\rho}
\newcommand{\gs}{\sigma}
\newcommand{\gt}{\theta}
\newcommand{\gx}{\xi}

\newcommand{\sw}{\sigma}
\newcommand{\SW}{\Sigma}
\newcommand{\ew}{\lambda}
\newcommand{\EW}{\Lambda}

\newcommand{\Del}{\Delta}
\newcommand{\gD}{\Delta}
\newcommand{\gG}{\Gamma}
\newcommand{\gO}{\Omega}
\newcommand{\gL}{\Lambda}
\newcommand{\gS}{\Sigma}

% Formatting shortcuts
\newcommand{\red}[1]{\textcolor{red}{#1}}
\newcommand{\blue}[1]{\textcolor{blue}{#1}}
\newcommand{\htext}[2]{\texorpdfstring{#1}{#2}}

% Statistics
\newcommand{\distr}{\sim}
\newcommand{\stddev}{\sigma}
\newcommand{\covmatrix}{\Sigma}
\newcommand{\mean}{\mu}
\newcommand{\param}{\gt}
\newcommand{\ftr}{\phi}

% General utility
\newcommand{\todo}[1]{\footnote{TODO: #1}}
\newcommand{\exclaim}[1]{{\textbf{\textit{#1}}}}
\newcommand{\tbc}{[\textbf{Incomplete}]}
\newcommand{\chk}{[\textbf{Check}]}
\newcommand{\oprob}{[\textbf{OP}]:}
\newcommand{\core}[1]{\textbf{Core Idea:}}
\newcommand{\why}{[\textbf{Find proof}]}
\newcommand{\opt}[1]{\textit{#1}}


\DeclareMathOperator*{\argmin}{arg\,min}
\DeclareMathOperator{\rank}{rank}
\newcommand{\redcol}[1]{\textcolor{red}{#1}}
\newcommand{\bluecol}[1]{\textcolor{blue}{#1}}
\newcommand{\greencol}[1]{\textcolor{green}{#1}}


\renewcommand{\~}{\htext{$\sim$}{~}}


%opening
\title{Number Theory: Quick reference}
\author{vishvAs vAsuki}

\begin{document}
\maketitle
\tableofcontents

\part{Notation}
[n]: Set of first n natural numbers.

\part{Themes}
About Z.

\part{Rigorous patterns of ideas and solution strategies}

\section{Properties of Numbers}
Evenness and odness. Primes and composites. Unique factorization of n as product of primes: $n=p_{1}^{e_{1}} ..$.

\section{GCD}
gcd(x,y). $gcd(x, y) |x-y$.

\subsection{Euclid's algorithm}
To find gcd(x, y): if $y|x$ return y else return gcd(x, y-x).

From Euclid's alg, GCD(x,y)=ax+by. If 1 = ax+by, a $\equiv$ multiplicative inverse of x mod y.

\subsection{Extended Euclid's alg}
Find a,b using Euclid's alg.

\subsection{Diophontine equation}
Indeterminate polynomial eqn with integer solutions: eg: gcd(x, y) = ax + by in Euclid's alg.

\section{Conjectures}
Goldbach conjecture: $\forall x \in N, x>4$, x = sum of 2 primes.

\section{Primes}
\subsection{Special primes}
Marsenne prime: writ as $2^{n}-1$.

\subsection{Prime number theorem}
Num of primes under k = $\Pi(k) = (1 + o(1))\frac{k}{\ln k}$. \why

(Green, Tao) Number of arithmatic progressions of primes of length $\geq k$ is $\geq 1$.

\subsection{Primality testing of n}
Don't try to factor: assumed hard.

\subsection{Randomized primality test}
(Miller Rabin) Pick rand x in $Z^{+}_{n} - \set{0}$. If $x | n$ reject. See if (Fermat's little th, Lucas-Lehmer) $\forall x \in Z_{n}^{*}: x^{n-1} = 1 \mod n$ holds: do it in polylog time with repeated squaring. Repeat test with many x's. In failure, reject. Else, check if it is a Carmichael composite number: see if 1 has a non-trivial square root: Write $n-1=2^{s}d$; pick $x\neq \pm 1$; repeatedly square and check if $x \mod n = 1$: if so reject, else if $x \mod n = -1$: try starting with another x.

\subsection{Picking some prime below N}
Pick a random number below n, check if it is a prime: if not prime fail. By Prime number th, this alg has $\approx (\ln n)^{-1}$ success rate, which can then be amplified.

\section{Special numbers}
\subsection{Square free integers}
Aka quadratfrei. Divisible by no perfect square except 1.

\subsection{Carmichael composite number}
Let prime factorization: $n=p_{1}^{e_{1}} ..$. Aka Fermat pseudoprimes: They're Fermat liers: $\forall a: a^{n-1} \equiv 1 \mod n$. n is Carmichael iff it is square free; for all $p_{i}$ $p_{i}-1|n-1$. \why Eg: 561 = 3*11*17; $\forall a: a^{560} = 1 \mod 3, \mod 17, \mod 11$ as $2, 10, 16 | 560$.

\section{Modulo arithmatic}
The remainder fn. $-3 \equiv 2 \mod 5$. $ab \mod n \equiv (a \mod n) (b \mod n) \mod n$. So, congruence relation over $\mathbb{Z}$ wrt +, *. If $a \equiv b \mod n \implies n|a-b$.

\subsection{Cancellation law}
$ka \equiv kb \mod n \implies k(a-b) \equiv 0 \mod p \implies a \equiv b \mod n$.

\subsection{Chinese remainder theorem}
Let $n_{i}$'s coprime, $N=\prod_{j} n_{j}$. System of simultaneous congruences $x = a_{i} \mod n_{i}$ for $i=1 \dots k$ has a unique solution for x in $\mathbb{Z}_{N}$.

\subsubsection{Uniqueness}
If $\forall i, x \equiv x_{i} \mod n_{i}$, and $y \equiv x_{i} \mod n_{i}$, $x - y \equiv 0 \mod N$.

\subsubsection{Solving for x}
Use Extended Euclid's alg on $1=r_{i}n_{i}+s_{i}\frac{N}{n_{i}}$ to find $r_{i}$ and $s_{i}$, let $e_{i}=s_{i}\frac{N}{n_{i}}$; then $e_{i} \equiv 1 \mod n_{i}$ but $0 \mod n_{j}$; thence find $x=\sum_{i=1}^{k}a_{i}e_{i}$.

\subsubsection{Equivalent statements and implications}
$|Z_{N}| \to |\times_{i} Z_{n_{i}}|$. Map $x \to (x \mod n_{1}, ..)$ from $Z_{N} \to \times_{i} Z_{n_{i}}$ is both one to one and onto. Also, Isomorphism by Chinese remainder fn: $\mathbb{Z}_{n} \cong \times_{i}\mathbb{Z}_{n_{i}}$ preserves +, *.

\subsubsection{Utility}
Useful for manipulating composite numbers. An airthmatic question mod N reduced to arithmatic questions modulo $n_{i}$, if we know $\set{n_{i}}$.

\section{Additive group: \htext{$Z^{+}_{p}$}{}}
A prime order group. Does not have any subgroups.

\section{Multiplicative group \htext{$Z_{N}^{*}$}{}}
$Z^{*}_{N}$ : N's coprimes in $\{1, \dots, N-1\}, * \mod N$. Proof: GCD with N is 1, so use extended Euclid's alg to find inverses. If p prime; -1 := p-1; $\sqrt{1} \equiv \pm 1 \mod p$.

\subsection{Order}
N=pq; p, q primes: order = totient function: $|Z^{*}_{N}|=\varphi(N)=(p-1)(q-1)$: we discard multiples of p, q. Also, if $N= \prod p_{i}^{e_{i}}$, $\varphi(N)=\prod (p_{i}-1)p_{i}^{e_{i}-1}$.

(Euler's theorem). $a^{\varphi(N)} \equiv 1 \mod N$: Take $a, a^{2} \dots a^{k} = e$; this is a subgroup of $Z^{*}_{N}$; by Lagrange (see group theory in algebra ref), $k|\varphi(N)$.

$N= \prod p_{i}^{e_{i}}$. $\frac{|Z_{N}^{*}|}{|Z_{N}^{+}|} = \frac{\varphi(N)}{N} = \prod_{i = 1}^{t}\frac{p_{i} - 1}{p_{i}} \geq \prod_{i = 1}^{t}\frac{i}{i+1} = \frac{1}{1 + t} \geq \frac{1}{1 + \log_{2}N}$.

So, \textbf{Fermat's little theorem}: p prime: $a^{p} \equiv a \mod p$.

\subsection{Primitive roots}
Aka generator. If $S = Z_{n}^{*}$, g is primitive root of n. $Z_{p}^{*}$ for prime p always has primitive root \why. 7 has primitive roots 3, 5. $1,2,4,p^{k},2p^{k}$ have primitive roots for p odd prime and $k \geq 1$.\\
\why \tbc

The number of primitive roots, if there are any, is $\phi(\phi(n))$. (See group theory in algebra ref)

\subsection{Primitive root test}
g is primitive root of n iff its multiplicative order is $\phi(n)$: else it generates a subgroup. Efficiently see if g is a generator: find prime factors of $\phi(n) = \prod_{i} p_{i}$, keep seeing if $g^{\frac{\phi(n)}{p_{i}}} = 1$.

\section{Quadratic residues}
$QR_{n}$: set of squares mod n. Quadratic non residues. If $a \in QR_{n}, a R n$, else a N n.

Finding $\sqrt{x}$ same as solving $y^{2} = x \mod n$, or factoring $(y^{2}-x) \mod n$.

\subsection{\htext{$QR_{p}$}{QR} for odd prime p}
As structure of $Z_{p}^{*}$ cyclic; writable as $\set{g^{i}}$ for primitive root g; only even powers $\set{g^{2i}}$ are squares. So, $|QR_{p}| = |Z_{p}^{*}|/2$.

1 has exactly 2 roots: $\pm 1$, and no more: $x^{2} - 1 \mod p = (x-1) \mod p (x+1) \mod p = 0$ so x-1 = 0 mod p or x+1 = 0 mod p. $g^{\frac{p-1}{2}} = -1$. $\sqrt{g^{2i}} = \pm g^{i}$ by Euler thm.

\subsubsection{Jacobi symbol}
$(\frac{a}{p})$ = 0 if $p|a$; +1 if a R p, $p\nmid a$; -1 if a N p.

Legendre: generalization to n=pq; $(\frac{a}{n})$ = -1 if a N n; if a R n, $(\frac{a}{n}) = 1$, but can't tell if a R n given $(\frac{a}{n}) = 1$.

\subsection{\htext{$QR_{n}$}{..} for p, q odd primes; n = pq (Blum integer)}
$\exists 4\ \sqrt{1}$: take $x^{2} = 1 \mod n$; $\pm 1$ are obvious roots; Chinese remainder thm solutions s for $x = 1 \mod p; x = -1 \mod q$ and t for $x = -1 \mod p; x = +1 \mod q$ are the other two. As square roots appear in pairs, s = -t. To find the non trivial square roots, must know p, q.

Similarly, for any odd $m = m_{1}m_{2}$, 1 has $\geq 4$ roots.

So, any $a^{2} \in QR_{n}$ has $\geq 4$ roots: $a\sqrt{1}$. So, $4^{-1} |Z_{n}^{*}| \leq |QR_{n}|$.

%\bibliographystyle{plain}
%\bibliography{colt}

\end{document}
