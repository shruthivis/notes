\section{Higher-Order Discrete Graphical Models}\label{SecHigherOrder}

Consider the general higher-order MRF from \eqref{EqnGenMRF}
\begin{align*}
\mathbb{P}(x) & \propto & \exp \biggr \{\sum_{\clique \in \CliqueSet; v \in \{1, \ldots, \statenum-1 \}^{|\clique|}} \theta^*_{\clique;v} \, \I[x_\clique = v] \biggr \},
\end{align*}
parameterized by the collection of vectors $\theta^*_{\clique} \in \real^{(\statenum-1)^{|\clique|}}$ associated with the cliques $\clique \in \CliqueSet$. 

As before, we fix a node $r$, and define the long vector $\Theta*_{\backslash \svert} \in \real^{\sum_{j=1}^{c-1} \binom{p-1}{j} (m-1)^{j+1}}$ as the concatenation of the parameter vectors $\theta^*_{rC}$ for all $C \subseteq \vertex \backslash \svert;\, |C| < c$. Let $\bar{\Theta}^*_P\in\mathbb{R}^{(m-1)^2 d_\svert}$ be the vector containing only neighbor-pairwise parameters $\bar{\theta}^*_{\svert t;jk}$ for all $t\in\N(\svert)$. Accordingly, let $\bar{\Theta}^*_{P^c}$ represent all non-zero non-pairwise entries.

\myparagraph{Hierarchical Models}

A common assumption imposed on such higher-order MRFs is that they be hierarchical models~\citep{Lauritzen}. Specifically, any MRF of the form \eqref{EqnGenMRF} is hierarchical if for any clique $C$, $\theta^*_{C} = 0$ implies that  $\theta^*_{B} = 0$ for any clique $B \supseteq A$ containing $A$. This has an importance consequence: the set of pairwise effects 
\begin{eqnarray*}
\N(\svert) & = & \biggr \{ u \in \vertex \backslash \{\svert\} \, \mid
\, \|\theta^*_{\svert u}\|_0 \neq 0 \biggr \},
\end{eqnarray*}
completely characterizes the set of edges. 


Thus, if we are able to estimate just the pairwise parameters of the entire higher-order model, we would still be able to recover the edge-set. Thus, we study the 
estimator in \eqref{EqnPairwiseGroup} but now when the observations are actually drawn from $\bar{\Theta}^*_{\backslash\svert}$. The hope is that this solution would still estimate the  pairwise parameters of the underlying higher-order model well. 

\subsection{Assumptions}
For fixed positive values $C_{\min}$, $D_{\max}$ and $\alpha\in\left(0,\frac{1}{2}\right)$, let $\gamma:=\frac{D_{\max}}{C_{\min}}\norm{\bar{\Theta}^*_{P^c}}{1}$ and $\tau=\frac{\alpha+\gamma(\sqrt{d_\svert}+1)}{1+\gamma}.$ We impose the following assumptions on the truth:
\begin{itemize}
\item [(C0)] {\bf Mismatch Factor}: $\gamma\leq\left(\frac{\alpha}{2-\alpha}\right)^2\frac{C_{\min}}{100\sqrt{2}(m-1) d_\svert}$.\\

This condition is required because of the mismatch of the true underlying model and our pairwise model. In other words, we have a non-zero mean noise, caused by model mismatch, that needs to be small. Moreover, since $C_{\min}\leq (m-1)\sqrt{d_\svert}$ (see section ~\ref{Hessian_Section}), this condition ensures that $\tau\in\left(0,\frac{1}{2}\right)$ for suitable choice of $\alpha$.\\

\item [(C1)] {\bf Invertibility}: $\Lambda_{\min}\Biggr(\mathbb{E}\Bigg[\nabla^2\log\Big(\mathbb{P}_{\bar{\Theta}^*_{\backslash\svert}}\left[X_\svert\mid X_{\backslash\svert}\right]\Big)\Bigg]_{\support_\svert\support_\svert}\Biggr)\geq C_{\min}(1+\gamma)$.

\item [(C2)] {\bf Incoherence}: $\displaystyle\norm{\mathbb{E}\Bigg[\nabla^2\log\Big(\mathbb{P}_{\bar{\Theta}^*_{\backslash\svert}}\left[X_\svert\mid X_{\backslash\svert}\right]\Big)\Bigg]_{\support_\svert^c\support_\svert}\!\!\!\!\mathbb{E}\Bigg[\nabla^2\log\Big(\mathbb{P}_{\bar{\Theta}^*_{\backslash\svert}}\left[X_\svert\mid X_{\backslash\svert}\right]\Big)\Bigg]_{\support_\svert\support_\svert}^{-1}}{\infty,2}\!\!\!\!\leq \frac{1-2\tau}{\sqrt{d_\svert}}.$\\

\item [(C3)] {\bf Boundedness}: $\Lambda_{\max}\left(\J^*\right)\leq D_{\max}<\infty$,\\

 where $\J^*\in\mathbb{R}^{\sum_{i=1}^{c-1}(m-1)^j(p-1)^j\times\sum_{i=1}^{c-1}(m-1)^j(p-1)^j}$ is a matrix defined as $\J^*=\mathbb{E}\left[\I[X_{S_1}=x_{S_1}]\I[X_{S_2}=x_{S_2}]^T\right]$ for any subset of nodes $S_1$ and $S_2$, and $c$ is the size of the maximum clique in the true graphical model.
\end{itemize} 

\noindent In order to analyze the derivative of the Hessian of the log-likelihood function, we need to bound the maximum eigen value of matrix $\Im^*=\J^*\otimes \mathbf{1}_{\sum_{j=1}^{c-1}(m-1)^j\times\sum_{j=1}^{c-1}(m-1)^j}$. For future reference, note that by Kronecker product properties, we have $\Lambda_{\max}\left(\Im^*\right)=\Lambda_{\max}\left(\J^*\right)$. Next Lemma illustrates the effect of these assumptions on the statistics of the samples.

\begin{lemma}
Assumptions (C0) - (C3) imply the following bounds on the pairwise parameters
\begin{itemize}
\item [(D1)] $\mathbb{P}\left[\Lambda_{\min}\Bigg(\left[\nabla^2\ell\Big(\bar{\Theta}^*_P;\Data\Big)\right]_{\support_\svert\support_\svert}\Bigg)\leq C_{\min}-\epsilon\right]$\\ .$\qquad\qquad\qquad\qquad\qquad\qquad\leq 2\exp\left(-\frac{1}{8}\left(\epsilon\sqrt{n}-\sqrt{\sum_{j=1}^{c-1}d_\svert^{\,j}}\,\right)^2 +\log\left(\sum_{j=2}^{c}(m-1)^jd_\svert^{j-1}\right)\right).$

\item [(D2)] $\displaystyle\mathbb{P}\left[\norm{\nabla^2\ell\left(\bar{\Theta}^*_P;D\right)_{\support_\svert^c\support_\svert}\!\! \left(\nabla^2\ell\left(\bar{\Theta}^*_P;D\right)_{\support_\svert\support_\svert}\right)^{\!-1}}{\infty,2} > \frac{1-\alpha}{\sqrt{d_\svert}}+\epsilon\right]$\\
$\leq 6\exp\!\left(\!\!-\frac{1}{8}\!\!\left(\bar{C}_{\min}\!\!\left(\frac{\tau}{3\sqrt{\sum_{j=1}^{c-1}d_\svert^{\,j}}}+\epsilon\right)\!\!\sqrt{n} -\!\!\left(\!1\!+\!\frac{\sqrt{\sum_{j=1}^{c-1}d_\svert^{\,j}}}{C_{\min}^2\sqrt{n}}\right) \!\!\sqrt{\sum_{j=1}^{c-1}d_\svert^{\,j}}\,\right)^{\!2}\!\!\!\!\! +\!\log\left(\sum_{j=2}^c(m-1)^j(1-p)^{j-1}\right)\right)$.

\item [(D3)] $\mathbb{P}\left[\Lambda_{\max}\left(\hat{\mathbb{E}}\left[\I[X_{t_1}=k_1]\I[X_{t_2}=k_2]^T\right]\right)
\geq D_{\max}+\epsilon\right]$\\ .$\qquad\qquad\qquad\qquad\qquad\qquad\leq 2\exp\left(-\frac{1}{8}\left(\epsilon\sqrt{n}-\sqrt{\sum_{j=1}^{c-1}d_\svert^{\,j}}\,\right)^2 +\log\left(\sum_{j=2}^{c}(m-1)^jd_\svert^{j-1}\right)\right)$.
\end{itemize}
\label{concentration_clique}
\end{lemma}

\subsection{Main Theorem}
\noindent The following theorem shows that if the graphical model satisfies hierarchical assumption, then pairwise estimation \underline{exactly} recovers the underlying graphical model provided that the higher order dependency parameters are not too large.

\begin{theorem}
Consider an $m$-ary graphical model with parameter $\bar{\Theta}^*$ and associate edge set $\bar{\edge}$ such that conditions (C0)-(C3) and hierarchical assumption are satisfied. Suppose the size of the largest clique in the graph is $c$ and the regularization parameter satisfies
\begin{equation}
\regpar_n\geq \frac{8(2-\alpha)}{\alpha}\left(\sqrt{\frac{\log(p-1)}{n}}+\frac{m-1}{4\sqrt{n}} +\frac{1}{4}\norm{\bar{\Theta}^*_{P^c}}{1}\right).
\end{equation}
Then, there exist positive constants $K$, $c_1$ and $c_2$ such that for 
\begin{equation}
n\geq K (m-1)^2 d_\svert^{\,\frac{3}{2}c-1} \log\left((m-1)^c(p-1)^{c-1}\right),
\end{equation}
with probability $1-c_1\exp\left(-c_2\left(\regpar_n-2\norm{\bar{\Theta}^*_{P^c}}{1}\right)^2n\right)$ we are guaranteed
\begin{itemize}
\item [(a)] For each node $\svert\in\vertex$, the $\ell_1/\ell_2$ regularized logistic regression \eqref{EqnPairwiseGroup} has a unique solution and hence specifies a neigborhood $\widehat{\N}(\svert)$.
\item [(b)] For each node $\svert\in\vertex$ correctly excludes all edges not in the true neighborhood $\N(\svert)$. Moreover, it includes all edges $(\svert,t)$ such that $\norm{\bar{\theta}^*_{\svert t;jk}}{2}\geq\frac{10}{C_{\min}}\lambda_n$.\\
\end{itemize}
\end{theorem}

\begin{proof}
We prove each part separately.
\begin{itemize}
\item [(a)] The proof consists of the construction of a valid primal-dual pair in the following three steps:
\begin{itemize}
\item [(i)] {\bf Constructing an Optimal Primal Candidate}: Let $\hat{\Theta}_{\backslash \svert}$ be the optimal solution of the restricted problem
\begin{equation}
\label{OracleEqnPairwiseGroup_general}
\hat{\Theta}_{\backslash \svert}=\arg\min_{\left(\Theta_{\backslash \svert}\right)_{\support_\svert^c}=\mathbf{0}} \biggr\{ \ell(\Theta_{\backslash \svert}; \Data) +
	\regpar_\numobs \norm{\Theta_{\backslash \svert}}{1,2}
	\biggr\}.
\end{equation}
Respectively, for any column $u\in\support_\svert$ set $\left(\hat{\dual}_{\backslash \svert}\right)_{u}=\frac{1}{\left\|\left(\hat{\Theta}_{\backslash \svert}\right)_u\right\|_2}\left(\hat{\Theta}_{\backslash \svert}\right)_u$.
\item [(ii)] {\bf Satisfying the Necessary Optimality Condition}: Using the result of the step (i), we solve \eqref{NecessaryOptCondition} for $\left(\hat{\dual}_{\backslash \svert}\right)_{\support_\svert^c}$.
\item [(iii)] {\bf Satisfying the Sufficient Optimality Condition}: We need to show that the constructed primal-dual pair $\left(\hat{\Theta}_{\backslash \svert},\hat{\dual}_{\backslash \svert}\right)$ satisfy the conditions of the Lemma~\ref{SuffOptCond}. By Lemma~\ref{OptCertificate_general}, these conditions are satisfied with high probability.
\end{itemize}
Hence, the solution to \eqref{EqnPairwiseGroup} is unique.\\
\item[(b)] By uniqueness of the solution shown in part [(a)], the method excludes all edges that are not in the set of edges. To show that all correct edges are included, i.e., to show the correct correct sign recovery, it suffices to show that
\begin{equation}
\norm{\widehat{\Theta}_{\support_\svert}-\widehat{\Theta}^*_{\support_\svert}}{\infty,2} \leq\frac{\theta_{\min}}{2},
\nonumber
\end{equation} 
where, $\displaystyle\theta_{\min}=\min_{t\in\vertex\backslash\{\svert\}}\norm{\theta_{\svert t;jk}}{2}$. From \eqref{err_bound_general}, we have
\begin{equation}
\begin{aligned}
\frac{2}{\theta_{\min}}\norm{\widehat{\Theta}_{\support_\svert}-\widehat{\Theta}^*_{\support_\svert}}{\infty,2}
&\leq\frac{2}{\theta_{\min}}\frac{5}{C_{\min}}\regpar_n\\
&\leq 1,
\end{aligned}
\nonumber
\end{equation}
provided that $\theta_{\min}>\frac{10}{C_{\min}}\regpar_n$.\\
\end{itemize}
\end{proof}

