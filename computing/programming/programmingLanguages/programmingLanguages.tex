\documentclass[oneside, article]{memoir}
\usepackage{amsmath, amssymb}
\usepackage{hyperref, graphicx, verbatim, listings, multirow, subfigure}
\usepackage{algorithm, algorithmic}
% \usepackage[bottom]{footmisc}
\lstset{breaklines=true}
\setcounter{tocdepth}{3}

% Lets verbatim and verb environments automatically break lines.
\makeatletter
\def\@xobeysp{ }
\makeatother
% \lstset{breaklines=true,basicstyle=\ttfamily}

% Configuration for the memoir class.
\renewcommand{\cleardoublepage}{}
% \renewcommand*{\partpageend}{}
\renewcommand{\afterpartskip}{}
\maxsecnumdepth{subsubsection} % number subsections
\maxtocdepth{subsubsection}

\addtolength{\parindent}{-5mm}
% Packages not included:
% For multiline comments, use caption package. But this conflicts with hyperref while making html files.
% subfigure conflicts with use with memoir style-sheet.

% Use something like:
% % Use something like:
% % Use something like:
% \input{../../macros}

% groupings of objects.
\newcommand{\set}[1]{\left\{ #1 \right\}}
\newcommand{\seq}[1]{\left(#1\right)}
\newcommand{\ang}[1]{\langle#1\rangle}
\newcommand{\tuple}[1]{\left(#1\right)}

% numerical shortcuts.
\newcommand{\abs}[1]{\left| #1\right|}
\newcommand{\floor}[1]{\left\lfloor #1 \right\rfloor}
\newcommand{\ceil}[1]{\left\lceil #1 \right\rceil}

% linear algebra shortcuts.
\newcommand{\change}{\Delta}
\newcommand{\norm}[1]{\left\| #1\right\|}
\newcommand{\dprod}[1]{\langle#1\rangle}
\newcommand{\linspan}[1]{\langle#1\rangle}
\newcommand{\conj}[1]{\overline{#1}}
\newcommand{\gradient}{\nabla}
\newcommand{\der}{\frac{d}{dx}}
\newcommand{\lap}{\Delta}
\newcommand{\kron}{\otimes}
\newcommand{\nperp}{\nvdash}

\newcommand{\mat}[1]{\left( \begin{smallmatrix}#1 \end{smallmatrix} \right)}

% derivatives and limits
\newcommand{\partder}[2]{\frac{\partial #1}{\partial #2}}
\newcommand{\partdern}[3]{\frac{\partial^{#3} #1}{\partial #2^{#3}}}

% Arrows
\newcommand{\diverge}{\nearrow}
\newcommand{\notto}{\nrightarrow}
\newcommand{\up}{\uparrow}
\newcommand{\down}{\downarrow}
% gets and gives are defined!

% ordering operators
\newcommand{\oleq}{\preceq}
\newcommand{\ogeq}{\succeq}

% programming and logic operators
\newcommand{\dfn}{:=}
\newcommand{\assign}{:=}
\newcommand{\co}{\ co\ }
\newcommand{\en}{\ en\ }


% logic operators
\newcommand{\xor}{\oplus}
\newcommand{\Land}{\bigwedge}
\newcommand{\Lor}{\bigvee}
\newcommand{\finish}{$\Box$}
\newcommand{\contra}{\Rightarrow \Leftarrow}
\newcommand{\iseq}{\stackrel{_?}{=}}


% Set theory
\newcommand{\symdiff}{\Delta}
\newcommand{\union}{\cup}
\newcommand{\inters}{\cap}
\newcommand{\Union}{\bigcup}
\newcommand{\Inters}{\bigcap}
\newcommand{\nullSet}{\phi}

% graph theory
\newcommand{\nbd}{\Gamma}

% Script alphabets
% For reals, use \Re

% greek letters
\newcommand{\eps}{\epsilon}
\newcommand{\del}{\delta}
\newcommand{\ga}{\alpha}
\newcommand{\gb}{\beta}
\newcommand{\gd}{\del}
\newcommand{\gf}{\phi}
\newcommand{\gF}{\Phi}
\newcommand{\gl}{\lambda}
\newcommand{\gm}{\mu}
\newcommand{\gn}{\nu}
\newcommand{\gr}{\rho}
\newcommand{\gs}{\sigma}
\newcommand{\gt}{\theta}
\newcommand{\gx}{\xi}

\newcommand{\sw}{\sigma}
\newcommand{\SW}{\Sigma}
\newcommand{\ew}{\lambda}
\newcommand{\EW}{\Lambda}

\newcommand{\Del}{\Delta}
\newcommand{\gD}{\Delta}
\newcommand{\gG}{\Gamma}
\newcommand{\gO}{\Omega}
\newcommand{\gL}{\Lambda}
\newcommand{\gS}{\Sigma}

% Formatting shortcuts
\newcommand{\red}[1]{\textcolor{red}{#1}}
\newcommand{\blue}[1]{\textcolor{blue}{#1}}
\newcommand{\htext}[2]{\texorpdfstring{#1}{#2}}

% Statistics
\newcommand{\distr}{\sim}
\newcommand{\stddev}{\sigma}
\newcommand{\covmatrix}{\Sigma}
\newcommand{\mean}{\mu}
\newcommand{\param}{\gt}
\newcommand{\ftr}{\phi}

% General utility
\newcommand{\todo}[1]{\footnote{TODO: #1}}
\newcommand{\exclaim}[1]{{\textbf{\textit{#1}}}}
\newcommand{\tbc}{[\textbf{Incomplete}]}
\newcommand{\chk}{[\textbf{Check}]}
\newcommand{\oprob}{[\textbf{OP}]:}
\newcommand{\core}[1]{\textbf{Core Idea:}}
\newcommand{\why}{[\textbf{Find proof}]}
\newcommand{\opt}[1]{\textit{#1}}


\DeclareMathOperator*{\argmin}{arg\,min}
\DeclareMathOperator{\rank}{rank}
\newcommand{\redcol}[1]{\textcolor{red}{#1}}
\newcommand{\bluecol}[1]{\textcolor{blue}{#1}}
\newcommand{\greencol}[1]{\textcolor{green}{#1}}


\renewcommand{\~}{\htext{$\sim$}{~}}


% groupings of objects.
\newcommand{\set}[1]{\left\{ #1 \right\}}
\newcommand{\seq}[1]{\left(#1\right)}
\newcommand{\ang}[1]{\langle#1\rangle}
\newcommand{\tuple}[1]{\left(#1\right)}

% numerical shortcuts.
\newcommand{\abs}[1]{\left| #1\right|}
\newcommand{\floor}[1]{\left\lfloor #1 \right\rfloor}
\newcommand{\ceil}[1]{\left\lceil #1 \right\rceil}

% linear algebra shortcuts.
\newcommand{\change}{\Delta}
\newcommand{\norm}[1]{\left\| #1\right\|}
\newcommand{\dprod}[1]{\langle#1\rangle}
\newcommand{\linspan}[1]{\langle#1\rangle}
\newcommand{\conj}[1]{\overline{#1}}
\newcommand{\gradient}{\nabla}
\newcommand{\der}{\frac{d}{dx}}
\newcommand{\lap}{\Delta}
\newcommand{\kron}{\otimes}
\newcommand{\nperp}{\nvdash}

\newcommand{\mat}[1]{\left( \begin{smallmatrix}#1 \end{smallmatrix} \right)}

% derivatives and limits
\newcommand{\partder}[2]{\frac{\partial #1}{\partial #2}}
\newcommand{\partdern}[3]{\frac{\partial^{#3} #1}{\partial #2^{#3}}}

% Arrows
\newcommand{\diverge}{\nearrow}
\newcommand{\notto}{\nrightarrow}
\newcommand{\up}{\uparrow}
\newcommand{\down}{\downarrow}
% gets and gives are defined!

% ordering operators
\newcommand{\oleq}{\preceq}
\newcommand{\ogeq}{\succeq}

% programming and logic operators
\newcommand{\dfn}{:=}
\newcommand{\assign}{:=}
\newcommand{\co}{\ co\ }
\newcommand{\en}{\ en\ }


% logic operators
\newcommand{\xor}{\oplus}
\newcommand{\Land}{\bigwedge}
\newcommand{\Lor}{\bigvee}
\newcommand{\finish}{$\Box$}
\newcommand{\contra}{\Rightarrow \Leftarrow}
\newcommand{\iseq}{\stackrel{_?}{=}}


% Set theory
\newcommand{\symdiff}{\Delta}
\newcommand{\union}{\cup}
\newcommand{\inters}{\cap}
\newcommand{\Union}{\bigcup}
\newcommand{\Inters}{\bigcap}
\newcommand{\nullSet}{\phi}

% graph theory
\newcommand{\nbd}{\Gamma}

% Script alphabets
% For reals, use \Re

% greek letters
\newcommand{\eps}{\epsilon}
\newcommand{\del}{\delta}
\newcommand{\ga}{\alpha}
\newcommand{\gb}{\beta}
\newcommand{\gd}{\del}
\newcommand{\gf}{\phi}
\newcommand{\gF}{\Phi}
\newcommand{\gl}{\lambda}
\newcommand{\gm}{\mu}
\newcommand{\gn}{\nu}
\newcommand{\gr}{\rho}
\newcommand{\gs}{\sigma}
\newcommand{\gt}{\theta}
\newcommand{\gx}{\xi}

\newcommand{\sw}{\sigma}
\newcommand{\SW}{\Sigma}
\newcommand{\ew}{\lambda}
\newcommand{\EW}{\Lambda}

\newcommand{\Del}{\Delta}
\newcommand{\gD}{\Delta}
\newcommand{\gG}{\Gamma}
\newcommand{\gO}{\Omega}
\newcommand{\gL}{\Lambda}
\newcommand{\gS}{\Sigma}

% Formatting shortcuts
\newcommand{\red}[1]{\textcolor{red}{#1}}
\newcommand{\blue}[1]{\textcolor{blue}{#1}}
\newcommand{\htext}[2]{\texorpdfstring{#1}{#2}}

% Statistics
\newcommand{\distr}{\sim}
\newcommand{\stddev}{\sigma}
\newcommand{\covmatrix}{\Sigma}
\newcommand{\mean}{\mu}
\newcommand{\param}{\gt}
\newcommand{\ftr}{\phi}

% General utility
\newcommand{\todo}[1]{\footnote{TODO: #1}}
\newcommand{\exclaim}[1]{{\textbf{\textit{#1}}}}
\newcommand{\tbc}{[\textbf{Incomplete}]}
\newcommand{\chk}{[\textbf{Check}]}
\newcommand{\oprob}{[\textbf{OP}]:}
\newcommand{\core}[1]{\textbf{Core Idea:}}
\newcommand{\why}{[\textbf{Find proof}]}
\newcommand{\opt}[1]{\textit{#1}}


\DeclareMathOperator*{\argmin}{arg\,min}
\DeclareMathOperator{\rank}{rank}
\newcommand{\redcol}[1]{\textcolor{red}{#1}}
\newcommand{\bluecol}[1]{\textcolor{blue}{#1}}
\newcommand{\greencol}[1]{\textcolor{green}{#1}}


\renewcommand{\~}{\htext{$\sim$}{~}}


% groupings of objects.
\newcommand{\set}[1]{\left\{ #1 \right\}}
\newcommand{\seq}[1]{\left(#1\right)}
\newcommand{\ang}[1]{\langle#1\rangle}
\newcommand{\tuple}[1]{\left(#1\right)}

% numerical shortcuts.
\newcommand{\abs}[1]{\left| #1\right|}
\newcommand{\floor}[1]{\left\lfloor #1 \right\rfloor}
\newcommand{\ceil}[1]{\left\lceil #1 \right\rceil}

% linear algebra shortcuts.
\newcommand{\change}{\Delta}
\newcommand{\norm}[1]{\left\| #1\right\|}
\newcommand{\dprod}[1]{\langle#1\rangle}
\newcommand{\linspan}[1]{\langle#1\rangle}
\newcommand{\conj}[1]{\overline{#1}}
\newcommand{\gradient}{\nabla}
\newcommand{\der}{\frac{d}{dx}}
\newcommand{\lap}{\Delta}
\newcommand{\kron}{\otimes}
\newcommand{\nperp}{\nvdash}

\newcommand{\mat}[1]{\left( \begin{smallmatrix}#1 \end{smallmatrix} \right)}

% derivatives and limits
\newcommand{\partder}[2]{\frac{\partial #1}{\partial #2}}
\newcommand{\partdern}[3]{\frac{\partial^{#3} #1}{\partial #2^{#3}}}

% Arrows
\newcommand{\diverge}{\nearrow}
\newcommand{\notto}{\nrightarrow}
\newcommand{\up}{\uparrow}
\newcommand{\down}{\downarrow}
% gets and gives are defined!

% ordering operators
\newcommand{\oleq}{\preceq}
\newcommand{\ogeq}{\succeq}

% programming and logic operators
\newcommand{\dfn}{:=}
\newcommand{\assign}{:=}
\newcommand{\co}{\ co\ }
\newcommand{\en}{\ en\ }


% logic operators
\newcommand{\xor}{\oplus}
\newcommand{\Land}{\bigwedge}
\newcommand{\Lor}{\bigvee}
\newcommand{\finish}{$\Box$}
\newcommand{\contra}{\Rightarrow \Leftarrow}
\newcommand{\iseq}{\stackrel{_?}{=}}


% Set theory
\newcommand{\symdiff}{\Delta}
\newcommand{\union}{\cup}
\newcommand{\inters}{\cap}
\newcommand{\Union}{\bigcup}
\newcommand{\Inters}{\bigcap}
\newcommand{\nullSet}{\phi}

% graph theory
\newcommand{\nbd}{\Gamma}

% Script alphabets
% For reals, use \Re

% greek letters
\newcommand{\eps}{\epsilon}
\newcommand{\del}{\delta}
\newcommand{\ga}{\alpha}
\newcommand{\gb}{\beta}
\newcommand{\gd}{\del}
\newcommand{\gf}{\phi}
\newcommand{\gF}{\Phi}
\newcommand{\gl}{\lambda}
\newcommand{\gm}{\mu}
\newcommand{\gn}{\nu}
\newcommand{\gr}{\rho}
\newcommand{\gs}{\sigma}
\newcommand{\gt}{\theta}
\newcommand{\gx}{\xi}

\newcommand{\sw}{\sigma}
\newcommand{\SW}{\Sigma}
\newcommand{\ew}{\lambda}
\newcommand{\EW}{\Lambda}

\newcommand{\Del}{\Delta}
\newcommand{\gD}{\Delta}
\newcommand{\gG}{\Gamma}
\newcommand{\gO}{\Omega}
\newcommand{\gL}{\Lambda}
\newcommand{\gS}{\Sigma}

% Formatting shortcuts
\newcommand{\red}[1]{\textcolor{red}{#1}}
\newcommand{\blue}[1]{\textcolor{blue}{#1}}
\newcommand{\htext}[2]{\texorpdfstring{#1}{#2}}

% Statistics
\newcommand{\distr}{\sim}
\newcommand{\stddev}{\sigma}
\newcommand{\covmatrix}{\Sigma}
\newcommand{\mean}{\mu}
\newcommand{\param}{\gt}
\newcommand{\ftr}{\phi}

% General utility
\newcommand{\todo}[1]{\footnote{TODO: #1}}
\newcommand{\exclaim}[1]{{\textbf{\textit{#1}}}}
\newcommand{\tbc}{[\textbf{Incomplete}]}
\newcommand{\chk}{[\textbf{Check}]}
\newcommand{\oprob}{[\textbf{OP}]:}
\newcommand{\core}[1]{\textbf{Core Idea:}}
\newcommand{\why}{[\textbf{Find proof}]}
\newcommand{\opt}[1]{\textit{#1}}


\DeclareMathOperator*{\argmin}{arg\,min}
\DeclareMathOperator{\rank}{rank}
\newcommand{\redcol}[1]{\textcolor{red}{#1}}
\newcommand{\bluecol}[1]{\textcolor{blue}{#1}}
\newcommand{\greencol}[1]{\textcolor{green}{#1}}


\renewcommand{\~}{\htext{$\sim$}{~}}



%opening
\title{Programming and programming languages}
\author{vishvAs vAsuki}

\begin{document}
\maketitle
\tableofcontents

\part{Introduction}
\chapter{Parallel programming programming models}
General parallel programming concerns are considered in the distributed computing survey.

\section{Thread abstraction}
Parallel programs can be considered in terms of threads of computation. When data is shared, one should take care of race conditions.

\subsection{Fork and join}
With this paradigm, a thread can fork to form a new thread. \tbc

\section{Mapreduce}
\subsection{Computation flow}
\subsubsection{Input preparation}
Master splits the input into multiple chunks, as necessary. It collects counter updates from mappers and reducers. May provide a html status page.

If applicable, the master also joins disparate input sources to produce a (key, valueList) pair.

There can be multiple disparate input source tables - each may be fed to a different mapper. Outputs from disparate mappers are combined based on key before reduce stage.

\subsubsection{General flow per mapper}
(key, value) - mapper - ((key, value)), counter updates - [combiner - ([same key], valueList)]

A combiner is a special type of reducer which may combine values from the outputs of a single mapper task before sending them accross the network for combination by reducers.

\subsubsection{Flow per reducer}
There is one shuffler per reducer. One reducer per shard.

(key, value) from various mappers - shuffler - (key, valueList) - reducer - ([same key], valueList), counter updates in output (file) shard.

\subsubsection{Output storage}
A file location in a distributed file system, together with the desired number of shards and output format is usually specified.

\subsubsection{Thread safety}
There is one thread running through the map function generally. So, in case of object oriented programming languages: the mapper object is usually thread-safe, Although class variables are not.

\tbc

\chapter{Translation to machine code}
\section{Machine instructions and their files}
Machine instructions are understood by the processor. Assembly level code provides a way to write machine instructions using words instead of hexadecimal instructions.

\subsection{Executable files}
When an operating system is being used, machine instructions, which are stored in a file or on fixed locations in the hard-disk, should be associated with suitable meta-data/ headers. Such files are called executable files.

\subsection{Object files}
Object files contain named segments of machine instructions - library functions. These files may be linked to executable files, from which, using a mechanism like the 'call stack', data is processed using the library functions.

\section{Translation from High level programming language}
Writing machine instructions directly is excruciating - most of the work is mechanical - it is best done automatically by specialized programs which read files with precise yet high-level instructions and create machine instructions.

\subsection{Compiled vs interpreted languages}
In the case of interpreted languages, this is done one line at a time. In case of compiled languages, this is done one program or code block at a time.

Some languages, like Java are compiled to a byte-code, which is then interpreted (translated to machine language) during execution.

Run-times of programs written in Interpreted languages can slow down due to the cost of translating each line of code at run-time. This problem is assuaged with the use of 'just in time' compilers - now, loops can potentially be translated entirely to machine language before they are entered.

\subsubsection{Freedom from error}
Compiled langauges tend to be more error free, because successful compilation guarantees: a] every variable used is known  to be clearly defined. b] Every function is passed the right sequence of arguments.

In case of a interpreted language, since they lack the compilation checks, errors are caught only at run-time, that too if execution of the erroneous part of the code is attempted. This can be assuaged by use of special linting tools.

\subsection{Steps in translation}
During lexical analysis, lexical tokens are identified from the high level program; variable names are distinguished from constants and operators. During syntactic analysis, or parsing, the tokens are put together into a bunch of instructions. Finally, addresses are assigned to variable names, and translation happens.

During this, linking is done by a linker: references to functions described in other object files are resolved.

\section{Memory allocation}
\subsection{Compile-time memory allocation}
This is memory allocated to data at compile time.

Memory can be considered as having been allocated at the time of function call and deallocated when the funciton exits. So, this is often visualized/ reffered to/ implemented as 'stack space'.

Stack memory is often more limited compared to heap space (although both can be increased).

\subsection{Run-time memory allocation}
This is done at run-time. One cannot be absolutely certain beforehand about the exact amount of memory a program will require. There are overheads in dynamic allocation: there can be problems such as fragmentation: data used by the same chunk of code may be located in different memory blocks (either in the main memory or in the cache on the hard disk). Allocation algorithms take atleast 50 instructions for making one allocation.

The space used for dynamic memory allocation is often called 'heap space'.

\chapter{Character encoding}
Displaying, accepting and writing to files characters readable by humans are common tasks in many programs, irrespective of language.

So, common standards have evolved to represent these characters with natural numbers or characters visible on most English keyboards or arbitrary bytes.

\section{ASCII}
This represents the Latin alphabet plus some common characters. Range: 0:255.

Special characters include control characters.

\subsection{Control characters}
Carriage return (move cursor to beginning of current line), line-feed (start a new line), tab (a long horizontal space). Their popular latin/ symbolic representations are: '\\r\\n\\t'. Carriage return without line-feed is often used to overwrite text.

\section{Unicode}
This is capable of accomodating symbols in many of the world's scripts. It uses multiple bytes. Since the number of bytes actually needed may vary, variants such as UTF-8 are used.

UTF-8 is backward compatible with much of ASCII. If extra bytes are needed, it is indicated by a special bit.



\part{Development tools}
\chapter{Build tool design}
\section{Build Targets}
A programmer may want to do certain specific actions with his code.

For example, downloading dependencies for being able to compile, compiling code, running test cases, running the code, packaging the code, deploying a package on a web-server. These actions are called build targets.

\subsection{Continuous mode}
With some dependencies, one can execute a specific action while constantly scanning for source code changes..

\subsection{Target Dependencies}
There may be various dependencies amongst these build targets. For example, to compile one may need to download some libraries from the internet.

\subsection{Library dependencies}
Some build tools offer automated library dependency management, where if a repository and the dependencies are specified, the dependencies are automatically downloaded from the internet.

\section{Quality metrics}
A good build system provides a concise way of declaratively expressing actions to be performed for various build targets: So, various commands for actions like downloading should be available.

Some build systems are also integrated with IDE's'.

\chapter{Unit test tools}
\section{Mox for Python}
The tester stubs out various functions that the method being tested is supposed to call. Then in the test, you make calls to those stubbed out functions with the arguments that you expect to be passed to them if the function being tested is working properly, and mox records it. Then, when the function is actually called, mox asserts that the order and arguments to the stubbed out functions are what was expected.



\chapter{Build tools}
\section{Cross-platform makefile generator}
CMake uses makefiles, wherein compilation tasks are defined. It can be used to generate eclipse and visual studio projects - eg: with cmake -G"Eclipse CDT4 - Unix Makefiles" dirPath.

It is configured using cmakelist files.

\section{sbt}
Scala build tool is written in scala. Configuration files tend to be very concise. See a ready empty sbt project for a quick start.

Important commands include: update (To satisfy dependencies), actions, compile, package. Some commands may be prefixed by \~ (Which constantly scans for source code changes and recompiles automatically when necessary.)

\subsection{Dependencies - automation}
One can automate the problem of downloading certain versions of external libraries on which the project depends, and adding the jars to the CLASSPATH if necessary.

\subsubsection{Repository specification}
By default, Maven2 and Scala Tools Releases repositories are used. The former includes many sourceforge libraries. One can add other repositories: resolvers += name at location.

\subsubsection{Library specification}
In build.sbt in the project folder, add the following line:
\begin{verbatim}
libraryDependencies += groupID % artifactID % revision
\end{verbatim}
If you are using a dependency that was built with sbt, double the first \% to be \%\%.

The revision argument could be: 1.3.2 or "[1.3,)" etc..

One can explicitly specify the url above by adding 'from "url1"' to the line above. The url is used only if it is necessary.

One can also add 'sbtAction' to indicate that the library dependency need only be included if a certain sbtAction is to be performed. \chk

\subsection{Tasks: definition}
lazy val collectJars = task \{ collectJarsTask; None \} dependsOn(compile)

Here, collectJarsTask is a scala method.

\subsubsection{File IO tasks}
Use FileUtilities.

\subsubsection{Documentation}
sbt doc

\section{maven}
Build configuration files are in xml, which tends to lead to bloated configuration files.

\chapter{Task management}
These services/ software are used to manage bugs, features, issues. Issue management is often bundled with project hosting websites which offer version control.

\chapter{Version control software}
\section{svn}
General syntax: svn command arguments.

For help, see svn help.

Commands include: commit/ ci, diff, update/ up, merge, resolved, status/ st.

To rollback a file to a previous revision: svn merge -r newRev:oldRev filename.

\verb'export SVN_EDITOR=vi'

\section{git}
Enables distributed development by letting people maintain local repositories which may be merged with a central repository periodically. It also emphasizes security, speed. It also enables non-linear development by allowing branching and merging.

\subsection{Architecture}
There is a remote repository R, local repository L and a cache/ local index/ changeset C. Finally, there is the working directory W.

The repositories may contain various branches besides the master branch.

\subsubsection{(Un)Ignoring changes}
Add untracked files to .gitignore.

For tracked files:
\begin{verbatim}
git update-index --assume-unchanged file
git update-index --no-assume-unchanged file
git ls-files -v|grep ^[a-z]

git update-index --skip-worktree [FILE]
git update-index --no-skip-worktree [FILE]
git ls-files -v|grep ^S

\end{verbatim}


\subsubsection{Repository commits and logs}
Every change committed to a particular repository is given a unique hash (we call this hashname) Eg: 53da224. One can use this as an identifier to rollback or compare changes.

The sequence of commits is stored in two places: logs and reflogs. Issuing 'rebase' commands, for example, fixes logs so that they reflect the desired sequence of check-points; yet it does not touch the reflogs sequence of commits. Thus the older history is retained. The reflog is periodically cleaned up to concur with logs.

\subsection{Update Commands}
git branch branchName creates a new branch.

C can be updated using add (for both new and modified files) and rm commands. git -A adds all altered files not excluded by .gitignore.

C can be commited to L using commit. L is synced with R using push and fetch.

R is synced with (W, L) using pull. By default the master branch is used, but one can specify a branch as an argument.

W can be created from R using clone. W can be checked out from L using checkout.

\subsubsection{Merges and stashes}
Differences between local and remote branches show up when doing a pull. One cannot push changes to a repository without first pulling and merging locally.

Conflicts between W and R may be resolved by the following action sequence: a] git stash b] git pull c] git stash apply - or in case the stash is disposable, git stash merge.

git pull --rebase pulls remote changes and does surgery on local git history to put your local changes on the top. This is dangerous - you can loose previous local commits. If that happens, one can use git reflog to retrieve the hashname for the change; and then do git checkout (if that hashname has not been garbage-collected).

\subsubsection{Checking differences, tool configuration}
git difftool [-t kdiff3 HEAD\~2]

git config --global merge.tool kdiff3 (creates and) modifies .gitconfig file in the home directory.


\subsection{UI}
git branches may be visualized using a UI.

gitk (optionally supplied a path-name) does this. gitk --all shows all branches. Thence git-gui may be started, which provides a good UI for updates and commiting.

git-gui provides a good ui for examining changes and checking in. One can add tools: for assume-unchanged \$FILENAME, for example.

\tbc

\section{hg / mercurial}
For help, see hg help. Similar to git.

General syntax: svn command arguments.

Commands: clone repo [destination]: checks out a repository to create a 'local' repository.

status/ st.

commit/ ci -m "msg": commits to the local repository. push: commits to the central repository.

\section{Version control websites}
\begin{itemize}
\item sourceforge uses svn.
\item Google projects uses hg, svn and  git.
  \subitem Shows activity level, number of active contributors, lines of code prominently.
\item github hosts git projects
\item bitbucket hosts hg projects.
\end{itemize}

\chapter{Visualization toools}
\section{Graph}
Gephi does the job, and supports a wide variety of input file formats including csv edge list.

\chapter{Integrated Development Environment (IDE)}
\section{Eclipse}
Eclipse allows development in many languages: C/ C++, Python, Java, Scala.

\subsection{Startup configuration}
This can be done in the eclipse.ini file. Heap size can be increased using -Xmx2048m after -vmargs. \chk Is it necessary to set heap size in the run configuration too?

Java library path can be specified by -Djava.library.path=path.

\subsection{Installing and updating components}
Various plugins and the core software itself can be installed and upgraded using the GUI, after first enabling various online repositories using another UI.

\subsection{Views and perspectives}
The UI paradigm itself considered in the software architecture survey. Common views include: File, Console, Resource tree, package explorer, type hierarchy etc.. Common perspectives include Java, Resource etc..

\subsection{Common operations and shortcuts}
Ctrl + Shift + R : search and open a resource fast.

Well known: Format. Go to line. Undo, redo.

\subsection{Functionalities}
Various views provide nifty functionalities. We list some useful ones below.

\subsubsection{Java}
\tbc

\part{General purpose programming}
\chapter{C}
\subsection{Debugging with gdb}
file reads in a source file.

\subsubsection{Setting breakpoints}
To set breakpoint: break mexFunction.

stop in, stop at: Stop in a particular function, stop at a particular line

\subsubsection{Debugging after stopping}
Once the debugger has stopped, can use: step, continue, exit etc.

where, whereami: Show the call stack, show the current location.

print: Display the value of a variable.

\subsubsection{Memory allocation}
Stack memory is often more limited compared to heap space (although both can be increased). int i[4000] uses stack memory - so it should be avoided.

\subsubsection{Segmentation faults}
Segmentation faults happen when the program tries to access out-of-bounds memory.

Dealing with mysterious errors (eg: segmentation faults) : sometimes even stuff printed to STDOUT and STDERR may not be printed properly in a bad crash, leading to ambiguity in pinpointing the source of the error. Commenting out code may then reveal it.

\chapter{Simplified wrapper, interface generator (SWIG)}
\section{Purpose and mechanism}
To use C or C++ code with python or java, swig is used.

One first writes an interface file describin the signature of the functions called, and of the objects passed.

Then, one runs a special program to automatically generate wrapper code (modules or classes) in the higher level language which provides functions and classes with similar signature, and which act as a bridge to the C code.

\subsection{Persistence and memory}
C++ objects persist between successive calls from the other language only if the code in the other language retains references to these objects.

Rather than keeping track of multiple such objects, it is simpler to keep track of a container object. So, the C code is often encapsulated within a class.


\tbc

\chapter{Python}
\subsubsection{Syntax conciseness}
Clean and concise syntax: compare to perl; so more readable. As a general rule, python tries to use the same operators and function names for similar operations over different types of operands: Eg: +, len, dir.

\subsection{Speed}
An interpreted language. Speed is many times slower than C or Java. For vector operations, speed comparable with Matlab; but for looping, may be faster.

\subsection{Linting}
Python is an interpreted language - so many silly errors which would have been caught by a compiler are noticed only at run-time, if at all that code is run. So, Linting (and testings) is especially important.

\subsubsection{Pylint commands}
In code, one can say:
\# pylint: disable-msg=W0613

\section{Writing, Building and executing code}
Make .py files; begin with \verb'#!/usr/bin/python'.

\subsection{Important env variables}
PYTHONHOME: location of the std libraries.

PYTHONPATH: default search path for modules/ package libraries, may refer to zipfiles containing pure Python modules (in either source or compiled form).

PYTHONSTARTUP.

\subsection{Good IDE's}
Eclipse with pydev. idle.

\subsection{Point of entry}
Can use interpreter. Or first line in file.py.

"python -c command [arg] ...", "python -m module [arg] ...", which executes the source file for module

\subsubsection{Arguments}
sys.argv, a list of strings has the script name and additional arguments from shell; an empty string if no argument is given.

\subsection{Installing External libraries}
Place a link in the site-packages directory.

Or run python setup.py build, python setup.py install.

Or do: sudo pip install pkgName or easy\_install pkgName.

\section{Help}
help(object/ module)

\section{Variables and data types}
\subsubsection{Matlab format}
import scipy.io. \\
X = scipy.io.loadmat('mydata.mat'), scipy.io.savemat().

\section{Interfacing with other languages, the OS}
\subsection{Numeric programming}
numpy.

\subsection{RPy or RPy2 for R}
No easy plotting in sage.\\
 from rpy import *. r.library ("..",  \verb'lib_loc =os.path.join(lib_path, "R"))'.

\subsection{With Matlab}
In sage use 'matlab.eval()'.

\section{Other libraries}
Regular expression: re.search(r'asdf', text) returns the first match object . m.group() returns the matching text.

re.findall returns a list of match objects. If regular expression contains pattern groups, tuple-list is returned.

\chapter{Java}
\section{Classes}
\subsection{Important Superclasses, interfaces}
\subsubsection{Serializable}
Implementing Serializable ensures that an object instance can be stored in a file and retrieved later.

\subsubsection{Clonable}
Implementing the Cloneable interface ensures that the traits of a given object can be copied using the copy() definition. ArrayList internally uses an array, but if copy were implemented such that only the reference to the internally used arraylist were copied, then both copies would be affected by any change in the arraylist. Hence, it is important to override copy() as appropriate.

\section{IO}
Contained in the package java.io.

\subsection{Stream}
IO is handled using streams; which usually need to be opened and closed explicitly for efficiency.

\subsubsection{OutputStream}
Output streams can output bytes or readable text. Commonly used is PrintStream and PrintWriter (For writing characters rather than bytes); which include : println and print.

\subsubsection{Output to files}
Initializing PrintStream or PrintWriter to write to a file can be painful because one needs to first initialize a FileOutputStream(fileName).

\subsubsection{Standard output}
System.in and System.out are always open.

\subsection{File listing}
java.io.File has useful functions: exists(fileName), list() to list contents in case it is a directory.

\subsection{Read configuration files}
Reading parameters from properties files, which store (key, value) pairs in lines with the format key= value is a very common task. So, there are specialized functions to deal with them: props = new java.util.Properties();  props.load(fileStream);.


\section{CGI programming}
\subsection{Writing servlets}
One must use the servlet API (preferably provided by the webserver being used) and create classes which extend HttpServlet, in which doGet and/or doPost methods are overriden.

\subsubsection{Reading input}
To decode HTML escaped characters, one can use the apache.commons.text package.

getParam, getParamMap.


\subsubsection{Output}
outStream object will have methods like setContentType, setEncoding, println.

\subsection{Mapping to URL}
One then maps a url to a servlet in the WEB-INF.xml file, using the \verb'<servlet>' and other tags.

\chapter{Lua}
\section{Paradigms and extensibility}
Multi-paradigm language. Functions can be treated as values, higher-order functions exist.

Highly extensible.

\subsection{Building}
Compiled to bytecode, executed.

\subsection{Metatable}
This can be used to effect inheritence.
\begin{verbatim}
 fibs = { 1, 1 }
setmetatable(fibs, {
  __index = function(name, n)
  -- Call this function if fibs[n] does not exist.
    name[n] = name[n - 1] + name[n - 2]
    return name[n]
  end
})
\end{verbatim}


\chapter{Perl}
\section{Distinctive features}
This is an interpreted language. It is loosely typed.

It is very good at text processing.

\subsection{Richness of syntax}
It is very rich, so that there are multiple ways to accomplish the same thing; to the point where code written by one programmer may be unintelligible to another.

\section{Running, building}
\subsection{Running}
See perlrun on the internet. Commands to be interpreted are either entered in a special shell, or is passed in a file to the interpreter, or is passed in the command line.

\subsection{Getting help}
Use perldoc, perlsyn. Inbuilt function reference: perlfunc.

\part{Mathematical programming}
\chapter{Fundamental concerns}
When doing mathematical computations - especially while using the floating point hardware representation, concerns dealing with overflow and underflow arise. These problems and various methods of dealing with them are described elsewhere.

\chapter{Special libraries}
\section{Linear algebra libraries}
\subsection{Tuning to fit memory, processor}
Matrix factorizations, matrix vector multiplications etc.. are tuned specially to the hardware, in order to achieve maximal speed-up. For example, they may break matrices up to fit the cache and combine the results.

\subsection{Dense linear algebra}
BLAS, LAPACK: fortran routines. clapack provides c interface to some lapack routines. atlas implements blas and some of lapack.

\subsection{Sparse linear algebra}
UMFPACK, SPOOLES. The standards for these computations is yet to be finalized, and it is an active area of engineering.

\chapter{Matlab/ Octave}
\section{Introduction}
\subsection{Distinctive features of the language}
Excellent plotting facilities; many libraries; interactive sessions good for exploring features of data; easy I/O: don't need to spend as much time in getting I/O right as in actual computation; excellent debugging facilities.

Octave is an open source clone of matlab.

\subsubsection{Purpose of design}
Matrix computations. This is meant to be a rapid prototyping language : It is not meant for fast implementations.

\subsubsection{Versions}
Certain OOP methods, properties don't work with 2007 version.

\subsection{Speed}
Matrix/ vector multiplications are implemented by calling efficient linear algebra implementations like LAPACK etc.. So, operations which can be specified using matrices/ vectors should never be specified using loops.

Also, certain functions like repmat, ones, sub2ind etc.. could be faster. Use packages like lightspeed which implement faster versions of these oft used functions.

\subsubsection{Speeding up loops}
When unavoidable, loops are known to be slow, even when efficient linear algebra routines are used internally. This is because of costs associated with interpreting each line of code in the loop at runtime.

\paragraph{Preallocate memory}
You can speed up loops by preallocating the memory used therein - that way matlab will not have to repeatedly allocate a new block of memory.

\paragraph{Use C}
Implementing the logic in C not only removes the overhead due to MATLAB interpretation; it also generally forces the programmer to be conscious of and avoid memory allocation overheads.

However, there is an additional cost involved in packaging arguments for use by the C function: So it is best not to call the same C function in a loop - rather it is better to code even the loop in C.

\paragraph{Make function logic inline}
This can avoid repeated memory allocation and overhead due to the function calling infrastructure and increase speed. BUT for unknown reasons, the reverse can be true sometimes!

\subsection{Using External libraries}
\subsubsection{Locating libraries}
Go to \url{http://www.mathworks.com/matlabcentral/}, \\ search, look under appropriate tags.

\subsubsection{Installing and using}
unzip package, add it to path: maybe can use addpath(genpath(dir)) to add subdirectories.


\subsection{Building and executing code}
Can write directly in matlab command window.

Or can write scripts (a sequence of commands in a file) or functions; Add directory to path; Use matlab to invoke.

\subsubsection{Useful options}
No GUI: matlab -nodesktop.

Command history: diary on. Others: addpath, savepath.

\subsubsection{Good IDE's}
Matlab.

\subsection{Debugging}
Can view matrices in tables, as an image etc..

Introducing break points: Use GUI or 'keyboard' can be inserted in the midst of any function. Then use dbquit or dbcont or dbstep or dbstack.

\subsubsection{Profiling code for bottlenecks}
profile on
myfun;
profile report

\subsection{Help}
help commandName.

For library documentation, see its docs.

\section{Constants and data types}
\subsection{Data types of constants}
Everything is an object. The most common object is an n-dimensional array.

boolean numbers are represented by 0 and 1.

str = 'stringValue': a character array.

\subsection{Important constants}
Inf, NaN.

isNaN(a).

\subsection{Arrays}
A = zeros(2,3,2,3) makes a 0 tensor of those dimensions.

\subsubsection{Matrix}
\begin{verbatim}
A = [2 1;
     1 8]
\end{verbatim}
; separates columns.

Block matrices: B = [A  A+32; A+48  A+16].

\subsubsection{Data generation functions}
$-14:0.2:14$: Produces vector of equispaced numbers in the interval.

\paragraph*{Basic constant arrays}
zeros, ones, trues, falses.

eye, speye.

\paragraph*{Random array generation}
randn, random, unifrnd.

\paragraph*{Sparse array generation}
spconvert(mat) returns sparse matrix with indices/ values specified in mat's columns. sparse().

\paragraph*{Generate random permutations}
randperm(4) may return [2 3 1 4].

\subsubsection{Array indexing}
Array indexing is used to access a subarray in an array.

\paragraph*{Indexing with scalars}
Array index starts at (1,1, ..). Shortcut for last element along any dimension: end. Eg: A(end, 5).

\paragraph*{Indexing with index-vectors}
Entire row or col of an array can be selected using : symbol: A(:,3).

\paragraph*{Indexing with one scalar}
Aka linear indexing

A(5) or A(v) are also meaning-ful, especially straight-forward for vector A.

\paragraph*{Indexing with logical vector}
Eg: $A(A>12)$.

\paragraph*{Indexing tricks}
Indexing upper triangular portion of a matrix: create indices using triu(ones(m, n)).

\subsection{Cell array}
A = cell(m, n); creates a cell array of pointers (cells), usually pointing to matrices.

\subsubsection{Cell elements: indexing, altering}
A\{i, j\}: returns the contents of the appropriate cell in the array, whereas A(i, j) returns a cell (pointer).

Delete element from cell array: $cell(i) = []$.


\subsubsection{Cell values}
\{23 34\} creates a 1*2 cell array.

\subsection{Data-type conversion}
Use logical(), num2str(number, precision).

\subsection{Structures}
\subsubsection{struct}
s = struct('field1', values1, 'field2', values2, ...). isstruct.

isfield(s, 'fieldx'). rmfield.

get and put are abbreviated by syntax like s.field1.

\paragraph*{Dimensions/ size of struct}
The size of the resulting structure is the same size as the value cell arrays or 1-by-1 if none of the values is a cell.

\subsubsection{Use Java objects}
\begin{verbatim}
mtype = java.util.Hashtable;
mtype.put ( 'numNodes', numNodes);
s = mtype.get('numNodes');
\end{verbatim}

Note: Java heap space runs out of memory much more easily.

\paragraph*{Checking emptyness}
mtype.get('asdf') returns [].

\subsection{Function handles}
Suppose fn f defined already. Then can use g = @f, can pass it to other fns just like variables.

\subsubsection{Reparametrization}
Can reparametrize functions while you are at it. Eg: If poly(x, b, c) is defined and b, c are specified, can use @(x)poly(x, b, c) to get a handle to a function with parameter x.

\subsubsection{Anonymous functions}
Can also create anonymous fn using an expression; Eg: $f = @(x)x^3+x$.

\subsubsection{Clearing memory}
clear varName;

\subsection{Objects}
\subsubsection{Object creation}
c= ClassName(..)

\subsubsection{Access/ manipulate objects}
Use object.member.

\paragraph*{Introspection methods}
properties(obj) or fieldnames(obj) \\
lists public properties; methods(obj) lists public methods. class(obj).


\section{Memory, variable management}
List variables in use: whos.

\subsection{Clearing, recompiling}
clear classes; rehash pathreset;. Need to do this if you alter class signature.

\section{Data manipulation}
\subsection{Operators}
\subsubsection{Unary}
transpose\', minus -.

\subsubsection{Boolean valued operators}
\begin{verbatim}
 <  >  <=  >=  == ~=
\end{verbatim}

\subsubsection{Boolean operations on matrices}
Boolean ops also work with matrices to make boolean matrices.

all(A) checks if $\forall i,j A_{i,j} \neq 0$.

\subsubsection{Arithmatic operators}
* + -

\subsubsection{Multi-ary}
Range specifiers: 1:5 1:2:5.

\subsubsection{Component-wise operations}
\begin{verbatim}
y = sin(x)./x;
\end{verbatim}

\subsubsection{Matrix operators}
\verb A/B : roughly $AB^{-1}$.

\verb A\B : roughly $A^{-1}B$.

\subsection{Operation with constant arrays: shortcuts}
In A + 1: 1 is evaluated as a matrix of 1's.

\subsection{Array processing functions}
\subsubsection{Array statistics}
size, numel.

sum (along any dimension: 1 for sum along rows/ sum of columns, 2 for sum along columns/ row sum etc..).

max, min.

nz, norm, unique.

find: find non-0 elements of matrix.

\paragraph*{Checking emptiness}
use isEmpty().

\subsubsection{Reshape array}
Delete row: A(2, :) = [].

If you store a value in an element outside of the array, the size increases to accommodate the newcomer.

Alter array shape: reshape.

\subsubsection{Alter matrix}
normr/c, tril, triu.

\subsection{Vector processing functions}
Vector statistics: histc().

dot(a, b) gets the dot product of a and b without having to worry about whether a and b are both column vectors in doing a'b.

\subsubsection{String manipulation}
c=['sdf' 'asdf'] or strcat() concatenates.

[firstToken reminder] = strtok(string, delimiters).

strrep : string replacement.

\subsection{Matrix functionals}
\subsubsection{svd and eig}
svd, svds (get only a few $\sw_i$), eig, eigs.

eig and eigs can return vectors differing in signs. \why

\paragraph*{Low rank approximations of symmetric A}
If A is symmetric, eigs is faster than doing svds, as it probably constructs $B = \mat{0 & A\\ A^{T} & 0}$ to compute svd. Can pass options which specify the structure of A (eg: symmetry), so that ew decomposition can be computed faster. Actually, $\abs(\ew) = \sw$, but if ew's are close together, $\ew$ and $\sw$ returned by eigs and svds can be very different --- but the resulting low rank approximations are both equally good.

\section{Code structure}
\subsection{Sentence syntax}
Sentence ends with ';' or new line: if former, execution is silent. Can continue writing in next sentence with '...'.

\subsubsection{Commenting}
\% comment.

\subsection{Functions and commands}
\subsubsection{Commands}
Commands usually don't take vlaues/ variables, assume everything is a string etc..

Functions can be nested.

\subsubsection{Functions without side-effects}
Matlab functions are almost functions in the mathmatical sense: accept values and variables as arguments, return value. If they modify an object passed in the argument, they *must* return the object.

\subsubsection{Syntax of function defn}
\begin{verbatim}
function outputValue = fnName(arg1, arg2)
%  fnName(x,y)  function documentation
end

\end{verbatim}

arg1 could be a function handle: you can then use arg1(..) to invoke the corresponding function.

To return multiple values, use function [a, b] = fnName(..).

To invoke functions from outside the file, they should be public: they should have same name as the file. Else, ye have a private function. Can't declare a fn inside a script.

\subsubsection{Anonymous / temporary functions}
\begin{lstlisting}
f = inline('sqrt(x.^2+y.^2)','x','y');
f = @(x, y)sqrt(x.^2+y.^2);
\end{lstlisting}

Reparametrization is another way of defining temporary functions: this is described under the function handles section.

\subsubsection{Nested functions}
\paragraph{Definition}
Functions can be nested - by defining them as usual within a function body. But they cannot be nested within control structures.

\paragraph{Variable scope}
They can access and modify variables and utilize functions, if they are defined within functions defined at a higher level.

Variables defined within a nested function are not available outside.

\paragraph{Restrictions on modifying runtime workspace}
While debugging, if you break execution in a function in which a nested function is defined, you cannot modify the workspace by defining new variables. But you can define 'global' variables to get around this.

\subsubsection{Checking input and output structure}
Behaviour of a function can be made to vary based on the number of output variables expected by the caller, and the number of input variables required.

\paragraph*{Variable inputs}
Use nargin to check number of arguments passed. Or, you can use structures to pass arguments.

\paragraph*{Variable outputs}
Use nargout to check the number of output arguments expected. Then, add variable number of outputs like this: varargout(1) = {op1}; varargout(2) = {op2}; .

\subsubsection{Invocation}
\paragraph*{Invocation of functions}
output = functionName(arg1, arg2);\\
\verb'[m, n] = size(X)'. Strings must be enclosed with ''.

Usually, reference is passed. Only the field being modified by the function will be passed 'by value'. If modified structure is returned as output, only changed fields will be altered upon function-return.

\subparagraph*{Asking functions for fewer outputs}
A function can return variable number (upto n) of outputs. But, when you invoke the function asking for k outputs, you cannot predict whether you get the first k of the n outputs or some other set - that is up to the function definition. Eg: qr().

\subparagraph*{To ignore return value}
\verb'[~, n] = size(X)'.


\paragraph*{Invocation of commands}
commandName arg1 arg2 . Arguments assumed to be strings: even if not enclosed by ''.

\paragraph*{Invocation of methods}
Call superclass method: methodName@super(obj);

\paragraph*{Invocation over multiple elements}
cellfun(fun, cell), arrfun(fun, array).

\subsection{Define classes, packages}
Base classes: value, handle.

\begin{verbatim}
classdef ClassName < package.superClass1 & superClass2
 properties (Access = ..)
  ...
 end
 methods
  % Constructor
  function s = ClassName(..)
   s = s@package.superClass1(...)
   s = s@superClass2(...)
  end
  function delete(objRef)
  ...
  end
 end
 events
  EventName
 end
end
\end{verbatim}

In all non-static methods, the first arg is the object itself. object.fn(a, b) $\equiv$ fn(object, a, b).

No overloading functions within the same class.

If function modifies the object, it must return the object, like any fn in the mathematical sense!

\subsubsection{Modifiers}
Property modifiers: setAccess, getAccess = protected/ private/ public, Constant (not available in 2007 version), Static. Method modifiers: Access, Abstract = true, Static.

Property access methods: called when ye get or set property, before setting/ getting:
\begin{verbatim}
function value = get.PropertyName(object)
...
end
\end{verbatim}

\subsubsection{Directory structure}
Under a directory in the path, +packageName dir, @ClassName dir or ClassName.m file. If a @ClassName folder is used, then ClassName.m containing the class definition should be placed within that folder.

\paragraph*{mex files}
Packages can contain mex files: \\
invocation: packageName.functionName.

\subsubsection{Ceveat about 2007 version}
The modifier syntax (Static) can't be used in 2007 version: must use (Static = true): does not even give error message.

Bad error messages.

\subsubsection{Check errors in definition}
Try to construct an object or Use fieldnames on an object, or invoke a test method.

\subsubsection{Specifying events}
Make a subclass of handle. Broadcast event using notify(obj,'EventName').

Listen for events: addlistener(eventObj,'EventName',@myCallback).

\subsection{Branching structures}
\begin{verbatim}
if n == 0
    T = t0;
elseif n == 1;
    T = t1;
else
    T = 3;
end

switch x
case {1,2}
disp('Probability = 20%');
case {3,4,5}
disp('Probability = 30%');
otherwise
disp('Probability = 50%');
end
\end{verbatim}

\subsection{Loop structures}
\begin{verbatim}
for k= (Any *row* vector like 2:n)
  T = [2*t1 0] - [0 0 t0];
  t0 = t1;
  t1 = T;
end

while booleanCondition
  statements;
end
\end{verbatim}

\subsubsection{break and continue}
Can break out of a loop or skip an iteration.

\subsubsection{Avoiding loops in place of matrix ops}
For efficient implementation, whenever matlab does large matrix operations, matlab uses PROPAC and other specially designed libraries to do things efficiently. So, if matlab has a matrix operation, always use that rather than writing your own loop.

\subsection{Error handling}
\subsubsection{Throwing errors}
\begin{verbatim}
ME = MException('component:mnemonic', ...
       'message');
throw(ME)
\end{verbatim}


\subsubsection{Throwing warnings}
Use warning('msg') or warndlg(). These are displayed when you do warning('on').

\subsubsection{Assertions about input}
assert(condition, errorMessage) generates an exception with an error message if a condition is violated. This is useful.


\section{Visualization}
\subsection{Creating a figure}
figureHandle = figure() creates a figure.

Can later save this: saveas (figureHandle, name With Extension).

gcf returns handle to the current figure.

figureHandle = hgload(figure path) or open() loads a figure.

\paragraph*{Set properties}
colormap gray;


\subsection{Plotting data}
The plotting functions accept suitable data, figure properties, return a figure handle.

subplot(m,n,p) breaks up the figure into m*n table, selects the pth cell for the current plot.

hist(vector, nbins) without output arguments. bar(vector).

plot (xAxis, yAxis, LineSpec, 'PropertyName', value, ...);

semilogy plots y axis in log scale.

\subsubsection{Line specifications}
It is composed of the following, in that order. Eg: '-+r'

Line styles: -, --, -., : .

Markers: + o * . x s d p h \^ v .

Colors: r g b c m y k g.


\subsection{Plot properties}
xlabel, ylabel, title.

grid on/ off/ minor.

\subsection{Visualize matrix}
\subsection{Visualize matrices}
spy a sparse matrix, for use with full matrices (aka non-sparse) use $>0.0001$. imagesc(A), image(A).

\section{Prallel programming}
\subsection{Multi-threaded programming}
Automatically enabled after 2008. Can be enabled with maxNumCompThreads = 2. Makes large matrix operations faster.

\subsection{With matlab toolbox}
Tightly integrated with many other mathworks toolboxes.

\subsubsection{Processor pool}
Worker processes can be enable on a cluster or on the local multicore machine : controlled by a distributed computing server.

Use: matlabpool open/ close/ size.

\subsubsection{parfor loop}
Operates like for loop, but there may be no variable dependency between loops - otherwise the program will fail to compile.

\subsubsection{Other tools}
Look in matlabcentral.

\section{System IO}
Use: [returnValue, commandOutput] = system(commandName);

To get environment variables: env(variableName).

\section{Other library functions}
Very rich in functions for matrix operations. Try to find pre-written functions before writing code from scratch.

\subsection{Useful external toolboxes/ packages}
software from Kevin Murphy and collaborators. Minka's lightspeed.

\subsection{Solving Matrix problems}
\verb A\b  solves system of equations Ax=b. lsqnonneg: Non negative least squares.

Fitting polynomials to x and y: polyfit.

\subsection{File I/O}
\subsubsection{Log files}
Can use file = fopen(fileName, mode); Mode can be w for overwrite, a for append ...
fprintf(file, 'asdf');
fclose(file).

\subsubsection{Loading formatted data}
\begin{verbatim}
load someFile.mat;
save data.mat A1;
\end{verbatim}
load can read ascii files with two space separated columns and make sense of it.


\subsubsection{C like functions}
fprintf, \verb'fscanf(fid,"%d ...")', fopen(fid, 'w'), fclose.

\subsection{User interaction I/O}
To print stuff: disp(). pause().

format long : set extra accuracy.

\subsection{Solve general optimization problem}
Described among optimization software.

\subsection{Algebra}
\begin{verbatim}
symadd('x^2 + 2*x + 3','3*x+5')
\end{verbatim}

\subsection{Measure time}
tic toc.

\section{mex: Using C functions}
Write c files for efficiency.

\subsection{Interface}
\begin{verbatim}
#include "mex.h"
void mexFunction(int numOutputArrays, mxArray *pOutput[],
int numInputArrays, const mxArray *pInput[])
 ...
\end{verbatim}

Compile them in matlab prompt: mex fileName.

\subsubsection{Debugging}
See tips online.

\subsection{mxarray datatype}
This includes the array's type, matlab variable name, dimensions. Non-sparse arrays contain pr and pi to store the real and imaginary parts of the data: these are one dimensional arrays.

\subsubsection{Creation}
Use functions like: mxCreateNumericArray, mxCreateCellArray, mxCreateCharArray.

\begin{verbatim}
 mxArray *myarray = mxCreateNumericArray(..);
 mxCreateDoubleMatrix(1,3,mxREAL);
\end{verbatim}

\subsubsection{Array access}
mxGetPr, mxGetPi, mxGetData, mxGetCell.

mxGetM, mxGetN: get matrix dimensions.

\subsubsection{Array modification}
mxSetPr, mxSetPi, mxSetData, mxSetField. Can use memcpy to copy arrays at one shot.

\subsubsection{Memory management}
mxMalloc, mxCalloc, mxFree, mexMakeMemoryPersistent, mexAtExit, mxDestroyArray, memcpy

\subsection{Executing matlab commands}
Both mexCallMATLAB and mexEvalString execute MATLAB commands. Use mexCallMATLAB for returning results (left-hand side arguments) back to the MEX-file. The mexEvalString function cannot return values to the MEX-file.

\subsection{User IO}
mexPrintf, mexWarnMsgTxt, mexErrMsgTxt.

\subsection{Input validation}
mxIsChar, mxIsClass(prhs[0], "sparse"), mxIsComplex.

\section{Java}
\subsection{Advantages}
Using loop operations can be 2x faster in java. For matrix operations, the fortran routines used by matlab are faster; but using java libraries like colt is said to be not too much worse.

Also, Arrays and values are automatically converted between matlab and java formats.


\subsection{Classpath}
javaclasspath shows static and dynamic portions of class path. The former is loaded from [matlabroot \verb '\toolbox\local\classpath.txt' ]
. The latter is set using javaclasspath or javaaddpath. The class path is refreshed using [clear java].

\chapter{Optimization software}
\section{cvx}
\subsection{Distinctive features}
\subsubsection{Disciplined convex programming framework}
Makes specification easy. Abstracts away the mathematical and methodological details (like what underlying solver to call?).

Tries to mimic the way people formulate convex programs: drawing from a library of convex functions, combining them in various valid ways to preserve convexity. Similarly, it has a rich, expandable library of functions with known properties, plus it specifies valid ways of composing and combining them.

Good way to check convexity of the problem!

\subsubsection{Defects}
A generic solver. Not suitable for large problems.

\subsubsection{Other features}
It is closely integrated with matlab.

Possible to specify the underlying solver used.

\subsection{Help}
See cvx manual on website, many examples, quickstart file.

\subsection{Specification structure}
\begin{lstlisting}
cvx_begin
    variable p(numParams);
    minimize norm(A*p - ones(numRows,1), Inf);
    [[or minimize(a + b), not minimize a + b]]
    subject to
    p >= 0;
    p <= 1;
cvx_end

Other modes:
cvx_begin sdp : interprets matrix inequalities as conic.
cvx_begin gp
\end{lstlisting}

All constraints and optimization function for minimization should be convex - else, cvx will reject the problem and try to provide a helpful message trying to explain the reason for the rejection.

\subsubsection{Matrix variables and constraints}
Multiple constants/ variables are specified using matrices.

\begin{verbatim}
Matrix variables:
variable X(n, n) symmetric;

Semidefiniteness constraints:
X == semidefinite(n);
\end{verbatim}

\subsection{Error reporting and dimensions}
To stop verbosity: \verb cvx_quiet(true) .

\subsubsection{Scalars and 1*1 matrix}
If you try to add a scalar and a 1*1 matrix lke $a^{T}b$, cvx fails with a cryptic error message.


\paragraph*{Assuming dimensions of variables automatically}
If you declare variable b intending for it to be a scalar; but then try to add it to a vector in the constraints, cvx does not report an error.

\subsection{Solution, program variable}
\subsubsection{Datatype of the program variable}
The program variable is different from matlab variables. While the optimization is underway, p is a cvx object. After optimization is done, p is the optimal solution - just a normal matlab variable.

\subsubsection{Solution variables}
opt\_val, tolerance etc..


\section{Matlab optimization toolbox}
Built into matlab.

\subsection{Unconstrained minimization}
\subsubsection{fminsearch}
Solves unconstrained optimization problem, where objective is assumed to be continuous. fminsearch can often handle discontinuity, particularly if it does not occur near the solution.

fminsearch uses the simplex search method. This is a direct search method that does not use numerical or analytic gradients as in fminunc.

\subsubsection{fminunc}
fminunc does the same thing, except it cannot handle discontinuities. \chk

\subsection{fmincon}
Finds a local minimum of a constrained nonlinear multivariable continuous function. Solves $\min f(x): c(x) \leq 0; c_{nl}(x) \leq 0; Ax \leq b; A'x = b'; l \leq x \leq u$.

It uses finite difference approximation of the gradient when the gradient is not available.

\begin{lstlisting}
[x,fval,exitflag,output,lambda,grad,hessian] = fmincon(fun,x0,A,b,Aeq,beq,lb,ub,nonlcon,options,P1,P2, ...)
\end{lstlisting}

Parameters to ignore can be specified by passing [].

\subsection{Other functions}
fminbnd : minimizes $f:R \to R$.

\subsection{Options}
A structure with options.

\begin{lstlisting}
options = optimset('Display','iter','TolFun',1e-2);
options = optimset(options,'TolX',1e-2);
\end{lstlisting}


\chapter{R}
\section{Introduction and use}
\subsection{When to use}
Statistics is the emphasis, not matrix manipulation. It is an expression language.

\subsection{When not to use}
Text processing and general purpose programming are painful.

\subsection{Writing, Building and executing code}
\subsubsection{Script execution}
source("/path/file.R").

\subsubsection{Reloading changed code}
library("R.utils");

sourceDirectory("work", modifiedOnly=TRUE, \\pattern="[A-Z]*[.]R\$", recursive=FALSE);

\subsubsection{Important environment variables}
\verb R_LIBS : the place where R libraries are installed.

\subsubsection{Working environment}
getwd(), setwd(dir)

options(): Example options(digits=3)

\subsubsection{Command history}
history(), savehistory, loadhistory.

Can use Ctrl+R as  in BASH.

\subsubsection{Good IDE's}
\paragraph*{Native GUI}
Graphical data entry: data.entry(x), edit(x).

\subsection{Listing and memory}
objects() lists objects.

rm(objA, objB) removes certain objects from memory. rm(list=ls()) removes all objects.


exists() checks for the existance of an object.

\subsection{Development environment}
RStudio is good. Interactivity with RapidMiner is useful.

\subsection{Debugging}
The following can be inserted in the midst of code:

browser(): breaks execution and allows debugging with arbitrary code: like keyboard in matlab. Use cont to continue the program. Q halts execution.

debug(): marks function for debugging.

trace() function modifies a function to allow debug code to be temporarily inserted.

setBreakpoint("fun.R\#20")

traceback() is useful when error is encountered.

\subsection{Help}
help(command), ?command, \\
apropos("log"), help.search("log"), example(commandName).

\subsection{libraries}
\subsubsection{Seeking}
help.search(fnName);

\subsubsection{Using}

\subsubsection{Installing}
Many packages are listed in cran internet repository.

chooseCRANmirror()

install.packages("igraph"). For usage, see the packages subsection.

\section{Data: types, values, variables}
\subsection{Names and namespace}
\subsubsection{Valid names}
a.b and a\_b are valid name; but by doing so, you are not creating a structure named a. A name can start with '.', but if so it cannot be followed by a digit.

\subsubsection{Namespace}
Uses: avoid name conflicts, structure code well.

To call a function b in a namespace a, use: a::b(). If b is hidden, then use a:::b().

\subsubsection{Accessing list members directly}
Use attach(Lst). detach() reverses this.

\subsection{Object}
Everything is actually an object with attributes.

\subsubsection{Accessing attributes}
attrName(objName)

\subsubsection{Setting attributes}
attr(object, attribute) is used to get or set an attribute.

Or one can use structure(object, attr=value).

\subsubsection{Important attributes}
mode: details about the type of data contained.

length.

\subsection{Modes and type}
Mode: details about the type of data contained. This is distinct from the type of the object itself (data.frame, or vector ..). So, type is akin to a generic/ template/ meta class in Java made concrete by specification of mode arguments corresponding to contents.

\subsubsection{Basic modes}
Boolean. Numeric. Integer.

\subsection{Scalar values and operations}
\subsubsection{Special values}
TRUE, FALSE. Inf. NA: 'not available' or missing values. NaN is also a special case of NA.

These can be checked using is.na() and isNaN functions.

\subsection{Data conversion}
as.array(), as.data.frame etc..

Use methods(as) for a list of such methods.

\subsection{Vectors}
\subsubsection{The basic math object}
A scalar is actually a vector with one element in R. An array is also internally a vector. A string is a character array.

\subsubsection{Homegeneity}
Vectors are homogenous: their 'mode' attribute is character, numeric, logical.

\subsubsection{Named entries}
names(x) = stringList is a way of giving names to indices.

\subsubsection{Indexing}
A vector can be indexed with a vector of a] logical elements, b] positive integers, c] negative integers.

Eg: x[1:10] picks 1st 10 elements. x[-2]: all but x[2].

x['abc'] is valid if entries are named.

tail(x, k) picks the k last elements.

To append to a vector: append(v, val)

\subsubsection{Creating vectors}
c(1,2) : Using the concatenation function.

seq(); This can be abbreviated using the : operator.

rep(vector, timesToRepeat) is the replicator function.

vector() creates an empty vector.

\subsubsection{Vector statistics functions}
length, max, min, sum, mean, median, cummax, cummin, cumsum, cumprod, range, prod. cor: correlation.

which: gather 'TRUE' values from boolean vector.

\subsubsection{Apply scalar functions}
lapply returns a list. sapply, a simpler wrapper around lapply, returns a vector by default. vapply is an

apply returns an array or a vector but acts on arrays.

\subsection{Factor}
Factors are ways of storing a label-vector. They can be created using factor(labelColumn).

An important attribute is levels, which contains the set of labels used.

\subsection{Strings}
These are character vectors. Enclosed in "" for brevity.

\subsubsection{String manipulation}
concatenation: paste(vector). substr gets a substring.

strsplit(v, sep) returns a list, having split each element in v.

\subsection{Dates}
\subsubsection{String connection}
\verb'd<-as.Date("1995/12/01",format="%Y/%m/%d")' converts string to date.

format(dt) converts date to string.

\subsubsection{Arithmetic operations}
Then, one can add days with d+20.

\subsection{Arrays/ matrices}
\subsubsection{As special vectors}
Arrays are just vectors which support multiple subscript indexing. So all operations and restrictions that apply to vectors apply to arrays.

Arrays are stored column by column.

The dimensionality is stored as a vector in the dim attribute.

\subsubsection{Indexing}
A[a, ...]. Any subscript can be replaced with the sort of vector used to index vectors. If a subscript is omitted all values in that dimension are chosen.

Eg: A[,c(3:5)]: picks 3 cols.

\subsubsection{Data creation}
array(vector, dimensions). Or just set the dim attribute after creating a vector.

Or concatenate various vectors or arrays: c(v1, v2)

\paragraph{Matrices}
x = matrix(vector, nrow = 3).
matrix(0,nrow=n,ncol=n)

Row or column binding functions: rbind(vec1, vec2); cbind.

\subsubsection{Dimension-wise oeprations}
apply applies functions to margins of an array; apply(x,1,max): gives row max.

sweep(m, 2, colSums(m), FUN="/")

\subsubsection{Matrix functions}
t(A): transpose.\\
\%*\%, \%\^\%: matrix mult and exp.

diag(A) extracts diagonal, creates a diagonal matrix depending on the argument.

sum, rowSums, colSums, rowMeans, colMeans. cor: correlation.

\paragraph{Linear algebra}
eigen, svd, qr. solve(A, b). lsfit(A, b).

\subsection{Lists}
Lists are heterogenous. They are a combination of a Hashmap and a list. They are very convenient to use as structures.

\subsubsection{List creation}
$L <-list(1, a, b)$ or $L <-list(a1= 1, a2 = a, xyz = b)$ for named lists.

\subsubsection{Accessing list items}
L[[4]], L[["fieldName"]], L\$fieldName. While indexing, list names can be abbreviated: Eg: 'cov' instead of 'covar', as long as the interpreter is still able to uniquely identify the intended member.

TO see if a member exists, use: fieldName \%in\% colnames(Lst).

\subsubsection{Concatenation}
c(L1, L2) returns a single list with members from both.

\subsection{Data frame}
Data frames are the R concept for data tables or matrices which can consist of columns of mixed types which can also have a name.

\subsubsection{Creation}
They are often read in from files - using read.csv for example.

By concatenating vectors: df = data.frame(n, s, b).

\subsubsection{Conversion}
Or from a matrix: data.frame(A). It can be reconverted to a matrix using data.matrix.

\subsubsection{Indexing}
Indexing is done as in the case of two dimensional arrays. If column headers exist, they can be used for indexing.

\begin{verbatim}
drops <- c("x","z")
DF[,!(names(DF) %in% drops)]
DF$colName
\end{verbatim}


\subsubsection{Searching}
\verb which(sbux.df$Date == "3/1/1994")

\subsubsection{Concatenation}
As in the case of matrices

\subsection{Model Formulae}
\~ operator is used to separate left and right sides of formulae.

Syntax: response \~ predictor variables (separated by +).

\subsection{Functions}
Functions are actually objects.

\subsubsection{Definition}
\begin{verbatim}
my.mean <- function(x1 = defaultValue, y1) codeBlock
\end{verbatim}

The value returned is the value evaluated by the last expression in the function definition. Multiple return values are usually handled using lists.

\subsubsection{Variables used}
Usually variables have a local scope: they cannot be accessed outside a function.

They can use variables from the parent scope, say g. The value of g is bound from the parent scope. If a normal assignment is made to g: $g <- 0$, g is then taken to refer a local variable. If g is not bound to a value, but is required in the function definition, there is an error; as it is not bound either in the definition or by an argument. However, if $g <<- 0$ is used, then the parent-scope variable is updated, and g acts as the 'state' of the function.

\subsubsection{Operator defintions}
The function name can be replaced by \%*\% for example.

\subsubsection{Invocation}
Invocation can be done as in C, using a sequence of values. Together with positionally specified values, one can pass named arguments in any order. Eg: f(3, a=1, b = 2) or f(a=1).

\section{Operator}
\subsection{Assignment}
$<-, ->$, assign(). $<<-$ is used to make global assignments: assignment to a variable outside the local scope.

\subsection{Scalar operators}
\subsubsection{Arithmetic operators}
As in C.

\subsubsection{Operators on booleans}
\verb'|, &, ||, &&',  where the latter result in 'short-circuit' evaluation, where the second argument is evaluated only if necessary.

\subsubsection{Operators which produce booleans}
\verb'>, <, =='.

\subsection{General Vector operators}
\subsubsection{Mapping over elements}
lapply can be used to apply scalar functions to vector elements, while apply() can be used to apply vector functions on array rows.

\subsubsection{Arithmetic and boolean ops}
All scalar arithmetic operators are extended to be meaningful when provided vector arguments: even when they are not of the same size.

\subsubsection{Lengthening of arguments}
All shorter arguments are extended by repetition to have the size of the longest vector: Thus 1+c(2 3) is define.

\paragraph{Examples}
c(1,2,3,4)/c(4,3,2,1). c(1,2,3,4) + c(4,3) yields 5 5 5 8.

\subsubsection{ifelse op}
ifelse operation uses a logical vector as a condition.

\subsubsection{Set membership}
\begin{verbatim}
drops <- c("x","z")
DF[,!(names(DF) %in% drops)]
\end{verbatim}


\subsubsection{Missing value identification}
is.na(x) returns true for both NA and NAN values. Note that this is different from the syntactically undecidable expression x == NA.



\section{Code structure}
Every line of code is an expression or a sentence.

\subsection{Code blocks}
\{\} encloses code block.

\subsection{Sentence syntax}
Sentences end with newline or ;.

\# comments.

\subsection{Decision structures}
if(..) codeBlock; optionally followed by else codeBlock. for(var in vector) codeBlock. while(cond) codeBlock.

\subsection{Packages}
To be able to use a package, one says: library(packageName). Standard packages are automatically available. Note that this is distinct from the idea of a namespace.

\subsection{Organization with lists}
Can group functions in lists which are declared in separate files. These files are then (re)loaded using commands like source or sourceDirectory.

\section{IO}
\subsection{File I/O}
To load data from a table, use read.table(), read.csv (fileName, header=FALSE, stringsAsFactors = FALSE), write.csv(x, file).

Write in matlab format: library(R.matlab), writeMat(filename, A=mat).

\subsection{User interaction I/O}
\subsubsection{output}
a+b prints a value, which is then lost.

print. printf is available in the base package.

sink('fileName') diverts output to a file. sink() restores it to STDIO.

\subsubsection{Input}
x= scan(): keyboard input, no commas.

\subsection{Plotting and tables}
\verb' plot(y ~ x) ' produces a scatter plot.

Tables can be produced with \verb'xtabs(y ~ x)'.

\section{Data preparation and exploration}
scale(x, center = TRUE, scale = TRUE) normalizes columns using the mean and standard deviation.

Getting covariance matrix: cor(x, y = NULL, use = "pairwise.complete.obs")

\section{Modeling}
Several useful functions are provided to evaluate fitted models in the package stats, which is loaded by default.



\subsection{Classification}
Decision tree learning: rpart

\subsubsection{Logistic regression}
\begin{verbatim}
glm(model, data=tblName, family = binomial())
ret <- glm.fit(x=X, y=z, family=binomial())

\end{verbatim}
Options for family include: binomial(link=logit)

The return value is a list which includes coefficients and fitted.values.


\subsubsection{Logistic regression with l1/l2 regularization}
Use the glmnet package (requires gfortran).

Example:
\begin{verbatim}
returnList <- cv.glmnet(X, y, family = "binomial");
\end{verbatim}

returnList contains the following vectors: lambda (corresponding to the lagrangian multiplier for the l1-norm), cvm - the corresponding mean cross-validated errors, glmnet.fit: the fit weights.

\subsubsection{Decision tree: rpart and tree}
\begin{verbatim}
ret <- rpart(model, tblRet)
print(ret)
\end{verbatim}


\section{Other library functions}
\subsection{Random sampling}
Sampling from distributions: runif. sample(x, size, [replace=TRUE])

\section{Write C extensions}
\subsection{Write C code}
Useful libraries: \#include <R.h> \#include <Rmath.h>

Signature: void getSamples(int *input, int *output)

\subsection{Calling C code}
Compile C code: R CMD SHLIB foo.c

Loading: dyn.load("foo.so")

Calling: .C("foo", n=as.integer(5), x=as.double(rnorm(5)))

This returns a list of return.

\chapter{Rapidminer}
\section{Introduction}
\subsection{Purpose}
Rapidminer provides convenient tools for data loading, exploration, modelling, visualization.

\subsection{UI and API}
It comes with a nice GUI. It provides Java API. It also has a scripting interface.

\subsection{Data types}
(Bi or Poly) Nominal, Integer, Double, Varchar.

\section{Processes}
One can visualize the data mining process as a tree of operators. The leaf nodes correspond to the result nodes. Processes can be edited in the design view : here it looks like an electronic circuit. This is effectively programming using a GUI.

\subsection{Operators}
Operators are grouped under various categories: like Modeling, Data transformation, repository access, evaluation/ validation, process control (looping, conditions). These groups contains various subgroups.

Each operator has an input and output, and may require additional settings.

An operator may be nested: itself composed of sub-operators.

\section{Data view}
The data view has different tabs to show meta-data, the actual data, a good interface for plotting.

The meta-data view shows useful summary like mean and standard deviation for numeric data and, mode /min classes for nominal data.

\chapter{Spreadsheet programs}
\section{Google docs}
\subsection{Cell referencing}
Columns are numbered with letters and rows with numbers. A cell can be specified using a latter and a number.

When cell references are copied (perhaps as part of formulae), the references may be automatically changed in a certain way - this is called relative reference: Eg: The reference A20 used in cell A2, when copied to B2 will automatically be changed to B20.

When such change does not happen, we have absolute reference to a cell. Eg A\$20, and \$A20 (the former indicating that the row is fixed and the latter fixing the column), \$A\$20.

\subsection{Range specification}
A:A, A2:A30.

\subsection{Filtering}


\subsubsection{Condition specification}
">0", value.

\subsection{Counting}
COUNTIF(rangeToCheck, condition, rangeToCount). rangeToCount is optional.

COUNTBLANK. COUNT.

\subsection{Summation}
SUMIF, SUM. These have syntax similar to countif and count.


\subsection{Averaging}
AVG(RANGE)

\subsection{Looking up values}
vlookup()

\part{Shell Scripting and OS work}
\chapter{General tasks}
Web browsing is described elsewhere.

\section{Scanning documents with camera}
The following 3 steps can be accomplished online on picasa's piknik editor.

\subsection{Cropping}
The image may need to be cropped to the interesting region.

\subsection{Enhancement}
When the camera captures the image of a document, because of inadequate light, there is a shift in the white-content of  colors captured - eg: white becomes a shade of grey. So, processing the camera picture to increase brightness and contrast of the image (aka Enhancing) is required. This ensures that, when reprinted, the colors appear closer to the original - ie they are not darkened.

\subsection{Compression}
For efficient storage and transmission, the image may need to be compressed.

The color range and image size may be reduced. 65\% quality JPEG compression is usually adequate.

\section{File synchronization}
Dropbox service syncs files across computers; but it does not preserve deleted files for more than a month.

\section{VOIP}
Skype, google chat.

\section{Booting to special modes}

WIndows safe mode: Press ctrl during startup.

Jawbone headset pairing mode: Turn on while holding speak button.

\subsection{Atrix boot modes}
\subsubsection{Fastboot}
Turn on while pressing volume-down button until you see 'fastboot' on the top.

Pressing the down button shows various other options, including "Early USB Enumeration". up button selects an option. Pressing the up button upon seeing 'fastboot' selects the fastboot mode.

you only get about 1-2min to do anything before the phone reboots on its own. \chk

\subsubsection{Recovery}
One can enter the recovery mode using adb reboot recovery. Thence, udpates may be manually installed.

\section{Virtual machines, emulators}
An emulator provides the same interface as another operating system to some application. A virtual machine simulates computer hardware.

\subsection{Emulators on linux}
WINE emulates windows.

mono emulates standardized part of the .Net framework. Command: mono something.exe.

\subsubsection{Java virtual machines}
\tbc

\chapter{Pocket computer OS usage}
\section{Special tasks}
\subsection{Syncing media}
This sometimes involves more than merely copying files. Special directory structures, thumbnails may be created.

For syncing media: banshee on ubuntu, windows itunes for apple devices.



\subsection{Podcasts}
Mediafly is a good podcast listening service - one can store list of channels/ podcasts online; one can arrange to preload them when access to WiFi is  available.

\subsection{Scanning}
Genius scan uses camera and does necessary post-processing to increase image brightness and contrast.

\subsection{Offline navigation}
Applications are available to create maps from online sources for offline use on the pocket device. Eg: mobile atlas creator.



\subsection{Ebook conversion}
\subsubsection{Calibre}
ebook-convert a.html a.mobi . This reads and uses creates table of content in a.html.

ebook-viewer

web2disk URL

\subsection{Keypad commands}
Pocket computers equipped with cell-phone radios often have diagnostic/ system information routines accessible through dialing special codes.

\subsubsection{Android}
\verb *#*#checkin#*#* Phones home to check for updates.

\verb *#*#info#*#* - Enters a detailed phone information menu. Lets you turn off cell phone radio (while enabling bluetooth).

\verb *#*#1472365#*#* : Access to the GPS config menu.

More info \htext{here}{http://android.stackexchange.com/questions/1468/do-you-know-other-android-keypad-commands}

\subsection{Unlock bootloader}
The bootloader may be replaced - making rooting easier.

\subsubsection{Atrix running Gingerbread}
The procedure is as follows.

Install android sdk

Reboot phone in fastboot mode.

Connect to computer.

Type fastboot oem unlock

\tbc

\subsection{Exploring files from a computer}
Usually, when connected to another computer, one can browse and manipulate only a subset of files on the pocket computer - this makes file operations safer and avoids 'bricking' or complete software failure.

\subsubsection{Android}
One enables usb connections in debugging mode on the pocket computer, installs the android sdk, does sudo adb start-server, and then sudo adb shell etc..

\subsection{Enable and use root access}
This allows user programs the ability to gain root access - a feature usually disabled by default.

To use root from adb, just type su.

\subsubsection{Atrix running gingerbread}
Enter the fastboot mode, copy certain files, reboot and gain access using the sudo ./adb shell command; then replace su, SuperUser.apk + some other stuff (see ??).

\subsection{File transfer: Android}

\subsubsection{Enable Writeability}
adb shell. adb mount - gives mount points like /dev/block/mmcblk0p12 /system .  Then one does: mount -o remount,rw /system

If the above fails, one can transfer to /sdcard-ext and then move the file to any location under superuser mode.


\section{Kindle keyboard}
\subsection{Diagnostic pages}
411 page: Alt + RQQ. 611 page: Alt + YQQ.

\subsection{Shortcuts}
Alt + F: next song. Pan: alt + arrows. Text to speech: Alt+Sym. Numbers: Alt + qwerty.

\chapter{Linux OS}
Map tasks and commands.

\subsubsection{perl}
Running perl from the command line is a very popular option. Various switches are described in \htext{perlrun}{http://perldoc.perl.org/perlrun.html}.

-e expr executes expressions provided.

-n ensures that the expression runs on each line from file specified either in the command line or passed in stdin.

-p Same as -n, except output of each line is printed.

-a ensures that each line is automatically split to produce a token array (whose name may be arbitrary).

\section{Graphical user interface}
GUI software stack is described elsewhere.

\subsection{Common settings}
All possible sessions are represented by a file in a certain location for the login-session manager to find.

All available applications are listed in a single location.

\subsection{Gnome}
After changes, gnome can be restarted by entering 'r' in the 'Run command' interface - usually invokable by Alt+F2.

\subsubsection{Editing configuration}
gconf-editor provides a good 'registry'-like interface to all gnome applications'/ desktop environment settings.

'System settings' and ubuntu tweaker provide more polished but limited GUI's.

\subsection{KDE}
General settings are edited using systemsettings.

\section{Others}
xfce was tried successfully with Ubuntu 11.10.

\section{Editing}
\subsection{Editors}
Useful features include: copying/ pasting, automatic indentation, ability to easily (un)comment, brackets which auto-close.


\begin{itemize}
\item general: kate.
\item bib editor: jabref.
\item html editor: bluefish, kompozer.
\item latex editor: kile. texmaker for indic script.
\end{itemize}

\subsection{Programmatic editing}
Pipe file list to \verb'|xargs perl -w -i -p -e "s/str1/str2/g"'. (find path -type f will produce a list of all files if that is required!)

perl -pi -e 's/FINDTEXT/REPLACETEXT/' file*
if FINDTEXT is one line.

perl -p0777i -e 's/FINDTEXT/REPLACETEXT/m' file*
if FINDTEXT is many lines

\subsection{vi}
\subsubsection{Modes}
vi is used in one of many modes: command, find, edit/ insert, replace and view.

To shift between view and edit mode: press e or Escape. TO go to replace mode, press r.

To shift between view and command modes, press : or Escape. To shift between view and find mode, press / and Escape.

\subsubsection{Commands and pattern matching}
:g/Pattern/command executes command on every line matching a pattern.

\verb':1,$s/pattern/replacement/g'

text replacement with regular expressions in vi: use \^C\^M for new line.

\section{Resource check}
cpuinfo, cat /proc/meminfo.

du -h --max-depth=1 gets disk usage.

du -hs gets total space used by directory files.

\section{Image processing}
Gimp, the gnome image viewer and editor.

\section{Installation}
Add a repository using the commandline or synaptic UI. Import necessary keys. update repository lists. Then install anything.

\chapter{BASH scripting}
\section{Characteristic features}
All environmental variables are imported into the context.

\subsection{Command construction, execution}
Very simple syntax for execution of commands: simply use a line which says: commandName. Construction of command-strings using variable names is also simple: someText \$varName otherText

\subsection{Logic limitations}
Not suitable for logic more involved than an if else statement.


\section{Writing and executing code}
Bash commands and programs can either be run by providing/ typing the program in STDIN, or it can be run from a file.

\subsection{Context}
Every sequence of Bash commands is executed in a certain context. Different contexts do not share the variables (including environment variables like PWD and PATH) - so changing variables in one context does not affect another.

\subsubsection{Current context}
source fileName executes the file in the current context.

\subsubsection{New context}
One can run a file using: bash fileName.

\paragraph{Invocation as a command}
One can execute the file merely by saying fileName at the command prompt, if it begins with: \#!/usr/bin/bash . This line is used by many interpreters to identify and use the interpreter appropriate to the file when one runs the file.

If \#!/usr/bin/bash -x is used, all executed lines are echoed.

\subsection{Environment variables}
Some variables, called the environment variables, are set automatically when a shell context is created.

These variables are important because they are used while searching for various purposes affecting terminal display and command interpretation/ execution. Eg: \verb LD_LIBRARY_PATH , PATH (a : separated list of directories where an executable file is to be sought), PWD (present working directory).

\subsubsection{Setting}
export VAR=value sets an environment variable.

To set these at startup, edit .bashrc and \verb .bash_aliases  in the home directory. env lists the environment variables.

\section{Variables and data}
Dynamic typing.

\subsection{Assignment}
VARNAME=value.
Array: area2=( zero one two three four )
A particular element is set with area2[0]=val

Note that there should not be any space around =.

\subsection{Reference}
\verb|$VAR_NAME ${VAR_NAME}| are references to the variable.

Array Element reference: \verb'${area2[0]}'.

Reference the entire array: \verb'${colors[@]}'.


\chapter{Android development}
Pocket computer usage is considered in another chapter.

\section{Java SDK}
Android development is done in Java. One can install the android SDK and google API following instructions from google. This also provides platform-tools for command-line access to the device.

This also creates a keystore for use while debugging (perhaps in signing google-api use agreements.)

\subsection{C++ compiler}
Native code can be used through JNI - C code will have to be compiled using ndk. Official NDK does not allow exceptions in code. Instead one can use "crystax NDK".

\subsection{Logging}
All prints to stdout and stderr done in the Java code is redirected to a common formatted log file. By default similar output of native code is sent to /dev/null. This can be changed getting root access and creating a file /data/local.prop  with the following content:

log.redirect-stdio=true


\subsection{Command line access to device}
The command adb is useful. Parameters push and pull move files to and from the device. Parameter shell would connect to a very limited version of the linux environment available on the device.

\tbc

\section{Eclipse-plugin}
Android development happens in Java, and there are nifty plugins for debugging the application using Eclipse. These tools are provide a layer over the command-line platform-tools provided by the Android SDK.

\subsection{Debugging}
Debugging can be launched from the java perspective and stopped from the Devices view in the DDMS perspective.

In the LogCat viewer in the DDMS perspective one can see various logger messages and lines printed out to stdout and stderr. The Console viewer in the same perspective tells us when the application has been built and sent to the device.

It includes a file explorer.

\tbc


\section{Application architecture}
See the relavant note \htext{online}{http://developer.android.com/guide/topics/fundamentals.html}.

\subsection{Components}
An app consists of components. Their types are Activities (UI), Services (background), ContentProviders (shared data management), BroadcastReceivers.

\subsubsection{Communication with components}
One expresses an intent to get a certain result from a component using Activity.startActivityForResult(intent, reqCode).

When the external component returns the result together with the intent object and a resultCode, Activity.onActivityResult is called.

\subsection{Security and isolation}
Following the principle of allowing minimal access, the applications must declare what resources they need in advance in AndroidManifest.xml. Every application is signed.

Every application runs in its own virtual machine (dalvik-VM).

\tbc

\subsection{Threading}
Threads have MessageQueue-s, which are handled by Handler.handleMessage.

\subsection{Resources}
Resources required by the application are located in 'res' - including sound, image and xml files.

\subsubsection{UI design}
The look/ placement of UI components for various activity windows are recorded in xml files in the layout subfolder, which can be edited graphically using the interface provided by the Eclipse plugin.

Their functionality is defined in corresponding Activity classes. Java objects and classes which can be accessed and manipulated from code (which may be within the activity class) are generated automatically from the XML files.

\part{Distributed computing}
\chapter{Parallel computing paradigms}
Many programs are 'embarassingly parallel' - so easy to parallelize.

There are 3 steps in any parallel algorithm: specification of the problems which must be solved in parallel, executing the problems in parallel, combining the results of these parallel executions.

\chapter{Condor}
This is a common job-scheduler. Once you specify the job in a certain file, condor tries to execute it on some processor (in a cluster) and retrieve the result.

\section{Priority and restarts}
Commonly jobs dispatched by condor are low priority - so if a higher priority process comes in, the condor job is stopped and moved to a different processor. If the program has checkpointing facilites, the job simply continues from where it stopped : Eg: Compiled languages. But if checkpointing is not available, the process restarts when moved to a different node: Eg: most interpreted languages.

\chapter{ORC}
\section{Distinctive features of the language}
\subsection{Purpose of design}
Inspired by functional programming languages.

Distributed computing: There'll be many services on the internet, need a language to orchestrate them. So, good internet mashup language.

Good for concisely reasoning about distrubted systems.

\subsection{Sites}
Everything is a site: a possibly remote function without side-effect which may not respond. Even +, -, if(..) etc.. are sites. These may return sites too. Actual site call is not executed until all arguments are available.

\subsection{Functions}
The language allows/ needs functions, but they're not site calls. It is simply a parametrized expression.

\subsection{Parallelism}
Highly parrellelizable.

Even a+b: a and b are evaluated independently.

\subsection{Speed}

\section{Writing, Building and executing code}
Run in browser, or using an eclipse plugin.

\subsection{Debugging}


\section{Help}
The website.

\section{Pre-compilation processing}

\section{Variable and data types}
\subsection{Data types}
signal is a unary data-type. boolean, numeric etc.. are other data types.

[] is an empty list.

a = Ref() yields a pointer: then do a.read(), a.write().

\subsection{Declaration syntax}
val valName = expression.

This boils down to pruning operator. Eg: $a|valName$ is actually $a|b<b<expression$.

It is scoped to the expression in the next line.

Pattern matching: $a:b=[1, 2, 3]; (\_, a) = (2, 3)$.

\section{Data manipulation}
\subsection{Operators}
/= is 'not equal to' boolean operator.

\subsection{Accessing object methods}
channel.get() actually is shorthand for calling a function which retrieves the get() function and then invokes it.

\section{Code structure}
\subsection{Sentence syntax}
There is actually just one sentence/ expression. Rest is syntactic sugar. Line breaks don't matter.

\subsubsection{Commenting}
\{- asdf -\}

\subsection{Combinators}
Use to stitch together expressions.

Parallel combinator: $a|b$.

Sequential/ push combinator: $>$. Eg: $prompt()>b>c(b)$. Syntactic sugar: $prompt()>b>c$ is same as $prompt()>>c$.

Pruning/ pull operator: $a<b$. Both sides begin to be evaluated simultaneously; if LHS needs a value, then it blocks.

\subsection{Syntax of functions}
Can be nested.

def fnName(arglist) = expression.

Function definition which uses pattern matching, usual syntactic sugar to specify termination condition:

fn([]) = asdfasdf

fn((a:asdf)) = asdfasdf

\subsection{Decision, timing sites}
if(). stop() never returns any value. RTimer().

Iteration accomplished through recursion.

\subsection{Function invocation}
fn(arguments) or fn.

fn(expression): syntactic sugar for $f(x)<x<expression$.

\section{Error handling}

\section{Other library functions}
\subsection{User interaction I/O sites}
prompt(), print().

\part{Serving web-pages}
\chapter{Common Gateway Interface (CGI)}
\section{HTTP server and dynamic content}
A http server serves files; of which html files are normally viewed using web-browsers. These files may be static, or they may be generated dynamically, at the time of the request. For dynamic generation of such files, the http server should coordinate with some external program. At the minimum, it passes the request made by the client to this program and retrieves its output.

These programs may be precompiled - they may be written in C; or they may be written in a scripting language, and the http server may use an appropriate interpreter when the request is made.

\section{Scripting mixed with html}
The dynamic programs often dynamically generate html pages, so the dynamic portions of such pages can potentially be written in another language, while the static portions are written in html. Depending on the language used for the dynamic portion, this mixed-language is called by different names, like perlscript or JSP (or java server pages) or ASP.

The http server may utilize these scripts/ server pages by first converting it into a program of the corresponding language and then using the appropriate interpreter. Eg: JSP's are often first converted to Java servelets.

\chapter{Client side scripting}
Most browsers support javascript.

\chapter{Yahoo pipes}
With this service one can 'rewire the internet': that is, process data available on the web as we like.

\section{Interface}
Programming the processing pipeline is done with a simple graphical interface; where one connects boxes (representing various processing modules) with pipes (representing flow of information).

\section{Looping}
Processing modules may be of the following types: source modules which gather initial data published on the internet in various forms (html, rss feeds etc..),  user input modules, various operator modules for modifications (splitting data, looping, filtering, string replacement, reformatting etc..).

\chapter{Web API}
Some web-services expose API to allow algorithmic access to the information they serve.

\section{Request/ response format}
They specify the serialization formats in which input/ request and output/ response objects (structured data) are to be encoded. This is usually in JSON or XML format.

XML, being more verbose, results in slower computation and transmission.

\section{Underlying Protocol}
Usually, they use http or rpc protocol.

\section{Web interface Protocols}
\subsection{Simple Object access protocol (SOAP)}
It specifies the use of XML for representing objects, certain Message Exchange Patterns (MEP).

\chapter{Particular Web Api's}
\section{Wikimedia}
The mediawiki api is specified \htext{here}{http://en.wikipedia.org/w/api.php}. Also, the \htext{bot-info}{http://en.wikipedia.org/wiki/Wikipedia:Creating_a_bot#Java} page has more information.

\subsection{Java wrappers}
\begin{itemize}
\item \htext{jwpl}{http://code.google.com/p/jwpl/}
  \subitem Not useful in querying live data.
\item JavaWikiBotFramework (\htext{jwbf}{http://jwbf.sourceforge.net/pw/})
  \subitem Seems to have good reviews, poor web documentation.
\item \htext{gwtwiki}{http://code.google.com/p/gwtwiki/} bliki engine.
  \subitem Focuses mainly on converting between wiki text, plain text, google code wikitext and html etc..
  \subitem Has useful \htext{wiki}{http://code.google.com/p/gwtwiki/wiki/MediaWikiAPISupport} (\htext{2}{http://code.google.com/p/gwtwiki/wiki/Mediawiki2HTML}) page with examples.
\item \htext{java wiki api}{http://jwikiapi.sourceforge.net/}
  \subitem Has some examples.
\end{itemize}

Wiktionary frameworks:
\begin{itemize}
\item \htext{jwktl}{http://www.ukp.tu-darmstadt.de/software/jwktl/}
  \subitem Not useful in querying live data.
\end{itemize}


\chapter{Flash}
Flash player security settings can be adjusted by visiting a special \htext{webpage}{http://www.macromedia.com/support/documentation/en/flashplayer/help/settings_manager06.html}.



\part{Programming within browsers}
\chapter{Web Browsers}
Many popular browsers enable one to synchronize the following across different computers: bookmarks, stored user information and passwords.

google-chrome is a fast browser - it benefits from Google's heavy efforts at making browsing fast. One can configure keywords to map to search engines - this makes searching easy. It has problems rendering hyperlinked indic text -esp dEvanAgarI- properly.

\section{Extensions/ plugins}
Users tend to write nifty plugins for their browsers. These are listed elsewhere.

\section{User scripting}
\subsection{Interoperability}
firefox, chrome, and IE all support user-scripting. The trend became popular with Greasemonkey scripts in firefox. Firefox user-scripts are managed using the greasemonkey extension, while similar extensions are available for chrome - but I haven't been able to make them work.

There are libraries which take browser differences into account.

\section{Enhancements by websites}
Very useful shortcuts are available with google mail and reader. They can be listed by pressing ?.

\section{Observe browser activity}
While programming, debugging or bug reporting, it is useful to observe details of what is happening internally. Eg: What web requests are being made; what javascript errors are observed.

In chrome: right click -> inspect element. The network and scripts tabs are very useful.


\chapter{User scripts}
\section{Motivation}
Modern browsers, perhaps aided by extensions like greasemonkey, allow users to write scripts which are activated for certain websites. The language is essentially javascript.

Many scripts can be found in userscripts.org.

\section{Header format}
\begin{verbatim}
// ==UserScript==
// @name           sanskrit vocabulary trainer
// @description    sanskrit vocabulary trainer
// @author         vishvAs vAsuki Iyengar
// @include        http://*vasuki*
// @version        1.0
// ==/UserScript==
\end{verbatim}



\part{Databases}
\chapter{Database management systems}
\section{Relational Database management system (RDBMS)}
All records can be viewed as consisting of some fields or attributes.  Every attribute has a fixed domain. There are relations among these fields which impose constraints on what forms a valid record.

Tuples are a collection of values assigned to attributes. A bunch of tuples sharing the same attributes form a relation (or a table).

\subsection{Relational algebra}
Relations can undergo the following operations: Union, intersection, difference, cross product, selection/ restriction, projection, inner join, relational division (?).

\subsection{Database normalization}
What relations should exist and what attributes should various relations contain so that they do not redundantly store data?

To do this there are various normal forms (Boyce Codd). However, normalizing a database too much causes retrieval and update operations to slow down as they will involve joins.

\section{Sharding}
Databases can be too large to store in a single machine. In that case, the tables may be split horizontally and stored in different machines.

The different shards can be named using the convention tblName\@ shardNum.

Some query engines like dremel can query multiple shards simultaneously.

\section{Pooling writes}
Database writes, when pooled, reduce client to database server traffic. Intelligent pools collect writes and flush when a sufficient load is accumulated.

\chapter{Structured query language (SQL)}
A standard language for dealing with RDBMS. So, data can be visualized as being stored in tables.

\section{Help and tutorials}
w3schools has a good tutorial.

\section{Values}
Strings are enclosed as here: 'asdf'.

\section{Query language}

\subsection{Query table}
The query table may be specified in several ways. In the most basic case, it is a single table.

\subitem table1 (as t1), table2 represents a cross-product, aka cross join. This returns a table whose rows are members of tbl1 X tbl2. In specifying temporary table names, as in table1 as t1, some dialects allow omission of 'as'.

\subitem tableList may be of the form: tbl1 join-command tbl2 [on condition]

\subitem Or it can be: (selectStatement)


\subsection{Post selection processing}
\subitem 'group by fieldName' groups the data together before displaying.

\subitem Multiple select statements may be combined using keywords like: UNION ALL, INTERSECT.

\subsection{Temporary tables}
Temporary or inline tables are automatically created whenever a query table involves a join of any sort, or when it is defined by some select statement.

However, one may want to explicitly name and store the results of a query. Syntax is same as in create statement, with additional keyword: create temporary/ inline table.

\subsubsection{Storage}
These tables are stored in the RAM of the database client or on the database server.

The tables are dropped automatically at the end of the session.

\section{Stored procedures}
A piece of sql code which can be easily called.


\chapter{Flat file DB}
The entire database may be stored within a single file. Such databases may be useful when a database server is not required and when access is local. Eg: sqlite, berkeley db (not a relational db), recutils.

\section{sqlite}
sqlite3 databases cannot be opened with sqlite.

sqlite offers a limited set of field types.

sqlite does not allow joins during updates - making the process of copying columns a bigger chore.

\subsection{GUI}
sqlitebrowser offers a good GUI for sqlite - but the import from csv feature is very slow as of 2012.

\subsection{Shell}
\subsubsection{Importing}
\begin{verbatim}
.mode csv
.import fileName.csv tableName
\end{verbatim}




\chapter{MySql}
\section{Server}
The database is stored in a file. Access to the database is mediated by a server. Different users can have different abilities. A root user is created at the time of installation.

By default the following databases/ schemata exist:
\begin{itemize}
 \item mysql : Having tables such as user ..
 \item information\_schema.
\end{itemize}


\section{Clients}
\subsection{mysql shell}
Within the shell, all commands not beginning with \verb'\' ends with a ;.

\subsubsection{Invocation}
Format: mysql [options] [database].

Important options include:

-u username -p.

\subsubsection{General commands}
\verb'\q' quits.

General commands comply with the sql standard.

\subsubsection{Meta-commands}
show databases/ tables

'use database' selects a particular database to work with.

\subsubsection{Administration}
\verb[SET PASSWORD FOR 'root'@'localhost' = PASSWORD('secret_password');p[

\begin{verbatim}
GRANT priv_type [(column_list)] [, priv_type [(column_list)] ...]
      ON {tbl_name | * | *.* | db_name.*}
      TO user_name [IDENTIFIED BY 'password']
      [, user_name [IDENTIFIED BY 'password'] ...]
      [WITH GRANT OPTION]
\end{verbatim}

Correspondingly there is the REVOKE command.

\subsection{Ancilliary commandline tools}
mysqladmin provides easy access to some commands also available in mysql.

mysqldump creates database backups. mysqlhotcopy does the same, except it locks the databse while doing its job.

\tbc

\subsection{GUI}
mysql administrator provides a useful interface to manage the server itself, the users, backups etc..

mysql querytool provides easy command reference and a window to easily enter commands and see results.

eclipse sql explorer \tbc.


\section{Past experience}
Provides unfriendly error messages, rejects sql commands accepted by sqlite.


\part{Remote procedure call and serialization}

\part{Document typesetting}
\chapter{Declarative vs imperative features}
Many document display specification formats provide are declarative. On one hand, they specify the logical components of a document. On the other hand, they enable specification of display rules or styles for each logical element of the document.

Yet, despite this, sometimes finer imperative control over document display is desirable.

Some languages provide both features, while others provide one or the other.

\chapter{latex}
Both declarative (rule-based) and imperative specification of document display is possible.

A document element is defined to either be a command and or an environment - which are distinguished based on the input they accept.

\section{Common commands, environments}
Common commands include section, subsection, paragraph, subsubsection, title, author. Common environments include list (itemize and enumerate), tabular etc..

\subsection{Templates}
Rules which map commands and environments to imperative display rules are collected together in templates. These rules may be overriden. Every document must be based on some template.

Common templates include: report, article.

\subsection{Packages}
Often display rules may be redefined in files called packages, which may be used/ invoked using usepackage command.

\subsubsection{Page setup}
microtype for full justification of text. fullpage for using the maximum possible printable area.

\subsubsection{List compaction}
enumitem and shortlst make lists more configurable.

\section{Bibliography}
This part of the document is often dynamically generated based on citations used and a bibliography file. In this process, aux and bbl files are generated and used. \exclaim{In case of error, try deleting these files!}

Two things affect document display here: citation style and bibliographystyle. These are usually specified using packages, possibly followed by minor redefinitions of commands.

\subsection{Packages}
natbib is configurable and popular.

\section{(Re)definition}
Commands can be defined using: \verb'\newcommand{cmd}[args]{def}'.

Environments can be defined using \verb'\newenvironment{env}[args][default]{begdef}{enddef}'.

Older definitions can be overridden using renewcommand and renewenvironment.

\section{Imperative display commands}
textit, textbf.

center, raggedleft, raggedright.

\subsection{Font size}
In order:     tiny
    scriptsize
    footnotesize
    small
    normalsize
    large
    Large
    LARGE
    huge
    Huge .

Using them resets the font size till the end of the current text block.

\subsection{Display parameters}
They can be set using addtolength and setlength commands.

Common parameters include textwidth, oddsidemargin, evensidemargin, topsidemargin, parskip.

One can define colors using: \verb'\definecolor{mygrey}{gray}{.95}'.

\subsection{Space management}
\verb'\vspace{-3ex}' sets the vertical spacing to a third the size of 'x'. This should ideally be used towards the end of the document.

hfill and vfill are used to fill spaces.

\section{Drawing figures}
pstricks. tikz: good for drawing graphs with nodes and edges, but not as advanced as pstricks.

\chapter{html}



% \bibliographystyle{plain}
% \bibliography{programmingLanguages}
\end{document}
