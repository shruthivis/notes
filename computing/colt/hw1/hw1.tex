\documentclass[10pt]{amsart}
\usepackage{amssymb}

% Use something like:
% % Use something like:
% % Use something like:
% \input{../../macros}

% groupings of objects.
\newcommand{\set}[1]{\left\{ #1 \right\}}
\newcommand{\seq}[1]{\left(#1\right)}
\newcommand{\ang}[1]{\langle#1\rangle}
\newcommand{\tuple}[1]{\left(#1\right)}

% numerical shortcuts.
\newcommand{\abs}[1]{\left| #1\right|}
\newcommand{\floor}[1]{\left\lfloor #1 \right\rfloor}
\newcommand{\ceil}[1]{\left\lceil #1 \right\rceil}

% linear algebra shortcuts.
\newcommand{\change}{\Delta}
\newcommand{\norm}[1]{\left\| #1\right\|}
\newcommand{\dprod}[1]{\langle#1\rangle}
\newcommand{\linspan}[1]{\langle#1\rangle}
\newcommand{\conj}[1]{\overline{#1}}
\newcommand{\gradient}{\nabla}
\newcommand{\der}{\frac{d}{dx}}
\newcommand{\lap}{\Delta}
\newcommand{\kron}{\otimes}
\newcommand{\nperp}{\nvdash}

\newcommand{\mat}[1]{\left( \begin{smallmatrix}#1 \end{smallmatrix} \right)}

% derivatives and limits
\newcommand{\partder}[2]{\frac{\partial #1}{\partial #2}}
\newcommand{\partdern}[3]{\frac{\partial^{#3} #1}{\partial #2^{#3}}}

% Arrows
\newcommand{\diverge}{\nearrow}
\newcommand{\notto}{\nrightarrow}
\newcommand{\up}{\uparrow}
\newcommand{\down}{\downarrow}
% gets and gives are defined!

% ordering operators
\newcommand{\oleq}{\preceq}
\newcommand{\ogeq}{\succeq}

% programming and logic operators
\newcommand{\dfn}{:=}
\newcommand{\assign}{:=}
\newcommand{\co}{\ co\ }
\newcommand{\en}{\ en\ }


% logic operators
\newcommand{\xor}{\oplus}
\newcommand{\Land}{\bigwedge}
\newcommand{\Lor}{\bigvee}
\newcommand{\finish}{$\Box$}
\newcommand{\contra}{\Rightarrow \Leftarrow}
\newcommand{\iseq}{\stackrel{_?}{=}}


% Set theory
\newcommand{\symdiff}{\Delta}
\newcommand{\union}{\cup}
\newcommand{\inters}{\cap}
\newcommand{\Union}{\bigcup}
\newcommand{\Inters}{\bigcap}
\newcommand{\nullSet}{\phi}

% graph theory
\newcommand{\nbd}{\Gamma}

% Script alphabets
% For reals, use \Re

% greek letters
\newcommand{\eps}{\epsilon}
\newcommand{\del}{\delta}
\newcommand{\ga}{\alpha}
\newcommand{\gb}{\beta}
\newcommand{\gd}{\del}
\newcommand{\gf}{\phi}
\newcommand{\gF}{\Phi}
\newcommand{\gl}{\lambda}
\newcommand{\gm}{\mu}
\newcommand{\gn}{\nu}
\newcommand{\gr}{\rho}
\newcommand{\gs}{\sigma}
\newcommand{\gt}{\theta}
\newcommand{\gx}{\xi}

\newcommand{\sw}{\sigma}
\newcommand{\SW}{\Sigma}
\newcommand{\ew}{\lambda}
\newcommand{\EW}{\Lambda}

\newcommand{\Del}{\Delta}
\newcommand{\gD}{\Delta}
\newcommand{\gG}{\Gamma}
\newcommand{\gO}{\Omega}
\newcommand{\gL}{\Lambda}
\newcommand{\gS}{\Sigma}

% Formatting shortcuts
\newcommand{\red}[1]{\textcolor{red}{#1}}
\newcommand{\blue}[1]{\textcolor{blue}{#1}}
\newcommand{\htext}[2]{\texorpdfstring{#1}{#2}}

% Statistics
\newcommand{\distr}{\sim}
\newcommand{\stddev}{\sigma}
\newcommand{\covmatrix}{\Sigma}
\newcommand{\mean}{\mu}
\newcommand{\param}{\gt}
\newcommand{\ftr}{\phi}

% General utility
\newcommand{\todo}[1]{\footnote{TODO: #1}}
\newcommand{\exclaim}[1]{{\textbf{\textit{#1}}}}
\newcommand{\tbc}{[\textbf{Incomplete}]}
\newcommand{\chk}{[\textbf{Check}]}
\newcommand{\oprob}{[\textbf{OP}]:}
\newcommand{\core}[1]{\textbf{Core Idea:}}
\newcommand{\why}{[\textbf{Find proof}]}
\newcommand{\opt}[1]{\textit{#1}}


\DeclareMathOperator*{\argmin}{arg\,min}
\DeclareMathOperator{\rank}{rank}
\newcommand{\redcol}[1]{\textcolor{red}{#1}}
\newcommand{\bluecol}[1]{\textcolor{blue}{#1}}
\newcommand{\greencol}[1]{\textcolor{green}{#1}}


\renewcommand{\~}{\htext{$\sim$}{~}}


% groupings of objects.
\newcommand{\set}[1]{\left\{ #1 \right\}}
\newcommand{\seq}[1]{\left(#1\right)}
\newcommand{\ang}[1]{\langle#1\rangle}
\newcommand{\tuple}[1]{\left(#1\right)}

% numerical shortcuts.
\newcommand{\abs}[1]{\left| #1\right|}
\newcommand{\floor}[1]{\left\lfloor #1 \right\rfloor}
\newcommand{\ceil}[1]{\left\lceil #1 \right\rceil}

% linear algebra shortcuts.
\newcommand{\change}{\Delta}
\newcommand{\norm}[1]{\left\| #1\right\|}
\newcommand{\dprod}[1]{\langle#1\rangle}
\newcommand{\linspan}[1]{\langle#1\rangle}
\newcommand{\conj}[1]{\overline{#1}}
\newcommand{\gradient}{\nabla}
\newcommand{\der}{\frac{d}{dx}}
\newcommand{\lap}{\Delta}
\newcommand{\kron}{\otimes}
\newcommand{\nperp}{\nvdash}

\newcommand{\mat}[1]{\left( \begin{smallmatrix}#1 \end{smallmatrix} \right)}

% derivatives and limits
\newcommand{\partder}[2]{\frac{\partial #1}{\partial #2}}
\newcommand{\partdern}[3]{\frac{\partial^{#3} #1}{\partial #2^{#3}}}

% Arrows
\newcommand{\diverge}{\nearrow}
\newcommand{\notto}{\nrightarrow}
\newcommand{\up}{\uparrow}
\newcommand{\down}{\downarrow}
% gets and gives are defined!

% ordering operators
\newcommand{\oleq}{\preceq}
\newcommand{\ogeq}{\succeq}

% programming and logic operators
\newcommand{\dfn}{:=}
\newcommand{\assign}{:=}
\newcommand{\co}{\ co\ }
\newcommand{\en}{\ en\ }


% logic operators
\newcommand{\xor}{\oplus}
\newcommand{\Land}{\bigwedge}
\newcommand{\Lor}{\bigvee}
\newcommand{\finish}{$\Box$}
\newcommand{\contra}{\Rightarrow \Leftarrow}
\newcommand{\iseq}{\stackrel{_?}{=}}


% Set theory
\newcommand{\symdiff}{\Delta}
\newcommand{\union}{\cup}
\newcommand{\inters}{\cap}
\newcommand{\Union}{\bigcup}
\newcommand{\Inters}{\bigcap}
\newcommand{\nullSet}{\phi}

% graph theory
\newcommand{\nbd}{\Gamma}

% Script alphabets
% For reals, use \Re

% greek letters
\newcommand{\eps}{\epsilon}
\newcommand{\del}{\delta}
\newcommand{\ga}{\alpha}
\newcommand{\gb}{\beta}
\newcommand{\gd}{\del}
\newcommand{\gf}{\phi}
\newcommand{\gF}{\Phi}
\newcommand{\gl}{\lambda}
\newcommand{\gm}{\mu}
\newcommand{\gn}{\nu}
\newcommand{\gr}{\rho}
\newcommand{\gs}{\sigma}
\newcommand{\gt}{\theta}
\newcommand{\gx}{\xi}

\newcommand{\sw}{\sigma}
\newcommand{\SW}{\Sigma}
\newcommand{\ew}{\lambda}
\newcommand{\EW}{\Lambda}

\newcommand{\Del}{\Delta}
\newcommand{\gD}{\Delta}
\newcommand{\gG}{\Gamma}
\newcommand{\gO}{\Omega}
\newcommand{\gL}{\Lambda}
\newcommand{\gS}{\Sigma}

% Formatting shortcuts
\newcommand{\red}[1]{\textcolor{red}{#1}}
\newcommand{\blue}[1]{\textcolor{blue}{#1}}
\newcommand{\htext}[2]{\texorpdfstring{#1}{#2}}

% Statistics
\newcommand{\distr}{\sim}
\newcommand{\stddev}{\sigma}
\newcommand{\covmatrix}{\Sigma}
\newcommand{\mean}{\mu}
\newcommand{\param}{\gt}
\newcommand{\ftr}{\phi}

% General utility
\newcommand{\todo}[1]{\footnote{TODO: #1}}
\newcommand{\exclaim}[1]{{\textbf{\textit{#1}}}}
\newcommand{\tbc}{[\textbf{Incomplete}]}
\newcommand{\chk}{[\textbf{Check}]}
\newcommand{\oprob}{[\textbf{OP}]:}
\newcommand{\core}[1]{\textbf{Core Idea:}}
\newcommand{\why}{[\textbf{Find proof}]}
\newcommand{\opt}[1]{\textit{#1}}


\DeclareMathOperator*{\argmin}{arg\,min}
\DeclareMathOperator{\rank}{rank}
\newcommand{\redcol}[1]{\textcolor{red}{#1}}
\newcommand{\bluecol}[1]{\textcolor{blue}{#1}}
\newcommand{\greencol}[1]{\textcolor{green}{#1}}


\renewcommand{\~}{\htext{$\sim$}{~}}


% groupings of objects.
\newcommand{\set}[1]{\left\{ #1 \right\}}
\newcommand{\seq}[1]{\left(#1\right)}
\newcommand{\ang}[1]{\langle#1\rangle}
\newcommand{\tuple}[1]{\left(#1\right)}

% numerical shortcuts.
\newcommand{\abs}[1]{\left| #1\right|}
\newcommand{\floor}[1]{\left\lfloor #1 \right\rfloor}
\newcommand{\ceil}[1]{\left\lceil #1 \right\rceil}

% linear algebra shortcuts.
\newcommand{\change}{\Delta}
\newcommand{\norm}[1]{\left\| #1\right\|}
\newcommand{\dprod}[1]{\langle#1\rangle}
\newcommand{\linspan}[1]{\langle#1\rangle}
\newcommand{\conj}[1]{\overline{#1}}
\newcommand{\gradient}{\nabla}
\newcommand{\der}{\frac{d}{dx}}
\newcommand{\lap}{\Delta}
\newcommand{\kron}{\otimes}
\newcommand{\nperp}{\nvdash}

\newcommand{\mat}[1]{\left( \begin{smallmatrix}#1 \end{smallmatrix} \right)}

% derivatives and limits
\newcommand{\partder}[2]{\frac{\partial #1}{\partial #2}}
\newcommand{\partdern}[3]{\frac{\partial^{#3} #1}{\partial #2^{#3}}}

% Arrows
\newcommand{\diverge}{\nearrow}
\newcommand{\notto}{\nrightarrow}
\newcommand{\up}{\uparrow}
\newcommand{\down}{\downarrow}
% gets and gives are defined!

% ordering operators
\newcommand{\oleq}{\preceq}
\newcommand{\ogeq}{\succeq}

% programming and logic operators
\newcommand{\dfn}{:=}
\newcommand{\assign}{:=}
\newcommand{\co}{\ co\ }
\newcommand{\en}{\ en\ }


% logic operators
\newcommand{\xor}{\oplus}
\newcommand{\Land}{\bigwedge}
\newcommand{\Lor}{\bigvee}
\newcommand{\finish}{$\Box$}
\newcommand{\contra}{\Rightarrow \Leftarrow}
\newcommand{\iseq}{\stackrel{_?}{=}}


% Set theory
\newcommand{\symdiff}{\Delta}
\newcommand{\union}{\cup}
\newcommand{\inters}{\cap}
\newcommand{\Union}{\bigcup}
\newcommand{\Inters}{\bigcap}
\newcommand{\nullSet}{\phi}

% graph theory
\newcommand{\nbd}{\Gamma}

% Script alphabets
% For reals, use \Re

% greek letters
\newcommand{\eps}{\epsilon}
\newcommand{\del}{\delta}
\newcommand{\ga}{\alpha}
\newcommand{\gb}{\beta}
\newcommand{\gd}{\del}
\newcommand{\gf}{\phi}
\newcommand{\gF}{\Phi}
\newcommand{\gl}{\lambda}
\newcommand{\gm}{\mu}
\newcommand{\gn}{\nu}
\newcommand{\gr}{\rho}
\newcommand{\gs}{\sigma}
\newcommand{\gt}{\theta}
\newcommand{\gx}{\xi}

\newcommand{\sw}{\sigma}
\newcommand{\SW}{\Sigma}
\newcommand{\ew}{\lambda}
\newcommand{\EW}{\Lambda}

\newcommand{\Del}{\Delta}
\newcommand{\gD}{\Delta}
\newcommand{\gG}{\Gamma}
\newcommand{\gO}{\Omega}
\newcommand{\gL}{\Lambda}
\newcommand{\gS}{\Sigma}

% Formatting shortcuts
\newcommand{\red}[1]{\textcolor{red}{#1}}
\newcommand{\blue}[1]{\textcolor{blue}{#1}}
\newcommand{\htext}[2]{\texorpdfstring{#1}{#2}}

% Statistics
\newcommand{\distr}{\sim}
\newcommand{\stddev}{\sigma}
\newcommand{\covmatrix}{\Sigma}
\newcommand{\mean}{\mu}
\newcommand{\param}{\gt}
\newcommand{\ftr}{\phi}

% General utility
\newcommand{\todo}[1]{\footnote{TODO: #1}}
\newcommand{\exclaim}[1]{{\textbf{\textit{#1}}}}
\newcommand{\tbc}{[\textbf{Incomplete}]}
\newcommand{\chk}{[\textbf{Check}]}
\newcommand{\oprob}{[\textbf{OP}]:}
\newcommand{\core}[1]{\textbf{Core Idea:}}
\newcommand{\why}{[\textbf{Find proof}]}
\newcommand{\opt}[1]{\textit{#1}}


\DeclareMathOperator*{\argmin}{arg\,min}
\DeclareMathOperator{\rank}{rank}
\newcommand{\redcol}[1]{\textcolor{red}{#1}}
\newcommand{\bluecol}[1]{\textcolor{blue}{#1}}
\newcommand{\greencol}[1]{\textcolor{green}{#1}}


\renewcommand{\~}{\htext{$\sim$}{~}}


\newtheorem{thm}{Theorem}[subsection]
\newtheorem{cor}[thm]{Corollary}
\newtheorem{lem}[thm]{Lemma}
\newtheorem{claim}[thm]{Claim}

\theoremstyle{remark}
\newtheorem{defn}[thm]{Definition}
\newtheorem*{notation}{Notation}
\newtheorem*{ack}{Acknowledgement}
\newtheorem{alg}[thm]{Algorithm}
\newtheorem{rem}[thm]{Remark}

%opening
\title{Computational Learning Theory: Answer to Homework 1}
\author{vishvAs vAsuki}

\begin{document}

\maketitle

\section{}
\begin{notation}
Boolean function f is k closed := f can be written as both a DNF d and as a CNF c; each of which have max term length k.

Input x is of length n. Algorithm L evaluates f(x). L knows c and d.
\end{notation}

\begin{thm}
$\exists$ L which evaluates f(x) by looking only at $O(k^{2})$ variables.
\end{thm}
\begin{proof}
Don't know. Hint: Begin evaluating the variables DNF, use this to reduce the number of terms in the CNF and vice versa.
\end{proof}

\section{}
\begin{thm}
There exists an algorithm L which can learn DNF formalae on n variables with term length t with mistake bound and running time $O(n^{t})$.
\end{thm}
\begin{proof}
This can easily be solved by using the canonical disjunction learning algorithm after feature expansion. We create new variables representing all possible combinations of n variables. These new variables are $O((2n+1)^{t}) = O(n^{t})$ in number.

We know that disjunctions of $O(n^{t})$ variables can be learnt with mistake bound and running time $O(n^{t})$ using the algorithm which starts by including all literals in the hypothesis disjunction, and removes all literals evaluated to be true on an example rejected by the target concept.
\end{proof}

\section{}
p(x) and q(x) are polynomials. Rational function $R(x) = \frac{p(x)}{q(x)}$. Degree(R) = max(Degree(p), Degree(q)). Halfspace $f = sign(\sum_{i=1^{n}}a_{i}x_{i} - c) = sign(l_{f}(x))$, $a_{i}, c \in Z$. $g(x) = sign(l_{g}(x))$ is another halfspace. Weight of halfspace = $\sum |a_{i}| + |c|$. f and g are of weight W. $h = f \wedge g$, the AND of f and g, is a boolean function.

\begin{rem}
There exists R(x) of degree $O(l\log t)$:\\
If $x \in [1,2^{l}]$, $R(x) \in (1,1+t^{-1})$.
If $x \in [-2^{l}, -1]$, $R(x) \in (-1-t^{-1}. -1)$.

$R(x) = \frac{p(x)}{q(x)} > 0$ if and only if $p(x)q(x) > 0$. $R(x) = \frac{p(x)}{q(x)} < 0$ if and only if $p(x)q(x) < 0$. 
\end{rem}

\begin{thm}
h is learnable in the mistake bounded model in time and mistake bound $n^{(O(\log W))}$.
\end{thm}
\begin{proof}
\begin{rem}
For the rest of the proof, let R(x), p(x), q(x) be degree $O(t \log W)$ functions useful in approximating the sign function of the numbers $|x| \in [1,W]$, with the accuracy $t=1/10$.
\end{rem}
Let p()q() be used in stead of the sign function in f(x) and g(x).

Consider the polynomial $p_{h}(x) = (f(x)+1)(g(x)+1) - 2^{-1}$. $p_{h}(x)>1$ if and only if p()q() shows that both f(x) and g(x) are greater than 1; and  $p_{h}(x)<1$ if and only if p()q() shows that either f(x) or g(x) or both are lesser than 1.

$degree(f(x)) = degree(p(l_{f}(x))q(l_{f}(x))) = O(\log W$).\\
$degree(g(x)) = degree(p(l_{g}(x))q(l_{g}(x))) = O(\log W$).

So, $degree(p_{h}(x)) = O(\log W)$. $sign(p_{h}(x))$ is our PTF.

We know that PTF's of the degree $O(\log W)$ can be learnt in $n^{O(\log W)}$ time and mistake bound (by reduction to a halfspace learning problem, which can inturn be reduced to a linear programming problem).
\end{proof}

\section{}
An algorithm L in the mistake bounded model is 'careful' if it does nothing after amking a correct prediction. It only updates itself after making a mistake. C is a concept class.

\begin{thm}
If C is efficiently learnable by algorithm A with mistake bound t, then there exists a careful algorithm L which learns C with mistake bound t.
\end{thm}
\begin{proof}
An algorithm A has mistake bound M(c) for learning some concept $c \in C$ if A makes at most M mistakes on any sequence that is consistent with concept c. A has mistake bound $M(C) = max_{c \in C} M(c)$. M(C) = t.

Let h be the hypothesis maintained by A. Let c be the target concept.

Let L be constructed as follows: Whenever L sees an example x, L calls A as a subroutine, returns h(x) and keeps a copy of h. Then L informs A of whether $c(x) ?= h(x)$. Then, A may or may not update h to hypothesis h'. If $c(x) = h(x)$, L restores h. If $c(x) \neq h(x)$, L does allows A to update h.

Suppose that L makes a sequence of $m>t$ mistakes on examples $\set{x_{i}}$. Then, due to the construction of L, this implies that A would have made $m>t$ mistakes on that same sequence of examples. But this would contradict our initial assumption that the mistake bound of A is t. Hence, L cannot make more than t mistakes. Hence L has mistake bound t.
\end{proof}

\section{}
\begin{ack}
The algorithm L' is due to Alex Tang.
\end{ack}

Infinite feature model for learning decision lists is like the mistake bounded model, with the following changes: Infinite literals $\set{x_{i}}$ may appear in a decision list. But, any fixed decision list depends on only k of them. Any example is a list of at most n literals set to 1, the others are 0.

\begin{rem}
Hints: "Problem 4 may help you think about problem 5". "This problem has very little to do with decision llists: there is a general transformation that will work for almost any concept class."
\end{rem}

\begin{thm}
If we have an efficient algorithm L, in the mistake bounded model, for learning decision lists of length k over a universe of n literals with mistake bound k log n: This implies that decision lists would be learnable in the infinite feature model with mistake bound polynomial in n and polynomial running time.
\end{thm}
\begin{proof}
\begin{notation}
Let t(n) be the mistake bound of L. Let N be the set of literals used by L. Only k literals appear in a decision list. Let L' be the algorithm learning decision lists in the infinite feature model.
\end{notation}

L' is constructed as follows:

L' uses L and N to respond to label the examples. Initially, N is empty.

When L makes more than $t(|N|)$ mistakes, we know that N does not include all the relevant variables. When L makes more than $t(|N|)$ mistakes, we would have seen atleast 1 relevant literal not already in N. (This is proved in the lemma below.) So, we add the $nt(|N|)$ literals, corresponding to the $t(|N|)$ mistakes, to N. So, N now has $n+nt(n) \leq cnt(n)$ literals, where c is a constant. We restart L with this new set of literals, N.

After each failure, we are adding at least 1 relevant literal. So, after k such iterations, we should have all the relevant literals in
N. So, the total mistake bound is $mb = t(n) + t(cnt(n)) + t(cnt(cnt(n))) ... $. As t(n) = O(klog n), we have a mistake bound of $O(k^{2} \log kn)$ for the algorithm. In general, if t(n) = O(n), we have $mb = O(kn) + O(k^{2}n^2) .. O(k^{n}n^k) = O(k^{n}n^k)$.

As L' uses L to process examples, the computation required per example remains the same. Hence, L' has running time polynomial in n and k.

\end{proof}

\begin{lem}
Consider the algorithm L' described above. When L makes more than $t(|N|)$ mistakes, we would have seen atleast 1 relevant literal not already in N.
\end{lem}
\begin{proof}
\begin{notation}
Let R be the set of relevent variables in N. Let c be the target decision list. Let c' be a 'truncated' decision list, where every literal not in R is removed from c. As the length of c is at most k, the length of c' is at most k.

Let M be the sequence of 'examples on which L makes mistakes'.
\end{notation}

Proof by contradiction. Suppose that L made more than $t(|N|)$ mistakes, but did not see even 1 relevant literal not in R.

We assume without loss of generality that L is a 'careful' algorithm, as defined in the previous section.

So, the labels corresponding to M can be seen as being equal to c'(x). So, L, given a situation where c' is the target concept and the sequence of examples it is presented corresponds to M; would have made more than t(n) mistakes. This contradicts our assumption that L has mistake bound t(n), leading us to absurdity.
\end{proof}

% \bibliographystyle{plain}
% \bibliography{../colt}


\end{document}
