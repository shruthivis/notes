\documentclass[11pt]{article}
\usepackage{caption}
\usepackage{amsmath, amssymb}
\usepackage{hyperref, graphicx, verbatim, listings, multirow, subfigure}
\usepackage{algorithm, algorithmic}
% \usepackage[bottom]{footmisc}
\lstset{breaklines=true}
\setcounter{tocdepth}{3}

% Lets verbatim and verb environments automatically break lines.
\makeatletter
\def\@xobeysp{ }
\makeatother
% \lstset{breaklines=true,basicstyle=\ttfamily}

% Use something like:
% % Use something like:
% % Use something like:
% \input{../../macros}

% groupings of objects.
\newcommand{\set}[1]{\left\{ #1 \right\}}
\newcommand{\seq}[1]{\left(#1\right)}
\newcommand{\ang}[1]{\langle#1\rangle}
\newcommand{\tuple}[1]{\left(#1\right)}

% numerical shortcuts.
\newcommand{\abs}[1]{\left| #1\right|}
\newcommand{\floor}[1]{\left\lfloor #1 \right\rfloor}
\newcommand{\ceil}[1]{\left\lceil #1 \right\rceil}

% linear algebra shortcuts.
\newcommand{\change}{\Delta}
\newcommand{\norm}[1]{\left\| #1\right\|}
\newcommand{\dprod}[1]{\langle#1\rangle}
\newcommand{\linspan}[1]{\langle#1\rangle}
\newcommand{\conj}[1]{\overline{#1}}
\newcommand{\gradient}{\nabla}
\newcommand{\der}{\frac{d}{dx}}
\newcommand{\lap}{\Delta}
\newcommand{\kron}{\otimes}
\newcommand{\nperp}{\nvdash}

\newcommand{\mat}[1]{\left( \begin{smallmatrix}#1 \end{smallmatrix} \right)}

% derivatives and limits
\newcommand{\partder}[2]{\frac{\partial #1}{\partial #2}}
\newcommand{\partdern}[3]{\frac{\partial^{#3} #1}{\partial #2^{#3}}}

% Arrows
\newcommand{\diverge}{\nearrow}
\newcommand{\notto}{\nrightarrow}
\newcommand{\up}{\uparrow}
\newcommand{\down}{\downarrow}
% gets and gives are defined!

% ordering operators
\newcommand{\oleq}{\preceq}
\newcommand{\ogeq}{\succeq}

% programming and logic operators
\newcommand{\dfn}{:=}
\newcommand{\assign}{:=}
\newcommand{\co}{\ co\ }
\newcommand{\en}{\ en\ }


% logic operators
\newcommand{\xor}{\oplus}
\newcommand{\Land}{\bigwedge}
\newcommand{\Lor}{\bigvee}
\newcommand{\finish}{$\Box$}
\newcommand{\contra}{\Rightarrow \Leftarrow}
\newcommand{\iseq}{\stackrel{_?}{=}}


% Set theory
\newcommand{\symdiff}{\Delta}
\newcommand{\union}{\cup}
\newcommand{\inters}{\cap}
\newcommand{\Union}{\bigcup}
\newcommand{\Inters}{\bigcap}
\newcommand{\nullSet}{\phi}

% graph theory
\newcommand{\nbd}{\Gamma}

% Script alphabets
% For reals, use \Re

% greek letters
\newcommand{\eps}{\epsilon}
\newcommand{\del}{\delta}
\newcommand{\ga}{\alpha}
\newcommand{\gb}{\beta}
\newcommand{\gd}{\del}
\newcommand{\gf}{\phi}
\newcommand{\gF}{\Phi}
\newcommand{\gl}{\lambda}
\newcommand{\gm}{\mu}
\newcommand{\gn}{\nu}
\newcommand{\gr}{\rho}
\newcommand{\gs}{\sigma}
\newcommand{\gt}{\theta}
\newcommand{\gx}{\xi}

\newcommand{\sw}{\sigma}
\newcommand{\SW}{\Sigma}
\newcommand{\ew}{\lambda}
\newcommand{\EW}{\Lambda}

\newcommand{\Del}{\Delta}
\newcommand{\gD}{\Delta}
\newcommand{\gG}{\Gamma}
\newcommand{\gO}{\Omega}
\newcommand{\gL}{\Lambda}
\newcommand{\gS}{\Sigma}

% Formatting shortcuts
\newcommand{\red}[1]{\textcolor{red}{#1}}
\newcommand{\blue}[1]{\textcolor{blue}{#1}}
\newcommand{\htext}[2]{\texorpdfstring{#1}{#2}}

% Statistics
\newcommand{\distr}{\sim}
\newcommand{\stddev}{\sigma}
\newcommand{\covmatrix}{\Sigma}
\newcommand{\mean}{\mu}
\newcommand{\param}{\gt}
\newcommand{\ftr}{\phi}

% General utility
\newcommand{\todo}[1]{\footnote{TODO: #1}}
\newcommand{\exclaim}[1]{{\textbf{\textit{#1}}}}
\newcommand{\tbc}{[\textbf{Incomplete}]}
\newcommand{\chk}{[\textbf{Check}]}
\newcommand{\oprob}{[\textbf{OP}]:}
\newcommand{\core}[1]{\textbf{Core Idea:}}
\newcommand{\why}{[\textbf{Find proof}]}
\newcommand{\opt}[1]{\textit{#1}}


\DeclareMathOperator*{\argmin}{arg\,min}
\DeclareMathOperator{\rank}{rank}
\newcommand{\redcol}[1]{\textcolor{red}{#1}}
\newcommand{\bluecol}[1]{\textcolor{blue}{#1}}
\newcommand{\greencol}[1]{\textcolor{green}{#1}}


\renewcommand{\~}{\htext{$\sim$}{~}}


% groupings of objects.
\newcommand{\set}[1]{\left\{ #1 \right\}}
\newcommand{\seq}[1]{\left(#1\right)}
\newcommand{\ang}[1]{\langle#1\rangle}
\newcommand{\tuple}[1]{\left(#1\right)}

% numerical shortcuts.
\newcommand{\abs}[1]{\left| #1\right|}
\newcommand{\floor}[1]{\left\lfloor #1 \right\rfloor}
\newcommand{\ceil}[1]{\left\lceil #1 \right\rceil}

% linear algebra shortcuts.
\newcommand{\change}{\Delta}
\newcommand{\norm}[1]{\left\| #1\right\|}
\newcommand{\dprod}[1]{\langle#1\rangle}
\newcommand{\linspan}[1]{\langle#1\rangle}
\newcommand{\conj}[1]{\overline{#1}}
\newcommand{\gradient}{\nabla}
\newcommand{\der}{\frac{d}{dx}}
\newcommand{\lap}{\Delta}
\newcommand{\kron}{\otimes}
\newcommand{\nperp}{\nvdash}

\newcommand{\mat}[1]{\left( \begin{smallmatrix}#1 \end{smallmatrix} \right)}

% derivatives and limits
\newcommand{\partder}[2]{\frac{\partial #1}{\partial #2}}
\newcommand{\partdern}[3]{\frac{\partial^{#3} #1}{\partial #2^{#3}}}

% Arrows
\newcommand{\diverge}{\nearrow}
\newcommand{\notto}{\nrightarrow}
\newcommand{\up}{\uparrow}
\newcommand{\down}{\downarrow}
% gets and gives are defined!

% ordering operators
\newcommand{\oleq}{\preceq}
\newcommand{\ogeq}{\succeq}

% programming and logic operators
\newcommand{\dfn}{:=}
\newcommand{\assign}{:=}
\newcommand{\co}{\ co\ }
\newcommand{\en}{\ en\ }


% logic operators
\newcommand{\xor}{\oplus}
\newcommand{\Land}{\bigwedge}
\newcommand{\Lor}{\bigvee}
\newcommand{\finish}{$\Box$}
\newcommand{\contra}{\Rightarrow \Leftarrow}
\newcommand{\iseq}{\stackrel{_?}{=}}


% Set theory
\newcommand{\symdiff}{\Delta}
\newcommand{\union}{\cup}
\newcommand{\inters}{\cap}
\newcommand{\Union}{\bigcup}
\newcommand{\Inters}{\bigcap}
\newcommand{\nullSet}{\phi}

% graph theory
\newcommand{\nbd}{\Gamma}

% Script alphabets
% For reals, use \Re

% greek letters
\newcommand{\eps}{\epsilon}
\newcommand{\del}{\delta}
\newcommand{\ga}{\alpha}
\newcommand{\gb}{\beta}
\newcommand{\gd}{\del}
\newcommand{\gf}{\phi}
\newcommand{\gF}{\Phi}
\newcommand{\gl}{\lambda}
\newcommand{\gm}{\mu}
\newcommand{\gn}{\nu}
\newcommand{\gr}{\rho}
\newcommand{\gs}{\sigma}
\newcommand{\gt}{\theta}
\newcommand{\gx}{\xi}

\newcommand{\sw}{\sigma}
\newcommand{\SW}{\Sigma}
\newcommand{\ew}{\lambda}
\newcommand{\EW}{\Lambda}

\newcommand{\Del}{\Delta}
\newcommand{\gD}{\Delta}
\newcommand{\gG}{\Gamma}
\newcommand{\gO}{\Omega}
\newcommand{\gL}{\Lambda}
\newcommand{\gS}{\Sigma}

% Formatting shortcuts
\newcommand{\red}[1]{\textcolor{red}{#1}}
\newcommand{\blue}[1]{\textcolor{blue}{#1}}
\newcommand{\htext}[2]{\texorpdfstring{#1}{#2}}

% Statistics
\newcommand{\distr}{\sim}
\newcommand{\stddev}{\sigma}
\newcommand{\covmatrix}{\Sigma}
\newcommand{\mean}{\mu}
\newcommand{\param}{\gt}
\newcommand{\ftr}{\phi}

% General utility
\newcommand{\todo}[1]{\footnote{TODO: #1}}
\newcommand{\exclaim}[1]{{\textbf{\textit{#1}}}}
\newcommand{\tbc}{[\textbf{Incomplete}]}
\newcommand{\chk}{[\textbf{Check}]}
\newcommand{\oprob}{[\textbf{OP}]:}
\newcommand{\core}[1]{\textbf{Core Idea:}}
\newcommand{\why}{[\textbf{Find proof}]}
\newcommand{\opt}[1]{\textit{#1}}


\DeclareMathOperator*{\argmin}{arg\,min}
\DeclareMathOperator{\rank}{rank}
\newcommand{\redcol}[1]{\textcolor{red}{#1}}
\newcommand{\bluecol}[1]{\textcolor{blue}{#1}}
\newcommand{\greencol}[1]{\textcolor{green}{#1}}


\renewcommand{\~}{\htext{$\sim$}{~}}


% groupings of objects.
\newcommand{\set}[1]{\left\{ #1 \right\}}
\newcommand{\seq}[1]{\left(#1\right)}
\newcommand{\ang}[1]{\langle#1\rangle}
\newcommand{\tuple}[1]{\left(#1\right)}

% numerical shortcuts.
\newcommand{\abs}[1]{\left| #1\right|}
\newcommand{\floor}[1]{\left\lfloor #1 \right\rfloor}
\newcommand{\ceil}[1]{\left\lceil #1 \right\rceil}

% linear algebra shortcuts.
\newcommand{\change}{\Delta}
\newcommand{\norm}[1]{\left\| #1\right\|}
\newcommand{\dprod}[1]{\langle#1\rangle}
\newcommand{\linspan}[1]{\langle#1\rangle}
\newcommand{\conj}[1]{\overline{#1}}
\newcommand{\gradient}{\nabla}
\newcommand{\der}{\frac{d}{dx}}
\newcommand{\lap}{\Delta}
\newcommand{\kron}{\otimes}
\newcommand{\nperp}{\nvdash}

\newcommand{\mat}[1]{\left( \begin{smallmatrix}#1 \end{smallmatrix} \right)}

% derivatives and limits
\newcommand{\partder}[2]{\frac{\partial #1}{\partial #2}}
\newcommand{\partdern}[3]{\frac{\partial^{#3} #1}{\partial #2^{#3}}}

% Arrows
\newcommand{\diverge}{\nearrow}
\newcommand{\notto}{\nrightarrow}
\newcommand{\up}{\uparrow}
\newcommand{\down}{\downarrow}
% gets and gives are defined!

% ordering operators
\newcommand{\oleq}{\preceq}
\newcommand{\ogeq}{\succeq}

% programming and logic operators
\newcommand{\dfn}{:=}
\newcommand{\assign}{:=}
\newcommand{\co}{\ co\ }
\newcommand{\en}{\ en\ }


% logic operators
\newcommand{\xor}{\oplus}
\newcommand{\Land}{\bigwedge}
\newcommand{\Lor}{\bigvee}
\newcommand{\finish}{$\Box$}
\newcommand{\contra}{\Rightarrow \Leftarrow}
\newcommand{\iseq}{\stackrel{_?}{=}}


% Set theory
\newcommand{\symdiff}{\Delta}
\newcommand{\union}{\cup}
\newcommand{\inters}{\cap}
\newcommand{\Union}{\bigcup}
\newcommand{\Inters}{\bigcap}
\newcommand{\nullSet}{\phi}

% graph theory
\newcommand{\nbd}{\Gamma}

% Script alphabets
% For reals, use \Re

% greek letters
\newcommand{\eps}{\epsilon}
\newcommand{\del}{\delta}
\newcommand{\ga}{\alpha}
\newcommand{\gb}{\beta}
\newcommand{\gd}{\del}
\newcommand{\gf}{\phi}
\newcommand{\gF}{\Phi}
\newcommand{\gl}{\lambda}
\newcommand{\gm}{\mu}
\newcommand{\gn}{\nu}
\newcommand{\gr}{\rho}
\newcommand{\gs}{\sigma}
\newcommand{\gt}{\theta}
\newcommand{\gx}{\xi}

\newcommand{\sw}{\sigma}
\newcommand{\SW}{\Sigma}
\newcommand{\ew}{\lambda}
\newcommand{\EW}{\Lambda}

\newcommand{\Del}{\Delta}
\newcommand{\gD}{\Delta}
\newcommand{\gG}{\Gamma}
\newcommand{\gO}{\Omega}
\newcommand{\gL}{\Lambda}
\newcommand{\gS}{\Sigma}

% Formatting shortcuts
\newcommand{\red}[1]{\textcolor{red}{#1}}
\newcommand{\blue}[1]{\textcolor{blue}{#1}}
\newcommand{\htext}[2]{\texorpdfstring{#1}{#2}}

% Statistics
\newcommand{\distr}{\sim}
\newcommand{\stddev}{\sigma}
\newcommand{\covmatrix}{\Sigma}
\newcommand{\mean}{\mu}
\newcommand{\param}{\gt}
\newcommand{\ftr}{\phi}

% General utility
\newcommand{\todo}[1]{\footnote{TODO: #1}}
\newcommand{\exclaim}[1]{{\textbf{\textit{#1}}}}
\newcommand{\tbc}{[\textbf{Incomplete}]}
\newcommand{\chk}{[\textbf{Check}]}
\newcommand{\oprob}{[\textbf{OP}]:}
\newcommand{\core}[1]{\textbf{Core Idea:}}
\newcommand{\why}{[\textbf{Find proof}]}
\newcommand{\opt}[1]{\textit{#1}}


\DeclareMathOperator*{\argmin}{arg\,min}
\DeclareMathOperator{\rank}{rank}
\newcommand{\redcol}[1]{\textcolor{red}{#1}}
\newcommand{\bluecol}[1]{\textcolor{blue}{#1}}
\newcommand{\greencol}[1]{\textcolor{green}{#1}}


\renewcommand{\~}{\htext{$\sim$}{~}}


%opening
\title{Affiliation recommendation using auxiliary friendship networks}
\author{Vishvas Vasuki, Nagarajan Natarajan, Zhengdong Lu, Inderjit Dhillon}
% \date{}

\begin{document}
\maketitle
% \tableofcontents

\section{Abstract}
\begin{enumerate}
 \item Introduce and motivate the problem
 \subitem Introduce social network analyis.
 \subitem Introduce the affiliation recommendation problem.
 \subitem Introduce affiliation networks.
 \subitem We consider a particular case: community recommendation.
 \subitem In general, affiliations can be items. Also, there are applications in biology: gene - disease networks.
 
 \item Key contributions
 \subitem Simple way of combining the networks. Based on this, we explore two classes of algorithms.
 \subsubitem Graph proximity based approaches common in link prediction. The use of such link prediction techniques in recommender systems is, to our knowledge, novel.
 \subsubitem Approaches based on the latent factor model.
 \subsubitem Graph proximity based approaches do better.
 \subitem Propose a way of evaluating affiliation recommendations. We demonstrate the importance of designing the right evaluation strategy.
 
 \item State that you evaluate on Youtube and Orkut, by seeing how good the top 50 recommendations are.
\end{enumerate}

\section{Introduction}
\begin{enumerate}
 \item Introduce and motivate problem, as in abstract, but in more detail.
 \item Describe key contributions, as in abstract, but in more detail.
\end{enumerate}

\subsection{Overview}
\begin{enumerate}
 \item The table of contents as a paragraph.
\end{enumerate}

\section{Models}
\begin{enumerate}
 \item Overview of the section.
 \item Establish notation.
 \item Pose the affiliation recommendation problem as a ranking problem. The methods we describe to solve this problem rely on assigning scores to various items. Describe the score matrix.
\end{enumerate}

\subsection{Prediction on the combined graph}
\begin{enumerate}
 \item This is our way of combining the social and affiliation networks. Describe the joint adjacency matrix.
 \item Briefly introduce the scoring approaches based on graph proximity and based on latent factors approach.
\end{enumerate}

\subsection{Graph proximity model}
\begin{enumerate}
 \item Introduce the basic Katz measure for general graphs.
 \item Extend Katz measure to bipartite graphs. Interpret it as a weighted combination of paths.
 \item Extend Katz measure to the combined graph C. Interpret it as a weighted combination of paths.
 \item Introduce truncated Katz measure.
 \item Analysis of computational cost.
\end{enumerate}

\subsection{Latent factors model}
\begin{enumerate}
 \item Overview of the section.
 \item Say that the 0 entries in A are actually unknown, but that there is a huge prior on them actually being 0.
 \item Model the adjascency matrix as arising out of inner products between user and group factors, which are low dimensional representations of users and groups. Users and groups with 'high' inner products are likely to be connected to each other.
 \item Build the combined matrix C with unknown entries in the A part, and model this as arising out of interactions between user and group factors.
 \item Describe the objective being minimized.
 \item Describe role of $\gl$ in C.
 \item Describe various potential choices for D. Observe that choice of D is not very important, according to experiments.
 \item Describe SVD(C) as the solution to the above optimization problem.
 \item Analysis of computational cost.
\end{enumerate}

\section{Related work}
\begin{enumerate}
 \item Overview of the section.
 
 \item Describe the area of recommendation systems research.
  \subitem Community recommendation is less studied.
 
 \item Probabilistic collaborative filtering: Prior work.
  \subitem Community recommendation using LDA by Chen et al: They have not used the social network in making recommendations.
  \subitem Relate their LDA based approach to the latent factors approach. Describe LDA's connections with pLSA and LSA, which is just SVD.
  \subitem Combinatorial collaborative filtering is also based on LSA, but uses text descriptions rather than social networks.
 
 \item Prior work in joint matrix factorization.
  \subitem Compare with Linked Matrix Factorization by Tang et al.
  \subitem Compare with Collective Matrix Factorization by Singh et al, which generalizes this.
  \subitem Latent factors approach we propose are much more efficient than them: involve computing SVD, rather than using optimization techniques based on alternating least squares.
  
 \item Prior work in modelling and studying the co-evolution of social and affiliation networks.
 \subitem Inspires our community recommendation effort.
 
 \item Prior work in social network analysis.
  \subitem Inspires our graph proximity based scoring models.
 
 \item Our attempt to use link prediction/ graph proximity based techniques is novel.
\end{enumerate}

\section{Experimental evaluation}
\begin{enumerate}
 \item Overview of the section.
\end{enumerate}
 
\subsection{Data}
\begin{enumerate}
 \item Describe datasets.
  \subitem Present some statistics. Use bar graphs like those in Orkut and Youtube. Make observations about them.
 \item Describe the test, training and validation sets.
  \subitem Describe how the (per-user) test set is created.
  \subitem Describe how validation set is created. Mention the number of predictions made during validation, in comparing various parameters.
\end{enumerate}

\subsection{Evaluation method}
\begin{enumerate}
 \item Overview of the subsection.
 \item Describe sensitivity, specificity, precision.
 \item Describe AUC, ROC, the use of the appropriate slice of ROC in evaluating the performance of a predictor in making the first 50 predictions.
 \item Note the robustness of the results: sensitivities and specificities were averaged over 9500 users in Orkut and 16000 users in Youtube.
 \item Note the importance of using the right evaluation method. Contrast with results seen while using link prediction-style evaluation methods : Include the bar graph.
  \subitem Show by algebra that different quantities are being measured in the two evaluation strategies. Note that global sensitivity measurement is same as taking a weighted average over sensitivities.
  \subitem Emphasize this as a contribution of this paper.
\end{enumerate}

\subsection{Results and discussion}
\begin{enumerate}
 \item Overview of the subsection.
 
 \item Describe performance of graph proximity based methods.
  \subitem For the average user, graph proximity methods are much better.
  \subitem Note that use of social network in recommendation is important.
  \subitem Specify learned parameters.
 
 \item Discuss performance of methods based on the latent factors model.
  \subitem Note that using C makes a big difference. Note that use of social network in recommendation is important.
  \subitem Note that using various choices for the group-group part does not make significant difference.
  \subitem Specify learned parameters.
 
 \item Conclude.
  \subitem Note general consistency of the results across different datasets.
  \subitem Remind that we explored two classes of algorithms, and that the graph-proximity based ones are better.
\end{enumerate}

\section{Conclusion and future work}
\begin{enumerate}
 \item Mention the problem again.
  \subitem Motivate the problem and its general applicability. Eg: gene-disease networks.
 \item Outline key contributions.
 \item Describe future work.
  \subitem Using the affiliation network in social network link prediction: Early experiments indicate that this is hard.
  \subitem Mention application in areas beyond community recommendation: biology example.
  \subitem Community recommendation using even more sources of information: Chen et al used text information.
  \subitem Combining predictors.
\end{enumerate}


\section{Acknowledgements}
\begin{enumerate}
 \item Acknowledge Prateek, Berkant, Alan Mislove.
\end{enumerate}


\section{Corrections to be made}
Flawed claim: 'The inherent low rank nature of the user and group factors conforms to the huge prior of the unobserved entries in A being 0.' Instead, say that most unknown entries in A are 0.

We sould probably have made this point in the paper.

There were some other points we wished to make: \\
Eg: Showing computation-cost data, mentioning that the different recommenders could be combined together to make better predictions, adding more pictures explaining features of the datasets etc..

In Figure 3: 'Comparison of latent factors based algorithms', dataset is not identified. Make caption more descriptive. Refer to the right section for discussion in the caption. Fix its discussion: ambiguous description.

In Table 2: 'best parameters learned', no learned parameter is specified for SVD(C) on orkut.

Fix / complete related work section to cite recommender systems literature.

Story about validation is incomplete: how many predictions are being made in comparing different parameters?

% \bibliographystyle{plain}
% \bibliography{../references}

\end{document}
