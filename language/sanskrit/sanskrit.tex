\documentclass[oneside, article]{memoir}
\usepackage{amsmath, amssymb}
\usepackage{hyperref, graphicx, verbatim, listings, multirow, subfigure}
\usepackage{algorithm, algorithmic}
% \usepackage[bottom]{footmisc}
\lstset{breaklines=true}
\setcounter{tocdepth}{3}

% Lets verbatim and verb environments automatically break lines.
\makeatletter
\def\@xobeysp{ }
\makeatother
% \lstset{breaklines=true,basicstyle=\ttfamily}

% Configuration for the memoir class.
\renewcommand{\cleardoublepage}{}
% \renewcommand*{\partpageend}{}
\renewcommand{\afterpartskip}{}
\maxsecnumdepth{subsubsection} % number subsections
\maxtocdepth{subsubsection}

\addtolength{\parindent}{-5mm}
% Packages not included:
% For multiline comments, use caption package. But this conflicts with hyperref while making html files.
% subfigure conflicts with use with memoir style-sheet.

\usepackage{fontspec, xunicode}
\setmainfont[Script=Devanagari]{Kalimati}

% Configuration for the memoir class.
\renewcommand{\cleardoublepage}{}
% \renewcommand*{\partpageend}{}
\renewcommand{\afterpartskip}{}
\maxsecnumdepth{subsubsection} % number subsections
\maxtocdepth{subsubsection}

\addtolength{\parindent}{-5mm}
% Packages not included:
% For multiline comments, use caption package. But this conflicts with hyperref while making html files.
% subfigure conflicts with use with memoir style-sheet.

% \setlength{\textwidth}{4.5in}

% Use something like:
% % Use something like:
% % Use something like:
% \input{../../macros}

% groupings of objects.
\newcommand{\set}[1]{\left\{ #1 \right\}}
\newcommand{\seq}[1]{\left(#1\right)}
\newcommand{\ang}[1]{\langle#1\rangle}
\newcommand{\tuple}[1]{\left(#1\right)}

% numerical shortcuts.
\newcommand{\abs}[1]{\left| #1\right|}
\newcommand{\floor}[1]{\left\lfloor #1 \right\rfloor}
\newcommand{\ceil}[1]{\left\lceil #1 \right\rceil}

% linear algebra shortcuts.
\newcommand{\change}{\Delta}
\newcommand{\norm}[1]{\left\| #1\right\|}
\newcommand{\dprod}[1]{\langle#1\rangle}
\newcommand{\linspan}[1]{\langle#1\rangle}
\newcommand{\conj}[1]{\overline{#1}}
\newcommand{\gradient}{\nabla}
\newcommand{\der}{\frac{d}{dx}}
\newcommand{\lap}{\Delta}
\newcommand{\kron}{\otimes}
\newcommand{\nperp}{\nvdash}

\newcommand{\mat}[1]{\left( \begin{smallmatrix}#1 \end{smallmatrix} \right)}

% derivatives and limits
\newcommand{\partder}[2]{\frac{\partial #1}{\partial #2}}
\newcommand{\partdern}[3]{\frac{\partial^{#3} #1}{\partial #2^{#3}}}

% Arrows
\newcommand{\diverge}{\nearrow}
\newcommand{\notto}{\nrightarrow}
\newcommand{\up}{\uparrow}
\newcommand{\down}{\downarrow}
% gets and gives are defined!

% ordering operators
\newcommand{\oleq}{\preceq}
\newcommand{\ogeq}{\succeq}

% programming and logic operators
\newcommand{\dfn}{:=}
\newcommand{\assign}{:=}
\newcommand{\co}{\ co\ }
\newcommand{\en}{\ en\ }


% logic operators
\newcommand{\xor}{\oplus}
\newcommand{\Land}{\bigwedge}
\newcommand{\Lor}{\bigvee}
\newcommand{\finish}{$\Box$}
\newcommand{\contra}{\Rightarrow \Leftarrow}
\newcommand{\iseq}{\stackrel{_?}{=}}


% Set theory
\newcommand{\symdiff}{\Delta}
\newcommand{\union}{\cup}
\newcommand{\inters}{\cap}
\newcommand{\Union}{\bigcup}
\newcommand{\Inters}{\bigcap}
\newcommand{\nullSet}{\phi}

% graph theory
\newcommand{\nbd}{\Gamma}

% Script alphabets
% For reals, use \Re

% greek letters
\newcommand{\eps}{\epsilon}
\newcommand{\del}{\delta}
\newcommand{\ga}{\alpha}
\newcommand{\gb}{\beta}
\newcommand{\gd}{\del}
\newcommand{\gf}{\phi}
\newcommand{\gF}{\Phi}
\newcommand{\gl}{\lambda}
\newcommand{\gm}{\mu}
\newcommand{\gn}{\nu}
\newcommand{\gr}{\rho}
\newcommand{\gs}{\sigma}
\newcommand{\gt}{\theta}
\newcommand{\gx}{\xi}

\newcommand{\sw}{\sigma}
\newcommand{\SW}{\Sigma}
\newcommand{\ew}{\lambda}
\newcommand{\EW}{\Lambda}

\newcommand{\Del}{\Delta}
\newcommand{\gD}{\Delta}
\newcommand{\gG}{\Gamma}
\newcommand{\gO}{\Omega}
\newcommand{\gL}{\Lambda}
\newcommand{\gS}{\Sigma}

% Formatting shortcuts
\newcommand{\red}[1]{\textcolor{red}{#1}}
\newcommand{\blue}[1]{\textcolor{blue}{#1}}
\newcommand{\htext}[2]{\texorpdfstring{#1}{#2}}

% Statistics
\newcommand{\distr}{\sim}
\newcommand{\stddev}{\sigma}
\newcommand{\covmatrix}{\Sigma}
\newcommand{\mean}{\mu}
\newcommand{\param}{\gt}
\newcommand{\ftr}{\phi}

% General utility
\newcommand{\todo}[1]{\footnote{TODO: #1}}
\newcommand{\exclaim}[1]{{\textbf{\textit{#1}}}}
\newcommand{\tbc}{[\textbf{Incomplete}]}
\newcommand{\chk}{[\textbf{Check}]}
\newcommand{\oprob}{[\textbf{OP}]:}
\newcommand{\core}[1]{\textbf{Core Idea:}}
\newcommand{\why}{[\textbf{Find proof}]}
\newcommand{\opt}[1]{\textit{#1}}


\DeclareMathOperator*{\argmin}{arg\,min}
\DeclareMathOperator{\rank}{rank}
\newcommand{\redcol}[1]{\textcolor{red}{#1}}
\newcommand{\bluecol}[1]{\textcolor{blue}{#1}}
\newcommand{\greencol}[1]{\textcolor{green}{#1}}


\renewcommand{\~}{\htext{$\sim$}{~}}


% groupings of objects.
\newcommand{\set}[1]{\left\{ #1 \right\}}
\newcommand{\seq}[1]{\left(#1\right)}
\newcommand{\ang}[1]{\langle#1\rangle}
\newcommand{\tuple}[1]{\left(#1\right)}

% numerical shortcuts.
\newcommand{\abs}[1]{\left| #1\right|}
\newcommand{\floor}[1]{\left\lfloor #1 \right\rfloor}
\newcommand{\ceil}[1]{\left\lceil #1 \right\rceil}

% linear algebra shortcuts.
\newcommand{\change}{\Delta}
\newcommand{\norm}[1]{\left\| #1\right\|}
\newcommand{\dprod}[1]{\langle#1\rangle}
\newcommand{\linspan}[1]{\langle#1\rangle}
\newcommand{\conj}[1]{\overline{#1}}
\newcommand{\gradient}{\nabla}
\newcommand{\der}{\frac{d}{dx}}
\newcommand{\lap}{\Delta}
\newcommand{\kron}{\otimes}
\newcommand{\nperp}{\nvdash}

\newcommand{\mat}[1]{\left( \begin{smallmatrix}#1 \end{smallmatrix} \right)}

% derivatives and limits
\newcommand{\partder}[2]{\frac{\partial #1}{\partial #2}}
\newcommand{\partdern}[3]{\frac{\partial^{#3} #1}{\partial #2^{#3}}}

% Arrows
\newcommand{\diverge}{\nearrow}
\newcommand{\notto}{\nrightarrow}
\newcommand{\up}{\uparrow}
\newcommand{\down}{\downarrow}
% gets and gives are defined!

% ordering operators
\newcommand{\oleq}{\preceq}
\newcommand{\ogeq}{\succeq}

% programming and logic operators
\newcommand{\dfn}{:=}
\newcommand{\assign}{:=}
\newcommand{\co}{\ co\ }
\newcommand{\en}{\ en\ }


% logic operators
\newcommand{\xor}{\oplus}
\newcommand{\Land}{\bigwedge}
\newcommand{\Lor}{\bigvee}
\newcommand{\finish}{$\Box$}
\newcommand{\contra}{\Rightarrow \Leftarrow}
\newcommand{\iseq}{\stackrel{_?}{=}}


% Set theory
\newcommand{\symdiff}{\Delta}
\newcommand{\union}{\cup}
\newcommand{\inters}{\cap}
\newcommand{\Union}{\bigcup}
\newcommand{\Inters}{\bigcap}
\newcommand{\nullSet}{\phi}

% graph theory
\newcommand{\nbd}{\Gamma}

% Script alphabets
% For reals, use \Re

% greek letters
\newcommand{\eps}{\epsilon}
\newcommand{\del}{\delta}
\newcommand{\ga}{\alpha}
\newcommand{\gb}{\beta}
\newcommand{\gd}{\del}
\newcommand{\gf}{\phi}
\newcommand{\gF}{\Phi}
\newcommand{\gl}{\lambda}
\newcommand{\gm}{\mu}
\newcommand{\gn}{\nu}
\newcommand{\gr}{\rho}
\newcommand{\gs}{\sigma}
\newcommand{\gt}{\theta}
\newcommand{\gx}{\xi}

\newcommand{\sw}{\sigma}
\newcommand{\SW}{\Sigma}
\newcommand{\ew}{\lambda}
\newcommand{\EW}{\Lambda}

\newcommand{\Del}{\Delta}
\newcommand{\gD}{\Delta}
\newcommand{\gG}{\Gamma}
\newcommand{\gO}{\Omega}
\newcommand{\gL}{\Lambda}
\newcommand{\gS}{\Sigma}

% Formatting shortcuts
\newcommand{\red}[1]{\textcolor{red}{#1}}
\newcommand{\blue}[1]{\textcolor{blue}{#1}}
\newcommand{\htext}[2]{\texorpdfstring{#1}{#2}}

% Statistics
\newcommand{\distr}{\sim}
\newcommand{\stddev}{\sigma}
\newcommand{\covmatrix}{\Sigma}
\newcommand{\mean}{\mu}
\newcommand{\param}{\gt}
\newcommand{\ftr}{\phi}

% General utility
\newcommand{\todo}[1]{\footnote{TODO: #1}}
\newcommand{\exclaim}[1]{{\textbf{\textit{#1}}}}
\newcommand{\tbc}{[\textbf{Incomplete}]}
\newcommand{\chk}{[\textbf{Check}]}
\newcommand{\oprob}{[\textbf{OP}]:}
\newcommand{\core}[1]{\textbf{Core Idea:}}
\newcommand{\why}{[\textbf{Find proof}]}
\newcommand{\opt}[1]{\textit{#1}}


\DeclareMathOperator*{\argmin}{arg\,min}
\DeclareMathOperator{\rank}{rank}
\newcommand{\redcol}[1]{\textcolor{red}{#1}}
\newcommand{\bluecol}[1]{\textcolor{blue}{#1}}
\newcommand{\greencol}[1]{\textcolor{green}{#1}}


\renewcommand{\~}{\htext{$\sim$}{~}}


% groupings of objects.
\newcommand{\set}[1]{\left\{ #1 \right\}}
\newcommand{\seq}[1]{\left(#1\right)}
\newcommand{\ang}[1]{\langle#1\rangle}
\newcommand{\tuple}[1]{\left(#1\right)}

% numerical shortcuts.
\newcommand{\abs}[1]{\left| #1\right|}
\newcommand{\floor}[1]{\left\lfloor #1 \right\rfloor}
\newcommand{\ceil}[1]{\left\lceil #1 \right\rceil}

% linear algebra shortcuts.
\newcommand{\change}{\Delta}
\newcommand{\norm}[1]{\left\| #1\right\|}
\newcommand{\dprod}[1]{\langle#1\rangle}
\newcommand{\linspan}[1]{\langle#1\rangle}
\newcommand{\conj}[1]{\overline{#1}}
\newcommand{\gradient}{\nabla}
\newcommand{\der}{\frac{d}{dx}}
\newcommand{\lap}{\Delta}
\newcommand{\kron}{\otimes}
\newcommand{\nperp}{\nvdash}

\newcommand{\mat}[1]{\left( \begin{smallmatrix}#1 \end{smallmatrix} \right)}

% derivatives and limits
\newcommand{\partder}[2]{\frac{\partial #1}{\partial #2}}
\newcommand{\partdern}[3]{\frac{\partial^{#3} #1}{\partial #2^{#3}}}

% Arrows
\newcommand{\diverge}{\nearrow}
\newcommand{\notto}{\nrightarrow}
\newcommand{\up}{\uparrow}
\newcommand{\down}{\downarrow}
% gets and gives are defined!

% ordering operators
\newcommand{\oleq}{\preceq}
\newcommand{\ogeq}{\succeq}

% programming and logic operators
\newcommand{\dfn}{:=}
\newcommand{\assign}{:=}
\newcommand{\co}{\ co\ }
\newcommand{\en}{\ en\ }


% logic operators
\newcommand{\xor}{\oplus}
\newcommand{\Land}{\bigwedge}
\newcommand{\Lor}{\bigvee}
\newcommand{\finish}{$\Box$}
\newcommand{\contra}{\Rightarrow \Leftarrow}
\newcommand{\iseq}{\stackrel{_?}{=}}


% Set theory
\newcommand{\symdiff}{\Delta}
\newcommand{\union}{\cup}
\newcommand{\inters}{\cap}
\newcommand{\Union}{\bigcup}
\newcommand{\Inters}{\bigcap}
\newcommand{\nullSet}{\phi}

% graph theory
\newcommand{\nbd}{\Gamma}

% Script alphabets
% For reals, use \Re

% greek letters
\newcommand{\eps}{\epsilon}
\newcommand{\del}{\delta}
\newcommand{\ga}{\alpha}
\newcommand{\gb}{\beta}
\newcommand{\gd}{\del}
\newcommand{\gf}{\phi}
\newcommand{\gF}{\Phi}
\newcommand{\gl}{\lambda}
\newcommand{\gm}{\mu}
\newcommand{\gn}{\nu}
\newcommand{\gr}{\rho}
\newcommand{\gs}{\sigma}
\newcommand{\gt}{\theta}
\newcommand{\gx}{\xi}

\newcommand{\sw}{\sigma}
\newcommand{\SW}{\Sigma}
\newcommand{\ew}{\lambda}
\newcommand{\EW}{\Lambda}

\newcommand{\Del}{\Delta}
\newcommand{\gD}{\Delta}
\newcommand{\gG}{\Gamma}
\newcommand{\gO}{\Omega}
\newcommand{\gL}{\Lambda}
\newcommand{\gS}{\Sigma}

% Formatting shortcuts
\newcommand{\red}[1]{\textcolor{red}{#1}}
\newcommand{\blue}[1]{\textcolor{blue}{#1}}
\newcommand{\htext}[2]{\texorpdfstring{#1}{#2}}

% Statistics
\newcommand{\distr}{\sim}
\newcommand{\stddev}{\sigma}
\newcommand{\covmatrix}{\Sigma}
\newcommand{\mean}{\mu}
\newcommand{\param}{\gt}
\newcommand{\ftr}{\phi}

% General utility
\newcommand{\todo}[1]{\footnote{TODO: #1}}
\newcommand{\exclaim}[1]{{\textbf{\textit{#1}}}}
\newcommand{\tbc}{[\textbf{Incomplete}]}
\newcommand{\chk}{[\textbf{Check}]}
\newcommand{\oprob}{[\textbf{OP}]:}
\newcommand{\core}[1]{\textbf{Core Idea:}}
\newcommand{\why}{[\textbf{Find proof}]}
\newcommand{\opt}[1]{\textit{#1}}


\DeclareMathOperator*{\argmin}{arg\,min}
\DeclareMathOperator{\rank}{rank}
\newcommand{\redcol}[1]{\textcolor{red}{#1}}
\newcommand{\bluecol}[1]{\textcolor{blue}{#1}}
\newcommand{\greencol}[1]{\textcolor{green}{#1}}


\renewcommand{\~}{\htext{$\sim$}{~}}

\title{॥संस्कृत-सूत्रं॥}
\author{विश्वासः वासुकेयः॥}

\begin{document}
\maketitle

\part{परिचयः॥}
श्रीमद्पाणिनि-वररुचि-पतञ्जलिभ्यः नमः॥ शब्दब्रह्मणे नमः॥ अक्षराणां, पदानां, वाक्यानां च शुद्धोच्चारणे च सुसंस्कारे पुण्यः।

\chapter{संस्कृते अद्भुतानि॥}
अक्षरोत्पत्तेः च उच्चारणस्य सूक्ष्म-अध्यायः (मन्त्र-शक्ति-मूलः एषः।)। उर्वरी शब्द-व्युत्पत्ति-व्यवस्था (तस्मात् व्युत्पत्तानां सूक्ष्मः विश्लेषणः च)। सूक्ष्माः पद-वाक्य-संस्कार-नियमाः (- तेषु अपि क्रियासु परस्मै-आत्मने-भेदः, द्वि-बहुवचनभेदः च)।

\section{विभागाः।}
संस्कार-दृष्ट्या - 'सुप्तिच्ङन्तं पदं।'

कारक-दृष्ट्या - नामपद-क्रियापद-तद्विशेषणानि च। सुबन्तानि अपि क्रियापदानि भवितुम् अर्हन्ति, उदाहरणाय 'गतम् धनम्', 'पुष्पैः विकसितम्'। कर्तृ-कर्मपदविशेषणैः अपि क्रियपदसंज्ञा योग्या  - 'गतम् धनम्'।

\section{पदे वर्णपरिवर्तनम् ॥}
वृद्धिः च गुणः‌ यथा वृद्धि-गुणसध्योः विवृतौ। संप्रसारणम् धातुखण्डे च।

\part{पदमूलाः॥}
\chapter{क्रिया-पदमूलाः धातवः॥}
\section{विशेषाणि अङ्गानि।}
उपधा इति धातोः अन्तिमपूर्वाक्षरम् ।

पुगन्तम् इति पुक् अन्ते यस्य तत् अङ्गम् । णिचि परे कदाचित् पुक्-आगमो विधीतः। ज्ञापयति -इत्यस्मिन् ज्ञाप पुगन्तम् ।

\section{गणविभक्तिः॥}
कर्मगणन-अनुसारम् विभक्तिः अन्यत्र विवृता। क्रियाफलदिशा-वाचक-प्रत्ययानाम् च सार्वधातुक-लकार-विकरण-प्रत्ययानाम् योग्यतानुसारा विभक्तिः अन्यत्र विवृता।

\chapter{प्रातिपदिकानि च तदुत्पत्तिः॥}
\section{समासान्तानि।}
\subsection{पदसन्धानम् ।}
समास-लक्षणम् अनेक-पद-सन्धानम्। एकपदोत्पन्नम् पदम् समासान्तम् न। तस्मात् व्यस्त-प्रयोगयुक्तस्य वाक्यस्य स्थाने समस्त-प्रयोगयुक्तात् वाक्यात् संक्षेपः।

वाक्ये परस्पर-अन्वययुक्तानाम् एव पदानाम् समासः साध्यः। उदाहरणाय  - राजा चोरम् ताडयति इत्यस्मिन् वाक्ये।

\subsubsection{विग्रहवाक्यम् ।}
लौकिकः विग्रह: - मयुरः नर्तकः यस्य सः = मयुरनर्तकः। लौकिकः विग्राहे सर्वेषु अन्तर्गतेषु पदेषु सत्सु सः स्वपदविग्रहः इति कथ्यते । वृक्षस्य समीपे - उपवृक्षम् - इत्येतत् वाक्यम् अस्वपदविग्रहः।

अलौकिकः विग्रहः - मयुर + सु + नायक + सु = मयूरनायक। अत्र 'यस्य सः' इत्येतौ वर्जितौ।

\subsubsection{क्रम-वचनम् ॥}
विंशतेः अधिके सति - एकविंशतितम

नवम। त्रयोदशः/शा/शम् ।

\section{कृदन्तानि॥}
धातुना सह कृत्-प्रत्यय-संयोगात् कृदन्ताः जायन्ते। कीदृशीम् चित् क्रियाम् सूचयति एव। 

सार्वधातुकत्व-आर्धधातुकत्व-प्रभावाः अन्यत्र विवृताः।

\subsection{क्त॥}
\subsubsection{शब्द-योगः॥}
क्त - पतित m. n. पतिता f। वह्-वोढम्। इष्- इष्टम् ()। 

\subsubsection{उपयोगः॥}
कर्तृ-कर्म-विशेषणम् वा/ तथा क्रियासूचकम् उत्पत्तम् पदम् ।

वाक्ये कर्मणि-भावे-प्रयोगयोः कर्मपद-विशेषणे उपयुज्यते - घटेन ग्रामः गतः (अस्ति), घटेन गतम् (ergative - सकर्मकधातोः अपि भाव-वचनम्)। 

\subsubsection{कर्तृ-विशेषणे उपयोगः॥}
कर्तरि-प्रयोगे कर्तृ-विशेषणाय उपयोगः कदाचित् शुद्धः। अकर्मकधातोः तु क्त-कृदन्त-उपयोगः शुद्धः - पुष्पाणि विकसितानि। सकर्मधातुनि गत्यर्थे च शयादिभ्यः धातुभ्यः एव शुद्धः। 

3\.4\.71 AdikarmaNi [ktaH kartari cha] .

3\.4\.72
gatyartha-akarmaka-shliShashI~NsthA.a.asavasajanaruhajIryatibhyashcha

भागवतपुराण-व्यवहारात् नास्ति पूर्णतः प्रामादिकम् 'देवम् प्रणतः अस्मि' इति प्रयोगः।

\subsection{वर्तमति क्रियायाम् कर्तृ-विशेषणम् ॥}
शतृँ - इत्-लोपानन्तरम् त्-आगमः- पतत् m. n. पतन्ती f।  लट्-साधर्म्यम् अस्य प्रत्ययस्य। अतः सार्वधातुकत्वम् अपि।

शानच् - (म्)आन-आदेशः - पतमान m. n. पतमाना f, कुर्वाण।

क्रु - रु-आगमः - भीरु।

घिनुण् - घञ् + इनि - योगिन्, भोगिन्।

तृच् - कर्तृ m. n., कत्री f॥

णिनि - इन्-आगमः - चारिन्।

ण्वुल्, वुञ्, वुन् - चल् -> चालक।

वरच् - ईश्वर।

वन् - ह्रस्व

\subsubsection{साधारणतः पुंसः॥}
'टु-इतः अथुच्' - वेपथुः।

ट - चरः॥

ड - अन्तिम-हलः लोपः - दूरग, मध्यग।

\subsection{समाप्तायाम् क्रियायाम् ॥}
क्तवत् - पतितवत् m. n. पतितवती f। वाक्ये कर्तृपद-विशेषणम् ।

वेदे वनिप् - यज्वन्, धीवन्।

 \subsubsection{परोक्षभूते क्रियायाम्॥}
लिडादेशः परस्मै-पदेषु - पेतिवः m. n. पेतुषी f.। लिडादेशः आत्मने-पदेषु - पेतान m. n. पेताना f.।

\subsection{भविष्यति क्रियायाम्॥}
लुटादेश परस्मै-पदेषु - पत्स्यत् m. n. पत्स्यन्ती f.।

लुटादेश आत्मने-पदेषु - पत्स्यमान m. n. पत्स्यमाना f.।

\subsection{कर्म-विशेषणम्॥}
घ - गोचरः।

नङ् - प्रश्नः, यत्नः।


\subsubsection{घुञ्-प्रत्ययः॥}
घुञन्तः पुंलिङ्गे एव साधारणतः प्रयुज्यते। पूर्वस्वरस्य अकारस्य वृद्धिः च  अन्यह्रस्वस्य गुणः भवति।

पठ्तः - पाठः। वद्तः वादः। ऊह्तः ऊहः। पद्तः पादः।

\subsection{करणे ॥}
ष्ठ्रङ् नपुंसकानि - योक्त्र सूत्र नेत्र पत्त्र ।

\subsection{क्रियायाः भावे एव॥}
\subsubsection{भाववचने नपुंसकानि॥}
भावे ल्युट् - नित्य-नपुंसकलिङ्गि - अन-आगमः - पठनं, स्नानं। ल्युट् इत्यत्र लकारस्य टकास्य च लोपः भवति (हलन्त्यम् (१.३.३), लशक्वतद्धिते (१.३.८), (तस्य लोपः १.३.९) । युवोरनाकौ (७.१.१) एतदनुसारं ’यु’ इत्यस्य स्थाने ’अन’ इत्यागमः।

णमुल् अव्ययं - कारं, भावं।

\subsubsection{युच्॥}
अनपुंसक-शब्दान् जनयति - तस्मिन्नपि साधारणतः स्त्री-लिङ्गे आप्-प्रत्ययेन सह उपयुज्यन्ते। उदाहरणाय - प्रार्थना, भावना।

सर्वेभ्यः णिजन्तेभ्यः योग्यः। तथा परिगणितेभ्यः अणिजन्त-धातुभ्यः अपि।

युवोरनाकौ (७.१.१) एतदनुसारं ’यु’ इत्यस्य स्थाने ’अन’ इत्यागमः।

\subsubsection{भाववाचने स्त्र्यः॥}
क्तिच्-न्- निय्त-स्त्री-लिङ्गि - कृतिः, मतिः ।

कि - शुच् + कि = शुचि

\subsubsection{योग्यतायां॥}
तव्य - पत्तव्य m. n. पत्तव्या f.। 

अनीयत् - पतनीय m. n. पतनीया f.।

ण्यत्, णिच्-यत् -पत्य m. n. पत्या f.।

उन् - साधु।

\subsubsection{उद्देश-वचनं॥}
तुमुँन् अव्यय-प्रत्ययः - पत्तुम्। वोढुम् । 

\subsubsection{क्रिया-उत्तर-काल-वाचकं अव्ययं॥}
क्त्वा - पतित्वा। उपसर्गः अस्ति चेत् ल्यप् - आगम्य।

\subsection{प्रवृत्ति-वचने॥}
इष्णुच् - सहिष्णु।

उकञ् - पातुक, स्थायुक।

कु - नु-आदेशः - क्षिप्नु।

\subsection{उण्-आदि-प्रत्ययाः॥}
एतेषां सूचिः उणादिसूत्रे। तेषां उपयोगः बहुसीमितः - सूचिकृतधातुभ्यः एव।

उण् - कृ, वा, स्वाद् - कारु, वायु, स्वादु।

\subsection{अन्याः।}
श्रि + क्विप् = श्रीः।

\section{अन्तिमाक्षर-विभजनम् अनव्यायानाम्॥}
\subsection{अन्तिमाक्षरत्व-सङ्ख्या॥}
अमरकोषे ८०\% नामशब्दानि अकारान्ते पुँल्लिङ्गे वा‌ नपुंसकलिङ्गे वा आ-ई-कारान्ते स्त्रीलिङ्गे सन्ति!

\section{अन्य-उपासंज्ञाः।}
\subsubsection{निपाताः॥}
(Exclamations/ utterences)

प्रादयः निपाताः॥

\subsubsection{उपसर्गाः॥}
क्रियापदेन सह योगे काश्चन निपाताः भवन्ति उपसर्गाः। उपसर्गात् धातोः अर्थः अन्यत्र नीयते प्रहार-विहार-संहार-वत्। परन्तु उपसर्गाः सुबन्ताः अव्ययानि एव।

\subsection{सर्वनामानि॥}
सर्वनामसंज्ञायुक्तेभ्यः भिन्नाः प्रक्रियाः च प्रत्ययाः विहिताः।

नञ्-समासे च न-समासे सर्वनामसंज्ञा अवशिष्यते । अनिक च नैक सर्वनामे एव।



\chapter{तिङ्-अन्त्यं॥}
\section{उपगणाः॥}
\section{काल-वाचकाः लकाराः॥}
\subsection{भूतकाल-वाचनं।}
लट्-उपयुज्य भूतकाल-वचनं : "सः तदा हसति स्म" वाक्य-इव। 

\end{document}
