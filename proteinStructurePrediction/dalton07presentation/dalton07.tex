\documentclass{beamer}


\usetheme{Warsaw}

\title{Presenting: `An evaluation of automated homology modelling methods at low target-template sequence similarity' by Dalton and Jackson}
\author{vishvAs vAsuki}
\date{\today}

\begin{document}

\frame{\titlepage}

\section[Outline]{}
\frame{\tableofcontents}

\section{Homology Modelling: Background}

\begin{frame}
\frametitle{Steps in homology modelling}
\begin{itemize}
\item Template selection
\item Alignment
\item Model construction
\end{itemize}
\end{frame}

\begin{frame}
\frametitle{Template selection}
\begin{itemize}
\item Blast E-Values: .0001 E-Value means that the structures are pretty close.
\item FASTA sequence identities
\item For multiple templates: Ensure structural similarity among templates. Use `Z-score'.
\end{itemize}
\end{frame}

\begin{frame}
\frametitle{Alignment: Critical}
\begin{itemize}
\item \textbf{Use sequence information only:} BLAST, FASTA and PSI-BLAST.
\item \textbf{Use structure information too}: Use threading (3D-Coffee), build and align locale profiles (Staccato) or penalize indels in secondary structure locales (SAlign).
\end{itemize}
\end{frame}


\begin{frame}
\frametitle{Model construction: Conserved region}
\begin{itemize}
\item \textbf{Rigid-body assembly (Nest, Builder, Swiss-Model):} Copy structure of conserved core from the template. \cite{galperin03}
\item \textbf{Segment-Matching (SegMod/ENCAD):} Assemble short (hexapeptide?) segments to fit atom positions inferred from the template.
\item \textbf{Satisfaction of spatial restraints (Modeller):} Extract restraints on structure from template. Add additional restraints. (Possibly NMR data too) Minimize violations. Then, use energy functions. \cite{galperin03} This beats the rest by a small margin.
\end{itemize}
\end{frame}

\begin{frame}
\frametitle{Model construction: Loop Modelling}
\begin{itemize}
\item \textbf{Ab initio} (Modeller and Nest): Don't understand. But, they beat the rest by a small margin.
\item \textbf{Database} (Builder and SegMod/ENCAD): Fragments from DB, energy minimization
\item Database if Ab-initio fails (Swiss-Model)
\end{itemize}
\end{frame}

\begin{frame}
\frametitle{Model construction: Sidechains}
\begin{itemize}
\item Put it close to template sidechain. Use torsion angle libraries. Then use Energy minimization.
\end{itemize}
\end{frame}

\section{The article}

\begin{frame}
\frametitle{Tests and Results}
\begin{itemize}
\item Test 3 new sequence-structure alignment programs: \textbf{3D-Coffee}, Staccato and SAlign.
\item Find out if using multiple templates makes things better (and not worse).
\item Test 5 homology modelling programs and their respective loop building methods: Builder, \textbf{Nest, Modeller}, SegMod/ENCAD and Swiss-Model.
\end{itemize}
\end{frame}

\begin{frame}
\frametitle{Test data}
\begin{itemize}
\item `The SABmark database provided 123 targets with at least five templates from the same SCOP family and sequence identities 50\%.' \cite{dalton07}
\item Loops range in length from 5 to 16 residues. (These include 2 rei on either side of the unmatched region.)
\end{itemize}
\end{frame}


\begin{frame}
\frametitle{Lessons}
\begin{itemize}
\item `There are two main areas of difficulty in homology modelling that are particularly important when sequence identity between target and template falls below 50\%: sequence alignment and loop building.' \cite{dalton07}
\item Use structure information in sequence alignment.
\item Sequence alignment quality decreases with decreasing sequence identity.
\item Overall, sequence alignment seems more critical than choice of modeller. (We can change this.)
\item Ab initio loop modellers seem to work \textbf{slightly} better. Loop accuracy decreases with increasing length.
\item Be wary about using multiple, rather than single, templates.
\end{itemize}
\end{frame}


\bibliographystyle{plain}
\bibliography{../informationInPSPBib}


\end{document}
