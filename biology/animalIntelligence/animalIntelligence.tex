\documentclass[oneside, article]{memoir}
\usepackage{amsmath, amssymb}
\usepackage{hyperref, graphicx, verbatim, listings, multirow, subfigure}
\usepackage{algorithm, algorithmic}
% \usepackage[bottom]{footmisc}
\lstset{breaklines=true}
\setcounter{tocdepth}{3}

% Lets verbatim and verb environments automatically break lines.
\makeatletter
\def\@xobeysp{ }
\makeatother
% \lstset{breaklines=true,basicstyle=\ttfamily}

% Configuration for the memoir class.
\renewcommand{\cleardoublepage}{}
% \renewcommand*{\partpageend}{}
\renewcommand{\afterpartskip}{}
\maxsecnumdepth{subsubsection} % number subsections
\maxtocdepth{subsubsection}

\addtolength{\parindent}{-5mm}
% Packages not included:
% For multiline comments, use caption package. But this conflicts with hyperref while making html files.
% subfigure conflicts with use with memoir style-sheet.

% Use something like:
% % Use something like:
% % Use something like:
% \input{../../macros}

% groupings of objects.
\newcommand{\set}[1]{\left\{ #1 \right\}}
\newcommand{\seq}[1]{\left(#1\right)}
\newcommand{\ang}[1]{\langle#1\rangle}
\newcommand{\tuple}[1]{\left(#1\right)}

% numerical shortcuts.
\newcommand{\abs}[1]{\left| #1\right|}
\newcommand{\floor}[1]{\left\lfloor #1 \right\rfloor}
\newcommand{\ceil}[1]{\left\lceil #1 \right\rceil}

% linear algebra shortcuts.
\newcommand{\change}{\Delta}
\newcommand{\norm}[1]{\left\| #1\right\|}
\newcommand{\dprod}[1]{\langle#1\rangle}
\newcommand{\linspan}[1]{\langle#1\rangle}
\newcommand{\conj}[1]{\overline{#1}}
\newcommand{\gradient}{\nabla}
\newcommand{\der}{\frac{d}{dx}}
\newcommand{\lap}{\Delta}
\newcommand{\kron}{\otimes}
\newcommand{\nperp}{\nvdash}

\newcommand{\mat}[1]{\left( \begin{smallmatrix}#1 \end{smallmatrix} \right)}

% derivatives and limits
\newcommand{\partder}[2]{\frac{\partial #1}{\partial #2}}
\newcommand{\partdern}[3]{\frac{\partial^{#3} #1}{\partial #2^{#3}}}

% Arrows
\newcommand{\diverge}{\nearrow}
\newcommand{\notto}{\nrightarrow}
\newcommand{\up}{\uparrow}
\newcommand{\down}{\downarrow}
% gets and gives are defined!

% ordering operators
\newcommand{\oleq}{\preceq}
\newcommand{\ogeq}{\succeq}

% programming and logic operators
\newcommand{\dfn}{:=}
\newcommand{\assign}{:=}
\newcommand{\co}{\ co\ }
\newcommand{\en}{\ en\ }


% logic operators
\newcommand{\xor}{\oplus}
\newcommand{\Land}{\bigwedge}
\newcommand{\Lor}{\bigvee}
\newcommand{\finish}{$\Box$}
\newcommand{\contra}{\Rightarrow \Leftarrow}
\newcommand{\iseq}{\stackrel{_?}{=}}


% Set theory
\newcommand{\symdiff}{\Delta}
\newcommand{\union}{\cup}
\newcommand{\inters}{\cap}
\newcommand{\Union}{\bigcup}
\newcommand{\Inters}{\bigcap}
\newcommand{\nullSet}{\phi}

% graph theory
\newcommand{\nbd}{\Gamma}

% Script alphabets
% For reals, use \Re

% greek letters
\newcommand{\eps}{\epsilon}
\newcommand{\del}{\delta}
\newcommand{\ga}{\alpha}
\newcommand{\gb}{\beta}
\newcommand{\gd}{\del}
\newcommand{\gf}{\phi}
\newcommand{\gF}{\Phi}
\newcommand{\gl}{\lambda}
\newcommand{\gm}{\mu}
\newcommand{\gn}{\nu}
\newcommand{\gr}{\rho}
\newcommand{\gs}{\sigma}
\newcommand{\gt}{\theta}
\newcommand{\gx}{\xi}

\newcommand{\sw}{\sigma}
\newcommand{\SW}{\Sigma}
\newcommand{\ew}{\lambda}
\newcommand{\EW}{\Lambda}

\newcommand{\Del}{\Delta}
\newcommand{\gD}{\Delta}
\newcommand{\gG}{\Gamma}
\newcommand{\gO}{\Omega}
\newcommand{\gL}{\Lambda}
\newcommand{\gS}{\Sigma}

% Formatting shortcuts
\newcommand{\red}[1]{\textcolor{red}{#1}}
\newcommand{\blue}[1]{\textcolor{blue}{#1}}
\newcommand{\htext}[2]{\texorpdfstring{#1}{#2}}

% Statistics
\newcommand{\distr}{\sim}
\newcommand{\stddev}{\sigma}
\newcommand{\covmatrix}{\Sigma}
\newcommand{\mean}{\mu}
\newcommand{\param}{\gt}
\newcommand{\ftr}{\phi}

% General utility
\newcommand{\todo}[1]{\footnote{TODO: #1}}
\newcommand{\exclaim}[1]{{\textbf{\textit{#1}}}}
\newcommand{\tbc}{[\textbf{Incomplete}]}
\newcommand{\chk}{[\textbf{Check}]}
\newcommand{\oprob}{[\textbf{OP}]:}
\newcommand{\core}[1]{\textbf{Core Idea:}}
\newcommand{\why}{[\textbf{Find proof}]}
\newcommand{\opt}[1]{\textit{#1}}


\DeclareMathOperator*{\argmin}{arg\,min}
\DeclareMathOperator{\rank}{rank}
\newcommand{\redcol}[1]{\textcolor{red}{#1}}
\newcommand{\bluecol}[1]{\textcolor{blue}{#1}}
\newcommand{\greencol}[1]{\textcolor{green}{#1}}


\renewcommand{\~}{\htext{$\sim$}{~}}


% groupings of objects.
\newcommand{\set}[1]{\left\{ #1 \right\}}
\newcommand{\seq}[1]{\left(#1\right)}
\newcommand{\ang}[1]{\langle#1\rangle}
\newcommand{\tuple}[1]{\left(#1\right)}

% numerical shortcuts.
\newcommand{\abs}[1]{\left| #1\right|}
\newcommand{\floor}[1]{\left\lfloor #1 \right\rfloor}
\newcommand{\ceil}[1]{\left\lceil #1 \right\rceil}

% linear algebra shortcuts.
\newcommand{\change}{\Delta}
\newcommand{\norm}[1]{\left\| #1\right\|}
\newcommand{\dprod}[1]{\langle#1\rangle}
\newcommand{\linspan}[1]{\langle#1\rangle}
\newcommand{\conj}[1]{\overline{#1}}
\newcommand{\gradient}{\nabla}
\newcommand{\der}{\frac{d}{dx}}
\newcommand{\lap}{\Delta}
\newcommand{\kron}{\otimes}
\newcommand{\nperp}{\nvdash}

\newcommand{\mat}[1]{\left( \begin{smallmatrix}#1 \end{smallmatrix} \right)}

% derivatives and limits
\newcommand{\partder}[2]{\frac{\partial #1}{\partial #2}}
\newcommand{\partdern}[3]{\frac{\partial^{#3} #1}{\partial #2^{#3}}}

% Arrows
\newcommand{\diverge}{\nearrow}
\newcommand{\notto}{\nrightarrow}
\newcommand{\up}{\uparrow}
\newcommand{\down}{\downarrow}
% gets and gives are defined!

% ordering operators
\newcommand{\oleq}{\preceq}
\newcommand{\ogeq}{\succeq}

% programming and logic operators
\newcommand{\dfn}{:=}
\newcommand{\assign}{:=}
\newcommand{\co}{\ co\ }
\newcommand{\en}{\ en\ }


% logic operators
\newcommand{\xor}{\oplus}
\newcommand{\Land}{\bigwedge}
\newcommand{\Lor}{\bigvee}
\newcommand{\finish}{$\Box$}
\newcommand{\contra}{\Rightarrow \Leftarrow}
\newcommand{\iseq}{\stackrel{_?}{=}}


% Set theory
\newcommand{\symdiff}{\Delta}
\newcommand{\union}{\cup}
\newcommand{\inters}{\cap}
\newcommand{\Union}{\bigcup}
\newcommand{\Inters}{\bigcap}
\newcommand{\nullSet}{\phi}

% graph theory
\newcommand{\nbd}{\Gamma}

% Script alphabets
% For reals, use \Re

% greek letters
\newcommand{\eps}{\epsilon}
\newcommand{\del}{\delta}
\newcommand{\ga}{\alpha}
\newcommand{\gb}{\beta}
\newcommand{\gd}{\del}
\newcommand{\gf}{\phi}
\newcommand{\gF}{\Phi}
\newcommand{\gl}{\lambda}
\newcommand{\gm}{\mu}
\newcommand{\gn}{\nu}
\newcommand{\gr}{\rho}
\newcommand{\gs}{\sigma}
\newcommand{\gt}{\theta}
\newcommand{\gx}{\xi}

\newcommand{\sw}{\sigma}
\newcommand{\SW}{\Sigma}
\newcommand{\ew}{\lambda}
\newcommand{\EW}{\Lambda}

\newcommand{\Del}{\Delta}
\newcommand{\gD}{\Delta}
\newcommand{\gG}{\Gamma}
\newcommand{\gO}{\Omega}
\newcommand{\gL}{\Lambda}
\newcommand{\gS}{\Sigma}

% Formatting shortcuts
\newcommand{\red}[1]{\textcolor{red}{#1}}
\newcommand{\blue}[1]{\textcolor{blue}{#1}}
\newcommand{\htext}[2]{\texorpdfstring{#1}{#2}}

% Statistics
\newcommand{\distr}{\sim}
\newcommand{\stddev}{\sigma}
\newcommand{\covmatrix}{\Sigma}
\newcommand{\mean}{\mu}
\newcommand{\param}{\gt}
\newcommand{\ftr}{\phi}

% General utility
\newcommand{\todo}[1]{\footnote{TODO: #1}}
\newcommand{\exclaim}[1]{{\textbf{\textit{#1}}}}
\newcommand{\tbc}{[\textbf{Incomplete}]}
\newcommand{\chk}{[\textbf{Check}]}
\newcommand{\oprob}{[\textbf{OP}]:}
\newcommand{\core}[1]{\textbf{Core Idea:}}
\newcommand{\why}{[\textbf{Find proof}]}
\newcommand{\opt}[1]{\textit{#1}}


\DeclareMathOperator*{\argmin}{arg\,min}
\DeclareMathOperator{\rank}{rank}
\newcommand{\redcol}[1]{\textcolor{red}{#1}}
\newcommand{\bluecol}[1]{\textcolor{blue}{#1}}
\newcommand{\greencol}[1]{\textcolor{green}{#1}}


\renewcommand{\~}{\htext{$\sim$}{~}}


% groupings of objects.
\newcommand{\set}[1]{\left\{ #1 \right\}}
\newcommand{\seq}[1]{\left(#1\right)}
\newcommand{\ang}[1]{\langle#1\rangle}
\newcommand{\tuple}[1]{\left(#1\right)}

% numerical shortcuts.
\newcommand{\abs}[1]{\left| #1\right|}
\newcommand{\floor}[1]{\left\lfloor #1 \right\rfloor}
\newcommand{\ceil}[1]{\left\lceil #1 \right\rceil}

% linear algebra shortcuts.
\newcommand{\change}{\Delta}
\newcommand{\norm}[1]{\left\| #1\right\|}
\newcommand{\dprod}[1]{\langle#1\rangle}
\newcommand{\linspan}[1]{\langle#1\rangle}
\newcommand{\conj}[1]{\overline{#1}}
\newcommand{\gradient}{\nabla}
\newcommand{\der}{\frac{d}{dx}}
\newcommand{\lap}{\Delta}
\newcommand{\kron}{\otimes}
\newcommand{\nperp}{\nvdash}

\newcommand{\mat}[1]{\left( \begin{smallmatrix}#1 \end{smallmatrix} \right)}

% derivatives and limits
\newcommand{\partder}[2]{\frac{\partial #1}{\partial #2}}
\newcommand{\partdern}[3]{\frac{\partial^{#3} #1}{\partial #2^{#3}}}

% Arrows
\newcommand{\diverge}{\nearrow}
\newcommand{\notto}{\nrightarrow}
\newcommand{\up}{\uparrow}
\newcommand{\down}{\downarrow}
% gets and gives are defined!

% ordering operators
\newcommand{\oleq}{\preceq}
\newcommand{\ogeq}{\succeq}

% programming and logic operators
\newcommand{\dfn}{:=}
\newcommand{\assign}{:=}
\newcommand{\co}{\ co\ }
\newcommand{\en}{\ en\ }


% logic operators
\newcommand{\xor}{\oplus}
\newcommand{\Land}{\bigwedge}
\newcommand{\Lor}{\bigvee}
\newcommand{\finish}{$\Box$}
\newcommand{\contra}{\Rightarrow \Leftarrow}
\newcommand{\iseq}{\stackrel{_?}{=}}


% Set theory
\newcommand{\symdiff}{\Delta}
\newcommand{\union}{\cup}
\newcommand{\inters}{\cap}
\newcommand{\Union}{\bigcup}
\newcommand{\Inters}{\bigcap}
\newcommand{\nullSet}{\phi}

% graph theory
\newcommand{\nbd}{\Gamma}

% Script alphabets
% For reals, use \Re

% greek letters
\newcommand{\eps}{\epsilon}
\newcommand{\del}{\delta}
\newcommand{\ga}{\alpha}
\newcommand{\gb}{\beta}
\newcommand{\gd}{\del}
\newcommand{\gf}{\phi}
\newcommand{\gF}{\Phi}
\newcommand{\gl}{\lambda}
\newcommand{\gm}{\mu}
\newcommand{\gn}{\nu}
\newcommand{\gr}{\rho}
\newcommand{\gs}{\sigma}
\newcommand{\gt}{\theta}
\newcommand{\gx}{\xi}

\newcommand{\sw}{\sigma}
\newcommand{\SW}{\Sigma}
\newcommand{\ew}{\lambda}
\newcommand{\EW}{\Lambda}

\newcommand{\Del}{\Delta}
\newcommand{\gD}{\Delta}
\newcommand{\gG}{\Gamma}
\newcommand{\gO}{\Omega}
\newcommand{\gL}{\Lambda}
\newcommand{\gS}{\Sigma}

% Formatting shortcuts
\newcommand{\red}[1]{\textcolor{red}{#1}}
\newcommand{\blue}[1]{\textcolor{blue}{#1}}
\newcommand{\htext}[2]{\texorpdfstring{#1}{#2}}

% Statistics
\newcommand{\distr}{\sim}
\newcommand{\stddev}{\sigma}
\newcommand{\covmatrix}{\Sigma}
\newcommand{\mean}{\mu}
\newcommand{\param}{\gt}
\newcommand{\ftr}{\phi}

% General utility
\newcommand{\todo}[1]{\footnote{TODO: #1}}
\newcommand{\exclaim}[1]{{\textbf{\textit{#1}}}}
\newcommand{\tbc}{[\textbf{Incomplete}]}
\newcommand{\chk}{[\textbf{Check}]}
\newcommand{\oprob}{[\textbf{OP}]:}
\newcommand{\core}[1]{\textbf{Core Idea:}}
\newcommand{\why}{[\textbf{Find proof}]}
\newcommand{\opt}[1]{\textit{#1}}


\DeclareMathOperator*{\argmin}{arg\,min}
\DeclareMathOperator{\rank}{rank}
\newcommand{\redcol}[1]{\textcolor{red}{#1}}
\newcommand{\bluecol}[1]{\textcolor{blue}{#1}}
\newcommand{\greencol}[1]{\textcolor{green}{#1}}


\renewcommand{\~}{\htext{$\sim$}{~}}



%opening
\title{Animal cognition and affect}
\author{vishvAs vAsuki}

\begin{document}
\maketitle
\tableofcontents

\part{Prelude}
\chapter{Theme}
Here we concern ourselves with facts and theories about cognition and affect discovered and proposed using the scientific method. Personal experience and insight is marked using \experience{}.

Biological mechanism behind cognition and affect is described \\elsewhere. Strategies to improve cognitive and affective performance are considered elsewhere.

\chapter{General properties}
\section{Associative nature of thought}
Thought about a certain concept is influenced strongly by (possibly weak) associations with other concepts.

\subsection{Physiological reason}
Neurons which fire together wire together - they are highly social. Nature of human thought is associative.

\subsection{Examples}
People, when asked to think about the past or future, tend to lean slightly back or forward. People when exposed to a crossword with polite words tend to exhibit more polite behavior. People asked: 'Which continent is Kenya in?' and 'What are the opposing colors in chess?', reply 'Zebra' when asked to name the first animal which comes to mind. When asked 'What does a cow drink?', the first word which comes to mind  is milk. Atheists exposed to the 10 commandments are more likely to exhibit moral behavior. People exhibit more moral behavior when in the presence of pictures of eyes/ faces.

\chapter{Approaches}
\section{Comparison to computer}
It is easy to think of the mind in relation to the body as being analogous to software being in relation to a computer. The state of a computer is ever changing - yes; and we can refer to specific things when necessary. Eg: "program X is running inside the computer, so it is presenting this output".

\section{Biology}
Understanding of biological mechanisms, especially of signalling chemicals like adrenaline, endorphine, oxytocin, dopamine, melatonin is very useful.

\section{Behaviorism}
\subsection{Academic history}
This was a reaction to focus on internal states and feelings rather than environmental stimuli in western study of human psyche and behavior. Major proponents: Skinner, perhaps Pavlov.

\subsection{Basic scenario}
An agent is placed in an environment. Its actions are called operants, and the environment may respond with a reinforcer/ stimulus.

\subsection{Avoidance of mentalist explanations}
They fastidiously avoid mentalist explanations for behavior. Unlike cognitive psychologists, they try not to postulate 'software/ circuits in the brain' which may be responsible for a certain behavior.

\subsubsection{Redefining emotion}
The behaviorist redifinition of emotion as the distribution over behavior is considered elsewhere.

They don't say 'X made him angry. He was angry so he hit the door.', but instead say 'X made him angry: hitting the door indicates that he is angry'.

\subsection{Control and prediction}
Their methodology relied mostly on single animal research. One focus was on 'control'/ training of behavior using environmental feedback. Self-control techniques use the same strategies. They did many experiments in reinforcement learning, they were able to train animals to do complex tasks etc..

\subsection{Methodology}
Peoples' reporting of emotions was considered unreliable - but this attitude predates technological advances like fMRI.

Skinner's methodological prescriptions -- single animal research, avoidance of the use of statistics, avoidance of inferred entities, to mention a few are tightly followed.


\part{Cognition}
Specific aspects of cognition making within the context of executive function and other behavior is considered elsewhere.

\chapter{Sensory perception}
\section{Automated processing}
Sensory perception involves not only the sense organs, but also specialized circuitry in the brain. This is akin to input devices + software drivers in the case of computers.

Perception often involves attention, but all processing is non-explicit and non-verbal.

\subsection{Signal sparsity}
The brain deduces most of what it knows from scant filtered signals.

Eyes keep moving around from nose to eye to lips etc.. So actual information observed is actually small, the brain does not know what is happening outside this small area. Instead, it constructs visual/ cognitive hypotheses far beyond what is observed.

\subsection{Perception at grosser levels}
\subsubsection{Situation awareness}
Situation awareness is perception of bodies in the environment at a grosser level, requiring some attention. This is observed while driving.

\subsubsection{Affects and actions}
Some understanding of the affective state of other animals is automatic, due to mirror neurons - when one carefully observes someone else performing an action, the same neurons which would have been involved in that action fire in the observer.

Similar is the case when the observed animal is in pain.

\section{Illusions}
Brain relies on scant signals, and is structurally prone to certain errors, despite being made cognizant of the illusion. This is true of both cognitive and sensory illusions.

\section{Development and acuity}
\subsection{Effect of activities}
Action (FPS) Video games are shown to increase visual acuity.

\subsection{Early childhood training}
Some perception is impossible if neural circuits to process those sensations are not developed in early childhood. For example:
\begin{itemize}
\item Infants deprived of vision in infancy remain blind.
\item People who loose eyesight later in life retain the concept of color and visualization, unlike people who are born blind. \chk
\end{itemize}

\chapter{Output/ Actuation}
This involves basic muscle movements, often in complex patterns, resulting in more complex automatic behavior like walking, running, using hands and speaking.

Large areas of the brain are associated with the use of hands.

\chapter{Declarative Memory}
Aka smRti.

\section{Procedural vs declarative knowledge}
Knowledge can be declarative or procedural depending on whether it is stored using some language (which may include words, sounds, pictures, videos, touch, taste etc..); here we are concerned with the former kind. Procedural memory is considered separately. 

\section{Importance}
Memory (short and long term) is fundamental to computation (mechanical or animal). Like thoughts, memory is like a mosaic, rather than linear.

\section{Classification, uses}
Depending on how quickly stored information (memory) is forgotten, memory is classified into short and long term. Activation of long term memory is crucial in innovative thought.

Important, frequently used information is stored for a long time.

Acquisition of procedural memory is considered elsewhere.

\subsection{Working memory}
Most people can hold 2-4 items in their memory. Working memory is fundamental to attention and behavior inhibition.

Exceeding this limit results in cognitive overload. So, prioritizing and picking information to store is critical. \experience{This may be accomplished by a process of repeated summarization.}

Greater working memory implies greater speed. It also entails stronger ability to recover attention towards a goal from distractions.

\section{Memory updates with recall}
As an memory is recalled, the memory stored in the brain is modified.

For example, 1] Whenever we remember a face, we actually reconstruct picture of the face. 2] in one case a rape victim in trying to identify the rapist from a lineup merged the memory of a similar face with the memorized face of the rapist.

\section{Long-term storage techniques}
General knowledge acquisition techniques are considered elsewhere.

Long term storage necessarily requires paying attention to the material.

\subsection{Location method}
Location method is also the technique to remember lists of objects where one, in imagination, places objects in different parts of a structure (eg: a room).

\subsection{Mnemonics}
Mnemonics are short, easier-to-remember sounds used to remember larger sentences and lists.

\subsection{Exercises}
Working harder while getting the information helps with storing memory - like associating it with previous ideas or using deliberately hard text.

\subsection{Repetition}
Spaced repetition strengthens memory. So, interleaving different information to be stored helps.

With retrieval practice, one tests and updates one's memory with attempts at recall. Retrieving a memory makes it stronger.

\section{Recall techniques}
Recall is a form of thought, and has an associative/ mosaic nature.

\subsection{Priming}
With priming, one exploits the social nature of the brain's working to access long-term memory. To remember something related to what happened while playing cricket, it helps to crouch into the posture adapted while playing cricket.

\section{Conneciton with age}
Stored memory does not dissipate during most of peoples' lives, but quickness in recall declines after youth. Storing ability may also decline. \tbc

\section{Effect of external memories}
With the advent of tools such as written books and computers, there is a (cultural and possibly biological) decline in accurate memorization and recall techniques. Eg: oral poetry in multiple cultures has declined.


\chapter{Procedural memory}
\section{Automaticity}
Some procedures/ skills may require theoretical knowledge and its conscious, planned implementation - this uses both long term and working factual memory.

Features of an automatic behavior are all or some of: efficiency, lack of awareness, unintentionality, uncontrollability. Such behavior may possibly be in response to stimulii. They are aka habits. When their optimality in accomplishing a certain task is measurable, they are called skills.

\subsection{Examples}
Piloting: not a conscious skill - so need to make mistakes on simulator, besides conscious understanding of mistakes. Math: a conscious skill - so, can learn to deliberately avoid mistakes.

\subsection{Independence from working memory}
Procedural memory is outside working memory: so some pianists can play a complex piece while holding a conversation.

Even highly skilled performers falter under stress - aka choking. Attention to step-by-step procedure, which bypasses prcedural memory and uses working + long term factual memory, disrupts automatic performance.

\subsection{Automation of conscious skills}
This is when a skill becomes automatic. Eg: trained morse code operators can copy a message while holding a conversation; a martial artist automatically kicks and punches optimally.

\section{Practice}
Repetition is fundamental to the development of skills and habits (habituation). Just as skills can be improved, habits can be removed or replaced by learning an an alternate response procedure.

Feedback and its attribution are considered elsewhere.

\subsection{Deliberate practice}
When combined with good feedback and done over a long term, we get deliberate practice. A certain study observed that in many varied fields, expert performance is attained after 10000 hours or roughly 10 years of practice.

\subsection{Effect of sleep}
REM Sleep is shown to be beneficial to procedural learning.

\section{Feedback in skill development}
Feedback - whether endogenous or exogenous, whether detailed or gross is necessary for learning skills.

\subsection{Attribution of feedback}
Attribution of some occurrence or affect (+ve or -ve) to prior occurrence (especially action) is an important part of the sequence prediction mechanism.

\subsubsection{Role of reasoning}
Attribution is often instinctive, but is sometimes mediated and modulated by reasoning.

\subsection{Power of automatic attribution}
\subsubsection{Grokking complex patterns}
The dopamine system recognizes complex temporal patterns in regular environment, even when the pattern does not have a verbal representation. Eg: Recognizing faces, balancing a cycle, The gulf war incident where a anti-missile commander, having stared at the radar for days, distinguished incoming missiles from returning friendly jets; but could not explain how.

\subsubsection{Limitations}
Trusting intuition good only in regular environments with good feedback (even if they are complex). In highly irregular environments, better not trust intuition.

\subsubsection{Arousal attribution bias}
A person is predisposed towards attributing arousal to another person/ agency. For example, a person aroused by his presence on a tall bridge is likely to attribute it to a female he talks to on a bridge; romantic dates involving roller-coaster rides are often successful.

\subsection{Reasoning based attribution}
\subsubsection{Superstition}
Behaviorists called this tendency to learn to associate an action with a result by the name ``law of effect'', which in many cases would be a 'post hoc ergo poster hoc' fallacy. In controversial experiments by Skinner, what could be superstition-motivated rituals such as bobbing head, rotating were developed by pigeons expecting food - even though food was mechanically provided at regular intervals. Compulsive gamblers - who often have an overactive dopamine system - develop superstitions and believe in omens.

\section{Connection to age}
\subsection{Early development skills}
Some skills can be learned only by young animals. For example:

\begin{itemize}
\item Children not exposed to language in their early childhood never learn it.
\item Perfect pitch recognition is only possible with musical education before the age of 7.
\item Perfect accent is possible only if the child is exposed to a language's phonemes in early childhood. Babies' movement rhythms vary with the language spoken around them.
\end{itemize}


\chapter{Simulation and modeling}
Aka vikalpa, vitarka, vichAra. Purpose could be amusement, prediction, evaluation, planning. It involves understanding, reasoning, empathy.

\section{Understanding, theorizing}
This may or may not be probabilistic. It may or may not involve explicit formal reasoning.

This includes empathy.

\subsection{Role of proability theory}
Facile and formal use of probability theory is a good idea. In any case, probability estimation in animal intelligence can be understood using the belief/ Bayesian interpretion of probability theory.


\subsection{Biases and heuristics}
\subsubsection{Recall-ease heuristic}
Aka Availability heuristic. "If it comes to my mind first, it is more probable." This causes people to underestimate odds of future pains.

Also, people overestimate their chances of experiencing very low-frequency events, including negative events.

\subsubsection{Simulation ease heuristic}
"If it is easier to imagine/ visualize, it is more likely to be true."

\subsubsection{Strength of preconceptions}
Aka Confirmation bias. The tendency to search for or interpret information in a way that confirms one's preconceptions.

“illusion of validity”: When a compelling impression of a particular event clashes with general knowledge, the impression commonly prevails.

\subsection{Sufficiency vs necessity}
\experience{Even when the intellect attempts to act according to the rules of logic, it often makes the mistake of confusing sufficiency for necessity. Eg: This was apparent in my case during GRE practice tests - especially in verbal reasoning.}

\section{Cognitive flexibility}
This is the capacity for out-of-box thinking. This depends on acceptence of new input and accessing long-term memory.

Its connection to periodic long term attention is considered elsewhere.

\section{Intelligent animals' behavior}
\subsection{Identity/ self-concept}
Aka ahaMkAra. 

Identity is a very useful model/ story used to understand animal behavior, affects and goals. This includes various (overlapping) components like self-concept, social identity, cultural identity, professional identity, and their corresponding ideals.

Identity deeply affects regret, conscience, impulse control, socially responsible behavior etc.. It is affected by habits, close relationships and declarative reasoning

\subsection{Biases}
The way animal behavior is usually modelled in informal thought changes depending on whether one's own self or someone else is being modelled. This is described in the chapter on decision.

\section{Prediction}
\subsection{Affective impact prediction}
\subsubsection{Affective impact prediction error}
Aka Impact bias: "the tendency for people to overestimate the length or the intensity of future feeling states. In other words, people seem to think that if disaster strikes it will take longer to recover emotionally than it actually does."

We are often not good at predicting what will make us happy, not at describing what made us happy in the past. 

\chapter{Evolutionary explantion}
We are evolved to be short-term, innumerate, self-centered; while we fancifully want to be long-term, dharma-centered and numerate.

\chapter{Attention}
aka manas by Hindus. Attention control is akin to task scheduling in computers

\section{Types}
Attention could be endogenous or exogenous (response to a loud noise). Endogenous attention is thought to involve frontal cortex and basal ganglia. Depending on whether it involves focusing sensors, attention is overt or covert.

\subsection{Number of tasks}
Attention may be directed at a single or multiple tasks. Usually multitasking is accomplished by focusing on each task for a short period of time. There is a cost associated with switching tasks.

\section{Impulsivity control}
This is the ability to ignore distractions.


\subsection{Attention span}
Greater attention span, to a certain extant, implies greater executive function. Response to exogenous attention: 8s. Sustained attention on a task is around 20 minutes on average, though people are able to renew attention after it.

\experience{Attention span can be measured and enforced, leading to better executive function. Eg: Pomodoro technique involves a cycle of 25 minute attention span rewarded by a 5 minute break.}

\subsection{Effect of tools and activities}
Nutritional and biological factors affecting the pre-frontal cortex are considered elsewhere.

\subsubsection{Hands}
Working with hands improves is known to improve executive function. Large areas of the brain are associated with the use of hands.

\subsubsection{Working on the web}
Working on a computer with access to the internet involves working in an environment rich in information and interruptions.

This environment rewards lower attention span with novelty. \experience{So, rewards associated impulsivity control should be stronger.}

Hence, one hypothesis is that, in the current generation, impulsivity control ability is decreasing on average. In the long term, strong executive function in an information / interruption rich environment, with the ability to handle lot of information may actually increase.

\subsection{Environmental factors}
Attention span varies with motivation, environment, fatigue and ability/ fluency. Tasks where people are in the 'flow' (which are neither too easy nor too challenging) are easier to attend to.

\subsubsection{Number of distractors}
Studies show that if there are many stimuli present (especially if they are task-related), it is much easier to ignore the non-task related stimuli, but if there are few stimuli the mind will perceive the irrelevant stimuli as well as the relevant.


\subsection{Connection with age}
Younger children have lower attention spans.

Ability to process multiple stimuli fully may decline with age after youth, leading to easier cognitive overload; but old folk compensate by focusing on the important information. \chk

Yet, people who have developed a greater 'cognitive reserve' show a slower decline with age and disease.

\section{Over the long term}
Deep human thought is to a great extant parallel. Switching attention between various tasks, while returning to the same task repeatedly/ periodically over the course of a long period of time has the beneficial effect of increasing cognitive flexibility, exploiting the associative/ abstract nature of thought.

\tbc

\chapter{Decision: evaluation and comparison}
Making a choice is fundamental to behavior. It fundamentally involves comparison.

\section{Reasoned vs automatic}
It may involve simulation, possibly with explicit verbal reasoning to various degrees.

\subsection{Preset Preferences}
Decisions not involving explicit reasoning are made using set preferences. Some preferences are formed due to developmental environment; while some tastes are acquired consciously due to habit formation. Right from the womb, the human baby learns the rhythm of its mother's language, the taste of food she ingests; and adapts its behavior upon being born.

\subsection{Efficacy}
In highly complex environments with many data-points (including in the case of automatic procedures/ skills), automatically made decisions are shown to be better than explicitly reasoned decisions.

\section{vs Awareness of decision}
MRI scans show that a decision is reached around 0.25s before the awareness of deciding. So, these are two separate things.

\section{Reasoning concepts}
Explicit reasoning behind decisions shows some common patterns and concepts.

\subsection{Identity}
This plays a major role in both reasoning leading to decisions and explanatory models. Identity itself is described elsewhere.

The Hindu ranking/ classification of goals into dharma, artha and kAma is considered elsewhere. 

\subsubsection{Fudge factor}
People allow themselves to cheat/ deviate from self-image just a little bit.

\subsubsection{Effect of moral reminders}
Atheists trying to remember 10 commandments experiment, and not cheating at all in subsequent tasks.

If people watch someone outside their group cheating, their cheating went down. In general, when reminded about morality, cheating becomes less.  If people watch someone from their own group cheating, they are more comfortable with cheating.

\subsection{Opportunity cost}
What is the cost of exploiting an opportunity? You could be loosing out on the chance to pursue other opportunities.

\subsection{Tabulation}
Tabulation is a very common and useful tool for comparing multiple alternatives. They enable comparison of alternatives along various dimensions. One can then assign various weights (implicitly or explicitly) to different features, thereby identifying the features relevant for comparison and simplifying the decision problem.

\experience{One can use tabulation to compare various alternative actions, using features like dharma, artha and kAma.}

\subsection{Comparison with past evaluations}
Comparing with past evalutations, rather than considering what is possible, to evaluate value of an outcome is a mistake. To check if price for an item is fair, ask "what other things you could do with the same money?", not "What was this item worth yesterday?".

\section{General Properties}
\subsection{Action-choices}
\subsubsection{Loss aversion}
Loss hurts much more than gains make you happy.

\subsubsection{Immediate gratification bias}
Aka anti-delayed gratification bias. If the same gratification is offered after a slight delay, people undervalue that option. To counter thus, one must imagine the future more vividly.

\subsection{General comparison}
\subsubsection{Influence of history}
Comparison with professed evaluations of others is also misleading. "The subjects consistently reported that the more expensive wines tasted better, even when they were actually identical to cheaper wines." That they actually feel greater pleasure in their brain has been confirmed by the use of MRI.

\subsubsection{Relativity bias}
Evaluating two houses side by side yields different results than evaluating three — A, B and a somewhat less appealing version of A. The subpar A makes it easier to decide that A is better — not only better than the similar one, but better than B.

\subsection{Comparing skill/ ability}
\subsubsection{Illusory superiority/ inferiority}
People with below average skill may wrongly think that they have above-average skills/ intellect etc.; this fallacy is often due to misjudging one's skills arising from one's poor understanding of the skill.

However, having consciously developed a certain skill, people often underestimate the goodness of their skills in relation to others - this often is due to the fact that others' skills are misjudged as being high by one who is very aware of one's own shortcomings.

\subsubsection{Personality-attribution error}
We often think that our success is heavily due to our qualities, while failure is due to adverse circumstances. But, we think the very opposite of other people. This can be parly seen as a combination of the confirmation bias and the 'illusory superiority' effect.

\subsubsection{Underestimation of luck, circumstances}
The planning fallacy is failing to think realistically about where one fits in the distribution of people. In irregular environments, luck and circumstance matters more than skills.
Eg: Stock market, leadership tests.

\section{Quality depletion}
Aka Ego depletion.

The more the number of decisions one makes, the less one is willing to consider deeply when making more decisions.

Food with glucose seems to temporarily increase/ restore decision making capacity and enthusiasm.

\subsection{Examples}
This is exploited by car sellers and fast food chains which offer a huge variety of choices, so that the taxed brain resorts to simply picking default options, after making a few choices. Also, Israeli judges were more likely to grant parole in the morning than in the afternoon. Also, in they were more likely to grant parole after being given glucose rich food.

\chapter{Affect}
Aka emotion, vikAra.

\section{Distribution over actions/ thoughts}
Emotion was defined by behaviorists in terms of probability distribution over responses (either to stimuli, or physical reactions/ sensations including in the brain) Eg: sadness increased certain type of responses; while others identify it purely with certain mental sensations or neurochemicals. One may extend this distribution to cover thoughts as well.

\subsection{Modes of cognition/ operation}
On may draw an analogy with a computer which can operate in different modes; and whose computation and output distribution changes according to the mode. So, affects are like modes of cognition.

\section{Flight or fight}
During this, hormones such as adrenaline and testosterone alter the body, brain and senses.

\section{Stress}
\subsection{Effects}
\subsubsection{Performance}
Stress in moderate quantity (Eustress) increases performance at a given task.

Stress in very high quantities (distress) negatively affects performance - memory begins to fail, social interactions turn dour.

\subsubsection{Long term stress}
Chronic stress leads to diseases like high blood pressure etc.. which then leads to secondary defects in the body/ brain.

\subsection{Common sources}
Time constraints (often self-imposed) for tasks.

Social/ relationship stress. Frustration at not gaining others' respect to the extant desired.

\subsection{Common destressors}
Exercise.

Time in nature.

Friendly socialization. Humor releases endorphines.

Meditation, chanting, ritual, focus on the present.

\tbc

\subsection{Limiting the levels}
Though short spikes in stress, and moderate levels of stress/ challenge are desirable, excessive or prolonged stress should be avoided. Especially, don’t leave the stress switch on for too long.

One should limit one's ambitions to be just beyond one's current abilities - thus avoiding stress caused due to failure to meet unrealistic expectations.

\section{Emotion regulation}
\tbc

\part{Behavior and action}
\chapter{Introduction}
\section{Degree of automation}
Behavior can be automatic or not, depending on the extant to which declarative memory is used; the latter class is called executive function, which involves planning, decision, error control. General aspects of cognition which underly this are considered elsewhere.

\subsection{Duration of executive function}
Goal directed effort can be considered short term and specific, or long term. This is the difference between a marathon and a sprint.

\section{Short term executive function}
3 components are critical in executive function: working memory, attention, cognitive flexibility. It is mediated by the pre-frontal cortex.

\subsection{Hindu terminology}
manas aka attention. buddhi is other aspects of executive function (working and long-term memory, decision making, prediction, inhibition).

\subsection{Importance}
This is an important measure of general intelligence. It is also economically valuable.

\section{Long term effort}
\subsection{Social encouragement/ approval}
\subsubsection{Effect of praise}
People show a big difference in their future efforts and accomplishment are praised a] for their effort and b] for their ability : the latter class fare worse than they would have otherwise. 

People subjected to praise are also more afraid to fail.

\subsubsection{Stereotype threat effect}
It turns out that when people are in a situation that defies stereotypes, reminding them of the stereotype diminishes their performance. In one study from NYU, students were given a math test. Asking men and women questions about their gender beforehand increased the performance gap substantially.

\subsubsection{Weakness in pursuing public goals}
IN a phenomenon which has been called 'substitution' among other things, we find that people who declare their goal in public tend to be less likely to actually make the effort to achieve the goal. This is because the motivational circuits are fooled into mistaking the high emotions experienced in declaring a goal for the emotions expected when the goal is actually achieved.

\subsection{Reluctance to quit}
The upside of quitting is underestimated. 'sunk-cost fallacy' often comes into play. To preserve a story about one self (ie in order to justify the past effort), sometimes one begins to like the unfruitful activity more than before.

\section{Addiction}
Addiction is a physical and psychological dependence on psychoactive substances (and possibly behavior). It is characterized by recurring compulsion to engage in the addiction despite -ve consequences (physical, social, cognitive etc..).

Psychological dependence is characterized by lack of impulse control, depression and craving upon abstinence.

Physical dependence is characterized by increased heart rate, blood pressure etc..

% \bibliographystyle{plain}
% \bibliography{ai}

\end{document}
