% Data description section for the grant proposal.
Substantial efforts have been spent over the years by many biologists and fellow researchers to collect data from biological systems. This includes information on how human genes interact with each other, relationships between genes in one organism with that of the genes in other organisms, what genes play a role in a certain human disease, how genes influence characteristics in other organisms and so on. The collected data is by no means complete. It could be highly unlikely to observe some of the natural phenomena in biological systems(like genes influencing a disease) even with state-of-the-art technology. Furthermore, time, money and other resource constraints may play great barriers that render such experiments impossible. The collected biological data not only give insights into understanding the fundamental processes in biological systems, but also form the basis for application of computational techniques to make interesting and useful inferences. In this section, we describe in detail the biological networks that form the basis of our proposed research in network analysis. Before we delve into details, let us introduce some basic terminology used in the document.\\
\exclaim{Gene.} A portion of DNA of an organism (species) responsible for hereditary transfer of ``traits''.\\
\exclaim{Phenotype.} A trait of an organism: morphological, physiological or behavioral. Phenotypes result from the expression of specific genes, among other factors. Examples of phenotypes are \emph{lung cancer, color blindness, abnormal sleep pattern}, etc.\\
\exclaim{Ortholog.} A gene in a different species that evolved from an ancestral gene when the two species diverged.  A number of human genes have orthologs in mouse, yeast and other species.

We will represent networks as graphs with the network entities as nodes and the ``communication'' between the entities as edges or links between the nodes. In social networks that we consider, e.g. a social networking site like \emph{Orkut} or movie rental site like Netflix, the entities are the people who are members of the network and the communities or groups they affiliate with. In biological networks, the entities are genes and phenotypes. Different types of networks can co-exist among a set of entities depending on the nature of relationships. To continue with our examples in social networks, we can conceive of a ``friendship'' network among people or an ``affiliation'' network where edges are from people to communities (or affiliations). Interactions between genes in a species and their expression in characteristic phenotypes or traits of the species can be conceptualized as biological networks for purposes of study and analysis. We now discuss in detail about the origin, representation, and characteristics of various biological networks.

\subsection{Biological networks: Description and Sources}
\label{section:dataset}
Biological networks arise naturally due to the interaction of genes in biological processes and influence of genes in morphological or physiological traits of different species. In biological systems, we are interested in the network of interaction of genes, i.e., \emph{gene-gene} networks and the expression of genes in phenotypes, i.e., \emph{gene-phenotype} networks. The edges may be weighted in order to indicate the strength of the interaction. Our biological networks comprise 6 different species namely human, plant(\emph{Arabidopsis}), worm, fly, mouse and yeast as shown in Table \ref{tab:data}. Note that we can conceive of a single biological network with the genes and phenotypes of all the species as entities. Figure \ref{fig:yeast} shows one set of interactions among genes in yeast species\cite{pmid15567862}.
\begin{figure}[ht]
  \begin{center}
  \subfigure[Yeast species]{\includegraphics[height=5cm, width=7cm]{figures/yeast.jpg}}
  \subfigure[Internet]{\includegraphics[height=5cm, width=7cm]{figures/wired.png}}
  \end{center}
\vspace{-5ex}
\caption{Networks from different domains}
\label{fig:yeast}
\end{figure}

\begin{table}[ht]
\centering
\begin{tabular}{| c | c | r | r | r |} \hline
\textsc{Species} & \textsc{Acronym} & \textsc{Genes} & \textsc{Phenotypes} & \textsc{\# Known Relations} \\ \hline
Human & Hs & 1,488 & 292 & 1,921\\ \hline
Plant & At & 4,074 & 890 & 13,968\\ \hline
Worm & Ce & 3,148 & 440 & 19,161\\ \hline
Fly  & Dm & 4,079 & 2,609 & 53,985\\ \hline
Mouse & Mm & 4,795 & 4,056 & 72,601\\ \hline
Yeast & Sc & 3,805 & 1,150 & 66,755\\ \hline
\end{tabular}
\caption{Species which can be used for disease modeling, and sizes of the biological networks for different species considering only orthologs of human genes.}
\label{tab:data}
\end{table}

\paragraph{Bipartite graphs perspective.}
A bipartite graph $G = (V, E)$ is a graph whose vertex set $V$ can be partitioned into two disjoint independent sets $V_{1}$ and $V_{2}$ such that for any edge $(u, v) \in E$, $u \in V_{1}$ and $v \in V_{2}$. The gene-phenotype network for a particular species can be viewed a bipartite graph where $V_{1}$ is the set of orthologs of human genes in that species and $V_{2}$ is the set of phenotypes of that species\cite{pmid19010805, pmid20215462}. Furthermore, the network of the genes and phenotypes of all species can be viewed a bigger bipartite graph. 

The biological data was collected from multiple sources including \cite{pmid18223650, pmid17912365, pmid18613949, pmid20118918}; additional data for flies have since been assembled from publicly available datasets. Detailed description on the extraction of the data sets can be found in \cite{McGarySI}. In particular, when a human gene is known to occur as multiple orthologs in a species, a single row represents the entire set of orthologs in the species-specific gene-phenotype association matrix. Note that the human gene-gene network is obtained from sources independent of gene-phenotype networks\cite{pmid19767740, pmid19728866, pmid19246570}. Networks are represented using adjacency matrices. Recall that the problem of interest is to predict the association of genes in human diseases (phenotypes are diseases in this case). Consequently, the genes in the network are only the human genes. In case of other species, we are interested in the orthologs of human genes in the species. The size of the preprocessed data sets is shown in Table \ref{tab:data}.

\subsection{Biological networks and social networks: an analogy}
In a social network like \emph{Orkut}, there exists a friendship network among its users and affiliation network among users and affiliations, which is a bipartite graph. It is interesting to note that this is analogous to the biological network with genes playing the role of users and phenotypes playing the role of affiliations. Hence, models for evolution and link formation in social networks could be adapted to explain biological networks. The study and analysis of social networks could be applied to biological networks. We will see in Section \ref{sec:preliminaryResearch} that our preliminary experiments indicate that our previous work on social network link prediction does well in predicting the association of genes in phenotypes.

It is interesting to observe some of the statistics of the data set. For example, Figure \ref{fig:genePhenotypeDist} shows the frequency distributions of genes observed in a human and mouse phenotype, and diseases or phenotypes in which a human gene (or its ortholog) is observed. Observe that only a very few diseases have more than 10 genes known to be involved. Importantly, we see that the number of genes associated with phenotypes in mouse is significantly greater than the number of genes associated with diseases in humans. In fact, gene-phenotype connections are relatively better studied in many other species as well. The observations imply that there is much more information in other species that can potentially be used for identifying human gene-disease connections.

\begin{figure}[ht]
  \begin{center}
%     \subfigure[Distribution of genes in human diseases]{\includegraphics[scale=0.3]{figures/NumGenesInDiseases1.jpg}}
%     \subfigure[Distribution of diseases of human genes]{\includegraphics[scale=0.3]{figures/NumDiseasesOfGenes1.jpg}}
%     \subfigure[Distribution of orthologs in mouse phenotypes]{\includegraphics[scale=0.3]{figures/NumGenesInPhenotypes5.jpg}}
%     \subfigure[Distribution of phenotypes of orthologs in mouse]{\includegraphics[scale=0.3]{figures/NumPhenotypesOfGenes5.jpg}}
    \subfigure[]{\includegraphics[height=4cm, width=7cm]{figures/NumGenesInDiseases1.jpg}}
    \subfigure[]{\includegraphics[height=4cm, width=7cm]{figures/NumDiseasesOfGenes1.jpg}}
    \subfigure[]{\includegraphics[height=4cm, width=7cm]{figures/NumGenesInPhenes5.jpg}}
    \subfigure[]{\includegraphics[height=4cm, width=7cm]{figures/NumPhenesOfGenes5.jpg}}
  \end{center}
\vspace{-5ex}
\caption{Statistics of the gene-phenotype network in humans(top) and mouse(bottom). (a) Only a few diseases have more than 10 genes known to be involved. (b) The number of genes associated with human diseases is rather small. (c) Many phenotypes in mouse have larger number of genes known to be associated as compared to human diseases. (d) The number of genes associated with mouse phenotypes is significantly higher than humans. }
\label{fig:genePhenotypeDist}
\end{figure}
