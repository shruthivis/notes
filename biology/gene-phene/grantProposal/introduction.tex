The search for cures to diseases is an ancient one. In recent years, there has been an enormous growth in the amount and variety of information that one can use in this endeavor. Mathematical modelling and data driven analysis can play a critical role both in generating and in exploiting this information. A fundamental problem in biology is to identify the genes which affect the expression of physiological traits of organisms, or phenotypes. The problem of associating genes with traits is interesting in its own right as it increases our understanding of the mechanism of life, but it is also important from the health-care perspective. Consider the case of genetic diseases as an example. Identification of the genes responsible for such diseases could potentially be of great use in drug diagnostic design. However, while the entire human genome has been sequenced, biologists still do not know the function of more than one-third of the genes in the human genome. Identification of genes which influence the development of a disease will greatly aid drug design and therapy. Thus, a promising and critical application of data driven analysis in bioinformatics lies in predicting genes that are responsible for various genetic human diseases and traits.

Among the roughly 20,000 known human genes, how does one identify the set of genes which influence a given human disease or trait? Such identification has, in recent years, been done using genome-wide association studies\cite{GWAurl} --- a relatively slow and laborious process based on surveying the genome of people with a disease for haplotypes which occur more frequently in such individuals. However, one can use information from a variety of other sources to identify gene-disease links much more quickly and accurately. Examples of such sources of information include homology relationships between human genes and genes of other species, and the relationship between genes and phenotypes in other organisms. The utility of this information is based on the fact that all life on Earth shares common ancestry, and therefore shares sets of genes. For example, genes have been found in plants, that are responsible for Waardenberg syndrome in humans. To exploit such diverse sources of information, using the power of mathematical modelling and bioinformatics is essential. Such an approach is being pioneered at the University of Texas at Austin \cite{McGaryOrthologousPhenotypes}, and preliminary results indicate great promise in predicting novel genes that are responsible for human diseases and subsequent drug design for better treatment of diseases such as breast cancer. Even as researchers have barely begun to scratch the surface of this promising line of research, they have already been able to predict novel genes associated with diseases such as cancer, angiogenesis effects and deafness \cite{newsMarcotteNY, newsMarcotte}.

In this proposal, we outline a mathematical and data-driven approach to answer the question of what genes influence a given disease, and furthermore, to advance radically new research in the area of disease modeling, computational medicine and (possibly) drug design. PI Dhillon's research is in the areas of large-scale data analysis, data mining and machine learning, where mathematical and computational tools are used for various predictive tasks. The proposed research will generalize methods for large-scale data analysis developed in Prof. Dhillon's lab, as well as develop novel methods, to the biological data available and inferred in Prof. Marcotte's lab. It should be noted that the techniques developed as part of the proposed research are not limited to this biological application. Indeed, the proposed techniques will be important for network analysis in other biological as well as non-biological applications.

The proposed research will develop mathematical models for relationships between genes and phenotypes. This will enable the development of novel mathematical techniques that derive their roots from areas as diverse as graph theory, network analysis, probabilistic graphical models and compressive sensing. For example, genes and phenotypes may be modelled as nodes in a graph, the relationships between them as edges, the nodes may further have annotations or features, and the problem at hand may be viewed abstractly as one of discovering new relationships between these nodes. Alternately, the gene-phenotype data may be viewed as a matrix with missing values, and techniques along the lines of matrix completion may be brought to bear on it. Though the proposed research will be targeted towards finding associations between genes and diseases, the mathematical models and techniques developed are expected to advance the state of the art in diverse areas such as recommendation systems, social network analysis, graphical models and matrix completion (compressive sensing).
