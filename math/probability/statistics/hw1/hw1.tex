\documentclass{article}
% Configuration for the memoir class.
\renewcommand{\cleardoublepage}{}
% \renewcommand*{\partpageend}{}
\renewcommand{\afterpartskip}{}
\maxsecnumdepth{subsubsection} % number subsections
\maxtocdepth{subsubsection}

\addtolength{\parindent}{-5mm}
% Packages not included:
% For multiline comments, use caption package. But this conflicts with hyperref while making html files.
% subfigure conflicts with use with memoir style-sheet.

% Use something like:
% % Use something like:
% % Use something like:
% \input{../../macros}

% groupings of objects.
\newcommand{\set}[1]{\left\{ #1 \right\}}
\newcommand{\seq}[1]{\left(#1\right)}
\newcommand{\ang}[1]{\langle#1\rangle}
\newcommand{\tuple}[1]{\left(#1\right)}

% numerical shortcuts.
\newcommand{\abs}[1]{\left| #1\right|}
\newcommand{\floor}[1]{\left\lfloor #1 \right\rfloor}
\newcommand{\ceil}[1]{\left\lceil #1 \right\rceil}

% linear algebra shortcuts.
\newcommand{\change}{\Delta}
\newcommand{\norm}[1]{\left\| #1\right\|}
\newcommand{\dprod}[1]{\langle#1\rangle}
\newcommand{\linspan}[1]{\langle#1\rangle}
\newcommand{\conj}[1]{\overline{#1}}
\newcommand{\gradient}{\nabla}
\newcommand{\der}{\frac{d}{dx}}
\newcommand{\lap}{\Delta}
\newcommand{\kron}{\otimes}
\newcommand{\nperp}{\nvdash}

\newcommand{\mat}[1]{\left( \begin{smallmatrix}#1 \end{smallmatrix} \right)}

% derivatives and limits
\newcommand{\partder}[2]{\frac{\partial #1}{\partial #2}}
\newcommand{\partdern}[3]{\frac{\partial^{#3} #1}{\partial #2^{#3}}}

% Arrows
\newcommand{\diverge}{\nearrow}
\newcommand{\notto}{\nrightarrow}
\newcommand{\up}{\uparrow}
\newcommand{\down}{\downarrow}
% gets and gives are defined!

% ordering operators
\newcommand{\oleq}{\preceq}
\newcommand{\ogeq}{\succeq}

% programming and logic operators
\newcommand{\dfn}{:=}
\newcommand{\assign}{:=}
\newcommand{\co}{\ co\ }
\newcommand{\en}{\ en\ }


% logic operators
\newcommand{\xor}{\oplus}
\newcommand{\Land}{\bigwedge}
\newcommand{\Lor}{\bigvee}
\newcommand{\finish}{$\Box$}
\newcommand{\contra}{\Rightarrow \Leftarrow}
\newcommand{\iseq}{\stackrel{_?}{=}}


% Set theory
\newcommand{\symdiff}{\Delta}
\newcommand{\union}{\cup}
\newcommand{\inters}{\cap}
\newcommand{\Union}{\bigcup}
\newcommand{\Inters}{\bigcap}
\newcommand{\nullSet}{\phi}

% graph theory
\newcommand{\nbd}{\Gamma}

% Script alphabets
% For reals, use \Re

% greek letters
\newcommand{\eps}{\epsilon}
\newcommand{\del}{\delta}
\newcommand{\ga}{\alpha}
\newcommand{\gb}{\beta}
\newcommand{\gd}{\del}
\newcommand{\gf}{\phi}
\newcommand{\gF}{\Phi}
\newcommand{\gl}{\lambda}
\newcommand{\gm}{\mu}
\newcommand{\gn}{\nu}
\newcommand{\gr}{\rho}
\newcommand{\gs}{\sigma}
\newcommand{\gt}{\theta}
\newcommand{\gx}{\xi}

\newcommand{\sw}{\sigma}
\newcommand{\SW}{\Sigma}
\newcommand{\ew}{\lambda}
\newcommand{\EW}{\Lambda}

\newcommand{\Del}{\Delta}
\newcommand{\gD}{\Delta}
\newcommand{\gG}{\Gamma}
\newcommand{\gO}{\Omega}
\newcommand{\gL}{\Lambda}
\newcommand{\gS}{\Sigma}

% Formatting shortcuts
\newcommand{\red}[1]{\textcolor{red}{#1}}
\newcommand{\blue}[1]{\textcolor{blue}{#1}}
\newcommand{\htext}[2]{\texorpdfstring{#1}{#2}}

% Statistics
\newcommand{\distr}{\sim}
\newcommand{\stddev}{\sigma}
\newcommand{\covmatrix}{\Sigma}
\newcommand{\mean}{\mu}
\newcommand{\param}{\gt}
\newcommand{\ftr}{\phi}

% General utility
\newcommand{\todo}[1]{\footnote{TODO: #1}}
\newcommand{\exclaim}[1]{{\textbf{\textit{#1}}}}
\newcommand{\tbc}{[\textbf{Incomplete}]}
\newcommand{\chk}{[\textbf{Check}]}
\newcommand{\oprob}{[\textbf{OP}]:}
\newcommand{\core}[1]{\textbf{Core Idea:}}
\newcommand{\why}{[\textbf{Find proof}]}
\newcommand{\opt}[1]{\textit{#1}}


\DeclareMathOperator*{\argmin}{arg\,min}
\DeclareMathOperator{\rank}{rank}
\newcommand{\redcol}[1]{\textcolor{red}{#1}}
\newcommand{\bluecol}[1]{\textcolor{blue}{#1}}
\newcommand{\greencol}[1]{\textcolor{green}{#1}}


\renewcommand{\~}{\htext{$\sim$}{~}}


% groupings of objects.
\newcommand{\set}[1]{\left\{ #1 \right\}}
\newcommand{\seq}[1]{\left(#1\right)}
\newcommand{\ang}[1]{\langle#1\rangle}
\newcommand{\tuple}[1]{\left(#1\right)}

% numerical shortcuts.
\newcommand{\abs}[1]{\left| #1\right|}
\newcommand{\floor}[1]{\left\lfloor #1 \right\rfloor}
\newcommand{\ceil}[1]{\left\lceil #1 \right\rceil}

% linear algebra shortcuts.
\newcommand{\change}{\Delta}
\newcommand{\norm}[1]{\left\| #1\right\|}
\newcommand{\dprod}[1]{\langle#1\rangle}
\newcommand{\linspan}[1]{\langle#1\rangle}
\newcommand{\conj}[1]{\overline{#1}}
\newcommand{\gradient}{\nabla}
\newcommand{\der}{\frac{d}{dx}}
\newcommand{\lap}{\Delta}
\newcommand{\kron}{\otimes}
\newcommand{\nperp}{\nvdash}

\newcommand{\mat}[1]{\left( \begin{smallmatrix}#1 \end{smallmatrix} \right)}

% derivatives and limits
\newcommand{\partder}[2]{\frac{\partial #1}{\partial #2}}
\newcommand{\partdern}[3]{\frac{\partial^{#3} #1}{\partial #2^{#3}}}

% Arrows
\newcommand{\diverge}{\nearrow}
\newcommand{\notto}{\nrightarrow}
\newcommand{\up}{\uparrow}
\newcommand{\down}{\downarrow}
% gets and gives are defined!

% ordering operators
\newcommand{\oleq}{\preceq}
\newcommand{\ogeq}{\succeq}

% programming and logic operators
\newcommand{\dfn}{:=}
\newcommand{\assign}{:=}
\newcommand{\co}{\ co\ }
\newcommand{\en}{\ en\ }


% logic operators
\newcommand{\xor}{\oplus}
\newcommand{\Land}{\bigwedge}
\newcommand{\Lor}{\bigvee}
\newcommand{\finish}{$\Box$}
\newcommand{\contra}{\Rightarrow \Leftarrow}
\newcommand{\iseq}{\stackrel{_?}{=}}


% Set theory
\newcommand{\symdiff}{\Delta}
\newcommand{\union}{\cup}
\newcommand{\inters}{\cap}
\newcommand{\Union}{\bigcup}
\newcommand{\Inters}{\bigcap}
\newcommand{\nullSet}{\phi}

% graph theory
\newcommand{\nbd}{\Gamma}

% Script alphabets
% For reals, use \Re

% greek letters
\newcommand{\eps}{\epsilon}
\newcommand{\del}{\delta}
\newcommand{\ga}{\alpha}
\newcommand{\gb}{\beta}
\newcommand{\gd}{\del}
\newcommand{\gf}{\phi}
\newcommand{\gF}{\Phi}
\newcommand{\gl}{\lambda}
\newcommand{\gm}{\mu}
\newcommand{\gn}{\nu}
\newcommand{\gr}{\rho}
\newcommand{\gs}{\sigma}
\newcommand{\gt}{\theta}
\newcommand{\gx}{\xi}

\newcommand{\sw}{\sigma}
\newcommand{\SW}{\Sigma}
\newcommand{\ew}{\lambda}
\newcommand{\EW}{\Lambda}

\newcommand{\Del}{\Delta}
\newcommand{\gD}{\Delta}
\newcommand{\gG}{\Gamma}
\newcommand{\gO}{\Omega}
\newcommand{\gL}{\Lambda}
\newcommand{\gS}{\Sigma}

% Formatting shortcuts
\newcommand{\red}[1]{\textcolor{red}{#1}}
\newcommand{\blue}[1]{\textcolor{blue}{#1}}
\newcommand{\htext}[2]{\texorpdfstring{#1}{#2}}

% Statistics
\newcommand{\distr}{\sim}
\newcommand{\stddev}{\sigma}
\newcommand{\covmatrix}{\Sigma}
\newcommand{\mean}{\mu}
\newcommand{\param}{\gt}
\newcommand{\ftr}{\phi}

% General utility
\newcommand{\todo}[1]{\footnote{TODO: #1}}
\newcommand{\exclaim}[1]{{\textbf{\textit{#1}}}}
\newcommand{\tbc}{[\textbf{Incomplete}]}
\newcommand{\chk}{[\textbf{Check}]}
\newcommand{\oprob}{[\textbf{OP}]:}
\newcommand{\core}[1]{\textbf{Core Idea:}}
\newcommand{\why}{[\textbf{Find proof}]}
\newcommand{\opt}[1]{\textit{#1}}


\DeclareMathOperator*{\argmin}{arg\,min}
\DeclareMathOperator{\rank}{rank}
\newcommand{\redcol}[1]{\textcolor{red}{#1}}
\newcommand{\bluecol}[1]{\textcolor{blue}{#1}}
\newcommand{\greencol}[1]{\textcolor{green}{#1}}


\renewcommand{\~}{\htext{$\sim$}{~}}


% groupings of objects.
\newcommand{\set}[1]{\left\{ #1 \right\}}
\newcommand{\seq}[1]{\left(#1\right)}
\newcommand{\ang}[1]{\langle#1\rangle}
\newcommand{\tuple}[1]{\left(#1\right)}

% numerical shortcuts.
\newcommand{\abs}[1]{\left| #1\right|}
\newcommand{\floor}[1]{\left\lfloor #1 \right\rfloor}
\newcommand{\ceil}[1]{\left\lceil #1 \right\rceil}

% linear algebra shortcuts.
\newcommand{\change}{\Delta}
\newcommand{\norm}[1]{\left\| #1\right\|}
\newcommand{\dprod}[1]{\langle#1\rangle}
\newcommand{\linspan}[1]{\langle#1\rangle}
\newcommand{\conj}[1]{\overline{#1}}
\newcommand{\gradient}{\nabla}
\newcommand{\der}{\frac{d}{dx}}
\newcommand{\lap}{\Delta}
\newcommand{\kron}{\otimes}
\newcommand{\nperp}{\nvdash}

\newcommand{\mat}[1]{\left( \begin{smallmatrix}#1 \end{smallmatrix} \right)}

% derivatives and limits
\newcommand{\partder}[2]{\frac{\partial #1}{\partial #2}}
\newcommand{\partdern}[3]{\frac{\partial^{#3} #1}{\partial #2^{#3}}}

% Arrows
\newcommand{\diverge}{\nearrow}
\newcommand{\notto}{\nrightarrow}
\newcommand{\up}{\uparrow}
\newcommand{\down}{\downarrow}
% gets and gives are defined!

% ordering operators
\newcommand{\oleq}{\preceq}
\newcommand{\ogeq}{\succeq}

% programming and logic operators
\newcommand{\dfn}{:=}
\newcommand{\assign}{:=}
\newcommand{\co}{\ co\ }
\newcommand{\en}{\ en\ }


% logic operators
\newcommand{\xor}{\oplus}
\newcommand{\Land}{\bigwedge}
\newcommand{\Lor}{\bigvee}
\newcommand{\finish}{$\Box$}
\newcommand{\contra}{\Rightarrow \Leftarrow}
\newcommand{\iseq}{\stackrel{_?}{=}}


% Set theory
\newcommand{\symdiff}{\Delta}
\newcommand{\union}{\cup}
\newcommand{\inters}{\cap}
\newcommand{\Union}{\bigcup}
\newcommand{\Inters}{\bigcap}
\newcommand{\nullSet}{\phi}

% graph theory
\newcommand{\nbd}{\Gamma}

% Script alphabets
% For reals, use \Re

% greek letters
\newcommand{\eps}{\epsilon}
\newcommand{\del}{\delta}
\newcommand{\ga}{\alpha}
\newcommand{\gb}{\beta}
\newcommand{\gd}{\del}
\newcommand{\gf}{\phi}
\newcommand{\gF}{\Phi}
\newcommand{\gl}{\lambda}
\newcommand{\gm}{\mu}
\newcommand{\gn}{\nu}
\newcommand{\gr}{\rho}
\newcommand{\gs}{\sigma}
\newcommand{\gt}{\theta}
\newcommand{\gx}{\xi}

\newcommand{\sw}{\sigma}
\newcommand{\SW}{\Sigma}
\newcommand{\ew}{\lambda}
\newcommand{\EW}{\Lambda}

\newcommand{\Del}{\Delta}
\newcommand{\gD}{\Delta}
\newcommand{\gG}{\Gamma}
\newcommand{\gO}{\Omega}
\newcommand{\gL}{\Lambda}
\newcommand{\gS}{\Sigma}

% Formatting shortcuts
\newcommand{\red}[1]{\textcolor{red}{#1}}
\newcommand{\blue}[1]{\textcolor{blue}{#1}}
\newcommand{\htext}[2]{\texorpdfstring{#1}{#2}}

% Statistics
\newcommand{\distr}{\sim}
\newcommand{\stddev}{\sigma}
\newcommand{\covmatrix}{\Sigma}
\newcommand{\mean}{\mu}
\newcommand{\param}{\gt}
\newcommand{\ftr}{\phi}

% General utility
\newcommand{\todo}[1]{\footnote{TODO: #1}}
\newcommand{\exclaim}[1]{{\textbf{\textit{#1}}}}
\newcommand{\tbc}{[\textbf{Incomplete}]}
\newcommand{\chk}{[\textbf{Check}]}
\newcommand{\oprob}{[\textbf{OP}]:}
\newcommand{\core}[1]{\textbf{Core Idea:}}
\newcommand{\why}{[\textbf{Find proof}]}
\newcommand{\opt}[1]{\textit{#1}}


\DeclareMathOperator*{\argmin}{arg\,min}
\DeclareMathOperator{\rank}{rank}
\newcommand{\redcol}[1]{\textcolor{red}{#1}}
\newcommand{\bluecol}[1]{\textcolor{blue}{#1}}
\newcommand{\greencol}[1]{\textcolor{green}{#1}}


\renewcommand{\~}{\htext{$\sim$}{~}}



%opening
\title{Data mining: Homework 1}
\author{vishvAs vAsuki}

\begin{document}

\maketitle

\section{1}
\subsection{Notation}
Given a training set $\set{(x_{i}, y_{i})_{i=1}^{N}}$. Let x be the vector of $(x_{i})$; and let y be the corresponding $(y_{i})$ vetor. Construct the $N \times 2$ matrix X = [1 x]. Take $w = (w_{0}, w_{1})$.

Below, we use the symbols $\bar{x}, \bar{y}, \stddev_{xy}, \stddev_{xx}$ as defined in the question.

\subsection{a}
Now, we want to solve the least squares problem: $Xw \approx y$.

Forming the normal equations, we get:
\begin{eqnarray*}
X^{T}Xw &=& X^{T}y\\
\mat{N & \sum x_{i}\\ \sum x_{i} & x^{T}x}w &=& \mat{\sum y_{i}\\ x^{T}y}\\
\mat{1 & \bar{x}\\ \bar{x} & \sum x_{i}^{2}/N}w &=& \mat{\bar{y}\\ \sum x_{i}y_{i}/N}\\
\mat{1 & \bar{x}\\ 0 & \sum x_{i}^{2}/N - \bar{x}^{2}}w &=& \mat{\bar{y}\\ \sum x_{i}y_{i}/N - \bar{x}\bar{y}}\\
\end{eqnarray*}

We have carried out Gaussian elimination above. Solving these equations for w, after some algebra, we find:
\begin{eqnarray*}
w_{1} &=& \frac{\sum x_{i}y_{i}/N - \bar{x}\bar{y}}{\sum x_{i}^{2}/N - \bar{x}^{2}}\\
&=& \frac{N^{-1}\sum(x_{i}-\bar{x})(y_{i}-\bar{y})}{N^{-1}\sum(x_{i}-\bar{x})^{2}}\\
&=& \frac{\stddev_{xy}}{\stddev_{xx}}\\
w_{0} &=& \bar{y} - w_{1}\bar{x}\\
\end{eqnarray*}

\subsection{b}
In general, if $x_{i}\in R^{d}$: If $x_{i,j}$ is the jth component of the vector $x_{i}$, and if $\bar{x_{j}} = \frac{\sum_{i=1}^{N}x_{i,j}}{N}$: $w_{0} = \bar{y} - \sum_{i=1}^{d} w_{i}\bar{x_{j}}$.

\subsubsection{Proof}
Consider the $N*(d+1)$ matrix $X = [1\ x_{i}^{T}]$, so that $X_{i,1} = 1, X_{i,j} = x_{i,j-1} \forall j>1$. Then, forming the normal equations: $N^{-1}X^{T}Xw = N^{-1}X^{T}y$.

Examine the first row of $N_{-1}X^{T}X$: you have $[1\ \bar{x_{1}} .. \bar{x_{d}}]$, and the first element in $N^{-1}X^{T}y$ is $\bar{y}$.

\section{2}
The proof is invalid as $\sum_{i=0}^{\infty} \gb^{i}A^{i}$ does not converge $\forall \gb \geq 0$ and multiplyication by $\infty$ is not well defined. But this will not happen if $\gb < \frac{1}{nt}$ where $\max_{i,j}|A_{i,j}| = t$.

\subsection{Proof}
Note that A, being an adjascency matrix, satisfies: $A_{i,j} \geq 0$.

Then take $a = \gb nt$. The series $\sum_{i=0}^{\infty}a^{i}$ is a geometric series which converges to $\frac{1}{1-a}$.

Now consider a matrix $T \in R^{n \times n}: T_{i,j} = t$. Now, $\sum_{i=0}^{\infty} T^{i}$ converges as $T^{i}_{j,k} = a^{i-1} \forall i>2$.

Now, $A_{i,j} \leq T_{i,j} = t$. So, the series corresponding to the (i,j)th element of the matrix sum:  $\sum_{k} \gb^{k} A^{k}_{i,j}$ is bounded, and this series is non-decreasing. Hence, the series, being non-decreasing and bounded, is convergent.

\section{3}
The code used:
\begin{verbatim} 
load dataset1 
Xtrain = [ones(30,1) Xtrain]
Xtest = [ones(120,1) Xtest]

A = Xtrain'*Xtrain
b = Xtrain'*Ytrain

w = A\b
sqError = (Ytrain - Xtrain*w)'*(Ytrain - Xtrain*w)
sqrt(sqError/30)
sqError = (Ytest - Xtest*w)'*(Ytest - Xtest*w)
sqrt(sqError/120)

[U S V] = svd(Xtrain)


w = V*inv(S'*S)*V'*Xtrain'*Ytrain
sqError = (Ytrain - Xtrain*w)'*(Ytrain - Xtrain*w)
sqrt(sqError/30)
sqError = (Ytest - Xtest*w)'*(Ytest - Xtest*w)
sqrt(sqError/120)

load dataset2
Xtrain = [ones(30,1) Xtrain]
Xtest = [ones(120,1) Xtest]

A = Xtrain'*Xtrain
b = Xtrain'*Ytrain

w = A\b
sqError = (Ytrain - Xtrain*w)'*(Ytrain - Xtrain*w)
sqrt(sqError/30)
sqError = (Ytest - Xtest*w)'*(Ytest - Xtest*w)
sqrt(sqError/120)

[U S V] = svd(Xtrain)
T=(S(1:4, 1:4)*S(1:4, 1:4))
b = inv(T)*V(:,1:4)'*Xtrain'*Ytrain

w = V(:,1:4)'\b
sqError = (Ytrain - Xtrain*w)'*(Ytrain - Xtrain*w)
sqrt(sqError/30)
sqError = (Ytest - Xtest*w)'*(Ytest - Xtest*w)
sqrt(sqError/120)
\end{verbatim}

\subsection{a}
RMS Error by solving Normal equations: Training error: 0.1566. Test error: 0.1726.

RMS Error using SVD: Training error: 0.1566. Test error: 0.1726.

\subsection{b}
RMS Error by solving Normal equations: Training error: 0.1576. Test error: 0.1822.

RMS Error using SVD: Training error: 0.1566. Test error: 0.1726.

Please see the attached code to see how I use the SVD: I drop the columns of V corresponding to $\sw_{i} \approx 0$.

Using SVD, as expected, yields more accurate results than using LU.

The second data-set contains no extra information compared to the first data-set. So, SVD's performance on the two datasets is identical.

\section{4}
\subsection{Notation}
$x = (x_{1}, x_{2}, x_{3})$ represents height, weight and age of data-point x.

Bob uses the units: inches, pounds, months. Alice uses the units centimeters, kilograms, days.

\subsubsection{Assumption about w and the linear model}
We assume that $w \in R^{3+1}$. w is indexed from 0 to 3. The linear model is $y \approx w_{0} + \sum_{i=1}^{3} w_{i}x_{i}$.

\subsubsection{The diagonal matrix D}
 Let $x_{A}, x_{B}$ be the observations of the same data point, as measured by Alice and Bob. Then, $x_{A}^{T} = x_{B}^{T}D' $ where D' is a diagonal matrix expressing the factors which relate the units used by Alice to the units used by Bob, like cm/in etc..

Arrange the observations of various data points $\set{x}$ by Alice and Bob as rows in the matreces A and B, but ensure that the first column of both these matrices is 1. Note that $A=BD$, where $D = \mat{1 & 0\\ 0 & D'}$.

\subsection{a}
The normal equations are $A^{T}Aw = A^{T}y$ and $B^{T}Bz = B^{T}y$ for Alice and Bob. The former can be rewritten as $D^{T}B^{T}BDw = D^{T}B^{T}y$ or $B^{T}BDw = B^{T}y$. Thus, we see that $Dw = z$ or $w = D^{-1}z$.

This tells us the relationship between z and w, the solutions of Bob and Alice to the least squares problem.

\subsection{b}
The regularization part of the objective in the ridge regression problem should ideally not include $w_{0}$. We assume that this is the case with the question, and that $\gl \norm{w}_{2}$ term considers only the $w_{1} .. w_{3}$  terms. For this reason, we rewrite the objective as: $\min_{w} \norm{Xw} + w^{T}I'w$, where $I' = \mat{0 & 0\\ 0 & I_{3}}$.

$(A^{T}A + \gl I')w = A^{T}y$ and $(B^{T}B + \gl I')z = B^{T}y$ express the solution to the ridge regression problem. The former can be rewritten as : $(D^{T}B^{T}BD + \gl I')w = D^{T}B^{T}y$.

So, $(D^{T}B^{T}BD + \gl I')w = D^{T}(B^{T}B + \gl I')z$. This tells us the relationship between z and w, the solutions of Bob and Alice to the ridge regression problem.

\subsection{c}
Suppose the label vector in the training set is changed to $\bar{y} = y + 1$. Let the modified least squares solution be w'. As we noted in the first problem's solution, $w'_{0} = \frac{\sum \bar{y_{i}}}{N} - \sum_{i=1}^{d} w_{i}\bar{x_{j}}$. $w'_{0} = 1 + \frac{\sum y_{i}}{N} - \sum_{i=1}^{d} w_{i}\bar{x_{j}} = 1 + w_{0}$.

However, $\forall i>0, w_{i}' = w_{i}$. This equality does not change when ridge regression is applied, as in the regularizer, $w_{0}$ is omitted.

% \bibliographystyle{plain}
% \bibliography{../linAlg}


\end{document}
