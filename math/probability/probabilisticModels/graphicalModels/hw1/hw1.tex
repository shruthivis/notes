\documentclass{article}
% Configuration for the memoir class.
\renewcommand{\cleardoublepage}{}
% \renewcommand*{\partpageend}{}
\renewcommand{\afterpartskip}{}
\maxsecnumdepth{subsubsection} % number subsections
\maxtocdepth{subsubsection}

\addtolength{\parindent}{-5mm}
% Packages not included:
% For multiline comments, use caption package. But this conflicts with hyperref while making html files.
% subfigure conflicts with use with memoir style-sheet.

% Use something like:
% % Use something like:
% % Use something like:
% \input{../../macros}

% groupings of objects.
\newcommand{\set}[1]{\left\{ #1 \right\}}
\newcommand{\seq}[1]{\left(#1\right)}
\newcommand{\ang}[1]{\langle#1\rangle}
\newcommand{\tuple}[1]{\left(#1\right)}

% numerical shortcuts.
\newcommand{\abs}[1]{\left| #1\right|}
\newcommand{\floor}[1]{\left\lfloor #1 \right\rfloor}
\newcommand{\ceil}[1]{\left\lceil #1 \right\rceil}

% linear algebra shortcuts.
\newcommand{\change}{\Delta}
\newcommand{\norm}[1]{\left\| #1\right\|}
\newcommand{\dprod}[1]{\langle#1\rangle}
\newcommand{\linspan}[1]{\langle#1\rangle}
\newcommand{\conj}[1]{\overline{#1}}
\newcommand{\gradient}{\nabla}
\newcommand{\der}{\frac{d}{dx}}
\newcommand{\lap}{\Delta}
\newcommand{\kron}{\otimes}
\newcommand{\nperp}{\nvdash}

\newcommand{\mat}[1]{\left( \begin{smallmatrix}#1 \end{smallmatrix} \right)}

% derivatives and limits
\newcommand{\partder}[2]{\frac{\partial #1}{\partial #2}}
\newcommand{\partdern}[3]{\frac{\partial^{#3} #1}{\partial #2^{#3}}}

% Arrows
\newcommand{\diverge}{\nearrow}
\newcommand{\notto}{\nrightarrow}
\newcommand{\up}{\uparrow}
\newcommand{\down}{\downarrow}
% gets and gives are defined!

% ordering operators
\newcommand{\oleq}{\preceq}
\newcommand{\ogeq}{\succeq}

% programming and logic operators
\newcommand{\dfn}{:=}
\newcommand{\assign}{:=}
\newcommand{\co}{\ co\ }
\newcommand{\en}{\ en\ }


% logic operators
\newcommand{\xor}{\oplus}
\newcommand{\Land}{\bigwedge}
\newcommand{\Lor}{\bigvee}
\newcommand{\finish}{$\Box$}
\newcommand{\contra}{\Rightarrow \Leftarrow}
\newcommand{\iseq}{\stackrel{_?}{=}}


% Set theory
\newcommand{\symdiff}{\Delta}
\newcommand{\union}{\cup}
\newcommand{\inters}{\cap}
\newcommand{\Union}{\bigcup}
\newcommand{\Inters}{\bigcap}
\newcommand{\nullSet}{\phi}

% graph theory
\newcommand{\nbd}{\Gamma}

% Script alphabets
% For reals, use \Re

% greek letters
\newcommand{\eps}{\epsilon}
\newcommand{\del}{\delta}
\newcommand{\ga}{\alpha}
\newcommand{\gb}{\beta}
\newcommand{\gd}{\del}
\newcommand{\gf}{\phi}
\newcommand{\gF}{\Phi}
\newcommand{\gl}{\lambda}
\newcommand{\gm}{\mu}
\newcommand{\gn}{\nu}
\newcommand{\gr}{\rho}
\newcommand{\gs}{\sigma}
\newcommand{\gt}{\theta}
\newcommand{\gx}{\xi}

\newcommand{\sw}{\sigma}
\newcommand{\SW}{\Sigma}
\newcommand{\ew}{\lambda}
\newcommand{\EW}{\Lambda}

\newcommand{\Del}{\Delta}
\newcommand{\gD}{\Delta}
\newcommand{\gG}{\Gamma}
\newcommand{\gO}{\Omega}
\newcommand{\gL}{\Lambda}
\newcommand{\gS}{\Sigma}

% Formatting shortcuts
\newcommand{\red}[1]{\textcolor{red}{#1}}
\newcommand{\blue}[1]{\textcolor{blue}{#1}}
\newcommand{\htext}[2]{\texorpdfstring{#1}{#2}}

% Statistics
\newcommand{\distr}{\sim}
\newcommand{\stddev}{\sigma}
\newcommand{\covmatrix}{\Sigma}
\newcommand{\mean}{\mu}
\newcommand{\param}{\gt}
\newcommand{\ftr}{\phi}

% General utility
\newcommand{\todo}[1]{\footnote{TODO: #1}}
\newcommand{\exclaim}[1]{{\textbf{\textit{#1}}}}
\newcommand{\tbc}{[\textbf{Incomplete}]}
\newcommand{\chk}{[\textbf{Check}]}
\newcommand{\oprob}{[\textbf{OP}]:}
\newcommand{\core}[1]{\textbf{Core Idea:}}
\newcommand{\why}{[\textbf{Find proof}]}
\newcommand{\opt}[1]{\textit{#1}}


\DeclareMathOperator*{\argmin}{arg\,min}
\DeclareMathOperator{\rank}{rank}
\newcommand{\redcol}[1]{\textcolor{red}{#1}}
\newcommand{\bluecol}[1]{\textcolor{blue}{#1}}
\newcommand{\greencol}[1]{\textcolor{green}{#1}}


\renewcommand{\~}{\htext{$\sim$}{~}}


% groupings of objects.
\newcommand{\set}[1]{\left\{ #1 \right\}}
\newcommand{\seq}[1]{\left(#1\right)}
\newcommand{\ang}[1]{\langle#1\rangle}
\newcommand{\tuple}[1]{\left(#1\right)}

% numerical shortcuts.
\newcommand{\abs}[1]{\left| #1\right|}
\newcommand{\floor}[1]{\left\lfloor #1 \right\rfloor}
\newcommand{\ceil}[1]{\left\lceil #1 \right\rceil}

% linear algebra shortcuts.
\newcommand{\change}{\Delta}
\newcommand{\norm}[1]{\left\| #1\right\|}
\newcommand{\dprod}[1]{\langle#1\rangle}
\newcommand{\linspan}[1]{\langle#1\rangle}
\newcommand{\conj}[1]{\overline{#1}}
\newcommand{\gradient}{\nabla}
\newcommand{\der}{\frac{d}{dx}}
\newcommand{\lap}{\Delta}
\newcommand{\kron}{\otimes}
\newcommand{\nperp}{\nvdash}

\newcommand{\mat}[1]{\left( \begin{smallmatrix}#1 \end{smallmatrix} \right)}

% derivatives and limits
\newcommand{\partder}[2]{\frac{\partial #1}{\partial #2}}
\newcommand{\partdern}[3]{\frac{\partial^{#3} #1}{\partial #2^{#3}}}

% Arrows
\newcommand{\diverge}{\nearrow}
\newcommand{\notto}{\nrightarrow}
\newcommand{\up}{\uparrow}
\newcommand{\down}{\downarrow}
% gets and gives are defined!

% ordering operators
\newcommand{\oleq}{\preceq}
\newcommand{\ogeq}{\succeq}

% programming and logic operators
\newcommand{\dfn}{:=}
\newcommand{\assign}{:=}
\newcommand{\co}{\ co\ }
\newcommand{\en}{\ en\ }


% logic operators
\newcommand{\xor}{\oplus}
\newcommand{\Land}{\bigwedge}
\newcommand{\Lor}{\bigvee}
\newcommand{\finish}{$\Box$}
\newcommand{\contra}{\Rightarrow \Leftarrow}
\newcommand{\iseq}{\stackrel{_?}{=}}


% Set theory
\newcommand{\symdiff}{\Delta}
\newcommand{\union}{\cup}
\newcommand{\inters}{\cap}
\newcommand{\Union}{\bigcup}
\newcommand{\Inters}{\bigcap}
\newcommand{\nullSet}{\phi}

% graph theory
\newcommand{\nbd}{\Gamma}

% Script alphabets
% For reals, use \Re

% greek letters
\newcommand{\eps}{\epsilon}
\newcommand{\del}{\delta}
\newcommand{\ga}{\alpha}
\newcommand{\gb}{\beta}
\newcommand{\gd}{\del}
\newcommand{\gf}{\phi}
\newcommand{\gF}{\Phi}
\newcommand{\gl}{\lambda}
\newcommand{\gm}{\mu}
\newcommand{\gn}{\nu}
\newcommand{\gr}{\rho}
\newcommand{\gs}{\sigma}
\newcommand{\gt}{\theta}
\newcommand{\gx}{\xi}

\newcommand{\sw}{\sigma}
\newcommand{\SW}{\Sigma}
\newcommand{\ew}{\lambda}
\newcommand{\EW}{\Lambda}

\newcommand{\Del}{\Delta}
\newcommand{\gD}{\Delta}
\newcommand{\gG}{\Gamma}
\newcommand{\gO}{\Omega}
\newcommand{\gL}{\Lambda}
\newcommand{\gS}{\Sigma}

% Formatting shortcuts
\newcommand{\red}[1]{\textcolor{red}{#1}}
\newcommand{\blue}[1]{\textcolor{blue}{#1}}
\newcommand{\htext}[2]{\texorpdfstring{#1}{#2}}

% Statistics
\newcommand{\distr}{\sim}
\newcommand{\stddev}{\sigma}
\newcommand{\covmatrix}{\Sigma}
\newcommand{\mean}{\mu}
\newcommand{\param}{\gt}
\newcommand{\ftr}{\phi}

% General utility
\newcommand{\todo}[1]{\footnote{TODO: #1}}
\newcommand{\exclaim}[1]{{\textbf{\textit{#1}}}}
\newcommand{\tbc}{[\textbf{Incomplete}]}
\newcommand{\chk}{[\textbf{Check}]}
\newcommand{\oprob}{[\textbf{OP}]:}
\newcommand{\core}[1]{\textbf{Core Idea:}}
\newcommand{\why}{[\textbf{Find proof}]}
\newcommand{\opt}[1]{\textit{#1}}


\DeclareMathOperator*{\argmin}{arg\,min}
\DeclareMathOperator{\rank}{rank}
\newcommand{\redcol}[1]{\textcolor{red}{#1}}
\newcommand{\bluecol}[1]{\textcolor{blue}{#1}}
\newcommand{\greencol}[1]{\textcolor{green}{#1}}


\renewcommand{\~}{\htext{$\sim$}{~}}


% groupings of objects.
\newcommand{\set}[1]{\left\{ #1 \right\}}
\newcommand{\seq}[1]{\left(#1\right)}
\newcommand{\ang}[1]{\langle#1\rangle}
\newcommand{\tuple}[1]{\left(#1\right)}

% numerical shortcuts.
\newcommand{\abs}[1]{\left| #1\right|}
\newcommand{\floor}[1]{\left\lfloor #1 \right\rfloor}
\newcommand{\ceil}[1]{\left\lceil #1 \right\rceil}

% linear algebra shortcuts.
\newcommand{\change}{\Delta}
\newcommand{\norm}[1]{\left\| #1\right\|}
\newcommand{\dprod}[1]{\langle#1\rangle}
\newcommand{\linspan}[1]{\langle#1\rangle}
\newcommand{\conj}[1]{\overline{#1}}
\newcommand{\gradient}{\nabla}
\newcommand{\der}{\frac{d}{dx}}
\newcommand{\lap}{\Delta}
\newcommand{\kron}{\otimes}
\newcommand{\nperp}{\nvdash}

\newcommand{\mat}[1]{\left( \begin{smallmatrix}#1 \end{smallmatrix} \right)}

% derivatives and limits
\newcommand{\partder}[2]{\frac{\partial #1}{\partial #2}}
\newcommand{\partdern}[3]{\frac{\partial^{#3} #1}{\partial #2^{#3}}}

% Arrows
\newcommand{\diverge}{\nearrow}
\newcommand{\notto}{\nrightarrow}
\newcommand{\up}{\uparrow}
\newcommand{\down}{\downarrow}
% gets and gives are defined!

% ordering operators
\newcommand{\oleq}{\preceq}
\newcommand{\ogeq}{\succeq}

% programming and logic operators
\newcommand{\dfn}{:=}
\newcommand{\assign}{:=}
\newcommand{\co}{\ co\ }
\newcommand{\en}{\ en\ }


% logic operators
\newcommand{\xor}{\oplus}
\newcommand{\Land}{\bigwedge}
\newcommand{\Lor}{\bigvee}
\newcommand{\finish}{$\Box$}
\newcommand{\contra}{\Rightarrow \Leftarrow}
\newcommand{\iseq}{\stackrel{_?}{=}}


% Set theory
\newcommand{\symdiff}{\Delta}
\newcommand{\union}{\cup}
\newcommand{\inters}{\cap}
\newcommand{\Union}{\bigcup}
\newcommand{\Inters}{\bigcap}
\newcommand{\nullSet}{\phi}

% graph theory
\newcommand{\nbd}{\Gamma}

% Script alphabets
% For reals, use \Re

% greek letters
\newcommand{\eps}{\epsilon}
\newcommand{\del}{\delta}
\newcommand{\ga}{\alpha}
\newcommand{\gb}{\beta}
\newcommand{\gd}{\del}
\newcommand{\gf}{\phi}
\newcommand{\gF}{\Phi}
\newcommand{\gl}{\lambda}
\newcommand{\gm}{\mu}
\newcommand{\gn}{\nu}
\newcommand{\gr}{\rho}
\newcommand{\gs}{\sigma}
\newcommand{\gt}{\theta}
\newcommand{\gx}{\xi}

\newcommand{\sw}{\sigma}
\newcommand{\SW}{\Sigma}
\newcommand{\ew}{\lambda}
\newcommand{\EW}{\Lambda}

\newcommand{\Del}{\Delta}
\newcommand{\gD}{\Delta}
\newcommand{\gG}{\Gamma}
\newcommand{\gO}{\Omega}
\newcommand{\gL}{\Lambda}
\newcommand{\gS}{\Sigma}

% Formatting shortcuts
\newcommand{\red}[1]{\textcolor{red}{#1}}
\newcommand{\blue}[1]{\textcolor{blue}{#1}}
\newcommand{\htext}[2]{\texorpdfstring{#1}{#2}}

% Statistics
\newcommand{\distr}{\sim}
\newcommand{\stddev}{\sigma}
\newcommand{\covmatrix}{\Sigma}
\newcommand{\mean}{\mu}
\newcommand{\param}{\gt}
\newcommand{\ftr}{\phi}

% General utility
\newcommand{\todo}[1]{\footnote{TODO: #1}}
\newcommand{\exclaim}[1]{{\textbf{\textit{#1}}}}
\newcommand{\tbc}{[\textbf{Incomplete}]}
\newcommand{\chk}{[\textbf{Check}]}
\newcommand{\oprob}{[\textbf{OP}]:}
\newcommand{\core}[1]{\textbf{Core Idea:}}
\newcommand{\why}{[\textbf{Find proof}]}
\newcommand{\opt}[1]{\textit{#1}}


\DeclareMathOperator*{\argmin}{arg\,min}
\DeclareMathOperator{\rank}{rank}
\newcommand{\redcol}[1]{\textcolor{red}{#1}}
\newcommand{\bluecol}[1]{\textcolor{blue}{#1}}
\newcommand{\greencol}[1]{\textcolor{green}{#1}}


\renewcommand{\~}{\htext{$\sim$}{~}}

% Use something like:
% % Use something like:
% % Use something like:
% \input{../../amsartMacros}

\newtheorem{thm}{Theorem}[subsection]
\newtheorem{cor}[thm]{Corollary}
\newtheorem{lem}[thm]{Lemma}
\newtheorem{fact}[thm]{Fact}
\newtheorem{claim}[thm]{Claim}

\newtheorem{assumption}[thm]{Assumption}

\newtheorem{defn}[thm]{Definition}

\theoremstyle{remark}
\newtheorem{alg}[thm]{Algorithm}
\newtheorem*{notation}{Notation}
\newtheorem*{rem}{Remark}
\newtheorem*{hint}{Hint}
\newtheorem*{ack}{Acknowledgement}

\newtheorem{question}[thm]{Question}
\newtheorem{answer}[thm]{Answer}


\newtheorem{thm}{Theorem}[subsection]
\newtheorem{cor}[thm]{Corollary}
\newtheorem{lem}[thm]{Lemma}
\newtheorem{fact}[thm]{Fact}
\newtheorem{claim}[thm]{Claim}

\newtheorem{assumption}[thm]{Assumption}

\newtheorem{defn}[thm]{Definition}

\theoremstyle{remark}
\newtheorem{alg}[thm]{Algorithm}
\newtheorem*{notation}{Notation}
\newtheorem*{rem}{Remark}
\newtheorem*{hint}{Hint}
\newtheorem*{ack}{Acknowledgement}

\newtheorem{question}[thm]{Question}
\newtheorem{answer}[thm]{Answer}


\newtheorem{thm}{Theorem}[subsection]
\newtheorem{cor}[thm]{Corollary}
\newtheorem{lem}[thm]{Lemma}
\newtheorem{fact}[thm]{Fact}
\newtheorem{claim}[thm]{Claim}

\newtheorem{assumption}[thm]{Assumption}

\newtheorem{defn}[thm]{Definition}

\theoremstyle{remark}
\newtheorem{alg}[thm]{Algorithm}
\newtheorem*{notation}{Notation}
\newtheorem*{rem}{Remark}
\newtheorem*{hint}{Hint}
\newtheorem*{ack}{Acknowledgement}

\newtheorem{question}[thm]{Question}
\newtheorem{answer}[thm]{Answer}

\lstset{language=matlab}


%opening
\title{Graphical Models: Homework 1}
\author{vishvAs vAsuki}

\begin{document}

\maketitle

\begin{notation}
V represent the set of vertices. E represents the set of edges.
\end{notation}


\section{Topological Numbering}
\begin{lem}
If a directed graph G is acyclic, it has a topological numbering.
\end{lem}
\begin{proof}
Take a DAG G = (V, E). We show that G has a topological numbering by constructing one.

For every node $v \in V$, define level l(v) to be the length of the longest directed path ending in v. If there are no directed paths terminating in v, define l(v) = 0.

Since there are a finite number of directed paths of length n or less in a DAG, and because there are no cycles in G, one can find $l(v) \forall v \in V$. Observe that, for any $u \in V$, there can only be an incoming path from v if $l(v) < l(u)$: otherwise, l(u) would be atleast l(v)+1.

\begin{algorithm}[h!]
   \caption{Topological Numbering Generator}
   \label{alg:top}
\begin{algorithmic}
   \STATE {\bfseries Input:} Graph G.
   \STATE Find $l(v) \forall v \in V$, set $i \assign 0$, $n \assign 1$.
   \REPEAT
   \STATE Take $S = \set{v: v \in V, l(v) = i}$. Assign topological numbers from $[n, n+|S|-1] \inters N$ arbitrarily to nodes in S.
   \STATE $i \assign i+1, n \assign n+|S|-1$.
   \UNTIL{i = n}
\end{algorithmic}
\end{algorithm}

Algorithm \ref{alg:top} produces a valid topological ordering, because, given a node v, no ancestor u of v has a gigher topological number assigned to it. This is inturn because u would have a lower level, and Algorithm \ref{alg:top} would have assigned it a smaller topological number due to its definition.

\end{proof}

\begin{lem}
If a directed graph G is cyclic, it does not have a topological numbering.
\end{lem}
\begin{proof}
Proof by contradiction.

Suppose that there were a topological ordering $f:V \to N$. Take two nodes u, v, paricipating in a cycle in G. Because of our assumption that f is a topological ordering, $f(u) < f(v)$ as there is a path from u to v. By the same assumption, $f(v) < f(u)$ as there is a path from u to v. Since both of these cannot be true, our supposition that there is a topological ordering f must have been wrong.
\end{proof}

\begin{thm}
 A directed graph has a topological numbering if and only if it is acyclic.
\end{thm}
\begin{proof}
This follows from the lemmata proved.
\end{proof}


\section{Consistency}
\subsection{Problem setup}
Take a DAG G. Let \gp(i) \dfn parents of i. For each i, let $f_i(x_i, x_{\gp(i)})$ be such that $\sum_{x_i} f_i(x_i, x_{\gp(i)}) = 1, f_i(x_i, x_{\gp(i)}) \geq 0$. p(x) \dfn $\prod_i f_i(x_i, x_{\gp(i)})$.

\begin{thm}
\label{thm_DAGSummation}
p(x) is a probability distribution, ie: $\sum_x p(x) = 1$.
\end{thm}
\begin{proof}
Consider the following expression.
\begin{eqnarray*}
\sum_x p(x) &=& \sum_x \prod_i f_i(x_i, x_{\gp(i)})\\
\end{eqnarray*}

We prove that this sums to 1 using induction on decreasing topological numbers. Take a topological numbering $t:V \to N$ for G. Let $x_{[i]}$ represent $x_s : t(x_s) = i$.

Let $g(s) \dfn \sum_{x_{[s]} .. x_{[n]}} \prod_i f_i(x_i, x_{\gp(i)})$. We know that g(n) = 1 from the definition of $f_i$. Make the inductive assumption that g(s) = 1. If this is the case, we see that g(s-1) = 1, as follows.

$g(s-1) = \sum_{x_{[s-1]}} f_{j=t^{-1}(s)}(x_j, x_{\gp(j)})\sum_{x_{[s]} .. x_{[n]}}\prod_{i = t^{-1}(u): \forall u \geq s} f_i(x_i, x_{\gp(i)}) = \sum_{x_{[s-1]}} f_{j=t^{-1}(s)}(x_j, x_{\gp(j)}) = 1$. Hence, by the principle of mathematical induction, we see that $g(1) = \sum_x p(x) = 1$.
\end{proof}

\begin{notation}
As we have proved p() to be a probability distribution, we will begin using Pr().
\end{notation}


\begin{lem}
$Pr(x_i|x_{\gp_i}) = f_i(x_i, x_{\gp_i}) \forall i$.
\end{lem}
\begin{proof}
Take a topological numbering $t:V \to N$ for G. For any i, we can sum out all $x_{j: t(j)> t(i)}$, as in the previous proof. We use this below.

\begin{eqnarray*}
Pr(x_{j:t(j) \leq t(i)}) &=& \prod_{j: t(j) \leq t(i)} f_j(x_j, x_{\gp_j})\\
Pr(x_{j:t(j) < t(i)}) &=& \prod_{j: t(j) < t(i)} f_j(x_j, x_{\gp_j})\\
\therefore Pr(x_i|x_{j:t(j) < t(i)}) &=& \frac{Pr(x_{j:t(j) \leq t(i)})}{Pr(x_{j:t(j) < t(i)})}\\
&=& f_i(x_i, x_{\gp_i})\\
\end{eqnarray*}
But, observe that, irrespective of what values $x_{j:t(j) < t(i) \land j \notin \gp_i}$, \\
$Pr(x_i|x_{j:t(j) < t(i)}) = f_i(x_i, x_{\gp_i})$, so $x_i \perp x_{j:t(j) < t(i) \land j \notin \gp_i}$. So, $Pr(x_i|x_{\gp_i}) = Pr(x_i|x_{j:t(j) < t(i)}) = f_i(x_i, x_{\gp_i})$
\end{proof}


\section{Tree Model Factorization}
\subsection{Problem setup}
Undirected graphical model T is a tree. $Pr(x)$ associated with T. $Pr(x_i) \dfn$ marginal probability for random variable i.

\begin{thm}
$$Pr(x) = \prod_i Pr(x_i) \prod_{i,j \in T}\frac{Pr(x_i, x_j)}{Pr(x_i)Pr(x_j)}$$.
\end{thm}
\begin{proof}
We prove this by induction on the size of the tree. The statement is obviously true for a tree of size 1. Assume that it is true for any T with $|V| = n$.

Consider T' with $|V| = n+1$. Pick any (leaf) node u with just 1 edge (u, v). Decompose T' into $T = T - u$ with n nodes, and u. Let $x_{T}$ represent values of variables represented in T.

We have proved in solution to another problem that, \\
$X \perp Y|Z \equiv Pr(x, y, z) = \frac{Pr(x, z)Pr(y, z)}{Pr(z)}$. We use this here. As $u \perp (T-v)|v$, we have 
\begin{eqnarray*}
Pr(X_{T'}) &=& \frac{Pr(X_T)Pr(u,v)}{Pr(v)} \\
&=& \frac{Pr(u, v)}{Pr(v)}\prod_{i \in T} Pr(x_i) \prod_{i,j \in T}\frac{Pr(x_i, x_j)}{Pr(x_i)Pr(x_j)} \\
&=& \prod_{i \in T'} Pr(x_i) \prod_{i,j \in T'}\frac{Pr(x_i, x_j)}{Pr(x_i)Pr(x_j)} \\
\end{eqnarray*}

\end{proof}


\section{Conditional Independence}
\subsection{a}
\begin{claim}
$X \perp Y \implies X \perp Y|Z$ is false.
\end{claim}
\begin{proof}
Take Z = X + Y with $Y, X \distr u[0,1]$.
\end{proof}


\subsection{b}
\begin{claim}
$X \perp Y|W \land X \perp Z|W \implies X \perp (Y, Z)|W$ is false.
\end{claim}
\begin{proof}
Take X = (Y+Z)W with $Y, Z, W \distr u[0,1]$.
\end{proof}


\subsection{c}
\begin{claim}
$X \perp (Y, Z) |W  \implies X \perp Y | W$ is true.
\end{claim}
\begin{proof}
\begin{eqnarray*}
X \perp (Y, Z) |W  &\implies&\\
Pr(X, Y, Z|W) &=& Pr(X|W) Pr(Y, Z|W)\\ 
\therefore \sum_Z Pr(X, Y, Z|W) &=& Pr(X|W) \sum_Z Pr(Y, Z|W)\\ 
\therefore Pr(X, Y|W) &=& Pr(X|W) Pr(Y|W)\\ 
&\implies& X \perp Y | W\\
\end{eqnarray*}
\end{proof}


\subsection{d}
\begin{claim}
$X \perp Y |Z \land W = f(X) \implies X \perp Y | Z, W$ is true.
\end{claim}
\begin{proof}
\begin{eqnarray*}
X \perp Y |Z  &\implies&\\
Pr(X, Y|Z) &=& Pr(X|Z) Pr(Y|Z)\\ 
\frac{Pr(X, Y|Z)}{Pr(W)} &=& \frac{Pr(X|Z) Pr(Y|Z)}{Pr(W)}\\ 
Pr(X, Y|Z,W) &=& Pr(X|Z,W) Pr(Y|Z)\\ 
\end{eqnarray*}

But, as W = f(X), $(Z, Y) \perp W | X$; so, $Pr(W|X, Y, Z) = Pr(W|X)$. Also $(Z) \perp W | X$; so, $Pr(W|X, Z) = Pr(W|X)$. We use these below.
\begin{eqnarray*}
Pr(Y, W|Z) &=& \sum_X Pr(X, Y, W | Z)\\
&=& \sum_X Pr(X, Y| Z)Pr(W|X, Y, Z)\\
&=& \sum_X Pr(Y| Z)Pr(X| Z)Pr(W|X)\\
&=& Pr(Y| Z)\sum_X Pr(X| Z)Pr(W|X, Z)\\
&=& Pr(Y| Z)Pr(W|Z)\\
\end{eqnarray*}

So, $Pr(Y|Z) = Pr(Y|Z, W)$. Using this in the earlier equation:
\begin{eqnarray*}
Pr(X, Y|Z,W) &=& Pr(X|Z,W) Pr(Y|Z)\\
&=& Pr(X|Z,W) Pr(Y|Z, W)\\
\end{eqnarray*}
This is what we wanted to show.
\end{proof}


\subsection{e}
\begin{claim}
$X \perp Y|Z \land X \perp Y | (W, Z) \implies X \perp (Y, W)|Z$ is true.
\end{claim}
\begin{proof}
\begin{eqnarray*}
Pr(X,Y,W|Z) &=& Pr(X|Z)Pr(Y|X, Z)Pr(W|X, Y, Z)\\
&=& Pr(X|Z)Pr(Y|Z)Pr(W|Y, Z)\texttt{ Using $\perp$ conditions.}\\
&=& Pr(X|Z)Pr(Y, W|Z)\\
\end{eqnarray*}
This is what we wanted to show.
\end{proof}

\section{Factorization}
\subsection{1}
\begin{lem}
If $X \perp Y | Z$, Pr(x, y, z) = Pr(x, z)Pr(y,z)/ Pr(z).
\end{lem}
\begin{proof}
If $X \perp Y | Z$, $Pr(y|x,z) = Pr(y|z)$. We use this below.

\begin{eqnarray*}
Pr(x, y, z) &=& Pr(x, z)Pr(y|x, z)\\
&=& Pr(x, z)Pr(y|x, z)\\
&=& Pr(x, z)Pr(y|z)\\
&=& Pr(x, z)Pr(y,z)/ Pr(z)\\
\end{eqnarray*}
\end{proof}

\begin{lem}
If Pr(x, y, z) = Pr(x, z)Pr(y,z)/ Pr(z), $X \perp Y | Z$.
\end{lem}
\begin{proof}
\begin{eqnarray*}
Pr(x, y, z) &=& Pr(x, z)Pr(y,z)/ Pr(z)\\
Pr(x, z)Pr(y|x, z) &=& Pr(x, z)Pr(y,z)/ Pr(z)\\
\therefore Pr(y|x, z) &=& Pr(y|z)\\
\therefore X \perp Y | Z\\
\end{eqnarray*}
\end{proof}

\subsection{2}
\begin{lem}
If $X \perp Y | Z$, Pr(x, y, z) = f(x, z)g(y, z).
\end{lem}
\begin{proof}
We showed in answer to an earlier question: Pr(x, y, z) = Pr(x, z)Pr(y,z)/ Pr(z). Take f(x, z) = Pr(x, z)/ Pr(z), g(y, z) = Pr(y,z).
\end{proof}

\begin{lem}
If Pr(x, y, z) = f(x, z)g(y, z), $X \perp Y | Z$.
\end{lem}
\begin{proof}
\begin{eqnarray*}
\frac{Pr(x, y, z)}{Pr(z)} &=& f(x, z)g(y, z)/Pr(z) \\
Pr(x, y | z) &=& f(x, z)g(y, z)/Pr(z)\\
\therefore Pr(y | z) &=& (\sum_x f(x, z))g(y, z)/Pr(z)\\
\therefore Pr(x | z) &=& f(x, z) (\sum_y g(y, z))/Pr(z)\\
\therefore Pr(x | z)Pr(y|z) &=& \frac{f(x, z)g(y, z)}{Pr(z)Pr(z)} (\sum_y g(y, z))(\sum_x f(x, z))\\
&=& \frac{f(x, z)g(y, z)}{Pr(z)Pr(z)} (\sum_{x,y} g(y, z) f(x, z))\\
&=& \frac{f(x, z)g(y, z)}{Pr(z)Pr(z)} (\sum_{x,y} Pr(x, y, z))\\
&=& \frac{f(x, z)g(y, z)}{Pr(z)}\\
&=& Pr(x, y | z)\\
\end{eqnarray*}
\end{proof}



\section{Positive Density}
\begin{thm}
$Pr(x, y, z) > 0$. $X \perp Y | Z \land X \perp Z|Y \implies X\perp (Y, Z)$.
\end{thm}
\begin{proof}
\tbc
\end{proof}

\begin{rem}
When the condition $Pr(x, y, z) > 0$ is relaxed, this does not necessarily hold. Consider: $X = Y + Z$, with $Y, Z \distr u[0,1]$.
\end{rem}


\section{Separation Example}
\subsection{Problem setup}
A graph was given, which is not reproduced here.

\subsection{Independent pairs of RV's}
It is the union of the following sets.

$\set{\set{u, v}: u \in \set{1, 6, 4}, v \in V - \set{1, 6, 4}}$.

$\set{\set{u, v}: u \in \set{2, 10, 3}, v \in \set{7, 8}}$.

$\set{\set{7, 8}}$.

\subsection{Independence from 1 given 2, 9}
$\set{7, 10, 3, 5, 8}$.

\subsection{Independence from 8 given 2, 9}
$\set{5, 6, 1}$.

\section{Bayesian Network Marginalization}
\begin{thm}
Let G = (V, E) be the DAG corresponding to Pr(x). The DAG corresponding to $Pr(x_{V-A})$ is obtained as follows: Take subgraph S in G induced by (V - A). For every $(u, v) \in (V-A)^{2}$, add a new edge if $\exists$ a directed path (u, s, v) in G, such that s is a sequence of vertices in A.
\end{thm}
\begin{proof}
S corresponds to $Pr(x_{V-A})$ iff the factorization obtained by summing out $x_A$ from factorization of Pr(x) according to G corresponds to the form derived using S. We show that this is the case.

\begin{eqnarray*}
Pr(x_{V-A}) &=& \sum_{x_A} Pr(x)\\
&=& \sum_{x_A} \prod_{i} Pr(x_i|\gp_i)\\
\end{eqnarray*}

Take a topological numbering t on V. All variables in A which are downstream from every node in V-A can be removed from the summation using the same logic used in proving theorem \ref{thm_DAGSummation} earlier.

Variables in A which are not downstream from any variables in V-A can be eliminated from the summation, using the same strategy explained for the class of nodes below.

Now, only nodes in A which lie on some directed path (u, s, v) in G, so that s is a sequence of nodes in A, remain to be removed. Let par(s) be the parents of s in V-A. Now, the factor for v can be replaced with $Pr(u, s|par(s))$, summing s out, we end up with the factor $Pr(u|par(s))$.
\end{proof}

% \bibliographystyle{plain}
% \bibliography{../linAlg}

\end{document}
