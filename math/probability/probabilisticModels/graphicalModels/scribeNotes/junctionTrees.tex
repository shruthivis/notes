\documentclass[12pt]{report}
\usepackage{sparsity_structures_and_algorithms_10, epsfig, hyperref}

\usepackage{tikz}
\usetikzlibrary{arrows,decorations,fit,backgrounds}
\tikzstyle{ct-shared-nodes}=[rectangle, draw, fill=black!30, thick, minimum size=1mm]
\tikzstyle{ct-clique}=[circle, draw, thick, minimum size=1mm]
\tikzstyle{gm-var-hidden}=[circle, draw, thick, minimum size=1mm]

% Use something like:
% % Use something like:
% % Use something like:
% \input{../../macros}

% groupings of objects.
\newcommand{\set}[1]{\left\{ #1 \right\}}
\newcommand{\seq}[1]{\left(#1\right)}
\newcommand{\ang}[1]{\langle#1\rangle}
\newcommand{\tuple}[1]{\left(#1\right)}

% numerical shortcuts.
\newcommand{\abs}[1]{\left| #1\right|}
\newcommand{\floor}[1]{\left\lfloor #1 \right\rfloor}
\newcommand{\ceil}[1]{\left\lceil #1 \right\rceil}

% linear algebra shortcuts.
\newcommand{\change}{\Delta}
\newcommand{\norm}[1]{\left\| #1\right\|}
\newcommand{\dprod}[1]{\langle#1\rangle}
\newcommand{\linspan}[1]{\langle#1\rangle}
\newcommand{\conj}[1]{\overline{#1}}
\newcommand{\gradient}{\nabla}
\newcommand{\der}{\frac{d}{dx}}
\newcommand{\lap}{\Delta}
\newcommand{\kron}{\otimes}
\newcommand{\nperp}{\nvdash}

\newcommand{\mat}[1]{\left( \begin{smallmatrix}#1 \end{smallmatrix} \right)}

% derivatives and limits
\newcommand{\partder}[2]{\frac{\partial #1}{\partial #2}}
\newcommand{\partdern}[3]{\frac{\partial^{#3} #1}{\partial #2^{#3}}}

% Arrows
\newcommand{\diverge}{\nearrow}
\newcommand{\notto}{\nrightarrow}
\newcommand{\up}{\uparrow}
\newcommand{\down}{\downarrow}
% gets and gives are defined!

% ordering operators
\newcommand{\oleq}{\preceq}
\newcommand{\ogeq}{\succeq}

% programming and logic operators
\newcommand{\dfn}{:=}
\newcommand{\assign}{:=}
\newcommand{\co}{\ co\ }
\newcommand{\en}{\ en\ }


% logic operators
\newcommand{\xor}{\oplus}
\newcommand{\Land}{\bigwedge}
\newcommand{\Lor}{\bigvee}
\newcommand{\finish}{$\Box$}
\newcommand{\contra}{\Rightarrow \Leftarrow}
\newcommand{\iseq}{\stackrel{_?}{=}}


% Set theory
\newcommand{\symdiff}{\Delta}
\newcommand{\union}{\cup}
\newcommand{\inters}{\cap}
\newcommand{\Union}{\bigcup}
\newcommand{\Inters}{\bigcap}
\newcommand{\nullSet}{\phi}

% graph theory
\newcommand{\nbd}{\Gamma}

% Script alphabets
% For reals, use \Re

% greek letters
\newcommand{\eps}{\epsilon}
\newcommand{\del}{\delta}
\newcommand{\ga}{\alpha}
\newcommand{\gb}{\beta}
\newcommand{\gd}{\del}
\newcommand{\gf}{\phi}
\newcommand{\gF}{\Phi}
\newcommand{\gl}{\lambda}
\newcommand{\gm}{\mu}
\newcommand{\gn}{\nu}
\newcommand{\gr}{\rho}
\newcommand{\gs}{\sigma}
\newcommand{\gt}{\theta}
\newcommand{\gx}{\xi}

\newcommand{\sw}{\sigma}
\newcommand{\SW}{\Sigma}
\newcommand{\ew}{\lambda}
\newcommand{\EW}{\Lambda}

\newcommand{\Del}{\Delta}
\newcommand{\gD}{\Delta}
\newcommand{\gG}{\Gamma}
\newcommand{\gO}{\Omega}
\newcommand{\gL}{\Lambda}
\newcommand{\gS}{\Sigma}

% Formatting shortcuts
\newcommand{\red}[1]{\textcolor{red}{#1}}
\newcommand{\blue}[1]{\textcolor{blue}{#1}}
\newcommand{\htext}[2]{\texorpdfstring{#1}{#2}}

% Statistics
\newcommand{\distr}{\sim}
\newcommand{\stddev}{\sigma}
\newcommand{\covmatrix}{\Sigma}
\newcommand{\mean}{\mu}
\newcommand{\param}{\gt}
\newcommand{\ftr}{\phi}

% General utility
\newcommand{\todo}[1]{\footnote{TODO: #1}}
\newcommand{\exclaim}[1]{{\textbf{\textit{#1}}}}
\newcommand{\tbc}{[\textbf{Incomplete}]}
\newcommand{\chk}{[\textbf{Check}]}
\newcommand{\oprob}{[\textbf{OP}]:}
\newcommand{\core}[1]{\textbf{Core Idea:}}
\newcommand{\why}{[\textbf{Find proof}]}
\newcommand{\opt}[1]{\textit{#1}}


\DeclareMathOperator*{\argmin}{arg\,min}
\DeclareMathOperator{\rank}{rank}
\newcommand{\redcol}[1]{\textcolor{red}{#1}}
\newcommand{\bluecol}[1]{\textcolor{blue}{#1}}
\newcommand{\greencol}[1]{\textcolor{green}{#1}}


\renewcommand{\~}{\htext{$\sim$}{~}}


% groupings of objects.
\newcommand{\set}[1]{\left\{ #1 \right\}}
\newcommand{\seq}[1]{\left(#1\right)}
\newcommand{\ang}[1]{\langle#1\rangle}
\newcommand{\tuple}[1]{\left(#1\right)}

% numerical shortcuts.
\newcommand{\abs}[1]{\left| #1\right|}
\newcommand{\floor}[1]{\left\lfloor #1 \right\rfloor}
\newcommand{\ceil}[1]{\left\lceil #1 \right\rceil}

% linear algebra shortcuts.
\newcommand{\change}{\Delta}
\newcommand{\norm}[1]{\left\| #1\right\|}
\newcommand{\dprod}[1]{\langle#1\rangle}
\newcommand{\linspan}[1]{\langle#1\rangle}
\newcommand{\conj}[1]{\overline{#1}}
\newcommand{\gradient}{\nabla}
\newcommand{\der}{\frac{d}{dx}}
\newcommand{\lap}{\Delta}
\newcommand{\kron}{\otimes}
\newcommand{\nperp}{\nvdash}

\newcommand{\mat}[1]{\left( \begin{smallmatrix}#1 \end{smallmatrix} \right)}

% derivatives and limits
\newcommand{\partder}[2]{\frac{\partial #1}{\partial #2}}
\newcommand{\partdern}[3]{\frac{\partial^{#3} #1}{\partial #2^{#3}}}

% Arrows
\newcommand{\diverge}{\nearrow}
\newcommand{\notto}{\nrightarrow}
\newcommand{\up}{\uparrow}
\newcommand{\down}{\downarrow}
% gets and gives are defined!

% ordering operators
\newcommand{\oleq}{\preceq}
\newcommand{\ogeq}{\succeq}

% programming and logic operators
\newcommand{\dfn}{:=}
\newcommand{\assign}{:=}
\newcommand{\co}{\ co\ }
\newcommand{\en}{\ en\ }


% logic operators
\newcommand{\xor}{\oplus}
\newcommand{\Land}{\bigwedge}
\newcommand{\Lor}{\bigvee}
\newcommand{\finish}{$\Box$}
\newcommand{\contra}{\Rightarrow \Leftarrow}
\newcommand{\iseq}{\stackrel{_?}{=}}


% Set theory
\newcommand{\symdiff}{\Delta}
\newcommand{\union}{\cup}
\newcommand{\inters}{\cap}
\newcommand{\Union}{\bigcup}
\newcommand{\Inters}{\bigcap}
\newcommand{\nullSet}{\phi}

% graph theory
\newcommand{\nbd}{\Gamma}

% Script alphabets
% For reals, use \Re

% greek letters
\newcommand{\eps}{\epsilon}
\newcommand{\del}{\delta}
\newcommand{\ga}{\alpha}
\newcommand{\gb}{\beta}
\newcommand{\gd}{\del}
\newcommand{\gf}{\phi}
\newcommand{\gF}{\Phi}
\newcommand{\gl}{\lambda}
\newcommand{\gm}{\mu}
\newcommand{\gn}{\nu}
\newcommand{\gr}{\rho}
\newcommand{\gs}{\sigma}
\newcommand{\gt}{\theta}
\newcommand{\gx}{\xi}

\newcommand{\sw}{\sigma}
\newcommand{\SW}{\Sigma}
\newcommand{\ew}{\lambda}
\newcommand{\EW}{\Lambda}

\newcommand{\Del}{\Delta}
\newcommand{\gD}{\Delta}
\newcommand{\gG}{\Gamma}
\newcommand{\gO}{\Omega}
\newcommand{\gL}{\Lambda}
\newcommand{\gS}{\Sigma}

% Formatting shortcuts
\newcommand{\red}[1]{\textcolor{red}{#1}}
\newcommand{\blue}[1]{\textcolor{blue}{#1}}
\newcommand{\htext}[2]{\texorpdfstring{#1}{#2}}

% Statistics
\newcommand{\distr}{\sim}
\newcommand{\stddev}{\sigma}
\newcommand{\covmatrix}{\Sigma}
\newcommand{\mean}{\mu}
\newcommand{\param}{\gt}
\newcommand{\ftr}{\phi}

% General utility
\newcommand{\todo}[1]{\footnote{TODO: #1}}
\newcommand{\exclaim}[1]{{\textbf{\textit{#1}}}}
\newcommand{\tbc}{[\textbf{Incomplete}]}
\newcommand{\chk}{[\textbf{Check}]}
\newcommand{\oprob}{[\textbf{OP}]:}
\newcommand{\core}[1]{\textbf{Core Idea:}}
\newcommand{\why}{[\textbf{Find proof}]}
\newcommand{\opt}[1]{\textit{#1}}


\DeclareMathOperator*{\argmin}{arg\,min}
\DeclareMathOperator{\rank}{rank}
\newcommand{\redcol}[1]{\textcolor{red}{#1}}
\newcommand{\bluecol}[1]{\textcolor{blue}{#1}}
\newcommand{\greencol}[1]{\textcolor{green}{#1}}


\renewcommand{\~}{\htext{$\sim$}{~}}


% groupings of objects.
\newcommand{\set}[1]{\left\{ #1 \right\}}
\newcommand{\seq}[1]{\left(#1\right)}
\newcommand{\ang}[1]{\langle#1\rangle}
\newcommand{\tuple}[1]{\left(#1\right)}

% numerical shortcuts.
\newcommand{\abs}[1]{\left| #1\right|}
\newcommand{\floor}[1]{\left\lfloor #1 \right\rfloor}
\newcommand{\ceil}[1]{\left\lceil #1 \right\rceil}

% linear algebra shortcuts.
\newcommand{\change}{\Delta}
\newcommand{\norm}[1]{\left\| #1\right\|}
\newcommand{\dprod}[1]{\langle#1\rangle}
\newcommand{\linspan}[1]{\langle#1\rangle}
\newcommand{\conj}[1]{\overline{#1}}
\newcommand{\gradient}{\nabla}
\newcommand{\der}{\frac{d}{dx}}
\newcommand{\lap}{\Delta}
\newcommand{\kron}{\otimes}
\newcommand{\nperp}{\nvdash}

\newcommand{\mat}[1]{\left( \begin{smallmatrix}#1 \end{smallmatrix} \right)}

% derivatives and limits
\newcommand{\partder}[2]{\frac{\partial #1}{\partial #2}}
\newcommand{\partdern}[3]{\frac{\partial^{#3} #1}{\partial #2^{#3}}}

% Arrows
\newcommand{\diverge}{\nearrow}
\newcommand{\notto}{\nrightarrow}
\newcommand{\up}{\uparrow}
\newcommand{\down}{\downarrow}
% gets and gives are defined!

% ordering operators
\newcommand{\oleq}{\preceq}
\newcommand{\ogeq}{\succeq}

% programming and logic operators
\newcommand{\dfn}{:=}
\newcommand{\assign}{:=}
\newcommand{\co}{\ co\ }
\newcommand{\en}{\ en\ }


% logic operators
\newcommand{\xor}{\oplus}
\newcommand{\Land}{\bigwedge}
\newcommand{\Lor}{\bigvee}
\newcommand{\finish}{$\Box$}
\newcommand{\contra}{\Rightarrow \Leftarrow}
\newcommand{\iseq}{\stackrel{_?}{=}}


% Set theory
\newcommand{\symdiff}{\Delta}
\newcommand{\union}{\cup}
\newcommand{\inters}{\cap}
\newcommand{\Union}{\bigcup}
\newcommand{\Inters}{\bigcap}
\newcommand{\nullSet}{\phi}

% graph theory
\newcommand{\nbd}{\Gamma}

% Script alphabets
% For reals, use \Re

% greek letters
\newcommand{\eps}{\epsilon}
\newcommand{\del}{\delta}
\newcommand{\ga}{\alpha}
\newcommand{\gb}{\beta}
\newcommand{\gd}{\del}
\newcommand{\gf}{\phi}
\newcommand{\gF}{\Phi}
\newcommand{\gl}{\lambda}
\newcommand{\gm}{\mu}
\newcommand{\gn}{\nu}
\newcommand{\gr}{\rho}
\newcommand{\gs}{\sigma}
\newcommand{\gt}{\theta}
\newcommand{\gx}{\xi}

\newcommand{\sw}{\sigma}
\newcommand{\SW}{\Sigma}
\newcommand{\ew}{\lambda}
\newcommand{\EW}{\Lambda}

\newcommand{\Del}{\Delta}
\newcommand{\gD}{\Delta}
\newcommand{\gG}{\Gamma}
\newcommand{\gO}{\Omega}
\newcommand{\gL}{\Lambda}
\newcommand{\gS}{\Sigma}

% Formatting shortcuts
\newcommand{\red}[1]{\textcolor{red}{#1}}
\newcommand{\blue}[1]{\textcolor{blue}{#1}}
\newcommand{\htext}[2]{\texorpdfstring{#1}{#2}}

% Statistics
\newcommand{\distr}{\sim}
\newcommand{\stddev}{\sigma}
\newcommand{\covmatrix}{\Sigma}
\newcommand{\mean}{\mu}
\newcommand{\param}{\gt}
\newcommand{\ftr}{\phi}

% General utility
\newcommand{\todo}[1]{\footnote{TODO: #1}}
\newcommand{\exclaim}[1]{{\textbf{\textit{#1}}}}
\newcommand{\tbc}{[\textbf{Incomplete}]}
\newcommand{\chk}{[\textbf{Check}]}
\newcommand{\oprob}{[\textbf{OP}]:}
\newcommand{\core}[1]{\textbf{Core Idea:}}
\newcommand{\why}{[\textbf{Find proof}]}
\newcommand{\opt}[1]{\textit{#1}}


\DeclareMathOperator*{\argmin}{arg\,min}
\DeclareMathOperator{\rank}{rank}
\newcommand{\redcol}[1]{\textcolor{red}{#1}}
\newcommand{\bluecol}[1]{\textcolor{blue}{#1}}
\newcommand{\greencol}[1]{\textcolor{green}{#1}}


\renewcommand{\~}{\htext{$\sim$}{~}}


\begin{document}
\scribe{vishvAs vAsuki, Nikita Sudan}		% required
\lecturenumber{6}			% required, must be a number
\lecturedate{Feb 10}		% required, omit year

\maketitle

\section{Topics covered}
\begin{itemize}
 \item Inference using Clique trees.
 \item Junction trees
 \subitem Relationship with triangulated graphs.
\end{itemize}


\section{Inference on clique trees}
\subsection{Clique trees}
We studied clique trees in the previous lecture. In a clique tree, the nodes are maximal cliques of the origninal graph, together with separator nodes. An example is shown in Figure~\ref{fig:junctionTree}.

\subsection{Marginalization}
\subsubsection{An example}
Consider the graphical model in Figure~\ref{fig:G1}, and its clique tree, shown in Figure~\ref{fig:junctionTree}. We want to do marginalization by passing messages in the clique tree, to evaluate $Pr(x_2, x_3)$.

The following messages are passed.
$$m_{21 \to 2}(x_2) = \sum_{x_1}f_{21}(x_2, x_1)$$
$$m_{24 \to 2}(x_2) = \sum_{x_2}f_{24}(x_2, x_4)$$

The node 2 then combines these messages:
$$m_{2 \to 23}(x_2) = m_{21 \to 2}(x_2)m_{24 \to 2}(x_2)$$

Clique 23 generates the belief:
$$b_{23}(x_2, x_3) = f_{23}(x_2, x_3)m_{2 \to 23}(x_2)$$

If $Pr(x) = f_{12}f_{23}f_{24}$, then it is easy to see that $b_{23}(x_2, x_3) = Pr(x_2, x_3)$.

\subsubsection{Update rules and beliefs in general}
Let c be a clique, and let it be connected to separator s. Then, the update rules are as follows:

$$m_{c \to s}(x_s) = \sum_{x_{c \setminus s}} f_c(x_c) \prod_{s' \subset c,\ s' \neq s}m_{s' \to c}(x_{s'})$$

$$m_{s \to c}(x_s) = \prod_{s \subset c,\ c' \neq c}m_{c' \to s}(x_{s})$$

The beliefs are as follows:
$$b_{c}(x_c) = f_{c}(x_c)\prod_{s \subset c}m_{s \to c}(x_s)$$


\subsection{Consistency and correctness}
Each variable i can appear in multiple cliques (say c, c') of the clique tree. So, unlike in the case of sum products, we should check for consistency: the marginal of a variable, as predicted by each clique should be the same, independent of the clique. So, if $b_i(x_i) = \sum_{x_{c \setminus i}}b_c(x_c)$ and $b_i'(x_i) = \sum_{x_{c' \setminus i}}b_c'(x_c')$, we want that $\forall i: b_i(x_i) = b_i'(x_i)$.

\subsubsection{Clique trees where inference fails}
Consider the graph in figure~\ref{fig:noJunctionTree} and its clique tree, shown in figure~\ref{fig:notJunctionTree}. Let us find $b_1(x_1) = \sum_{x_2}b_{12}(x_1, x_2)$ and $b_1'(x_1) = \sum_{x_4}b_{14}(x_1, x_4)$.

Below, we use $a_1$ in the place of $x_1$ when we compute $m_{14 \to 4}(x_4)$ to emphasize the fact that $x_1$ appears again, later in the computation.
\begin{eqnarray*}
m_{4 \to 34}(x_4) &=& m_{14 \to 4}(x_4) = \sum_{a_1}f_{14}(a_1, x_4)\\
m_{34 \to 3}(x_3) &=& m_{3 \to 23}(x_3) = \sum_{x_4,a_1}f_{14}(a_1, x_4)f_{34}(x_3, x_4)\\
m_{23 \to 2}(x_2) &=& m_{2 \to 12}(x_2) = \sum_{x_3, x_4,a_1}f_{14}(a_1, x_4)f_{34}(x_3, x_4)f_{23}(x_2, x_3)\\
b_{12}(x_1, x_2) &=& f_{12}(x_1, x_2)\sum_{x_3, x_4,a_1}f_{14}(a_1, x_4)f_{34}(x_3, x_4)f_{23}(x_2, x_3)\\
\end{eqnarray*}

By symmetry, we have:
$$b_{14}(x_1, x_4) = f_{14}(x_1, x_4)\sum_{x_3, x_2,a_1}f_{14}(a_1, x_2)f_{32}(x_3, x_2)f_{43}(x_4, x_3)$$.

Observe that $b_1(x_1) = \sum_{x_2}b_{12}(x_1, x_2)$ and $b_1'(x_1) = \sum_{x_4}b_{14}(x_1, x_4)$ are not equal in general. To see this, consider the case where $f_{14}(x_1, x_4)$ is a constant with respect to $x_1$, but $f_{12}(x_1, x_2)$ depends on $x_1$.



\begin{figure}
\centering
\begin{tikzpicture}[node distance=1cm,>=stealth',bend angle=15,auto]
\node (1)[gm-var-hidden]{1};
\node (2)[gm-var-hidden, right of = 1]{2};
\node (3)[gm-var-hidden, above right of = 2]{3};
\node (4)[gm-var-hidden, below right of = 2]{4};
\path [-] (1) edge (2)
(2) edge (3)
(2)  edge (4);
\end{tikzpicture}
\caption{An undirected graphical model}
\label{fig:G1}
\end{figure}


\begin{figure}
\centering
\begin{tikzpicture}[node distance=1cm,>=stealth',bend angle=15,auto]
\node (12)[ct-clique]{12};
\node (2)[ct-shared-nodes, right of = 12]{2};
\node (23)[ct-clique, above right of = 2]{23};
\node (24)[ct-clique, below right of = 2]{24};
\path [-] (12) edge (2)
(2) edge (23)
(2)  edge (24);
\end{tikzpicture}
\caption{Clique Tree corresponding to graph in Figure \ref{fig:G1}. This has the junction tree property.}
\label{fig:junctionTree}
\end{figure}

\begin{figure}
\centering
\begin{tikzpicture}[node distance=1cm,>=stealth',bend angle=15,auto]
\node (1)[gm-var-hidden]{1};
\node (2)[gm-var-hidden, right of = 1]{2};
\node (3)[gm-var-hidden, right of = 2]{3};
\node (4)[gm-var-hidden, below of = 2]{4};
\path [-] (1) edge (2)
(2) edge (3)
(3)  edge (4)
(4)  edge (1)
;
\end{tikzpicture}
\caption{There exists no Junction Tree possible for this graph.}
\label{fig:noJunctionTree}
\end{figure}

\begin{figure}
\centering
\begin{tikzpicture}[node distance=1cm,>=stealth',bend angle=15,auto]
\node (12)[ct-clique]{12};
\node (2)[ct-shared-nodes, right of = 12]{2};
\node (23)[ct-clique, right of = 2]{23};
\node (3)[ct-shared-nodes, right of = 23]{3};
\node (34)[ct-clique, right of = 3]{34};
\node (4)[ct-shared-nodes, right of = 34]{4};
\node (41)[ct-clique, right of = 4]{41};
\path [-] (12) edge (2)
(2) edge (23)
(23) edge (3)
(3) edge (34)
(34) edge (4)
(4) edge (41);
\end{tikzpicture}
\caption{Clique tree corresponding to graph in figure~\ref{fig:noJunctionTree}. This does not have the junction tree property.}
\label{fig:notJunctionTree}
\end{figure}

\begin{figure}
\centering
\begin{tikzpicture}[node distance=1cm,>=stealth',bend angle=15,auto]
\node (1)[gm-var-hidden]{1};
\node (2)[gm-var-hidden, right of = 1]{2};
\node (3)[gm-var-hidden, right of = 2]{3};
\node (4)[gm-var-hidden, below of = 2]{4};
\path [-] (1) edge (2)
(2) edge (3)
(3)  edge (4)
(4)  edge (1)
(2)  edge (4)
;
\end{tikzpicture}
\caption{The graph in Figure~\ref{fig:noJunctionTree} altered to be chordal.}
\label{fig:G3}
\end{figure}

\begin{figure}
\centering
\begin{tikzpicture}[node distance=1cm,>=stealth',bend angle=15,auto]
\node (124)[ct-clique]{124};
\node (24)[ct-shared-nodes, right of = 124]{24};
\node (234)[ct-clique, right of = 24]{234};
\path [-] (124) edge (24)
(24) edge (234)
;
\end{tikzpicture}
\caption{Clique tree corresponding to graph in figure~\ref{fig:G3}. This has the junction tree property.}
\label{fig:CT3}
\end{figure}


\subsection{Summary and fixes}
Suppose we altered the figure in Figure~\ref{fig:noJunctionTree} to get the Figure~\ref{fig:G3}. Then, the junction tree would be Figure~\ref{fig:CT3}. Now, $Pr(x) = f_{124}(x)f_{234}(x)$, and we do not encounter the problem of a variable being 'lost' in the process of being marginalized out, and being 'found' later in computations of some other clique in the clique tree. Next, we study, the property that distinguishes the cases in Figure~\ref{fig:noJunctionTree} and Figure~\ref{fig:CT3}.

\section{Junction Tree Property}

A clique tree is said to have the junction tree property if for every pair of cliques $c_{1}$ and $c_{2}$, the nodes in $c_{1}$ and $c_{2}$ appear in \textbf{all} cliques and seperators on the \textbf{unique} path between $c_{1}$ and $c_{2}$ on the clique tree. This is equivalent to saying for every node i of original graph, the cliques and seperators containing $i$ should form a connected sub-tree of clique tree. See Figure~\ref{fig:junctionTree} for an example of a tree that satisfies the junction tree property since the index ``2'' appears in each of the three connected nodes. Figure~\ref{fig:notJunctionTree} on the other hand is not a junction tree.


\begin{itemize}
\item Clique trees with Junction property are called Junction tree.
\item Note every graph has a junction tree.
\item Not every graph has a junction tree (i.e. a clique tree with the junction tree property. For e.g. the graph in Figure~\ref{fig:noJunctionTree}
\end{itemize}

%\newtheorem{lem}[thm]{Lemma}


\begin{lemma}
Let C be a leaf-clique in clique tree. Let S be the unique seperator connected to it. Let R be the union of all cliques that are not C. Then the Junction property implies that A $\land$ R = $\emptyset$.
\end{lemma}

\begin{proof}
Suppose $v \in A \land v \in R,$\newline
$\Rightarrow$ $v$ $\in$ $C_{i\backslash s}$ for some $C_{i}$ $\neq$ $C$ $\land$ $v$ $\in$ $C$\newline
$\Rightarrow$ $v$ $\in$ $S$ by Junction Tree property\newline
$\Rightarrow$ contradiction\newline
\end{proof}

\begin{proof}
By Induction,\newline
Initially: True if the Junction Tree for $f$ contains only one clique.\newline
Inductive Hypothesis: Suppose for all $f$ with junction having $<=$ $n - 1$ cliques we have that, $b_{c}(x_{c})$ $=$ $\sum$ ${_{x_{j\backslash c}}}\tilde{f}(x)$ $\forall$ $C$. Now consider a Junction Tree with $n$ cliques and let $C_{1}$ be a leaf clique and $S$ its (unique) seperator.  Let $A$ $=$ $C_{1}\backslash S$ $\land$ $R = \bigcup _{i \neq 1} (c_{i} \backslash S)$ represent all other cliques. By lemma, $A$ $\land$ $R$ are disjoint. Also, the Junction tree parameter is $f(x) = $ $f_{A,S}(x_{A}, x_{S})$ $f_{R, S}(x_{R}, x_{S})$. 
\begin{eqnarray}
m_{s\rightarrow c_{1}} = \prod _{c_{i}\neq c_{1}} m_{c_{i} \rightarrow s}(x_{s})
\end{eqnarray}
is constant for now i.e. for the tree $f_{R,S}$ without $C_{1}$. %\newline

Let $\tilde{b}$ be the beliefs when Junction Tree algorithm is run on this tree. Then, by the inductive assumption, 
\begin{eqnarray}
\tilde{b} _{(x_{s})} = \sum _{x_{R}} f _{R,S} (x_{R}, x_{S}). 
\end{eqnarray}

Let $C_{2}$ be some clique in R that is joined to (and hence contains) S. Then,
\begin{eqnarray*}
\tilde{b}(x_{s}) = \sum _{x_{C_{2}\backslash S}} \tilde{b}(x_{C_{2}}) \nonumber\\%\newline
= [\sum _{x_{C_{2}}\backslash S} f_{C_{2}}(x_{C_{2}}) \prod _{s' \neq s} \tilde{m} _{s'\rightarrow C_{2}} (x_{s'})] \tilde{m}_{S\rightarrow C_{2}}(x_{S}) \nonumber\\%$\newline
= \tilde{m} _{C_{2} \rightarrow S} (x_{S}) \tilde{m} _{S \rightarrow C_{2}} (x_{S}) \nonumber%\newline
\end{eqnarray*}
Now, $\tilde{m} _{C_{2} \rightarrow S}(x_{S}) = m _{C_{2} \rightarrow S}(x_{S})$ i.e. the message in bigger for the full Junction Tree and 
\begin{eqnarray}
\tilde{m} _{S \rightarrow C_{2}} (x_{S}) = \prod _{i \neq 1,2} m _{C_{i} \rightarrow S} (x_{S}) \nonumber\\
\therefore  \tilde{b} (x_{S}) = \prod _{i \neq 1} m _{C_{i} \rightarrow S} (x_{S}) 
\end{eqnarray}
Combining equations 6.1, 6.2 and 6.3, we get:
\begin{eqnarray*}
m_{s\rightarrow C_{i}(x_{S})} = \sum _{x_{R}} f_{R, S} (x_{R}, x_{S})\\
\therefore b_{C_{1}} (x_{C_{1}}) = f _{A,S}(x_{A}, x_{S}) \sum _{x_{R}} f_{R,S}(x_{R},x_{S}) = \sum _{x_{v \backslash C_{1}}} f(x)
\end{eqnarray*}
\end{proof}

Q: Can we always make a Junction Tree? If not, when?

A: No, as can be seen in Figure~\ref{fig:noJunctionTree}.

\section{Triangulated Graphs}
Every cycle in a triangulated graph has a chord. Such graphs are also called chordal graphs.

\begin{theorem}
$G$ has a Junction Tree $\Leftrightarrow$ $G$ is a triangulated graph.
\end{theorem}

\begin{proof}
Using constructive procedure by building junction tree

a) Choose clique nodes such that each is a maximal clique in original graph. Now, between every pair of cliques $C_{1}$ and $C_{2}$ there is a potentially empty seperator $S = C_{1} \land C_{2}$. Let ``weight'' of this ``seperator edge'' between $C_{1}$ and $C_{2}$ be $w_{12}$ $=$ $|S| = |C_{1} \land C_{2}|.$

b) Run max weight spanning tree algorithm (on clique-tree, with edge weights as above) to obtain a clique tree. This tree is a Junction tree of graph is triangulated. Consider any node $k$ in the original graph with $m$ clique nodes, then the number of seperators,
\begin{eqnarray*}
num\_seperators = by \sum _{j = 1} ^{m - 1} 1 _{X_{k} \in S_{j}} <= [\sum _{i = 1} ^{M} 1_{X_{k} \in C_{i}}] - 1 \\
weight\_tree = \sum _{k = 1} ^{N} (num\_seperators)\\
\sum _{k = 1} ^{N} (\sum _{j = 1} ^{m - 1} 1 _{k \in S_{j}})\\
<= \sum _{k = 1} ^{N} [(\sum _{i = 1} ^{M} 1_{k in C_{i}}) - 1]\\
= \sum _{i = 1} ^{M} (\sum _{k = 1} ^{N} 1 _{k \in C_{i}}) - M\\
= \sum _{i = 1} ^{M} |C_{i}| - M
%$\Leftrightarrow$ cliques containing node k form a connected subtree in T, for every k \newline
%$\Leftrightarrow$ depth of Junction Tree
\end{eqnarray*}
The above is equivalent to the cliques containing node k which form a connected subtree in T, for every k. This is nothing but the depth of the Junction Tree.
\end{proof}
\end{document}

